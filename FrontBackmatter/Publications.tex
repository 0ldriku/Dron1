%*******************************************************
% Publications
%*******************************************************
\pdfbookmark[1]{Publications}{publications}
\chapter*{Publications}

\section*{Journal Articles}
\subsection*{Peer-Reviewed}
\noindent\hangindent=0.5in
Lu, J., \& Sato, R. (2025). Linguistic dimensions of comprehensibility and perceived fluency in L2 speech across tasks of varying complexity. \textit{Journal of Second Language Pronunciation, 11}(2), 240--266. \url{https://doi.org/10.1075/jslp.24057.lu}

\noindent\hspace{0.5in}\textit{This research forms the foundation of Chapter 5.}

\vspace{0.1in}


\noindent\hangindent=0.5in
Lu, J., \& Sato, R. (2025). Effects of domain, modality, and linguistic complexity on cognitive load and self-efficacy in learning academic content in L2. \textit{IEEE Access, 13}, 162436--162454. \url{https://doi.org/10.1109/ACCESS.2025.3610786}

\noindent\hspace{0.5in}\textit{This research forms the foundation of Chapter 8.}


\section*{Conference Presentations}

\subsection*{Peer-Reviewed}

\noindent\hangindent=0.5in
Lu, J., \& Sato, R. (2023, October 14--15). \textit{Prioritization of complexity and accuracy in L2 speech under varied cognitive load conditions} [Poster presentation]. The 23rd International Conference of the Japan Second Language Association (J-SLA 2023), Tokyo, Japan.

\noindent\hspace{0.5in}\textit{This research forms the foundation of Chapter 4.}

\vspace{0.1in}
\noindent\hangindent=0.5in
\begin{CJK}{UTF8}{ipxm}
\noindent陸嘉良・佐藤礼子. (2021, May 22--23). 認知負荷が日本語学習者の発話に与える影響:口頭流暢性の客観指標に注目して [口頭発表]. 2021年度日本語教育学会春季大会, オンライン開催.
\end{CJK}

\noindent\hspace{0.5in}\textit{This research forms the foundation of Chapter 4.}

\vspace{0.1in}

\subsection*{Non-Peer-Reviewed}

\noindent\hangindent=0.5in
Lu, J., \& Sato, R. (2024, September 7--8). \textit{CAF-Annotator: A tool for complexity, accuracy, and fluency annotation in second language acquisition research} [Oral presentation]. The 45th Annual Conference of Japan Society for Educational Technology, Online.

\noindent\hspace{0.5in}\textit{This research forms the foundation of Chapter 3.}

\vspace{0.1in}
\noindent\hangindent=0.5in
\begin{CJK}{UTF8}{ipxm}
\noindent陸嘉良・佐藤礼子. (2022, August 20--30). 認知負荷の異なる課題に対する困難さと面白さの評価:クラスタリングによる変化パターンの比較 [口頭発表]. 日本教育心理学会第64回総会, オンライン開催.
\end{CJK}

\noindent\hspace{0.5in}\textit{This research forms the foundation of Chapter 7.}

\vspace{0.1in}