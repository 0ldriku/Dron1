%*******************************************************
% Abstract
%*******************************************************
%\renewcommand{\abstractname}{Abstract}
\pdfbookmark[1]{Abstract}{Abstract}
% \addcontentsline{toc}{chapter}{\tocEntry{Abstract}}
\begingroup
\let\clearpage\relax
\let\cleardoublepage\relax
\let\cleardoublepage\relax

\chapter*{Abstract}
This thesis argues that observable breakdowns in second language (L2) performance under high cognitive load reflect principled resource allocation patterns rather than simple deficits. Two experiments comprising five studies examined how manipulations of cognitive load shape L2 production and comprehension in Chinese L1 learners of Japanese (intermediate--advanced proficiency, N = 72).

Experiment 1 manipulated intrinsic load through element interactivity in argumentative speaking tasks. Thirty-six learners performed these tasks while completing a concurrent reaction-time task to validate differences in cognitive demand. Study 1’s multivariate analyses showed that complexity, accuracy, and fluency (CAF) form a joint allocation profile: under higher load, speakers slowed down and produced shorter clauses but increased lexical density without higher error rates, indicating shifts toward protecting lexical precision and monitoring at the expense of speed and elaboration. Study 2 demonstrated that native Japanese listeners’ comprehensibility and fluency judgments track these reallocations, with evaluative cues shifting under task demands: pauses and errors mattered more under lighter demands, whereas structural clarity became more critical under heavier demands.

Experiment 2 examined academic comprehension across linguistic complexity (simple vs. complex syntax), content domain (science vs. history), and modality (text vs. video) with another 36 learners. Study 3 showed that syntactic complexity and science content jointly elevate cognitive load and depress self-efficacy, with disciplinary background modulating which factor is most taxing; load spikes predicted subsequent drops in self-efficacy and, in turn, comprehension. Study 4’s eye-tracking analyses indicated that syntactic complexity dominates processing costs, forcing slower and more locally anchored reading, while domain expertise reshapes rather than simply reduces these costs. Study 5’s pupillometry revealed conditional modality effects: video reduced physiological effort only in specific content, complexity combinations and sometimes diverged from subjective difficulty ratings.

The findings extend Cognitive Load Theory by highlighting intrinsic load as the primary bottleneck, with expertise and modality acting as modulators that change what is difficult rather than eliminating difficulty. For limited-capacity models, they show that CAF dimensions covary under load, listener judgments adapt to speaker constraints, and physiological engagement can diverge from conscious appraisal. Pedagogically, the thesis supports a hierarchical instructional approach: manage linguistic complexity first, scaffold according to learner, domain alignment, and select modality to match disciplinary training, interpreting non-native-like fluency as adaptive reallocation within a constrained system rather than straightforward deficiency.

\textbf{Below is the short version}

When second language (L2) learners speak and read under demanding conditions, their dysfluencies and comprehension problems are often treated as simple competence deficits rather than as consequences of cognitive load. Without a process-level account of how task design, linguistic complexity, expertise, and modality shape performance, theories of cognitive load and pedagogical decisions about task and material design risk misinterpreting these breakdowns and mis-targeting instruction. Using two experiments with five studies on Chinese L1 learners of Japanese (N = 72) that combine CAF analyses, native-speaker judgments, eye movements, and pupillometry, this thesis shows that observable breakdowns under higher load are better described as systematic reallocations of limited processing resources in production and comprehension than as global failures. These findings extend Cognitive Load Theory and limited-capacity models and support a hierarchical instructional approach that manages linguistic complexity first, then aligns scaffolding and modality with learners’ disciplinary expertise, interpreting deviations from native-like fluency as adaptive reallocations within a constrained system rather than straightforward deficiencies.

\vfill

\begin{otherlanguage}{ngerman}
\pdfbookmark[1]{Zusammenfassung}{Zusammenfassung}
\chapter*{Zusammenfassung}
Kurze Zusammenfassung des Inhaltes in deutscher Sprache\dots
\end{otherlanguage}

\endgroup

\vfill
