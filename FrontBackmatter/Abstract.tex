%*******************************************************
% Abstract
%*******************************************************
%\renewcommand{\abstractname}{Abstract}
\pdfbookmark[1]{Abstract}{Abstract}
% \addcontentsline{toc}{chapter}{\tocEntry{Abstract}}
\begingroup
\let\clearpage\relax
\let\cleardoublepage\relax
\let\cleardoublepage\relax

\chapter*{Abstract}
This thesis argues that observable breakdowns in second language (L2) performance under high cognitive load reflect principled resource allocation patterns rather than simple deficits. Two experiments comprising five studies examined how manipulations of cognitive load shape L2 production and comprehension in Chinese L1 learners of Japanese (intermediate--advanced proficiency, N = 72).

Experiment 1 manipulated intrinsic load through element interactivity in argumentative speaking tasks. 36 learners performed these tasks. Study 1’s multivariate analyses showed that complexity, accuracy, and fluency form a joint allocation profile: under higher load, speakers slowed down and produced shorter clauses but increased lexical density without higher error rates, indicating shifts toward protecting lexical precision and monitoring at the expense of speed and elaboration. Study 2 demonstrated that native Japanese listeners’ comprehensibility and fluency judgments track these reallocations, with evaluative cues shifting under task demands: pauses and errors mattered more under lighter demands, whereas structural clarity became more critical under heavier demands.

Experiment 2 examined academic comprehension across linguistic complexity (simple versus complex syntax), content domain (science versus history), and modality (text versus video) with another 36 learners. Study 3 showed that syntactic complexity and science content jointly elevate cognitive load and depress self-efficacy, with disciplinary background modulating which factor is most taxing; load spikes predicted subsequent drops in self-efficacy and, in turn, comprehension. Study 4’s eye-tracking analyses indicated that syntactic complexity dominates processing costs, forcing slower and more locally anchored reading, while disciplinary background reshapes rather than simply reduces these costs. Study 5’s pupillometry revealed conditional modality effects: video reduced physiological effort only in specific content, complexity combinations and sometimes diverged from subjective difficulty ratings.

The findings extend Cognitive Load Theory by highlighting intrinsic load as the primary bottleneck, with expertise and modality acting as modulators that change what is difficult rather than eliminating difficulty. Pedagogically, the thesis supports a hierarchical instructional approach: manage linguistic complexity first, scaffold according to learner, domain alignment, and select modality to match disciplinary training, interpreting non-native-like fluency as adaptive reallocation within a constrained system rather than straightforward deficiency.



\endgroup

\vfill
