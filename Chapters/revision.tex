\chapter{Introduction and Roadmap}\label{ch:intro-roadmap}
\section{Problem framing}

Millions of adults worldwide communicate daily in second languages (L2) they have not fully mastered. During production, these learners experience observable breakdowns: mid-utterance pauses during lexical retrieval, syntactic simplification, and failure to deploy acquired grammatical structures under real-time constraints. These phenomena reflect a fundamental cognitive constraint: limited working memory capacity \parencite{Cowan2001}. This thesis examines how cognitive load—the processing demand imposed on limited working memory—constrains both L2 production and comprehension. Specifically, it investigates how processing demands redistribute attention during real-time performance and how instructional design can manage these demands to support fluent, accurate, and meaning-focused performance.


The implications extend beyond individual learner experiences. Second language proficiency has become essential for educational, professional, and social participation in globalized contexts, rendering effective L2 instruction both a practical necessity and an equity concern. Traditional approaches that rely primarily on exposure may overwhelm learners whose working memory resources are simultaneously allocated to listening, planning, formulating, and monitoring output. Instruction that fails to account for these constraints may lead learners to plateau at intermediate proficiency levels or disengage entirely \parencite{Han2004}.

Cognitive load theory \parencite{sweller1988} provides a systematic framework for analyzing these constraints. Working memory—the cognitive system responsible for temporarily holding and manipulating information—has severely limited capacity. The focus of attention can maintain only three to four meaningful units simultaneously \parencite{Cowan2001}, creating a fundamental bottleneck for language processing. The empirical work presented in subsequent chapters tests predictions derived from this theoretical account. Chapter~\ref{ch:cogload} provides a detailed specification of the working memory architecture, the sources of cognitive load, and classic design effects that follow from capacity constraints.

The remainder of this chapter is organized as follows. Section~\ref{subsec:cog-framework} distinguishes the sources of cognitive load and their operation in L2 contexts. Sections~\ref{subsec:production} and~\ref{subsec:comprehension} examine load effects in L2 production and comprehension, respectively. Section~\ref{subsec:unifying} articulates the rationale for treating production and comprehension within a unified framework. Section~\ref{subsec:questions} presents the research questions that guide the empirical investigations. Section~\ref{subsec:contributions} outlines the methodological, theoretical, and practical contributions. Section~\ref{subsec:roadmap-ch} provides a chapter-by-chapter roadmap for the thesis.


\subsection{Cognitive load as framework}\label{subsec:cog-framework}

Working memory imposes severe constraints on language processing. According to \textcite{Baddeley2003}'s multicomponent model, working memory comprises a central executive that coordinates information from phonological and visuospatial subsystems. The phonological loop maintains speech-based information, the visuospatial sketchpad maintains visual-spatial information, and the central executive allocates attention and coordinates processing across these stores. The focus of attention can hold only three to four meaningful units at once \parencite{Cowan2001}, creating a fundamental bottleneck for complex cognitive tasks, including language use. When task demands exceed available capacity, performance degrades in predictable ways \parencite{sweller1988}.

Cognitive load refers to the total processing demand imposed on working memory during a learning or performance task \parencite{sweller2010}. Three categories of cognitive load are distinguished by source. \textit{Intrinsic load} stems from the inherent complexity of the material—specifically, the number of elements that must be processed simultaneously and their degree of interactivity \parencite{sweller2010}. For example, understanding a simple sentence with a single clause imposes lower intrinsic load than understanding a sentence with multiple embedded clauses. \textit{Extraneous load} originates from suboptimal instructional design that imposes unnecessary processing demands. For instance, presenting text and related diagrams in spatially separated locations forces learners to mentally integrate information across sources, consuming working memory resources without contributing to learning \parencite{ChandlerSweller1992}. \textit{Germane load} refers to working memory resources devoted to schema construction and automation—that is, to learning itself \parencite{sweller2010}. Effective instruction minimizes extraneous load and manages intrinsic load to maximize capacity available for germane processing.



A critical distinction clarifies why L2 learners face elevated baseline demands: the automaticity of linguistic processing. In first language (L1) use, processes such as lexical access, syntactic assembly, and phonological encoding proceed largely automatically, imposing minimal demand on working memory. In second language (L2) use, these same operations require controlled attention and consume substantial working memory resources until they become automatized through extensive practice \parencite{McLaughlin1990,Segalowitz2010}. This shift has two consequences. First, it elevates baseline intrinsic load: materials that are routine for L1 users impose higher processing demands for L2 learners because controlled attention must be allocated to processes that are automatic in the L1. Second, it increases vulnerability to extraneous load, as any design-imposed demands compete for an already taxed working memory system. These differences make careful instructional design especially critical for L2 learners.

\subsection{L2 production under load}\label{subsec:production}

Spoken production makes capacity limits especially visible because planning, formulation, articulation, and monitoring must unfold simultaneously. Under rising demand, learners redistribute attention strategically: they streamline content, select simpler constructions, and tolerate disfluencies such as pauses, repetitions, or self-repairs to maintain message continuity. The complexity, accuracy, and fluency (CAF) triad provides a principled framework for describing this redistribution of attention, not as failure but as rational adaptation to capacity constraints \parencite{housen2009}. The balance among complexity, accuracy, and fluency shifts systematically depending on task demands and learner priorities, with continuity often receiving priority because maintaining communicative flow is essential for successful interaction.

This capacity-based perspective has instructional implications. Expecting novice learners to sustain spontaneous speech through immersion alone may exceed available working memory capacity. Explicit guidance that reduces avoidable coordination demands should yield steadier fluency and more reliable self-monitoring \parencite{Sweller2017TESL}. Experimental evidence suggests that task conditions nudge this balance in predictable directions. When online pressure is eased through planning time, clear goals, or materials that reduce last-second problem-solving, learners typically sustain steadier flow and can attempt more elaborated language. These reallocations are the raw material for listener judgments; the same timing choices that help speakers maintain continuity are the cues listeners use to decide whether speech is smooth and easy to follow. When pressure increases, maintaining flow and monitoring form become harder to coordinate, making conservative choices and small breakdowns more likely \parencite{EllisYuan2004}.

These temporal adjustments have perceptual consequences. Global timing and rhythm, the steadiness of advance, the length of continuous runs, and the proportion of talking time align with judgments of smoothness and comprehensibility. Familiar indices—for example, speech rate, mean length of runs, and phonation time ratio—summarize these timing properties and connect observable delivery to listener perception \parencite{KormosDenes2004}. Converging evidence shows that prosody itself shifts under validated load: in military simulator flights, higher rated cognitive load increased mean F0 by roughly 7--12 Hz, increased intensity by about 1--1.5 dB, and compressed F0 range, indicating that processing pressure is audibly realized in voice as well as in timing \parencite{Huttunen2011AviationProsody}. The CAF framework thus functions not as an evaluative scorecard but as a descriptive tool for understanding how speakers redistribute attention under processing pressure \parencite{housen2009}. The following subsection extends this capacity-based logic to L2 comprehension.


\subsection{L2 comprehension under load}\label{subsec:comprehension}

Comprehension, like production, operates under severe capacity constraints, though these constraints manifest differently across modalities. In listening or video comprehension, load rises with the transience of information, acoustic degradation, and the need to align rapid speech with changing visuals; sustaining understanding requires continuous decoding and mapping to meaning, which can curtail resources for higher-level integration \parencite{mattys2012speech,pichora2016hearing}. In this thesis, text-based comprehension serves as the primary window into online processing; the same capacity limits can be observed there as systematic adjustments in pace and strategy \parencite{rayner1998}.

As linguistic or conceptual difficulty increases, learners exhibit characteristic behavioral adjustments: they slow reading rate, advance in smaller increments, and revisit earlier material more frequently \parencite{rayner1998}. These patterns reflect the additional processing resources required to integrate information and resolve syntactic dependencies. Capacity-based accounts clarify why: when storage and processing demands exceed available resources, progress becomes less efficient and partial results may degrade, which in turn increases reliance on local cues and weakens cross-clause integration \parencite{just1992}. These pressures are commonly amplified in a second language; recognition, meaning selection, and syntactic assembly draw more heavily on attention in the L2, so lines that are routine in the L1 can impose substantial demands in the L2. As a result, differences in current exposure and proficiency modulate efficiency; richer experience lowers the costs of early recognition and later integration, whereas limited experience makes both stages more resource intensive, with slowdowns and rereads more likely \parencite{whitford2012second}.

Beyond overall speed, the path the eyes take through a sentence provides a spatial picture of how attention is organized. When sentences are straightforward, gaze tends to move forward in an orderly way; when sentences are harder, the path becomes less predictable and includes more targeted returns to earlier words or phrases \parencite{rayner1998}. Temporal signals—how long learners look and when they go back—indicate when effort increases. Spatial signals—how orderly the path is—indicate how attention is being rerouted to cope with load. Considering both provides a fuller account of what it means for text-based comprehension to become more demanding in an L2; not only does progress slow, the movement pattern reorganizes so that local checking and global integration can still be achieved under tighter capacity limits. This change in orderliness—from organized paths to more disrupted sequences—can be summarized by scanpath regularity, a general indicator of eye-movement stability \parencite{VonderMalsburg2015}. Taken together, temporal and spatial adjustments show how learners reorganize progress when capacity is tight in a second language. This joint perspective motivates the validation strategy described later: converging indicators are needed to verify that observed changes genuinely track increased demand rather than unrelated factors. In this program, Study~4 analyzes these temporal and spatial indices in the text modality, and Study~5 complements them with pupil dynamics across text and video.

\subsection{Unifying Production and Comprehension}\label{subsec:unifying}

Production and comprehension are two manifestations of the same capacity-limited system. In speaking, the need to maintain communicative flow, select lexical items, assemble syntactic structure, and monitor output draws on working memory resources; in comprehension, the need to retain recently processed information, integrate it with incoming input, and verify coherence draws on the same pool. Treating them together is therefore not a convenience but a theoretical requirement: a change that helps a speaker maintain continuity can alter the cues a listener relies on, and a change that makes sentences denser can force a reader into slower, more piecemeal progress. A joint research program makes these dependencies visible instead of inferring them across separate literatures.

The link is practical as well as theoretical. What speakers choose to protect under pressure—continuity, precision, or elaboration—determines what listeners can hear and use for judgment, for example whether the message remains easy to follow and whether delivery sounds smooth \parencite{DerwingMunro1997,IsaacsTrofimovich2012}. The complexity--accuracy--fluency framework provides a common vocabulary for describing those speaker choices without assuming that one pattern is always optimal \parencite{housen2009}. On the comprehension side, the same pressure points manifest as slower forward progress and more returns to earlier text, with the path of the eyes growing less orderly as difficulty rises \parencite{Rayner2009}. Placing these strands in one program allows direct connections: validated increases in load can be traced to shifts in how attention is allocated during speaking and during reading, then connected to what listeners report and to what readers actually learn, all under shared assumptions about limited capacity.

A unified design also improves interpretability of outcomes. If listener judgments change when speakers are under heavier demand, the same logic predicts that comprehension will change when readers face denser language or unfamiliar content. Conversely, when supportive design features ease online pressure, both listeners and readers should benefit: speech should be easier to follow and text should be easier to work through. An integrated program can test these parallel expectations with compatible measures and analysis choices, reducing the risk that differences across studies reflect incompatible tasks rather than genuine differences in how people cope with demand. Production and comprehension are thus coordinated expressions of the same constraints, and analyzing them together clarifies when a result is specific to a modality and when it reflects a broader property of capacity-limited processing \parencite{just1992,Rayner2009}.

\subsection{Open Questions and Gaps}\label{subsec:gaps}

Despite clear progress, several foundational issues remain open. The first concerns integration across modalities. Production and comprehension are often studied in separate traditions, which makes it difficult to test the system-level claim that both draw on a shared, capacity-limited workspace. If the same pool of resources supports maintaining speech flow and maintaining text coherence, then parallel manipulations should show parallel pressures; yet unified tests that place speaking and reading under comparable demands remain rare \parencite{just1992}. An integrated approach would not collapse the differences between modalities; rather, it would make visible what they share and where they legitimately diverge.

The second issue is precision about redistribution. Even when demand is successfully manipulated, general rules for how attention rebalances across complexity, accuracy, and fluency under different task conditions remain elusive. In practice, speakers seem to protect continuity when pressure rises and invest in elaboration when pressure eases; but the exact pattern depends on features such as planning opportunities and the kinds of information a task pushes to the foreground. Resource-directing demands, which channel attention to particular content or relations, may shift choices differently from resource-dispersing demands, which increase the amount to be handled, yet the field does not yet have stable predictions for each type across tasks and proficiency levels \parencite{housen2009,EllisYuan2004,Johnson2017}. A related gap concerns perception. Listeners do not evaluate fluency in the abstract. They respond to global timing and continuity, and they may reweight these cues when pressure changes. Solid links exist between timing patterns and perceived fluency; what is less clear is whether that mapping itself shifts when demand is higher—for example, whether listeners tolerate brief disruptions when speaking conditions are heavy \parencite{KormosDenes2004}. Clarifying these contingencies would let task designers anticipate how changes in online load are likely to surface both in production and in what audiences hear.

The third issue is breadth in reading diagnostics and outcomes. Temporal indices indicate when effort spikes, but a fuller account also requires a spatial picture of how attention is organized across the page. When sentences are harder, gaze paths become less orderly; this loss of orderliness can be summarized without committing to a single sequence, offering a complementary window on how readers manage difficulty \parencite{VonderMalsburg2015}. What remains underdeveloped are systematic links from these spatial patterns to comprehension outcomes in L2 settings, and tests of how those links interact with individual differences in lexical access and current exposure. Because L2 reading typically taxes recognition and integration more heavily than L1 reading, differences in experience can shift the entire balance of what must be done moment by moment, altering both speed and the organization of the path through text \parencite{whitford2012second}. Closing this gap requires designs that treat temporal and spatial signals as joint evidence about coordination, then relate that coordination to what readers ultimately retain.

\subsection{Thesis Purpose and Scope}\label{subsec:purpose}

This thesis examines how increases in cognitive load shape two sides of second language use: production—what speakers do and what listeners hear—and comprehension—how learners process input and what they understand. The aim is practical as well as theoretical. By tracing how task features alter moment-to-moment allocation of attention, and by linking those adjustments to outcomes that matter—comprehensibility, fluency judgments, comprehension accuracy, and self-efficacy—the project distinguishes performance changes that reflect pressure on limited resources from changes that reflect knowledge or ability.

This choice follows accounts that treat cognitive load as motivational cost, predicting that instruction which reduces extraneous demand elevates self-efficacy during learning and that necessary intrinsic demand can also strengthen expectancy when benefits are visible \parencite{Feldon2019}. The research program comprises two experiments that together yield five studies. Using two experiments allows parallel coverage of speaking and academic learning while keeping designs focused and interpretable. Experiment~1 concentrates on how speakers reallocate attention and how listeners respond to those changes. Experiment~2 concentrates on how learners comprehend academic content when linguistic and contextual demands vary. Running them together permits shared principles—validated demand, converging indicators, process traces—while letting each modality use the tools it needs.

The scope is broad enough to compare production and comprehension under pressure, yet structured enough to keep results interpretable across tasks and measures. The program adopts CLT-aligned supports where appropriate, consistent with applied guidance for adult L2 instruction that prioritizes explicit structure over unguided immersion at beginner levels \parencite{Sweller2017TESL}. Because Japanese is head-final with overt case marking, clause-final packaging and clear morphosyntax are expected to carry evaluative weight under processing pressure, a claim tested in Study~1.


\section{Research Program, Questions, and Roadmap}

\subsection{Research Questions Across Studies}\label{subsec:questions}

This research program comprises five linked studies that address complementary questions across production and comprehension. The sequence moves from production under validated cognitive load (Studies~1--2) to comprehension under factorial task demands (Studies~3--5), with each study building on insights from the previous investigation. Each later study elaborates mechanisms suggested by the earlier one while maintaining a common focus on how limited capacity is reallocated and how those reallocations register in behavior and judgment. The questions are designed to make the logic visible from beginning to end: what learners do when pressure rises, what listeners actually notice, how learners handle increased difficulty in academic materials, and how moment-to-moment traces reveal the underlying adjustments.

Experiment~1 manipulates integration demand while holding content constant. Specifically, an element-interactivity manipulation increases the number of relations that must be coordinated in real time, while keeping materials otherwise comparable. The logic is straightforward: if capacity limits shape how speech unfolds, tightening integration should force visible reallocations in how speakers balance message transmission, form control, and structural elaboration. Study~1 therefore asks which parts of the complexity, accuracy, and fluency triad learners protect under validated load, which parts they allow to give way, and under what conditions concurrent gains remain possible \parencite{Robinson2005,Skehan2009}. This answers a practical classroom and assessment question: when the same learner sounds different under heavier conditions, what exactly was traded off, and was any improvement achieved at the same time? Study~2 turns to the listener. Using the same recordings, it models which linguistic and prosodic features best predict comprehensibility and perceived fluency for native listeners, and tests whether the relative weight of those predictors shifts with task demand \parencite{Segalowitz2010,IsaacsTrofimovich2012}. Internal reallocations matter only insofar as they become audible; linking production data to listener judgments shows when a timing change that helps the speaker also helps—or hinders—the listener. Together, these two studies treat production and perception as two views of the same event, allowing claims about speaking under load to be tied directly to what audiences actually hear. The comprehension strand then applies the same capacity logic to academic learning under factorial task demands.


Experiment~2 employs a factorial design that manipulates three factors common in instructional contexts—linguistic complexity (simple vs.\ complex syntax), domain (history vs.\ science), and modality (text vs.\ video)—to examine their independent and interactive effects. This design allows examination of how language difficulty, prior knowledge, and presentation format independently and interactively influence cognitive load and comprehension. The factor structure separates questions that are often conflated: is it the language itself, the kind of knowledge at stake, or the way information is presented that most strongly drives load? Building on this shared experiment, Study~3 analyzes subjective cognitive load and self-efficacy together with comprehension accuracy; eye tracking is not analyzed in this study. Study~4 analyzes eye movements in the text modality to characterize reading patterns using position-anchored outcomes that require a stable spatial layout (e.g., first-fixation position, regression-path measures). Study~5 analyzes the complete pupil-diameter trajectory within each passage section across modalities (text and video), providing a time-resolved physiological perspective that complements Study~4 and goes beyond aggregate eye-movement indices. A continuously sampled physiological signal is included because load can fluctuate within a single trial; a summary score at the end of the trial can miss those fluctuations, whereas a time series can capture them.

Across both experiments, manipulation checks rely on immediate effort ratings to index subjective load and, where a secondary task is present, on combined speed and accuracy to index spare capacity. These checks keep the interpretation grounded; analyses proceed only when demand has in fact increased. Taken together, the five studies map one territory from two vantage points. The speaking strand shows how pressure affects delivery and how those changes are heard; the academic learning strand shows how pressure affects the path through information and what is ultimately learned. Production and comprehension are thus treated not as competing priorities but as coordinated expressions of the same capacity limits, which motivates considering both modalities within a single, coherent program.

\textbf{RQ1 (Study~1):} Which dimensions of complexity, accuracy, and fluency do learners protect as cognitive load rises, which do they allow to give way, and under what conditions are concurrent gains possible \parencite{Robinson2005,Skehan2009}? The purpose is descriptive and diagnostic. By observing how speakers redistribute attention when tasks become more demanding, the study separates changes that preserve the flow of communication from changes that reflect reduced control or reduced elaboration. The expectation is not a single preferred pattern, but a small set of adaptive responses that depend on what the task makes urgent in the moment. Documenting these responses establishes a baseline for interpreting any later judgments about fluency or comprehensibility.

\textbf{RQ2 (Study~2):} Which linguistic and prosodic features predict comprehensibility and perceived fluency for native listeners who evaluate the same recordings, and does the relative weight of those predictors change with task demands \parencite{Segalowitz2010,IsaacsTrofimovich2012,SuzukiKormos2020}? The aim is to align internal reallocations with external perception. If the distribution of attention in RQ1 shifts toward maintaining continuity, listeners should register that shift as changes in overall smoothness and ease of following. The study therefore models the mapping from observable features to judgments, while also testing whether harder conditions lead listeners to rely more on global timing and less on fine-grained cues, or the reverse. In this way, RQ2 turns the production patterns from RQ1 into predictions about what audiences actually hear.

\textbf{RQ3 (Study~3):} How do linguistic complexity, domain, and modality shape subjective cognitive load, self-efficacy, and comprehension accuracy in academic learning contexts? The goal is to separate the difficulty of the language itself, the kind of knowledge at stake, and the presentation format while modeling confidence alongside performance so that small score differences are not overinterpreted.

\textbf{RQ4 (Study~4):} Which temporal and spatial eye-movement indices best capture the interplay between linguistic complexity and domain expertise in L2 text comprehension? The analysis is restricted to the text modality to enable position-anchored outcomes. Considering temporal and spatial signals together provides a fuller picture of how attention is routed through text. The study links the task factors from RQ3 to a concrete record of how readers proceed, yielding a process-level account of why accuracy and effort moved the way they did, and it tests whether domain expertise moderates the impact of linguistic complexity on these indices.

\textbf{RQ5 (Study~5):} Does continuous pupil diameter provide unique information about moment-to-moment cognitive load beyond subjective ratings and eye movements, and how do these dynamics vary with domain, modality, and linguistic complexity? The rationale is straightforward: if demand fluctuates within a section, a continuously sampled physiological signal should register those fluctuations even when summary scores look similar. By comparing pupil time series with the behavioral and eye-movement evidence, the study tests whether a physiological stream adds unique leverage for distinguishing conditions that impose comparable totals of effort but differ in their temporal profiles. Study~5 thus complements Study~4 by providing a time-resolved physiological perspective that applies across modalities.

\subsection{Contributions}\label{subsec:contributions}

This thesis contributes to second language research in three domains: methodology, theory, and instructional practice. The overarching contribution is integrative: production and comprehension are analyzed within a unified capacity-based framework, with each modality examined using appropriate measurement tools. The result is a coherent account of how processing pressure changes what speakers do and what listeners hear, and how that same pressure changes how readers proceed and what they ultimately learn.

Methodologically, both experiments implement converging manipulation checks and complementary process measures so that inferences rest on multiple, low-intrusion signals rather than a single indicator. Across studies, subjective effort ratings, capacity-sharing or ocular traces, and physiological measures provide converging indices; analyses proceed only when verification holds. Where a secondary task is viable, a combined speed-and-accuracy composite serves as a barometer of spare capacity, providing a behavioral cross-check that demand has tightened \parencite{vandierendonck2017}. For comprehension, eye tracking delivers millisecond-level traces of how input is parsed and monitored, capturing slowdowns and regressions as they unfold across text \parencite{Rayner2009}. Pupillometry extends this record with a continuous physiological signal of moment-to-moment effort, allowing the detection of within-trial fluctuations that summary scores can miss. Taken together, these tools privilege interpretability: a manipulation is treated as effective only when multiple sources align, and process traces explain how any observed differences arose.

The theoretical contribution specifies when capacity limits force trade-offs within the complexity, accuracy, and fluency triad in speaking, and how reallocations among these components are reflected in listener judgments of comprehensibility and perceived fluency. In this view, shifts toward continuity, toward precision, or toward elaboration are not failures but rational adaptations to processing pressure, and the thesis maps the conditions under which each pattern emerges \parencite{Skehan2009,Isaacs2018}. On the reading side, the work clarifies boundary conditions on expertise: domain knowledge can buffer processing costs, but only to the extent permitted by the linguistic demands of the material. Beyond that threshold the benefits of expertise taper because the language itself becomes the binding constraint \parencite{kalyuga2007}. Across modalities, the account is explicitly process-based: it links validated increases in demand to adjustments in timing, sequencing, and attention, and then to outcomes that matter for communication and learning.

The practical contribution translates these insights into guidance for task and materials design. For speaking, the findings identify which linguistic and prosodic features listeners weight under different levels of demand. This informs the construction of tasks that reclaim attention—for example through clear goals or planning opportunities—and it sharpens feedback and assessment practices by focusing on the cues that most strongly drive judgments \parencite{Isaacs2018}. For academic reading with text and video, the results clarify how domain, linguistic complexity, and pacing jointly shape processing and accuracy. This supports concrete design choices about modality, segmentation, redundancy, and scaffolding so that materials remain learnable under realistic constraints \parencite{Mayer2005}. In brief, the thesis contributes dependable measurement, a capacity-aware explanation of observed trade-offs, and actionable design principles.

\subsection{Roadmap by chapter}\label{subsec:roadmap-ch}

This thesis moves from shared foundations to targeted experiments and then to an integrated account that connects production with comprehension. The structure is designed to keep readers oriented at each step: establishing common definitions and assumptions, developing methods and results for each modality, and drawing the strands together so that findings can be interpreted as parts of a single capacity-focused perspective. Concepts and measurement logic come first, empirical chapters follow in a sequence that mirrors the research questions, and the synthesis at the end makes explicit how choices made by speakers relate to what listeners experience and how textual demands shape the path of reading and what learners ultimately retain.

\textbf{Part A: Foundations (Chapters~1--2)} establishes the motivation for the work, positions the contribution within current theory and method, and reviews background that is shared by both modalities. Chapter~1 frames the problem, outlines the research questions, and previews the contributions. Chapter~2 specifies the cognitive architecture that underlies cognitive load theory, reviews empirical signatures of load in production and comprehension, describes design principles for managing cognitive demands, and examines the bidirectional relationship between cognitive load and self-efficacy. The goal is to ensure that later chapters employ the same terminology, the same assumptions about working memory and capacity limits, and the same rationale for manipulation checks and process tracing. With this groundwork in place, readers can treat the empirical analyses as applications of a single framework rather than as disconnected case studies.

\textbf{Part I: Production (Chapters~3--6)} addresses how cognitive load shapes L2 speaking and how listeners perceive that speech. Chapter~3 explains the Experiment~1 design and validation procedures. Chapter~4 reports Study~1 on how learners reallocate complexity, accuracy, and fluency under verified load. Chapter~5 reports Study~2, linking native listener judgments to observable timing, prosody, and form cues under different task demands. Chapter~6 synthesizes implications for task design and assessment.

\textbf{Part II: Comprehension (Chapters~7--11)} addresses how cognitive load shapes L2 reading and how learners comprehend academic materials under varying demands. Chapter~7 explains the Experiment~2 design, factors, and validation checks. Chapter~8 reports Study~3, analyzing subjective cognitive load and self-efficacy together with comprehension accuracy (eye tracking is not analyzed in this study). Chapter~9 reports Study~4, analyzing eye movements in the text modality using temporal and spatial, position-anchored indices to characterize how domain expertise moderates linguistic complexity effects. Chapter~10 reports Study~5, modeling the complete pupil-diameter trajectory within each passage section across modalities (text and video) as a continuous index of moment-to-moment effort. Chapter~11 synthesizes when expertise buffers processing costs and when linguistic demands dominate.

\textbf{Part C: Integration and Conclusions (Chapters~12--14)} closes the thesis by integrating findings across modalities. Chapter~12 draws explicit links from speaker choices to listener experience and from textual demands to the dynamics of reading. The synthesis is organized around common themes—validated demand, allocation of attention, and outcomes that matter for communication and learning—so that similarities and genuine differences across modalities are easy to identify. Chapter~13 presents limitations and future directions, identifying boundary conditions, methodological refinements, and extensions that follow naturally from the evidence. Chapter~14 concludes with a concise restatement of the central claims and contributions to method, theory, and practice, and with guidance for evaluation and instruction in settings where speaking and reading must improve together. Throughout, recurring measurement logic and shared terminology help readers carry insights forward from one chapter to the next, making the whole more coherent than the sum of its parts.
