%************************************************
\chapter{Production Synthesis (Studies 1 and 2)}\label{ch:prod-synthesis}
%************************************************

\section{Introduction}

The following synthesis integrates the production findings from Study 1 (\autoref{ch:study1}) and the listener-judgement findings from Study 2 (\autoref{ch:study2}). Table~\ref{tab:recap-studies12} summarizes the key findings that inform this synthesis.

% --- Chapter 6 table ---
\begin{table}[h]
\centering
\caption{Summary of Studies 1 and 2}

\footnotesize
\label{tab:recap-studies12}
\begin{tabular}{p{2.2cm}p{8.5cm}}
\toprule
\textsc{Study} & \textsc{Key Findings} \\
\midrule
Study 1: 
Production under Load & 
When the speaking task had more interacting parts (harder content), speakers slowed down and produced fewer content words per clause, but chose words more precisely. Importantly, error rates did not rise. This same trade-off appeared at all proficiency levels. \\
\addlinespace
Study 2: 
Listener Judgments & 
Ratings of “easy to understand” and “fluent” moved together very strongly. In easier tasks, two things mainly hurt ratings: long pauses at the ends of clauses and more errors. In harder tasks, comprehensibility relied on more cues: richer clause building and appropriate case particles helped; end-clause pauses, many auxiliaries, greater lexical variety, and more dysfluencies hurt. Perceived fluency overlapped partly with these cues.  \\
\bottomrule
\end{tabular}
\end{table}

The goal is to give a single account of how validated increases in intrinsic load reallocate attention during L2 speech, how those reallocations surface in native listener judgments, and what the joint pattern implies for task design and assessment in L2 Japanese.


\section{A Joint Account of Allocation and Perception}
\subsection{What Do Speakers Protect?}
Study 1 clarifies how speakers reallocate attention when intrinsic load rises. Phrasal elaboration and temporal speed give way, lexical density increases, and overt errorfulness is not the primary cost. In a capacity-limited system this is a sensible reshaping of priorities. As the number of relations that must be coordinated online increases, message formulation and lexical selection absorb more control, leaving less bandwidth for building larger phrasal units and for keeping tempo high. The absence of a global accuracy penalty indicates that speakers did not simply lose control; rather, they shifted where control was spent. The lack of moderation by proficiency shows that the direction of reallocation was similar across strata for this manipulation, although absolute levels differed.

Two features of Japanese help explain this pattern. First, clause-final packaging concentrates commitment at the verb, so building longer phrasal spans at speed is costly when more relations are active. Second, morphosyntactic marking carries much of the burden of reference and role assignment, which encourages a conservative protection of form under pressure. The observed rise in lexical density together with a contraction of phrasal span fits this profile, more precise wording, fewer extended runs.

\subsection{What Do Listeners Attend To?}
Study~2 indicates that listeners register the production reallocations in systematic ways that depend on demand. When demand is low, long or frequent clause-boundary pauses and overt errors are especially damaging, consistent with a general preference for steady advance and basic form control when nothing else strains processing.

When demand is high, two changes align with Study~1. First, structural clarity gains weight: explicit clause packaging and clear role marking make decoding easier when conceptual load is heavy. Second, behaviors that signal coordination strain become more costly: extended clause-final pausing and repair behavior depress judgments. Perceived fluency follows the same logic but remains more tightly linked to packaging and timing. In this dataset with experienced raters, comprehensibility and perceived fluency were not practically distinguished, reflecting largely shared cue families.

\subsection{How Production and Perception Align}

Taken together, the two studies show that higher intrinsic load reshapes both speaker allocation and listener weighting. When demand rises, speakers concentrate control on lexical choice and form maintenance, which narrows phrasal span and slows tempo. Listeners respond by giving more weight to cues that signal structure and roles, and by penalizing markers of coordination strain. In Japanese, clear case marking and compact clause formation support recovery under pressure, whereas extended boundary pausing, auxiliary-heavy phrasing, and broad lexical diversification make recovery harder. For production, the main cost of load is paid in timing and packaging; for perception, the main adjustment is a recalibration toward structural clarity, not a uniform penalty on complex speech.

The chapter-level picture therefore refines unitary talk about “complexity effects.” Complexity is not a single dimension in production or in perception. Under higher intrinsic load, lexical and phrasal facets moved in opposite directions in production, and the listener model reweighted structural clarity and boundary timing rather than applying a blanket penalty.

\section{Implications for Design and Assessment}


Three practical consequences follow directly from the joint pattern. First, when the instructional aim is to challenge learners with greater relational content, tasks should scaffold clause building and protect boundary timing. Light planning, explicit subgoal cues for packaging, or constrained prompts that favor compact clause formation buy back capacity without diluting conceptual load. Second, feedback and rating rubrics should foreground final boundary timing and role marking, since these cues carried the largest and most stable perceptual consequences across demand levels. Third, lexical goals should be calibrated to the load profile of a task. Under low demand, encouraging lexical range can support both judgments; under high demand, pushing lexical diversity without packaging support risks audible coordination strain.

For assessment, the evidence argues against interpreting fluency penalties in high complexity tasks as general deficits. Where structural clarity is preserved and errors remain controlled, lower speed and shorter runs are expected reallocations rather than failures. Rubrics that reward clause-level organization and explicit marking under pressure will better align with how experienced listeners actually judge L2 Japanese speech.


\section{Bridge to Comprehension}
The synthesis strengthens the thesis-wide claim that production and comprehension are two expressions of the same capacity limits. Under higher intrinsic load, speakers compress phrasal units and slow tempo to maintain control, and listeners rely more on explicit structure to recover the message. The reading studies that follow adopt the same logic. When language becomes denser or background knowledge is less aligned, readers advance in smaller steps and rely on cues that make structure explicit, and the path through the text becomes less orderly. The production results therefore foreshadow the comprehension results; the same constraints that reallocate speech also reshape the dynamics of reading. This continuity is the basis for the cross-pillar synthesis at the end of the thesis.

\section{Conclusion}
In L2 Japanese speaking under validated increases in intrinsic load, learners protect lexical precision and core form, and they pay with tempo and phrasal span. Experienced listeners register those reallocations through boundary timing and structural clarity, rewarding compact, well marked clauses and penalizing clause-final pausing and auxiliary-heavy reformulation. These are rational adjustments in a capacity-limited system. Designs that manage where coordination pressure is felt, at boundaries and at case marking, improve both the act of speaking and the experience of listening when tasks become harder.
