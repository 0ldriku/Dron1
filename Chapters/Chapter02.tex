\chapter{Cognitive Load: Architecture and Applications}\label{ch:cogload}


This chapter establishes the theoretical foundations of Cognitive Load Theory (CLT) and reviews empirical evidence for its application to second language learning. \autoref{sec:definitions} defines cognitive load within a specific model of human cognitive architecture. It distinguishes intrinsic, extraneous, and germane load, and traces how these distinctions generate classic instructional effects. \autoref{sec:consequences} examines behavioral signatures of cognitive load in L2 production and comprehension, connecting theoretical constructs to observable patterns in speech and reading. \autoref{sec:manipulations} reviews design principles for managing intrinsic demands, reducing extraneous demands, and fostering productive processing. \autoref{sec:verification} specifies the methodological approach to validating cognitive load manipulations, including subjective ratings, behavioral indices, and physiological measures. \autoref{sec:load-selfeff} examines the bidirectional relationship between cognitive load and self-efficacy, a motivational construct that shapes how learners allocate effort and interpret difficulty. 


\section{Definitions, Scope, and History}\label{sec:definitions}

This section establishes terminology and scope. The following subsections specify the cognitive architecture on which CLT is built, define the three categories of cognitive load, and explain how classic instructional effects follow from these architectural commitments. Boundary conditions and common misclassifications are identified to ensure precise application, and implications for second language learning are outlined.

\subsection{Cognitive Load and Cognitive Architecture}\label{subsec:architecture}

%TODO: add the figure from baddeley about working memory.
Cognitive Load Theory presumes a specific architecture. New information is processed in a short-lived, capacity-limited working memory, while consolidated knowledge in long-term memory can be accessed with little cost \parencite{sweller2019}. Within working memory, a central executive coordinates a phonological store and a visuospatial store \parencite{BaddeleyHitch1974,Baddeley2002,Baddeley2003}. The focus of attention typically maintains only four meaningful units \parencite{Cowan2001}, which creates a bottleneck for complex cognition, including language. When task demands exceed this capacity, performance changes in systematic ways \parencite{sweller1988}. This architectural distinction has direct instructional implications, because designs that respect working memory limits facilitate schema construction and automation in long-term memory \parencite{sweller2019}.




CLT focuses on culturally acquired, biologically secondary knowledge (such as reading, mathematics, and domain-specific concepts) that requires explicit instruction and for which working memory limitations are especially consequential \parencite{schnotz2007}. Within this framework, \textcite{sweller2010} defines cognitive load as the moment-to-moment demand imposed on working memory during learning activities.


A standard analytical framework distinguishes three categories of cognitive load \parencite{sweller2010}. \textit{Intrinsic load} stems from element interactivity in complex cognitive tasks. It reflects demands that are directly relevant for performing and learning the task. For example, understanding a simple sentence with a single clause imposes lower intrinsic load than understanding a sentence with multiple embedded clauses, because the latter requires coordinating more elements simultaneously. \textit{Extraneous load} arises from activities that are not productive for learning. It can be caused by task design features, learner characteristics such as intrusive thoughts about failure, or environmental factors such as distracting information. For instance, presenting text and related diagrams in spatially separated locations forces learners to mentally integrate information across sources, consuming working memory resources without contributing to schema construction \parencite{ChandlerSweller1992}. \textit{Germane processing} refers to working memory resources productively devoted to intrinsic cognitive load after extraneous load has been minimized. Intrinsic and extraneous load are additive, so reducing extraneous demand frees capacity for schema construction and schema automation \parencite{paas2020}. Figure \ref{fig:clt-load-profile} provides a conceptual model of this additive relationship, illustrating how intrinsic, extraneous, and germane load combine dynamically over time. This model also distinguishes instantaneous ``peak load'' from ``average load,'' providing a visual rationale for time-resolved measurement. 

\begin{figure}[ht]
  \centering
  \includegraphics[width=\textwidth]{Chapters/c1c2fig/clt_load_profile.pdf}
  \caption[Load profile and additive sources]{Instantaneous and average load over time with an assumed capacity limit, showing intrinsic load, extraneous load, and germane load as additive components and indicating free capacity. Adapted from \textcite{paas2003}.}
  \label{fig:clt-load-profile}
\end{figure}


Central to these distinctions is the notion of element interactivity: material is simple when elements can be processed independently and complex when elements must be coordinated simultaneously \parencite{sweller2010}. Increases in the number and strength of such interactions raise intrinsic demands for a given level of prior knowledge. The same lens helps identify extraneous demands, because formats that force unnecessary coordination inflate interactivity without changing what must be learned. This architectural grounding motivates the instructional effects reviewed in the following subsection.


Learning is the construction and tuning of schemas in long-term memory that recode multiple elements as a single functional unit, which reduces experienced element interactivity over time \parencite{sweller1994,paas2003}. As procedures consolidate and operations become automatic, working memory cost declines. In L2 contexts, this implies that fluent lexical access and morphosyntactic routines operate as schema-based chunks that free capacity for message planning and discourse integration. This architecture is foundational for the thesis: the phonological loop and central executive constrain L2 speaking (Studies~1--2), while the visuospatial store and central executive constrain L2 comprehension (Studies~3--5). These limits justify the interactivity and format manipulations in Experiments~1 and~2. 


\subsection{Foundational Design Principles}\label{subsec:effects}

The development of CLT was initially motivated by evidence that conventional problem-solving approaches impose high working memory demands. Specifically, means-ends analysis requires extensive search processes that consume substantial working memory resources \parencite{sweller1988}. Under these conditions, learners devote capacity to difference reduction and step selection rather than to noticing structure or forming schemas, which explains weak learning despite considerable effort \parencite{sweller1988,sweller2010}.

The formative response was to redesign activities to remove this unnecessary search so that remaining capacity could be invested in schema acquisition and automation. For example, goal-free problems minimize backward search, worked examples present solutions directly for study, and completion problems provide partial solutions that constrain search while still requiring essential processing \parencite{sweller2010,sweller2019}. Converging evidence came from format effects: physically or temporally integrating mutually referring sources reduces search and mapping costs, while removing redundant presentations avoids unnecessary reconciliation \parencite{chandler1991}. These split-attention and redundancy effects follow from element interactivity and are detailed as design principles in \autoref{subsec:extraneous}. 


\subsection{The Role of Learner Expertise}\label{subsec:boundaries}

While CLT aims to provide generalizable design principles, applying its core distinctions to specific instructional contexts requires careful justification. Intrinsic, extraneous, and germane demands cannot always be cleanly separated in theory or in practice, since the same feature may be necessary for learning in one context yet superfluous in another \parencite{schnotz2007}. Contemporary accounts emphasize that classifications shift with expertise and goals \parencite{sweller2010,sweller2019}.

A critical boundary condition is that optimal instructional formats vary with learner expertise \parencite{kalyuga2007}. Design features that support novices can become redundant or even counterproductive as learners gain knowledge. For example, a graphic that guides attention for novices may become superfluous once learners have encoded the relevant relations in long-term memory, at which point it imposes extraneous load \parencite{sweller2010}. In adult L2 instruction, this predicts that integrated supports such as glosses should benefit novices but may become redundant as learners internalize vocabulary mappings \parencite{Sweller2017TESL}. Empirically, both outcomes appear. High-knowledge readers can learn more from lower-cohesion materials because they supply bridging inferences, while low-knowledge readers benefit from increased cohesion, the classic reverse cohesion pattern \parencite{mcnamara1996are}. A broader expertise-reversal literature reports that guidance and scaffolds assist novices but can become neutral or counterproductive for knowledgeable readers, consistent with a shift in the optimal balance between externally provided structure and self-generated integration \parencite{tetzlaff2025}. These constraints guide Study~4, which tests whether domain expertise buffers the processing costs of complex language in L2 reading.




\section{Consequences of Cognitive Load}\label{sec:consequences}

Cognitive load manifests in observable behavioral patterns. The following subsections review empirical signatures of increased processing demand in production (Section~\ref{subsec:prod-sigs}) and comprehension (Section~\ref{subsec:comp-sigs}), connecting theoretical constructs to measurable indices.

\subsection{Production Signatures}\label{subsec:prod-sigs}
\subsubsection{Observable Speaking Signatures: Timing, Prosody, Repairs}

Under high task or presentation demands, speakers reallocate limited resources among competing performance dimensions. They may shift resources away from syntactic elaboration or error monitoring to maintain fluency, or conversely, they may slow their speech rate to preserve accuracy—producing the systematic trade-offs documented in L2 research \parencite{housen2009}. The balance among CAF shifts systematically depending on task demands and learner priorities, with continuity often receiving priority because maintaining communicative flow is essential for successful interaction. Explicit guidance that reduces avoidable coordination demands should yield steadier fluency and more reliable self-monitoring \parencite{Sweller2017TESL}. Experimental evidence suggests that task conditions nudge this balance in predictable directions. When online pressure is eased through planning time, clear goals, or materials that reduce last-second problem-solving, learners typically sustain steadier flow and can attempt more elaborated language. These reallocations are the raw material for listener judgments; the same timing choices that help speakers maintain continuity are the cues listeners use to decide whether speech is smooth and easy to follow. When pressure increases, maintaining flow and monitoring form become harder to coordinate, making conservative choices and small breakdowns more likely \parencite{EllisYuan2004}. 

Empirical studies of perceived fluency align with this account: temporal measures such as speech rate, phonation time ratio, and mean length of runs are strong predictors of raters' fluency judgments, indicating that when coordination costs increase, timing variables absorb the impact. These indices capture global timing and rhythm, the steadiness of advance, the length of continuous runs, and the proportion of talking time, all of which align with judgments of smoothness and comprehensibility \parencite{KormosDenes2004}. Converging evidence shows that prosody itself shifts under validated load: in military simulator flights, higher rated cognitive load increased mean F0 by roughly 7--12 Hz, increased intensity by about 1--1.5 dB, and compressed F0 range, indicating that processing pressure is audibly realized in voice as well as in timing \parencite{Huttunen2011AviationProsody}. Monitoring is integral to online speech production; process-proximal indicators such as repetitions, false starts, and reformulations complement global fluency metrics by indexing control costs and time-gaining strategies under load \parencite{albarqi2022}.


\subsubsection{Task Features, Sequencing, and Moderators}


Building on these signatures, task features modulate how attention is redistributed across CAF. Within the resource-directing versus resource-dispersing distinction, increasing resource-directing complexity is often associated with gains in accuracy and complexity, frequently with fluency costs, whereas tightening resource-dispersing constraints tends to depress performance unless support is provided \parencite{Robinson2005}. A limited-attention perspective likewise predicts trade-offs under higher conceptual demands, especially when learners must conceptualize new content while encoding it \parencite{skehan1997}. Recent syntheses report consistent increases in linguistic complexity when tasks include more elements and stronger reasoning demands, more variable effects on accuracy, and typical costs to fluency; consequently, sequencing rather than single-shot contrasts has become a central design focus \parencite{jin2025}.

Sequencing proposals such as SSARC (to stabilize, simplify, automatize, reconstruct, and complexify) make explicit commitments about ordering and expected outcomes over a syllabus \parencite{Robinson2005}. Experimental work on oral performance shows how sequencing interacts with load type. Using a resource-directing manipulation of the number of elements and a resource-dispersing manipulation of planning time, \textcite{chen2025} compared simple-to-complex and complex-to-simple sequences against task repetition. Complex-to-simple advantaged complexity and accuracy; simple-to-complex advantaged fluency. Both sequences exceeded pure repetition for syntactic complexity, and repetition favored lexical complexity and fluency. These patterns are consistent with the idea that conceptual difficulty can be increased to direct attention to form, provided that dispersing constraints are not tightened at the same time \parencite{Robinson2001,Robinson2005}.

Validation cautions remain important. Not all nominally complex tasks impose measurably higher load for a given population. A careful test of an elements manipulation found that only large increases in the number of interacting elements yielded differences on dual-task costs, duration estimates, and mental-effort ratings, with indications that proficiency moderates sensitivity \parencite{sasayama2016}.

Modality and individual differences further condition outcomes. When the same complexity manipulation is administered as speaking and as writing, complexity gains and accuracy costs can appear in both modes, yet baseline trade-offs differ. One study observed that speaking was more accurate but less fluent than writing and that working memory did not predict outcomes once complexity and modality were controlled \parencite{cho2018}. In L2 writing specifically, working memory shows limited direct effects on text-level performance once proficiency is taken into account, and proficiency often emerges as the more reliable predictor across simple and complex versions of the same task \parencite{manchon2023,kuiken2011}.

Classic dispersing supports such as pre-task planning can shift attention to form and buffer fluency costs under higher conceptual load, especially in speaking \parencite{ryu2018}. In practical terms, resource-directing manipulations (e.g., increasing reasoning demands) recruit attention to formulation, typically enhancing structural and lexical elaboration. In contrast, resource-dispersing manipulations (e.g., time pressure) heighten coordination demands, typically depressing temporal fluency \parencite{gilabert2007,albarqi2022}. For example, adding elements or reasoning demands often yields more lexical and syntactic elaboration with fluency costs, while dual-tasking reliably reduces temporal fluency and increases repair behavior \parencite{gilabert2007,albarqi2022}. These signatures define the CAF indices and the listener feature set in Studies~1 and~2.

\subsection{Comprehension Signatures}\label{subsec:comp-sigs}
\subsubsection{Predictive Processing and Language-Specific Constraints}

In comprehension, capacity constraints have well-documented ocular correlates: as task complexity or format-induced coordination increases, first-fixation duration and gaze duration rise on difficult regions, regressions become more frequent, and forward saccades shorten, reflecting effortful local integration rather than fluent parafoveal sampling \parencite{rayner1998}. These patterns follow from element interactivity: formats that inflate avoidable coordination, for example, mapping across separated sources or reconciling poorly aligned cues, raise extraneous demands and manifest in precisely these slowing and backtracking patterns \parencite{sweller2019}. Because these signatures unfold at different timescales, early measures, such as first-fixation duration, are sensitive to recognition and initial lexical access, whereas later measures, such as total fixation time, fixation counts, and regressions, index integration and discourse-level binding. Increases in either set of indices reveal the stage at which coordination costs emerge and whether a manipulation primarily obstructs identification or subsequent unification \parencite{rayner1998}.

Capacity-based accounts clarify why these patterns emerge: when storage and processing demands exceed available resources, progress becomes less efficient and partial results may degrade, which in turn increases reliance on local cues and weakens cross-clause integration \parencite{just1992}. These pressures are commonly amplified in a second language; recognition, meaning selection, and syntactic assembly draw more heavily on attention in the L2, so lines that are routine in the L1 can impose substantial demands in the L2. As a result, differences in current exposure and proficiency modulate efficiency; richer experience lowers the costs of early recognition and later integration, whereas limited experience makes both stages more resource intensive, with slowdowns and rereads more likely \parencite{whitford2012second}.



Understanding how linguistic complexity affects reading requires moving beyond traditional temporal measures to spatial indices that capture how attention is deployed across text. While these temporal measures are foundational, spatial measures reveal how attention is distributed across the text. Recent research shows that these spatial characteristics of eye-movement patterns provide converging evidence for processing difficulty \parencites[e.g.,][]{schad2012zoom,torres2021eye}. Radius of gyration (the root mean square distance of fixations from scanpath centroids) increases with linguistic complexity and correlates more strongly with self-reported effort than mean dwell time \parencite{schad2012zoom,Salvucci2000}. Convex hull area (the polygon encompassing all fixations) correlates with readability measures across different text types \parencite{torres2021eye}. Computational models that incorporate parallel word processing and continuous attentional gradients reproduce these patterns by modeling how processing difficulty broadens attentional gradients and suppresses forward saccade generation \parencite{engbert2002dynamical,engbert2005swift,schad2012zoom}. Including spatial dispersion alongside timing measures therefore strengthens inferences about complexity effects.



A growing body of work also points to discourse-level complexity: texts that lack cohesive devices or exhibit unpredictable topic shifts further inflate fixation durations, and a recent corpus analysis showed that a combined index of lexical rarity, syntactic embedding, and cohesion accounts for roughly one third of variance in L2 eye-movement measures, outperforming traditional readability formulas \parencite{zhang2024modelling}. These effects generalize across typologically diverse languages; large cross-linguistic corpora confirm that dependency-distance penalties apply in head-initial, head-final, and mixed orders alike \parencite{Futrell2020}. Electrophysiological studies show that processing long-distance dependencies is associated with a P600 component, indicating increased syntactic integration costs \parencite{fiebach2002wh}.




\subsubsection{Anticipatory Processing in the Visual World Paradigm}

In the visual world paradigm, listeners use verb semantics and real-world constraints to anticipate upcoming referents, often launching saccades to a target object before it is mentioned; the classic finding that \textit{eat} triggers anticipatory looks to an edible item exemplifies incremental prediction under minimal presentation costs \parencite{altmann1999incremental}. Reviews emphasize that fixations reflect integration of linguistic and visual information in real time, making the visual world paradigm well-suited for testing how load and context shape prediction during comprehension \parencite{huettig2011using}. For late L2 learners, anticipatory behavior can be reduced or delayed relative to native speakers, a difference that arises from frequency biases, competition, and representation quality rather than from a distinct processing architecture \parencite{kaan2014predictive}.

At the predictive processing level, visual world paradigms reveal that L2 learners often fail to use morphosyntactic cues predictively. For instance, one study showed that while native Japanese listeners used a nominative-dative sequence to launch anticipatory eye movements to a theme object before it was mentioned, English-speaking learners of Japanese did not \parencite{mitsugi2016use}. This suggests that the L2 learners did not use case-marker information to generate predictions about upcoming verb arguments. Building on this, \textcite{mitsugi2017incremental} showed that this deficit extends to voice prediction; in a similar visual world study, native listeners exploited the same case cues to anticipate a passive verb, whereas English-speaking learners did not shift their gaze until the voice morphology surfaced. Flexible word order creates additional processing vulnerabilities. Japanese permits object fronting and scrambling, which amplifies dependency distances and forces readers to maintain displaced constituents in working memory. In native speakers, eye tracking reveals disproportionate integration costs at clause-final verbs and increased regressions to case particles when non-canonical orders increase \parencite{ueno2008relative,tamaoka2019eye}. Linguistically complex passages containing multiple embedded structures therefore predict both heightened particle rereading and more dispersed fixation patterns.

\subsubsection{Physiological Indices: Pupillometry and Listening Effort}

Listening research converges on the same conclusion through independent indices of effort. Pupil dilation scales with sentence intelligibility, increasing when the acoustic or linguistic signal is degraded, which indicates greater allocation of limited resources to maintain comprehension \parencite{zekveld2010pupil}. Broader frameworks of effortful listening formalize this relation as a balance between task demands, available capacity, and motivation, and they explain why listeners can report fatigue even when accuracy is maintained \parencite{pichora2016hearing}. Reviews of speech recognition in adverse conditions catalogue how noise, competing talkers, and cognitive load at the receiver jointly compromise segmentation and lexical access; they also motivate design choices such as pacing and redundancy control that reduce extraneous demand \parencite{mattys2012speech}. These online measures and models provide the needed bridge to the present thesis, allowing manipulations to be linked to specific stages of comprehension by predicting longer early measures when decoding is strained and longer late measures when integration fails, and by tying increases in pupil size to the extra effort required to sustain understanding under load \parencite{rayner1998,zekveld2010pupil}. 

However, this relationship is not always linear. Comparable findings have been reported in listening and cognitive-effort studies where pupil size stayed relatively high for tasks that were well within participants’ processing capacity, indicating that larger mean pupil diameter does not always mean the nominal task was more complex \parencite{zekveld2018pupil}. At the same time, pupillometry work on effort regulation shows an inverted U-shaped response: dilation increases with task difficulty up to the point where the task is still manageable, and then declines once the task becomes effectively unmanageable or not worth the effort \parencite{herrmann2024pupil}. This drop at the “too hard” end has also been described in highly demanding or saturating visual/auditory tasks, where reduced dilation is interpreted as disengagement or overload rather than low demand \parencite{minassian2004pupillary}. In that framework, a lower mean pupil diameter in a complex condition can be read as an under-allocation of resources triggered by overload. This suggests learners may stop attempting full integration even though the nominal demand was higher, rather than providing evidence that the complex passages were easier \parencite{vanderwel2018pupil}.



\section{How Tasks Manipulate Cognitive Load}\label{sec:manipulations}

Having reviewed behavioral signatures of cognitive load, this section examines design principles for managing processing demands. It first reviews three complementary approaches from the general literature: managing intrinsic load through sequencing and worked examples (Section~\ref{subsec:intrinsic}), reducing extraneous load through contiguity and modality effects (Section~\ref{subsec:extraneous}), and fostering germane processing through explanation prompts and subgoal labeling (Section~\ref{subsec:productive}). The section then operationalizes these principles for the present thesis, detailing how intrinsic load is manipulated in the specific contexts of L2 production (Section~\ref{subsec:op-production}) and L2 comprehension (Section~\ref{subsec:op-comprehension}). Boundary conditions are made explicit throughout, including expertise reversal and limits imposed by temporal transience.


\subsection{Managing Intrinsic Load: Sequencing and Examples}\label{subsec:intrinsic}

Intrinsic load grows with element interactivity, that is, with the number of mutually dependent elements that must be coordinated at the same time by a learner with a given level of prior knowledge \parencite{sweller2010,swellerayres2011}. As knowledge increases and schemas form, many elements are recoded as a single functional unit, which lowers the experienced interactivity of the very same task \parencite{sweller1994}. Instruction manages this demand by altering what must be processed at once while preserving eventual exposure to the full set of relations.

Sequencing proceeds from lower to higher interactivity, isolates elements before integrating them, and schedules relational understanding for a later phase once prerequisites are secure \parencite{VanMerrienboer2004,jong2010}. A complementary progression begins with worked examples, shifts to completion problems, and culminates in independent problem-solving. This progression supports schema construction under high interactivity and yields transfer benefits when examples vary in surface features within a conceptual type \parencite{sweller1985,paas1994_2}.

A robust boundary condition, as established in Section~\ref{subsec:boundaries}, is the expertise-reversal effect. Supports that assist novices can become redundant or disruptive as knowledge grows because they compete with established schema-based processing. This pattern motivates adaptive fading of guidance across acquisition \parencite{kalyuga2007,sweller2010}. Section~\ref{subsec:productive} specifies how to channel freed capacity once supports reduce coordination costs. 



\subsection{Reducing Extraneous Load: Contiguity, Modality, Transience}\label{subsec:extraneous}

Extraneous load originates in how information is arranged and timed rather than in the conceptual relations to be mastered. Classic demonstrations show that spatial separation of mutually referring text and diagram forces unnecessary mapping and inflates avoidable coordination, the split-attention effect \parencite{ChandlerSweller1992}. In L2 settings, a concrete instance is vocabulary support: providing integrated translations adjacent to the target item removes avoidable search and mapping, whereas sending learners to a separate dictionary increases avoidable coordination \parencite{Sweller2017TESL}. Closely related formats that duplicate information can depress transfer through redundancy costs when learners must reconcile overlapping streams \parencite{Yeung1998}.

Figure~\ref{fig:ctml} presents the cognitive theory of multimedia learning as two processing channels, auditory verbal and visual pictorial, arranged across five stages of representation: physical input to the learner, sensory registration in ears and eyes, attended items in working memory, structured verbal and pictorial models in working memory, and activation of relevant knowledge in long term memory. The working memory stages are capacity limited, the physical input and long term stores are not. Arrows mark the active processes required for learning: selecting words and selecting images from the sensory streams, organizing words and organizing images into coherent models, and integrating those models with prior knowledge. This layout makes the design logic in this subsection concrete: contiguity, modality, redundancy control, and segmentation aim to keep selection, organization, and integration inside working memory limits while aligning the two channels. \textbf{TOOD: redraw the fig}


\begin{figure}[t]
  \centering
  \includegraphics[width=\textwidth]{Chapters/c1c2fig/fig_ctml.png}
  \caption[Cognitive theory of multimedia learning]{Two processing channels (auditory verbal, visual pictorial) progress through five representational stages: physical input to the learner, sensory registration in ears and eyes, selected items in working memory, organized verbal and pictorial models in working memory, and integration with prior knowledge in long term memory. Arrows indicate the active processes of selecting words and images, organizing them into coherent models, and integrating with prior knowledge, all under limited working memory capacity. This layout motivates the design levers in this section: spatial and temporal contiguity, nonredundant modality, redundancy control, and segmentation. Adapted from \textcite{mayer2003}.}
  \label{fig:ctml}
\end{figure}




Designs that force learners to divide attention across mutually referring sources impose avoidable processing. Spatial and temporal contiguity, for example, placing text next to its diagram or synchronizing narration with a visual change, reduces the need for mental alignment and frees resources for learning \parencite{chandler1991,ginns2006}. Adding on-screen text that repeats a visual or narration can depress transfer because the extra words add processing without adding structure \parencite{mayerbove1996,chandler1991}.

Within this framework, modality functions as a format variable that can either reduce avoidable coordination or inflate it, depending on alignment and redundancy. Using modality to distribute complementary information across channels lowers extraneous load when the streams must be processed together and are not duplicative. In worked-example learning with diagrams, presenting statements auditorily alleviates split attention and improves learning, the modality effect \parencite{mousavi1995}.

Auditory information is transient, however. When verbal streams are long or densely structured, spoken text can overwhelm limited capacity and reverse the benefit. Shortening or chunking narration and coordinating it with visual segments mitigates this transient information cost \parencite{leahy2011,mayer2003}.

Because animations and lectures unfold over time, segmentation and simple pacing control redistribute processing across manageable units. Segmenting dynamic visuals into meaningful parts improves learning for novices, and signaling guides attention to critical elements and relations \parencite{spanjers2012,alpizar2020}. In second language multimedia contexts, captioning and subtitling generally enhance listening comprehension and vocabulary learning, with meta-analytic evidence showing reliable benefits across proficiency bands and test types \parencite{monteroperez2013}. Eye-tracking work indicates that caption use is shaped by script familiarity, content difficulty, and pacing. Co-timing caption units with prosodic and semantic boundaries and providing pause or replay help avoid overload \parencite{winke2013,mayer2003}. These principles guide the modality (text vs.\ video), captioning, and segmentation design choices in Experiment~2.

Evidence from pupillometry complements these format effects and clarifies why differences are often time-localized rather than uniform shifts. In minimal detection paradigms, audiovisual stimuli can produce superadditive pupil dilation relative to unimodal inputs \parencite{Rigato2016}. In continuous speech, acoustic challenge reliably enlarges dilation, and combined difficulties do not necessarily add beyond the acoustic burden, which shows that one stream can dominate allocation \parencite{Kadem2020}. Conflict and integration timing are visible in the pupil, with increases shortly after fixation on mismatching facial cues \parencite{AriasSarah2023}. Under high intelligibility, clearer speech can sometimes elicit slightly larger dilation than casual speech, plausibly reflecting deliberate allocation for maximal accuracy rather than difficulty per se \parencite{Mechtenberg2023}. Together with contiguity and redundancy principles, these results imply that aligned, nonredundant streams can reduce extraneous demand, while misalignment or duplication can raise it, and that resulting differences should appear as localized divergences over time. This expectation motivates trajectory-based analyses in Study~5.

\subsection{Fostering Productive Processing: Explanations and Subgoal Labels}\label{subsec:productive}

Productive processing denotes effort directed at constructing and organizing schemas. Process tracing and training studies agree that principle-based self-explanations during example study are a reliable mechanism that supports deep learning. Prompting learners to explain why a step is warranted and how it instantiates a principle improves transfer beyond gains in recall \parencite{chi1989}. Classroom studies further show that the frequency and quality of spontaneously produced explanations predict problem-solving even when time on task is controlled \parencite{renkl1997}.

At the level of example construction, design syntheses translate these mechanisms into features of examples that elicit constructive activity. Three levers repeatedly improve far transfer. First, pair each worked example with a matched practice item to encourage immediate application. Second, vary examples within a type to promote abstraction. Third, place rationales and operations in close proximity so that learners encounter structure where it is needed \parencite{atkinson2000}.

Subgoal labeling makes the structure explicit at a useful grain size. Grouping steps under compact, purpose-focused labels helps learners form reusable categories and adapt procedures to novel problems. Abstract, principle-oriented labels outperform superficial headings because they cue the organization that experts use when planning solutions \parencite{catrambone1998}.

Prompts should target quality rather than quantity. Learners who articulate principle-to-step links and anticipate upcoming moves gain more than those who merely paraphrase visible actions. High-quality, condition-focused explanations are the proximal predictor of success in longitudinal classroom work \parencite{renkl1997}.

At the lesson level, fading consolidates these gains. Moving from fully worked examples to partially completed solutions and then to independent generation invites a gradual shift from comprehension to production, especially when structural rationales are co-located with operations so that explanation and application proceed together \parencite{atkinson2000,chi1989}. Building on these design principles, the next section specifies how to verify that manipulations produce intended effects.



\subsection{Intrinsic Load in L2 Production}\label{subsec:op-production}



Intrinsic load in L2 production is most commonly operationalized through task complexity manipulations. To this end, this thesis draws on Robinson's Triadic Componential Framework, the most influential model for classifying task complexity in Task-Based Language Teaching (TBLT). The framework distinguishes cognitive task complexity, task conditions, and task difficulty \parencite{Robinson2005}. Cognitive task complexity is specified along two sets of dimensions: \textit{resource-directing} dimensions, which relate to developmental complexity, and \textit{resource-dispersing} dimensions, which relate to performative complexity. These dimensions are summarized in Table~\ref{tab:robinson-framework}. Task conditions are the circumstances of performance, for example, time pressure or concurrent activities. Task difficulty refers to perceived demand for particular learners.

\begin{table}[ht]
\centering

\caption{Cognitive Task Complexity Dimensions in the Triadic Componential Framework (adapted from Robinson, 2005)}
\footnotesize
\label{tab:robinson-framework}
\begin{tabular}{ll}
\toprule
\textbf{Resource-Directing Dimensions} & \textbf{Resource-Dispersing Dimensions} \\
\textit{(Developmental Complexity)} & \textit{(Performative Complexity)} \\ \hline
+/- Few elements & +/- Planning time \\
+/- Reasoning demands & +/- Prior knowledge \\
+/- Here-and-Now vs. There-and-Then & +/- Single vs. Dual task \\
\bottomrule
\end{tabular}
\end{table}




Mapped to CLT, resource-directing manipulations raise intrinsic load as they require coordinating additional essential relations. In contrast, resource-dispersing manipulations (such as reduced planning time or a dual task) primarily increase extraneous load by forcing simultaneous processing or adding competing demands. Similarly, poor instructional formats add extraneous load. 

In this thesis, designed differences target intrinsic load via element interactivity while holding extraneous design factors constant. Verification proceeds with a minimal, modality-appropriate set: in production, immediate effort ratings paired with a light secondary task. These principles motivate the element-interactivity manipulation in Experiment~1.



\subsection{Intrinsic Load in L2 Comprehension}\label{subsec:op-comprehension}

Intrinsic load during comprehension arises primarily from the inherent complexity of the materials being processed \parencite{sweller2010}. Unlike production tasks, where intrinsic load is tied to conceptual and relational demands placed on the performer, comprehension load stems from the linguistic input itself. In domains such as L2 reading and multimedia learning, this means that intrinsic load is commonly operationalized by manipulating the element interactivity within the linguistic input itself. Element interactivity in language refers to the number of linguistic elements (e.g., words, phrases, clauses) and the relationships between them (syntactic dependencies, semantic links, anaphoric references) that must be simultaneously processed in working memory to achieve understanding \parencite{sweller2010}.

A principal determinant of element interactivity in linguistic material is its linguistic complexity. At a basic level, simple sentences with canonical word order and high-frequency vocabulary involve relatively few interacting elements, thus imposing lower intrinsic load. Conversely, increasing linguistic complexity, such as through the use of syntactically complex structures like embedded clauses, non-canonical word orders, or long-distance dependencies, or through lexically challenging features like low-frequency or abstract words, increases the number of elements and the complexity of their interrelationships that must be coordinated in working memory at any given moment \parencite{just1992,Futrell2020}. Successfully parsing complex syntax requires not only performing symbolic computations but also storing intermediate products (e.g., representations of earlier words or interrupted clauses) while processing subsequent input. As syntactic complexity increases, these concurrent storage and computational demands can strain or exceed the available working memory capacity, leading to processing slowdowns or comprehension failures \parencite{just1992}. Similarly, processing long-distance dependencies, where syntactically related words are separated by intervening material, is argued to incur greater processing effort due to the increased demands on memory for retrieving the representation of the earlier element when the later element is encountered \parencite{Futrell2020}. Recent work specifically confirms that various linguistic complexity indices, spanning lexical, syntactic, and discoursal levels, predict L2 learners' processing effort as captured by eye-movement measures, often outperforming traditional readability formulas \parencite{zhang2024modelling}. Therefore, Experiment 2 of this thesis adopts linguistic complexity (specifically, simple vs. complex syntax) as the primary manipulation of intrinsic load within the academic learning materials. This allows for a targeted investigation into how the inherent difficulty of the language itself impacts L2 learners' subjective cognitive load, self-efficacy, and ultimate comprehension accuracy. 

\section{Measurement and Validation of Cognitive Load}\label{sec:verification}

Having established how cognitive load can be manipulated through intrinsic and extraneous design features (\autoref{sec:manipulations}), this section outlines operational strategies for validating that these manipulations actually produce the intended increases in demand.
This section outlines operational strategies for validating experimentally induced cognitive load. Recent studies have begun using task-complexity manipulations to investigate the effects of cognitive load on L2 performance \parencite[e.g.,][]{Bergeron2017,crowther2018,fullana2024,SaitoLiu2022}. Nevertheless, as \textcite{sasayama2016} notes, designed differences between tasks of varying complexity do not always translate into actual cognitive differences. The present thesis therefore treats verification of cognitive load as a prerequisite for interpretation and, in each empirical strand, adopts a theory-aligned set of subjective and objective indicators to document that demand was actually elicited.

\subsection{Categories of Cognitive Load Measures}

Techniques for measuring task-induced cognitive load have been developed mainly in cognitive psychology and multimedia learning. A common scheme classifies measures by objectivity and causal proximity, yielding four types: subjective or objective, and direct or indirect \parencite{Brunken2003}.

First, subjective–indirect measures record invested mental effort. A standard approach asks participants to rate perceived effort on a nine-point scale from very low to very high immediately after a task \parencite{paas1994_1}. These ratings are efficient and sensitive to momentary strain; however, they require corroboration because a low reported effort score can reflect genuinely low demand or, conversely, a strategic minimization in the face of high demand \parencite{Brunken2003,vanGogPaas2008}.

Second, subjective–direct measures target perceived difficulty or stress explicitly attributed to the materials or delivery format. For example, participants can judge, on a graded scale, how easy or difficult computer-based instructions were to understand \parencite{Kalyuga1999}. The intended advantage is a closer linkage to the load imposed by the materials.

Third, objective–indirect measures register behavioral or physiological consequences of demand. Behavioral indices include time on task and error patterns, which typically covary with increases in cognitive complexity \parencite{Brunken2003}. Physiological indices include cardiac activity, pupil dilation (mean level and dilation range), blink rate, eye movements and fixations, and skin conductance. These measures avoid subjective bias, yet they remain indirect because multiple factors can intervene between cognitive state and observed behavior \parencite{Brunken2003}.

Fourth, objective–direct measures aim to index capacity sharing itself. Neuroimaging can, in principle, reveal task-related activation; however, links between such activation and cognitive load during elaborated L2 communication remain limited, and practical constraints are substantial \parencite{Brunken2003}. For applied settings, a dual-task paradigm is typically recommended. A simple secondary probe draws on the same general pool of attention; when primary-task demand rises, responses to the probe slow or become less accurate \parencite{brunken2002}. Probe design matters for sensitivity: if the secondary task is too easy (e.g., background-color change), participants can maintain performance regardless of primary-task load, but when selection demands are higher (e.g., detecting a letter-color change in a defined region), differences between presentation modes and task versions are detected more reliably \parencite{Schoor2012}.

Within task-based language teaching, these four families have been adapted to validate designed task complexity independently of linguistic performance. Self-rating questionnaires differentiate nominally simple from complex tasks, yet they primarily index perceived difficulty rather than the intended design complexity \parencite{Robinson2001}. Introspective methods, including stimulated recall and interviews, can demonstrate whether the more complex version actually recruited the intended reasoning and comparison operations even when perceived difficulty is similar \parencite{Kim_Payant_Pearson_2015}. Retrospective time-estimation tasks also distinguish complex from simple versions in the predicted direction: under higher reasoning demands, participants tend to overestimate elapsed time relative to the actual duration \parencite{Baralt_2013}. Eye tracking further shows that higher reasoning demands elicit more and longer fixations than simpler counterparts; in the same studies, dual-task probes yield slower and less accurate responses under the more complex condition, providing a capacity-based check on the subjective and introspective evidence \parencite{revesz2014}. 

\subsection{Requirements for Reliable Validation}
A recurring challenge is ensuring that designed task differences actually map onto cognitive load in practice \parencite{revesz2014}. Subjective self-ratings of difficulty or effort are informative but, on their own, insufficient for validating manipulations \parencite{crowther2018,fullana2024}.

Evidence from capacity-based validations reveals two critical boundary conditions:

First, manipulations must be large enough to register. Only substantial increases in element interactivity reliably produce measurable effects across dual-task, time-estimation, and self-rating measures. For example, in monologic speaking tasks, contrasts between one-element and nine-element tasks showed robust differences, whereas intermediate steps (e.g., three versus five elements) were inconsistently distinguished \parencite{sasayama2016}.

Second, subjective and objective indices can diverge. Self-ratings often show larger effects than dual-task or reaction-time measures. This divergence likely occurs because subjective ratings can be influenced by surface-level task features (e.g., visual clutter, time pressure), whereas objective probes may fail to detect genuine load if the secondary task is poorly calibrated or if participants disengage strategically \parencite{sasayama2016}.

The practical implication is straightforward: validation requires multiple, independent measures. Secondary probes should have sufficient attentional demands and irregular timing to prevent strategic anticipation, while avoiding sensory overlap with the primary task \parencite{Schoor2012,brunken2002}.

These principles motivate the converging-evidence approach throughout this thesis: subjective effort ratings are paired with secondary-task performance in the production studies (Studies 1--2) and with pupillometric indices in the comprehension studies (Studies 3--5).




\section{Cognitive Load and Self-Efficacy}\label{sec:load-selfeff}

Beyond verifying the existence of cognitive load through these objective measures, it is also critical to understand its consequences for learners' affective and motivational states. Cognitive load and self-efficacy, in particular, influence one another in both the short term and across longer spans. Recent syntheses integrate CLT with models of expectancy, value, and cost by treating extraneous load as motivational cost and by documenting direct, task-specific effects of load on self-efficacy \parencite{Feldon2019}. Under this view, reducing avoidable coordination lowers perceived cost and protects concurrent self-efficacy, while calibrated intrinsic demand can raise expectancy when it produces visible learning benefits.

Self-efficacy, an individual's belief in their capability to perform a specific task, is among the most powerful motivational predictors of academic achievement \parencite{bandura_self-efficacy_1997}. Meta-analyses demonstrate that higher self-efficacy predicts greater persistence, more adaptive strategy use, and higher achievement across educational levels and domains \parencite{honicke_influence_2016,richardson_psychological_2012}. Furthermore, longitudinal work has revealed a reciprocal relationship: performance success nurtures subsequent self-efficacy, while elevated self-efficacy feeds forward into future achievement, although performance tends to be the stronger driver \parencite{talsma_i_2018}.

Contemporary integrative accounts position experienced effort as the bridge between object-level processing and meta-level control; within this framework, self-efficacy functions simultaneously as an antecedent to cognitive processing and as an outcome that is updated by the experience of allocating effort under varying demands \parencite{de2020synthesizing,de_bruin_synthesizing_2020,wang_how_2023}. Although strategic behaviors such as planning, monitoring, and control can reduce extraneous load, they simultaneously impose metacognitive demands that consume the same cognitive resources they aim to liberate. The present thesis therefore samples effort and self-efficacy at section-level granularity where applicable, so that inferences respect the temporal profile of load and belief updating.

The following subsections synthesize empirical evidence for bidirectional influences between cognitive load and self-efficacy. Section~\ref{subsec:load-efficacy-links} reviews evidence across multiple time scales. Section~\ref{subsec:design-efficacy} examines instructional features that mediate or moderate this relationship. 

\subsection{Mutual Influences Between Cognitive Load and Self-Efficacy}\label{subsec:load-efficacy-links}

Experimental and longitudinal studies demonstrate that instructional load influences self-efficacy through pathways independent of achievement. For example, in an undergraduate science course, materials designed to reduce extraneous load produced larger pre-to-post gains in self-efficacy than comparison materials \parencite{feldon2018self}. These differences persisted after controlling for performance, suggesting that reduced cognitive burden directly influenced capability beliefs rather than operating solely through improved achievement. This temporal dissociation is consistent with observations that confidence can dip while learners work through difficult content, yet strengthen after sustained, supported success with necessary complexity.

Converging evidence across domains reinforces the robustness of this pattern. Recent physiological studies reveal cognitive load spikes at precise moments: pupillometry shows sharp effort increases during speaker or accent switches in sentence comprehension \parencite{mclaughlin_sequence_2024}, while fixation-locked EEG with eye-tracking captures transient peaks from task-irrelevant decorative images \parencite{scharinger_task-irrelevant_2024}. Diary studies indicate that perceived difficulty spikes precede immediate self-efficacy dips, especially when baseline self-efficacy is low \parencite{stoten_metacognition_2019}.

In creative problem-solving, higher creative self-efficacy corresponds to lower experienced load, and competence-relevant feedback reduces load while sustaining performance; mediation analyses indicate that part of feedback's benefit flows through reduced processing strain \parencite{redifer2021self}. In complex skills acquisition, guided formats such as worked examples or structured multimedia reduce perceived effort for novices and are associated with higher self-efficacy than unguided problem-solving when task complexity is high \parencite{zheng2008effects}. Field studies in authentic learning settings show that academic self-efficacy tracks working-memory-relevant demands, though sample sizes are sometimes small and subgroup differences counsel caution \parencite{vasile2011academic}. These micro-dynamics have cumulative consequences. Sustained high load leads to cognitive fatigue and performance declines \parencite{van_der_linden_mental_2003}. Conversely, optimally managed cognitive load fosters effective schema acquisition \parencite{sweller_cognitive_2011}, creating successful performance experiences that strengthen self-efficacy and adaptive strategy use \parencite{bandura_self-efficacy_1997,britner_sources_2006}.

Competing theoretical accounts exist: a proactive view posits that effective metacognitive monitoring guides regulation and performance \parencite{thiede_2003}, while a reactive view suggests that poor metacognition or maladaptive beliefs prompt disengagement, heightening overload vulnerability \parencite{schwonke_metacognitive_2015}. This latter view is supported by modern CLT, which formally identifies learner-internal states as a source of extraneous load. For example, working memory resources consumed by "intrusive worries about failure" constitute a form of extraneous load that can be reduced by "stress-suppressing activities," thereby freeing capacity for the primary task \parencite{paas2020}. The evidence provides strong support for a reactive pathway in which momentary increases in cognitive load are associated with subsequent declines in self-efficacy. 

\subsection{Instructional Mediators and Moderators of Cognitive Load and Self-Efficacy}\label{subsec:design-efficacy}

Several instructional features systematically shape the relation between experienced cognitive load and self-efficacy by altering where effort is spent and how that effort is interpreted. As demonstrated in the \citeauthor{feldon2018self}'s \citeyear{feldon2018self} study reviewed above, reducing extraneous processing demands through explicit task analysis and clear structure is associated with larger gains in self-efficacy than comparison instruction. In ordinary classroom implementations, a climate that combines clear structure with autonomy support corresponds to lower perceived extraneous and intrinsic load together with a stronger motivational environment that is conducive to self-efficacy growth \parencite{evans2024}.

Sequencing of task complexity moderates this process. Progressions that begin at moderate difficulty and then increase complexity foster germane processing and meta-awareness more reliably than uniformly high difficulty, and these process changes co-occur with gains in interest and performance that support later willingness to invest effort when interactivity increases \parencite{zeitlhofer2024complexity}. Models that integrate cognitive load with self-regulation clarify why these levers matter. Accounts that treat experienced effort as the bridge between processing and monitoring predict that learners require some bandwidth for regulation in order to interpret effort as informative rather than discouraging \parencite{de2020synthesizing}. Crucially, learners with stronger self-efficacy are typically more willing to expend the effort needed for self-regulated learning, thereby preventing overload; in contrast, learners with low self-efficacy often disengage, allowing intrinsic or extraneous load to overwhelm working memory \parencite{schwonke_metacognitive_2015}. Consistent with this prediction, interventions that keep difficulty near a moderate level elicit more adaptive strategy use than uniformly easy or uniformly hard tasks, and observed changes in strategy deployment are mediated by perceived load in an inverted-U-shaped pattern \parencite{seufert2024interplay}. In combination, these findings indicate that reductions in avoidable demand, calibrated sequencing, competence-focused feedback, and preserved regulation bandwidth function as mediators and moderators of the link between cognitive load and self-efficacy rather than as independent instructional goals \parencite{evans2024,Feldon2019}.

The literature reviewed above has several implications for the measurement approach adopted in this thesis. First, given that cognitive load and self-efficacy can fluctuate within a single instructional episode, both constructs are measured at section-level granularity rather than only at task completion \parencite{stoten_metacognition_2019}. This approach allows detection of transient spikes in processing demand and corresponding shifts in confidence that end-of-task ratings alone might miss \parencite{singer_reading_2017}.
\section{Remaining Questions}\label{sec:gaps-and-rqs}


This chapter established the theoretical and methodological foundations for investigating cognitive load in L2 contexts. It specified the cognitive architecture underlying CLT (\autoref{sec:definitions}), characterized behavioral signatures of load (\autoref{sec:consequences}), reviewed design principles (\autoref{sec:manipulations}), detailed measurement and validation (\autoref{sec:verification}), and examined load–efficacy coupling (\autoref{sec:load-selfeff}). Despite this progress, critical empirical and methodological gaps remain. This section synthesizes those gaps and maps them to the Research Questions presented in Chapter~\ref{ch:intro-roadmap}. A cross-cutting concern is generalizability beyond English. As noted in \autoref{subsec:jp-func-words}, this thesis targets Japanese, a typologically distant language with agglutinative morphology and SOV order, testing whether documented load effects and design principles extend to learners navigating these structural differences. Against this backdrop, three empirical gaps structure the work that follows.


\begin{description}
    \item[Gap 1: Production:] In production, while trade-offs among CAF under pressure are well described (\autoref{subsec:prod-sigs}), it is not clear how a verified increase in demand alters native listeners' weighting of linguistic dimensions when judging comprehensibility and fluency. Study 1 first identifies redistribution patterns in speaking. Study 2 then models how judgments of the same recordings are predicted by observable cues, testing whether cue weights shift with task demand.
    \item[Gap 2: Comprehension:] Although cognitive load and self-efficacy are linked (\autoref{sec:load-selfeff}), most studies measure these constructs over long periods (e.g., semester-long), obscuring short-term fluctuations that could reveal how they influence each other moment-to-moment. We address this by measuring both constructs at the section level within tasks (Study 3).
    \item[Gap 3: Comprehension—process dynamics:] Most studies focus on temporal signatures while neglecting spatial organization of attention in L2 text reading. Moreover, aggregate measures obscure within-passage fluctuations in cognitive load that could reveal how readers allocate attention. Study 4 integrates spatial scan-path metrics with temporal eye-movement indices in text to characterize how attention is spatially organized. Study 5 addresses the masking problem by modeling continuous pupil trajectories across text and video, revealing time-localized cognitive load dynamics that aggregate measures cannot capture.
    
\end{description}
