%************************************************
\chapter{Study 3: Cognitive Load and Self-Efficacy in Comprehension}\label{ch:study3}
%************************************************


\section{Introduction}



Modern L2 learners increasingly study disciplinary content through multimedia resources such as massive open online courses (MOOCs), flipped-classroom videos, and online explainers \parencite{macaro_systematic_2018}. However, few studies have systematically crossed domain, modality, and linguistic complexity within a single design to examine their joint effects on cognitive load, self-efficacy, and comprehension. From a CLT perspective, these factors influence both intrinsic load (domain conceptual density, linguistic complexity) and extraneous load (modality-specific demands), with dual-channel frameworks predicting interactions among them \parencite{mayer_multimedia_2002, moreno2006}. This gap leaves instructors without evidence-based guidance on how to combine design elements to support learning without inducing overload or undermining learner confidence.

Study 3 addresses this gap through a repeated-measures design in which Chinese L1 learners studying in Japan studied materials that varied systematically by domain (science, history), modality (text, video), and linguistic complexity (simple, complex). Cognitive load and self-efficacy were assessed after each of three passage sections to capture within-task fluctuations, and comprehension was assessed through multiple-choice and short-answer questions. This approach enabled us to examine how task factors shape section-level cognitive load and self-efficacy, whether learners' academic major moderates these effects, how load and efficacy interact dynamically across sections, and how this cognitive-motivational landscape predicts learning outcomes.


\section{Background}

\subsection{Domain-Specific Influences: Science Versus Humanities}

Task domain is a primary yet understudied determinant of cognitive and motivational demands. Scientific learning materials typically present dense expository prose replete with technical lexicon and hierarchically organized explanations, whereas history materials tend to adopt narrative formats organized around chronological sequences and causal linkages among various past events \parencite{shanahan2008teaching}. Comparative comprehension studies have shown that expository science passages impose higher intrinsic load and depend more heavily on prior conceptual knowledge than equivalent narrative history passages \parencite{kintsch_text_1994, wolfe_learning_2007}. These cognitive differences coincide with distinct motivational profiles: adolescents' self-efficacy is generally lower and more volatile in Science, Technology, Engineering, and Mathematics (STEM) than in humanities subjects \parencite{han_trajectories_2021}. Learners who doubt their scientific ability report higher perceived effort for identical tasks and are quicker to attribute difficulties to their lack of ability rather than to the material's inherent complexity \parencite{usher_sources_2009}. However, as few experimental studies have manipulated domain while concurrently measuring cognitive load and self-efficacy, the question of how content characteristics shape the cognitive-motivational landscape remains open. Nonetheless, the collective evidence implies that the combination of science content and high conceptual density places a dual burden on working memory and self-efficacy. These contrasts motivate our major \(\times\) domain test: science majors reading science should show different sensitivity to linguistic complexity than arts or social sciences majors, whose responses in science may be driven more by the domain’s conceptual density.




\subsection{Modality Effects: Text Versus Video}

As summarized in \autoref{subsec:extraneous}, well aligned dual channel presentations can reduce avoidable mapping, whereas transience can raise demands when pacing is fixed. However, direct comparisons often find parity, or even a text advantage, once pacing and presentation are matched. Research has shown no difference in immediate comprehension between modalities, though delayed performance was superior for text \parencite{tarchi_learning_2021}. Similarly, other studies have reported comparable comprehension, cognitive-effort, and calibration outcomes for self-paced video blogs versus written blogs \parencite{delgado_learning_2022}. These divergent findings indicate that the degree of learner control over pacing is a key moderating factor, as text inherently provides efficient, low-effort control while video's transience requires more effortful regulation. We therefore treat modality effects as conditional on intrinsic demand and learner control, predicting the largest text advantage when conceptual density and linguistic complexity are high and pacing cannot be slowed on demand \parencite{tarchi_learning_2021,delgado_learning_2022}.



\subsection{Linguistic Complexity and Processing Demands}
The linguistic complexity of instructional materials is a critical determinant of cognitive load, but psycholinguistic work has not reached consensus on the primary source of this burden for L2 learners. One perspective, rooted in dependency-locality theory, focuses on syntactic structure, predicting that processing cost rises with the linear distance between syntactically related words \parencite{Gibson1998, gibson_dependency_2000}. An alternative view, however, emphasizes that L2 learners often rely more on lexical-semantic cues than on syntactic parsing, making them particularly susceptible to overload from low-frequency vocabulary or unfamiliar terminology \parencite{clahsen_how_2006}. Regardless of the source of difficulty, heightened linguistic demands carry motivational costs: repeated struggles with complex passages have been reported to depress reading self-efficacy \parencite{wu_self-efficacy_2012}. While this debate is ongoing, it confirms that linguistic demands are critical. Thus, higher linguistic complexity is expected to tax working memory and erode learner self-efficacy, especially in an L2 context.



\subsection{Consequences for Comprehension Performance}

\autoref{sec:load-selfeff} established that cognitive load and self-efficacy influence performance bidirectionally. Two further considerations shape comprehension outcomes: question format and their joint predictive effects. Item-format studies have shown that multiple-choice and constructed-response questions draw on partially distinct cognitive resources \parencite{kuechler_why_2010}. High self-efficacy, by contrast, predicts deeper processing, more effective strategy deployment, and reduced anxiety, thereby improving performance \parencite{honicke_influence_2016}. Emerging evidence suggests that these factors interact: learners with robust self-efficacy can sometimes maintain germane load under moderately high intrinsic load, whereas those with low self-efficacy struggle even under reduced load \parencite{eitel_self-management_2020}.
However, the relative contributions of domain, modality, linguistic complexity, and momentary cognitive-motivational states to discrete comprehension outcomes—reasoning, factual recall, and question answering—remain poorly understood. Consequently, comprehension is likely to suffer most when heavy cognitive load converge with fragile self-efficacy, whereas robust self-efficacy can buffer the impact of cognitive load.




\section{Research Questions}


As reviewed in \autoref{subsec:load-efficacy-links}, cognitive load and self-efficacy influence each other bidirectionally across multiple timescales. Momentary increases in processing demand correspond to concurrent decreases in self-efficacy, while sustained engagement with calibrated intrinsic demands can yield later efficacy gains once learners succeed under those conditions. \autoref{subsec:design-efficacy} further establishes the value of measuring both constructs at section-level granularity rather than only at task completion. This approach allows detection of transient spikes in processing demand and corresponding shifts in confidence that end-of-task ratings alone would miss. Building on this framework, the present study examines how domain, modality, and linguistic complexity jointly shape these section-level trajectories and how the resulting cognitive-motivational landscape predicts comprehension.

The literature reviewed in the Background section highlights three key empirical gaps that motivate this investigation. First, domain, linguistic complexity, and modality have seldom been manipulated concurrently; thus, their individual and combined effects on cognitive load and self-efficacy remain unknown. Second, most existing studies have assessed these constructs only once per task, neglecting within-task fluctuations that could reveal causal ordering or critical tipping points. Third, limited research has modeled how section-level cognitive load and self-efficacy jointly predict multiple comprehension indicators across different domains and modalities.

To address these gaps, this study employed a repeated-measures design. Our primary research questions concerned how these task characteristics influence cognitive load and self-efficacy.


\begin{description}
    \item [RQ1:] How does cognitive load vary as a function of domain, modality, and linguistic complexity?
    \item [RQ2:] How does self-efficacy vary as a function of the same task factors?
\end{description}


Based on the literature, we expect the inherent conceptual density of science materials and the processing demands of complex syntax to be the primary drivers of cognitive load and to negatively impact self-efficacy. We also anticipate that the greater learner control afforded by text will provide a cognitive advantage over video, especially when intrinsic load is already high. Therefore, we propose the following hypotheses related to RQ1 and RQ2:

\begin{description}
    \item [H1:] Academic expertise will moderate intrinsic-load effects on both cognitive load and self-efficacy, yielding a domain $\times$ linguistic complexity $\times$ major interaction. Linguistic complexity costs will be pronounced for science majors within science, whereas arts/social sciences majors' responses in science will be driven primarily by the domain's high conceptual density, with attenuated sensitivity to syntactic complexity.
    \item [H2:] The effect of presentation modality on cognitive load and self-efficacy will be conditional on intrinsic load, yielding a domain $\times$ linguistic complexity $\times$ modality interaction; the modality that affords greater learner control will show its largest advantage under high intrinsic load (high conceptual density and linguistic complexity).
\end{description}



Next, we seek to understand the dynamic, moment-to-moment relationship between these two constructs during the learning process.

\begin{description}
    \item[RQ3:] How does cognitive load influence self-efficacy across task sections, and is this relationship consistent across domain, linguistic complexity, and modality?
\end{description}


Drawing on studies showing that spikes in cognitive load often precede dips in self-efficacy, we aim to test a reactive pathway where cognitive load directly impacts subsequent self-efficacy.

\begin{description}
    \item[H3:] A reactive control pathway will be evident, where momentary increases in cognitive load will negatively predict self-efficacy ratings in the subsequent task section.
\end{description}



Finally, we aim to model how these cognitive-motivational factors jointly predict learning outcomes, taking into account the demands of the comprehension questions themselves.

\begin{description}
    \item[RQ4:] How do domain, modality, linguistic complexity, question category, and section-level cognitive load and self-efficacy jointly predict performance accuracy?
\end{description}


Building on the premise that excessive cognitive load impairs performance while high self-efficacy can buffer against such demands, we predict an interactive effect on comprehension.

\begin{description}
    \item[H4:] The negative impact of cognitive load and the positive impact of self-efficacy on comprehension accuracy will be most pronounced in high-demand conditions, such as the science domain and for reasoning or comprehension items that require deeper processing.
\end{description}


%revise second 




\section{Statistical Analysis}




\begin{table*}[ht]
  \centering
  \caption{Descriptive statistics by conditions and major across sections.}
  \label{tab:descriptive_stats_major}
  \footnotesize
  \setlength{\tabcolsep}{1.5pt}
  \begin{threeparttable}
    \begin{tabular}{l c c c c c c}
      \toprule
      & \multicolumn{6}{c}{\textsc{Major: Arts/Social Sciences}} \\
      \cmidrule(lr){2-7}
      & \multicolumn{2}{c}{Section 1} & \multicolumn{2}{c}{Section 2} & \multicolumn{2}{c}{Section 3} \\
      \cmidrule(lr){2-3} \cmidrule(lr){4-5} \cmidrule(lr){6-7}
      Condition & CL & SE & CL & SE & CL & SE \\
      & \textit{M (SD)} & \textit{M (SD)} & \textit{M (SD)} & \textit{M (SD)} & \textit{M (SD)} & \textit{M (SD)} \\
      \midrule
      \multicolumn{7}{@{}l}{\textit{History Domain}} \\
      \addlinespace[0.5ex]
      Text + Simple & $4.67(1.36)$ & $4.11(1.63)$ & $4.67(1.73)$ & $4.11(1.69)$ & $4.44(1.53)$ & $4.11(1.89)$ \\
      Text + Complex & $4.38(1.53)$ & $4.62(0.88)$ & $4.12(1.65)$ & $4.50(1.25)$ & $4.25(1.67)$ & $4.75(1.22)$ \\
      Video + Simple & $4.00(1.91)$ & $5.62(1.01)$ & $3.88(1.73)$ & $5.62(1.24)$ & $3.75(1.89)$ & $5.38(1.53)$ \\
      Video + Complex & $5.22(1.05)$ & $4.78(0.93)$ & $5.78(0.93)$ & $4.11(1.55)$ & $5.56(0.97)$ & $3.89(1.55)$ \\
      \addlinespace[0.5ex]
      \multicolumn{7}{@{}l}{\textit{Science Domain}} \\
      \addlinespace[0.5ex]
      Text + Simple & $4.25(1.89)$ & $5.00(1.02)$ & $4.75(1.42)$ & $4.38(1.01)$ & $4.25(1.67)$ & $4.75(1.11)$ \\
      Text + Complex & $4.78(1.25)$ & $4.22(1.34)$ & $5.44(1.09)$ & $3.56(1.45)$ & $5.56(0.85)$ & $3.89(1.55)$ \\
      Video + Simple & $5.56(0.51)$ & $3.33(1.07)$ & $5.78(0.80)$ & $3.56(1.45)$ & $5.78(0.80)$ & $3.33(0.96)$ \\
      Video + Complex & $5.00(1.44)$ & $3.25(1.51)$ & $5.12(1.48)$ & $3.62(1.13)$ & $5.12(1.19)$ & $3.62(1.24)$ \\
      \midrule
      & \multicolumn{6}{c}{\textsc{Major: Science}} \\
      \cmidrule(lr){2-7}
      \multicolumn{7}{@{}l}{\textit{History Domain}} \\
      \addlinespace[0.5ex]
      Text + Simple & $4.91(1.53)$ & $4.09(1.86)$ & $5.09(1.33)$ & $4.00(1.84)$ & $5.27(1.44)$ & $4.18(1.55)$ \\
      Text + Complex & $4.25(1.75)$ & $4.88(1.48)$ & $4.25(1.82)$ & $5.25(2.03)$ & $4.12(1.39)$ & $4.62(1.84)$ \\
      Video + Simple & $4.38(1.24)$ & $4.62(1.01)$ & $4.25(1.59)$ & $4.75(0.68)$ & $5.12(1.65)$ & $4.12(1.30)$ \\
      Video + Complex & $5.55(1.25)$ & $3.73(1.63)$ & $5.55(1.33)$ & $3.45(1.52)$ & $5.55(1.33)$ & $3.36(1.52)$ \\
      \addlinespace[0.5ex]
      \multicolumn{7}{@{}l}{\textit{Science Domain}} \\
      \addlinespace[0.5ex]
      Text + Simple & $4.50(1.35)$ & $5.12(1.39)$ & $5.00(1.44)$ & $4.88(1.19)$ & $4.75(1.22)$ & $5.50(1.44)$ \\
      Text + Complex & $5.64(1.25)$ & $4.27(1.94)$ & $6.09(0.80)$ & $3.91(1.53)$ & $5.91(1.18)$ & $4.00(1.62)$ \\
      Video + Simple & $4.73(1.74)$ & $4.36(1.69)$ & $5.18(1.42)$ & $4.00(1.73)$ & $5.27(1.44)$ & $3.82(1.72)$ \\
      Video + Complex & $5.50(1.25)$ & $2.75(1.11)$ & $5.25(0.99)$ & $3.12(1.19)$ & $5.50(1.25)$ & $2.88(1.57)$ \\
      \bottomrule
    \end{tabular}
    \begin{tablenotes}[flushleft]
      \footnotesize
      \item \textit{Note.} Values are means with standard deviations in parentheses.
      \item CL = Cognitive Load; SE = Self-efficacy.
    \end{tablenotes}
  \end{threeparttable}
\end{table*}

All analyses were conducted in R (version 4.3.0; \parencite{RCoreTeam2023}). Linear mixed-effects models for research questions 1--3 were fitted with the lme4 package \parencite{bates2015}. The Bayesian multilevel mediation for research question 4 was implemented with the brms package \parencite{Burkner2017}. For the linear mixed-effects models, we conducted post hoc tests using the emmeans package \parencite{emm} to compute Tukey-adjusted pairwise comparisons and Cohen's d effect sizes from the estimated marginal means (EMMs). Descriptive statistics by domain, modality, major, and linguistic complexity are presented in Table \ref{tab:descriptive_stats_major}. This study was not preregistered.



The hypotheses regarding the effects of the task design features on cognitive load and self-efficacy (H1--H2) and the directional relationship between them (H3) were confirmatory, as they were derived directly from the literature reviewed. The analysis of how these factors jointly predict performance accuracy (H4) was considered exploratory, given the limited prior research on these specific multi-way interactions. 




%---revision---

To address research questions 1 and 2, we fitted separate linear mixed-effects models predicting cognitive load and self-efficacy. Our analytical approach was established as follows: an initial maximal model including all possible interactions was specified before data collection. To create a more focused, parsimonious model that directly tested our primary hypotheses (H1 and H2), this initial model was simplified. The key three-way interactions remained significant and the patterns of post-hoc comparisons were largely unchanged in the initial maximal model (see Supplementary Material). The final fixed effects structure was specified to include main effects for domain, modality, linguistic complexity, section, and major, as well as two key hypothesized three-way interactions: 1) a domain $\times$ linguistic complexity $\times$ major interaction (H1), to test how expertise moderates the effects of intrinsic cognitive load, and 2) a domain $\times$ linguistic complexity $\times$ modality interaction (H2), to test how the effect of modality is conditional on intrinsic load. 
Our analysis plan pre-specified key post hoc contrasts to test the nature of these interactions. For H1, we planned to test the simple effect of linguistic complexity within each domain, comparing the pattern for in-domain learners (e.g., science majors on science content) versus out-of-domain learners (e.g., science majors on history content). For the H2 interaction, we planned to test three key contrasts: the effect of modality within the two high intrinsic load conditions (\emph{complex + science} and \emph{complex + history}), and a key crossover comparison (\emph{simple + science + video} versus \emph{complex + history + text}) as the clearest test of the hypothesized three-way interaction. The models also included the length of stay in Japan and Japanese self-efficacy (summed across all 11 baseline scale items), both $z$-standardized, as well as gender, entered as covariates. A random intercept by participant was included to account for individual variability. Gender was included, given documented gender differences in academic self-efficacy, with males typically showing advantages in STEM domains, whereas females typically report higher confidence in language arts \parencite{huang2013}. Length of stay in Japan was entered as a covariate, serving as a proxy for learners' cumulative exposure to the L2 environment \parencite{flege_1995}. Japanese self-efficacy served as a baseline control, as general language self-efficacy consistently predicts academic achievement beyond other motivational factors \parencite{mills2007}. These covariates allowed for a more precise examination of how task-specific factors influence cognitive load and self-efficacy.



%---


For research question 3, a series of linear mixed-effects models was fitted to examine how cognitive load influences self-efficacy and whether this relationship is consistent across task factors. In contrast to the confirmatory approach for research questions 1 and 2, the analysis for research question 3 required a more focused model selection strategy to determine the optimal representation of the cognitive load predictor. An initial step involved comparing three specifications for the cognitive load predictor to determine its optimal representation: 1) raw ratings (for simple linear effects); 2) participant-mean centered ratings (each section's cognitive load rating minus the participant's mean; for within-participant effects); and 3) raw ratings with a quadratic term (raw cognitive load ratings with an added quadratic term; for nonlinear effects). The most appropriate and parsimonious specification was selected based on likelihood-ratio tests, the Akaike information criterion (AIC), and the Bayesian information criterion (BIC). The final model predicted self-efficacy from the participant-mean centered cognitive load. All models incorporated main effects for cognitive load and the five task factors (domain, modality, linguistic complexity, major, and section) to assess moment-to-moment variations, along with all two-way interactions between centered cognitive load and each task factor to examine moderation. The covariates (length of stay in Japan, Japanese self-efficacy, and gender) and a random intercept by participant were included, consistent with the models for research questions 1 and 2. Section was treated as an ordinal predictor to model a linear trend and maintain parsimony.



For research question 4, we first tested whether cognitive load directly predicted item-level performance accuracy using a generalized linear mixed-effects model (Bernoulli outcome, logit link). The fixed effects were grand-mean centered cognitive load (each section's cognitive load rating with the sample mean subtracted), fully interacted with domain, modality, linguistic complexity, major, section (1, 2, or 3, corresponding to the question section), and question category (recall, reasoning, or comprehension). This model also included covariates and a random intercept by participant. As cognitive load did not yield a reliable main effect on performance, we subsequently estimated an indirect pathway via self-efficacy using a Bayesian multilevel mediation framework. For the Bayesian multilevel mediation analysis involving the Bernoulli outcome (performance accuracy), it is important to note that the coefficients for the outcome model (paths $b$ and $c'$) are on the log-odds scale. Therefore, the indirect effect (the product $a\times b$) does not represent a direct change in probability. To aid interpretation, we visualize key effects on the probability scale in the figures and report the probability of direction (pd) alongside $95\%$ credible intervals (CrIs) in our results table. The mediator model predicted grand-mean centered self-efficacy from cognitive load and the five task factors, and the outcome model predicted item-level performance accuracy from cognitive load and self-efficacy, each fully interacted with the same moderators and question category. Because our continuous predictors were mean-centered but not scaled, we used weakly regularizing priors: $\mathcal{N}(0,1)$ on fixed effects and Student's $t(3,0,2.5)$ on group-level standard deviations. Both brms models ran four chains (8000 iterations with 4000 warm-ups for the mediator and 6000 iterations with 3000 warm-ups for the outcome). Posterior medians and 95\% CrIs summarized indirect ($a\times b$) and direct ($c'$) effects, while Bayesian $R^2$ quantified explained variance \parencite{gelman2019r}.


To enhance readability, we adopt a consistent reporting strategy for all mixed-effects models. The main text presents Type III ANOVA results computed with the lmerTest package \parencite{lmertest}, focusing on key main effects and statistically significant interactions. Comprehensive statistical details are provided in the Supplementary Material, including: (a) complete ANOVA tables with all model terms; (b) full fixed effect parameter estimates with coefficients, standard errors, and test statistics; and (c) complete post hoc comparisons for all significant interactions of RQ1 and RQ2.

\section{Results}


\subsection{Model diagnostics}




All key model diagnostics (residual distribution, homoscedasticity, influential observations, convergence behavior, and dispersion) were evaluated for every fitted model using the performance package \parencite{Ludecke2021}. For the three linear models (research question 1, and research question 2), residuals were approximately normal and homoscedastic; no influential outliers were detected on the basis of Cook's distance. The within-participant self-efficacy model for research question 3 showed some departure from normality ($p<.001$) but exhibited homoscedastic residuals ($p=.455$) and no influential cases. All inferences for fixed effects were based on Satterthwaite degrees of freedom and robust confidence intervals, which are resilient to moderate nonnormality. Residual nonnormality was handled with a 1000-iteration parametric bootstrap; the bootstrapped $95\%$ CIs corroborated every significant fixed effect. For the logistic mixed model addressing research question 4, the final specification converged without boundary singularities, yielded scaled residuals centered within the interval $[-3, 3]$, and showed no evidence of overdispersion; the optimizer returned convergence code 0. Across all Bayesian chains (mediation subanalyses), convergence was confirmed for all parameters with $\hat R < 1.01$ and a minimum bulk and tail Effective Sample Size (ESS) of approximately 4600; no divergent transitions were observed. Collectively, these checks indicate that the modeling assumptions were satisfactorily met, and subsequent statistical inferences are trustworthy.



\subsection{RQ1: Cognitive Load Across Domain, Modality, and Linguistic Complexity}



Fixed effects explained $17\%$ of the variance ($R^{2}_{\mathrm{m}} = 0.17$); the full model explained $50\%$ ($R^{2}_{\mathrm{c}} = 0.50$). As shown in Table~\ref{tab:anova_rq1}, the analysis revealed significant main effects for domain, modality, and linguistic complexity. Participants reported higher cognitive load for science passages, video modality, and complex language. In line with our hypotheses, the two primary three-way interactions were significant (see Figure \ref{fig:rq1_0}). First, supporting H1, a significant domain $\times$ linguistic complexity $\times$ major interaction emerged ($p = .002$). Second, supporting H2, a significant domain $\times$ linguistic complexity $\times$ modality interaction was also found ($p = .011$).


The H1 interaction revealed a symmetrical pattern based on learners' domain expertise. For science majors studying in-domain content, the cognitive load for \emph{simple + science} passages ($\mathrm{EMM}=4.95$) was significantly lower than for \emph{complex + science} passages ($\mathrm{EMM}=5.74$; $\Delta=-0.79$, $p<.001$, $d=-0.72$). Conversely, when studying out-of-domain history passages (\emph{complex + history} versus \emph{simple + history}), their cognitive load showed no reliable difference by linguistic complexity ($p=.995$). A parallel pattern emerged for arts/social sciences majors: their cognitive load was significantly higher for \emph{complex + history} ($\mathrm{EMM}=4.78$) than for \emph{simple + history} ($\mathrm{EMM}=4.13$; $\Delta=-0.65$, $p=.016$, $d=-0.59$), but their cognitive load was uniformly high and unaffected by linguistic complexity for out-of-domain science passages (\emph{complex + science} versus \emph{simple + science}; $p=.974$). This indicates that learners were only sensitive to linguistic complexity within their own field of expertise.


The H2 interaction demonstrated that the effect of modality depended on the intrinsic load of the task. For \emph{complex + history} passages, the cognitive load for video ($\mathrm{EMM} = 5.60$) was significantly higher than for text ($\mathrm{EMM} = 4.14$; $\Delta= 1.47$, $p = .008$, $d = 1.34$). This modality effect was not significant for \emph{complex + science} passages ($\Delta= -0.49$, $p = .904$, $d = -0.45$). The interaction was further clarified by a significant crossover pattern: \emph{simple + science + video} ($\mathrm{EMM} = 5.43$) produced significantly higher cognitive load than even \emph{complex + history + text} ($\mathrm{EMM} = 4.14$; $\Delta= 1.29$, $p = .029$, $d = 1.19$). This provides clear evidence that the impact of modality is conditional on the combined intrinsic load from domain and complexity. For a full breakdown of these pairwise comparisons, see the Supplementary Material Tables S4--5.


\begin{figure}[ht]
    \centering
    \includegraphics[width=0.8\linewidth]{Chapters/study3fig/s3f2.pdf}
    \caption{EMMs ($\pm$ 95\% CIs) of cognitive load ratings. (a) domain $\times$ linguistic complexity $\times$ modality. (b) domain $\times$ linguistic complexity $\times$ major.}
    \label{fig:rq1_0}
\end{figure}





\begin{table}[htbp]
\centering
\begin{threeparttable}
\caption{Summary of the cognitive load model.}
\label{tab:anova_rq1}
\setlength{\tabcolsep}{3pt}
\footnotesize
\sisetup{
input-signs = {-},
table-align-text-pre = false,
table-align-text-post = false
}
\begin{tabular}{
l % Effects
S[table-format=2.2]  % SS
S[table-format=2.2]  % MS
S[table-format=1.0]  % df_n
S[table-format=3.0]  % df_d
S[table-format=2.2]  % F
S[table-format=<1.3,table-comparator=true] % p-value
}
\toprule
& {\textit{SS}} & {\textit{MS}} & {\textit{$df_n$}} & {\textit{$df_d$}} & {\textit{F}} & {\textit{p}} \\
\midrule
% Main effects
Domain & 25.03 & 25.03 & 1 & 385 & 21.01 & {$<.001$} \\
Complexity & 16.56 & 16.56 & 1 & 385 & 13.90 & {$<.001$} \\
Major & 1.66 & 1.66 & 1 & 30 & 1.39 & {$\phantom{<}.247$} \\
Modality & 8.76 & 8.76 & 1 & 385 & 7.36 & {$\phantom{<}.007$} \\
Section & 3.63 & 1.81 & 2 & 385 & 1.52 & {$\phantom{<}.219$} \\
Length of stay in Japan & 0.16 & 0.16 & 1 & 30 & 0.13 & {$\phantom{<}.719$} \\
Japanese self-efficacy & 4.30 & 4.30 & 1 & 30 & 3.61 & {$\phantom{<}.067$} \\
Gender & 0.00 & 0.00 & 1 & 30 & 0.00 & {$\phantom{<}.960$} \\
\addlinespace

% Three-way interactions
Domain:Linguistic complexity:Major & 11.24 & 11.24 & 1 & 385 & 9.44 & {$\phantom{<}.002$} \\
Domain:Linguistic complexity:Modality & 8.82 & 8.82 & 1 & 30 & 7.40 & {$\phantom{<}.011$} \\
\bottomrule
\end{tabular}
\begin{tablenotes}[flushleft]
\footnotesize
\item \textit{Note.} All two-way interactions have been omitted for brevity. Full tables in Supplementary Material. 
\end{tablenotes}
\end{threeparttable}
\end{table}

\subsection{RQ2: Self-Efficacy Across Domain, Modality, and Linguistic Complexity}

Fixed effects explained $32\%$ of the variance ($R^{2}_{\mathrm{m}} = 0.32$); the full model explained $61\%$ ($R^{2}_{\mathrm{c}} = 0.61$). The analysis revealed significant main effects for domain, modality, linguistic complexity, and gender. Participants reported lower self-efficacy for science passages, video modality, and complex language, and female participants reported lower self-efficacy than male participants. In line with our hypotheses, the two primary three-way interactions were significant (see Fig. \ref{fig:rq2_0}). First, supporting H1, a significant domain $\times$ linguistic complexity $\times$ major interaction emerged ($p = .019$). Second, supporting H2, a significant domain $\times$ linguistic complexity $\times$ modality interaction was also found ($p = .034$).

The H1 interaction revealed an asymmetrical pattern of expertise moderating the impact of intrinsic load on self-efficacy. For \emph{science majors} studying in-domain content, self-efficacy was significantly higher for \emph{simple + science} passages ($\mathrm{EMM} = 4.68$) than for \emph{complex + science} passages ($\mathrm{EMM} = 3.57$; $\Delta = 1.10$, $p < .001$, $d = 1.08$). Conversely, when studying out-of-domain history passages (\emph{complex + history} versus \emph{simple + history}), their self-efficacy was unaffected by linguistic complexity ($p = .847$). A different pattern emerged for arts/social sciences majors: their self-efficacy was not significantly affected by linguistic complexity in either their in-domain history passages (\emph{complex + history} versus \emph{simple + history}; $p = .304$) or the out-of-domain science passages (\emph{complex + science} versus \emph{simple + science}; $p = .258$). This indicates that for self-efficacy, only learners with domain expertise showed a confidence level that was sensitive to linguistic complexity, and only within their own field.


The H2 interaction demonstrated that the effect of modality depended on the intrinsic load of the task. The modality effect was not significant for either \emph{complex + history} passages ($p = .093$) or \emph{complex + science} passages ($p = .753$). Instead, the interaction was primarily driven by the third key contrast, a significant crossover pattern: \emph{simple + science + video} condition produced significantly lower self-efficacy ($\mathrm{EMM} = 3.91$) than \emph{complex + history + text} condition ($\mathrm{EMM} = 5.13$; $\Delta = -1.22$, $p = .041$, $d = -1.19$). This supports H2 by confirming that modality effects are conditional on intrinsic load, though it reveals this through a complex crossover dynamic rather than a simple text advantage under high cognitive load. The complete set of post hoc comparisons is detailed in the Supplementary Material Tables S11--12.


%---------

\begin{figure}[ht]
  \centering
  \includegraphics[width=0.8\linewidth]{Chapters/study3fig/s3f3.pdf}
  \caption{EMMs ($\pm$ 95\% CIs) of self-efficacy ratings. (a) domain $\times$ linguistic complexity $\times$ modality interaction. (b) domain $\times$ linguistic complexity $\times$ major interaction.}
  
  \label{fig:rq2_0}
\end{figure}

\begin{table}[ht]
\centering
\begin{threeparttable}
\caption{Summary of self-efficacy model.}
\label{tab:anova_se_rq2}
\setlength{\tabcolsep}{3pt}
\footnotesize
\sisetup{
input-signs = {-},
table-align-text-pre = false,
table-align-text-post = false
}
\begin{tabular}{
l % Effects
S[table-format=2.2]  % SS
S[table-format=2.2]  % MS
S[table-format=1.0]  % df_n
S[table-format=3.0]  % df_d
S[table-format=2.2]  % F
S[table-format=<1.3,table-comparator=true] % p-value
}
\toprule
& {\textit{SS}} & {\textit{MS}} & {\textit{$df_n$}} & {\textit{$df_d$}} & {\textit{F}} & {\textit{p}} \\
\midrule
% Main effects
Domain & 24.10 & 24.10 & 1 & 385 & 23.06 & {$<.001$} \\
Complexity & 26.22 & 26.22 & 1 & 385 & 25.08 & {$<.001$} \\
Major & 2.29 & 2.29 & 1 & 30 & 2.19 & {$\phantom{<}.149$} \\
Modality & 26.39 & 26.39 & 1 & 385 & 25.25 & {$<.001$} \\
Section & 2.43 & 1.22 & 2 & 385 & 1.16 & {$\phantom{<}.314$} \\
Length of stay in Japan & 0.32 & 0.32 & 1 & 30 & 0.30 & {$\phantom{<}.586$} \\
Japanese self-efficacy & 2.42 & 2.42 & 1 & 30 & 2.31 & {$\phantom{<}.139$} \\
Gender & 13.16 & 13.16 & 1 & 30 & 12.59 & {$\phantom{<}.001$} \\
\addlinespace
% Three-way interactions
Domain:Linguistic complexity:Major & 5.77 & 5.77 & 1 & 385 & 5.52 & {$\phantom{<}.019$} \\
Domain:Linguistic complexity:Modality & 5.14 & 5.14 & 1 & 30 & 4.92 & {$\phantom{<}.034$} \\
\bottomrule
\end{tabular}
\begin{tablenotes}[flushleft]
\footnotesize
\item \textit{Note.} All two-way interactions have been omitted for brevity. Full tables in Supplementary Material. 
\end{tablenotes}
\end{threeparttable}
\end{table}





\subsection{RQ3: Within Person Coupling of Cognitive Load and Self-Efficacy}


Self-efficacy was predicted using a mixed-effects model with participant-mean centered cognitive load and the full set of task factors. Fixed effects explained 28\% of the variance in self-efficacy ($R^{2}_{\mathrm{m}}=0.28$); including a random intercept by participant increased the explained variance to 63\% ($R^{2}_{\mathrm{c}}=0.63$). Within-participant increases in cognitive load reliably lowered self-efficacy ($p<.001$). Furthermore, as shown in Table \ref{tab:anova_rq3}, the participants reported lower self-efficacy for science than for history materials ($p=.014$) and for video than for text ($p<.001$). Furthermore, they reported higher self-efficacy for simple language than for complex language ($p=.004$). Male participants reported higher self-efficacy than female participants ($p=.001$).



While increased cognitive load was generally associated with a decline in self-efficacy, the steepness of this negative relationship (i.e., the slope) varied by task domain. A significant cognitive load $\times$ domain interaction showed steeper self-efficacy declines for science than history passages ($p = .027; \Delta = 0.23, d = 0.23$, 95\% CI [0.02, 0.44]; see Figure \ref{fig:rq3_slope}).


\begin{table}[ht]
\centering
\begin{threeparttable}
\caption{Summary of the within-person centered model.}
\label{tab:anova_rq3}
\setlength{\tabcolsep}{3pt}
\footnotesize
\sisetup{
input-signs = {-},
table-align-text-pre = false,
table-align-text-post = false
}
\begin{tabular}{
l % Effects
S[table-format=2.2]  % SS
S[table-format=2.2]  % MS
S[table-format=1.0]  % df_n
S[table-format=3.2]  % df_d
S[table-format=2.2]  % F
S[table-format=<1.3,table-comparator=true] % p-value
}
\toprule
& {\textit{SS}} & {\textit{MS}} & {\textit{$df_n$}} & {\textit{$df_d$}} & {\textit{F}} & {\textit{p}} \\
\midrule
% Main effects
Cognitive load & 11.79 & 11.79 & 1 & 387.36 & 11.52 & {$<.001$} \\
Domain & 6.19 & 6.19 & 1 & 386.04 & 6.05 & {$\phantom{<}.014$} \\
Modality & 13.65 & 13.65 & 1 & 386.00 & 13.33 & {$<.001$} \\
LC & 8.81 & 8.81 & 1 & 386.08 & 8.61 & {$\phantom{<}.004$} \\
Section & 0.59 & 0.59 & 1 & 385.95 & 0.58 & {$\phantom{<}.448$} \\
Major & 2.17 & 2.17 & 1 & 30.79 & 2.12 & {$\phantom{<}.155$} \\
Length of stay in Japan     & 0.03 & 0.03 & 1 & 30.81 & 0.03 & {$\phantom{<}.859$} \\
Japanese self-efficacy & 0.89 & 0.89 & 1 & 31.17 & 0.87 & {$\phantom{<}.357$} \\
Gender & 12.49 & 12.49 & 1 & 30.85 & 12.20 & {$\phantom{<}.001$} \\
\addlinespace
% Interactions
Cognitive load:Domain & 5.03 & 5.03 & 1 & 397.76 & 4.91 & {$\phantom{<}.027$} \\
Cognitive load:Modality & 0.03 & 0.03 & 1 & 404.79 & 0.03 & {$\phantom{<}.861$} \\
Cognitive load:Linguistic complexity & 0.63 & 0.63 & 1 & 396.68 & 0.61 & {$\phantom{<}.435$} \\
Cognitive load:Section & 0.04 & 0.04 & 1 & 387.73 & 0.04 & {$\phantom{<}.843$} \\
Cognitive load:Major & 0.40 & 0.40 & 1 & 386.19 & 0.39 & {$\phantom{<}.534$} \\
\bottomrule
\end{tabular}
\end{threeparttable}
\end{table}


\begin{figure}[ht]
  \centering
  \includegraphics[width=0.8\linewidth]{Chapters/study3fig/s3f4.pdf}
  \caption{Predicted self-efficacy (EMMs ± 95\% CIs, shown as shaded bands) as a function of centered cognitive load, plotted separately for history and science materials. The steeper negative slope in science visualizes the significant cognitive load $\times$ domain interaction reported in the text.}
  \label{fig:rq3_slope}
\end{figure}

% revision---add , method comment 4


To provide converging behavioral evidence for the observed link between cognitive load and self-efficacy, we conducted an exploratory analysis of video control action logs (full descriptive statistics in Table~\ref{tab:log_data}). However, the distributional properties of these data (namely high zero-inflation and strong positive skew) precluded formal inferential testing. Therefore, our analysis focuses on descriptive patterns, and we make no claims about statistical differences between conditions. On average, rewinds were most frequent in \emph{complex + science} passages ($M = 10.55$), though this was driven by a few participants with extremely high counts. However, the median reveals a more typical pattern of behavior, with the highest frequency occurring in the \emph{complex + history} condition ($\mathrm{median}=4.00$). For all other actions, median usage was minimal ($\mathrm{median} \leq 1.00$). These observations suggest that while the most extreme compensatory behavior occurred in \emph{science + complex} passages, the most consistent use of rewinds by the typical participant occurred in \emph{history + complex} passages. This aligns with the finding that self-efficacy declined most steeply in response to cognitive load during science passages, as a few learners engaged in extensive compensatory action while others may have disengaged.

\begin{table}[ht]
  \centering
  \begin{threeparttable}
  \caption{Descriptive statistics of video control action frequencies by domain and linguistic complexity.}
  \label{tab:log_data}
  \setlength{\tabcolsep}{3pt}
  \footnotesize
  \sisetup{
    input-signs = {-},
    table-align-text-pre = false,
    table-align-text-post = false
  }
  \begin{tabular}{
    l                     % Domain
    l                     % Complexity
    l                     % Action type
    c                     % M(SD)
    c                     % Median
    c                     % n users
    c                     % Total n
  }
  \toprule
    \textsc{Domain} & \textsc{LC} & \textsc{Action} & {$M (SD)$} & {Median} & {$n$} & {Total $n$} \\
  \midrule
  \multirow{4}{*}{History} & \multirow{4}{*}{Simple} & Rewind & $6.89 (7.75)$ & $3.00$ & $8$ & \multirow{4}{*}{$16$} \\
   &  & Fast-forward & $1.00 (1.32)$ & $0.00$ & $4$ &  \\
   &  & Pause & $1.44 (2.07)$ & $0.00$ & $4$ &  \\
   &  & Play & $2.22 (3.27)$ & $1.00$ & $5$ &  \\
  \addlinespace
  \multirow{4}{*}{History} & \multirow{4}{*}{Complex} & Rewind & $6.67 (6.69)$ & $4.00$ & $8$ & \multirow{4}{*}{$20$} \\
   &  & Fast-forward & $0.22 (0.44)$ & $0.00$ & $2$ &  \\
   &  & Pause & $6.33 (13.94)$ & $1.00$ & $5$ &  \\
   &  & Play & $7.00 (13.90)$ & $1.00$ & $5$ &  \\
  \addlinespace
  \multirow{4}{*}{Science} & \multirow{4}{*}{Simple} & Rewind & $3.38 (3.11)$ & $2.50$ & $7$ & \multirow{4}{*}{$20$} \\
   &  & Fast-forward & $0.25 (0.71)$ & $0.00$ & $1$ &  \\
   &  & Pause & $1.62 (2.26)$ & $0.50$ & $4$ &  \\
   &  & Play & $1.62 (1.60)$ & $1.00$ & $6$ &  \\
  \addlinespace
  \multirow{4}{*}{Science} & \multirow{4}{*}{Complex} & Rewind & $10.55 (15.23)$ & $3.00$ & $11$ & \multirow{4}{*}{$16$} \\
   &  & Fast-forward & $0.18 (0.40)$ & $0.00$ & $2$ &  \\
   &  & Pause & $7.27 (16.73)$ & $0.00$ & $3$ &  \\
   &  & Play & $8.00 (18.23)$ & $0.00$ & $5$ &  \\
  \bottomrule
  \end{tabular}
  \begin{tablenotes}[flushleft]
  \footnotesize
    \item \textit{Note.} LC: Linguistic complexity; $M$: Mean frequency per participant; $SD$: Standard deviation; Median: Median frequency per participant; $n$: Number of participants who performed each action; Total $n$: Total participants in condition. Values include participants with zero actions.
  \end{tablenotes}
  \end{threeparttable}
\end{table}

%----




\subsection{RQ4: Predicting Accuracy from Design Factors, Cognitive Load, and Self-Efficacy}



A logistic mixed-effects model with a random intercept by participant (variance $= 0.224$, $SD = 0.474$) explained 11.8\% of the variance in trial-level performance accuracy through its fixed effects (\(R^2_{\mathrm{m}} = 0.118\)); including the random effect raised the explained variance to 17.4\% (\(R^2_{\mathrm{c}} = 0.174\)). 
As shown in Table~\ref{tab:full_results_rq4}, significant main effects emerged for question category and section. Post hoc comparisons revealed that recall items had higher accuracy than both reasoning ($p = .005$) and comprehension items ($p < .001$), while reasoning and comprehension did not differ significantly ($p = .054$). Both sections 1 and 2 outperformed section 3 (both $p < .001$), with no difference between sections 1 and 2 ($p = .992$). Self-efficacy showed a marginal positive association with accuracy ($p = .068$). No other main effects were significant.


Three two-way interactions reached significance. First, the cognitive load \(\times\) linguistic complexity interaction indicated that higher cognitive load reduced performance accuracy in complex passages but not in simple passages (simple versus complex: \(b = 0.212,\ z = 2.19,\ p = .029\)). Simple slope estimates confirmed this pattern: in complex passages, the slope was negative but fell short of conventional significance (\(\hat\beta = -0.134,\ p = .078\)), whereas it was near zero in simple passages (\(\hat\beta = 0.078,\ p = .286\)), as shown in Figure \ref{fig:rq4_fig}(a). Second, self-efficacy predicted accuracy more strongly for history passages than for science passages, as evidenced by the self-efficacy \(\times\) domain interaction (history versus science: \(b = 0.212,\ z = 2.31,\ p = .021\)). Simple slope analyses revealed a robust effect in history passages (\(\hat\beta = 0.286,\ p < .001\)) but not in science passages ($\hat\beta = 0.074,\ p = .310$), as shown in Figure \ref{fig:rq4_fig}(b). Third, the self-efficacy \(\times\) linguistic complexity interaction showed that the positive effect of self-efficacy was larger for simple passages (simple versus complex: \(b = 0.202,\ z = 2.21,\ p = .027\)); simple slope tests indicated a significant benefit in simple passages (\(\hat\beta = 0.281,\ p < .001\)) but not in complex passages ($\hat\beta = 0.079,\ p = .261$), as shown in Figure \ref{fig:rq4_fig}(c). The full simple slope analysis for each significant interaction is presented in Supplementary Table S18.
%revision--- comment 7
All other interactions were nonsignificant, including cognitive load $\times$ category; self-efficacy $\times$ category (both $p\geq.156$); cognitive load $\times$ self-efficacy; and all interactions involving major (all $p\geq.362$). This indicates that the effects of cognitive load and self-efficacy did not vary by question category, and that academic major did not moderate any effects.
%------

A Bayesian multilevel mediation analysis further investigated self-efficacy as a potential mediator of the effect of cognitive load on comprehension accuracy. (see Table~\ref{tab:mediation_rq4} for detailed posterior summaries). 
The overall indirect effect of cognitive load on performance via self-efficacy had a posterior median of $-0.09$ (95\% CrI = [$-0.20$, $0.01$]). Because the credible interval overlaps zero, the evidence remains suggestive rather than credible, even though the probability of direction is high (pd $= 96.4\%$). Throughout, we classify an effect as credible when the 95\% CrI excludes zero and as suggestive when the CrI includes zero but pd $\geq$ 95\%.

The direct effect of cognitive load, after controlling for self-efficacy, was uncertain; the total effect was also inconclusive. When linguistic complexity was considered as a moderator, the indirect path was credibly negative in simple passages but only suggestive in complex passages, because the credible interval for the complex condition included zero. The difference between these levels was uncertain, indicating no convincing moderation by linguistic complexity. A clearer distinction emerged for content domain: history passages showed a suggestive negative mediation, whereas science passages showed no effect. Critically, the contrast between domains was itself credible, suggesting that the ``cognitive load \(\rightarrow\) self-efficacy \(\rightarrow\) performance'' detour is likely stronger when learners engage with history content. Finally, regarding academic major, the indirect effect had an identical median for both arts/social sciences majors and science majors, though the credibility of the effect varied. It was credibly negative for the science majors but only suggestive for the arts/social sciences majors. Despite this slight variation, the formal difference between the majors centered precisely on zero, offering no evidence that disciplinary background moderates the mediation.


Overall, the data indicated a conditional route to performance. Cognitive load tended to impair accuracy only when syntactic complexity was high, while self-efficacy enhanced accuracy chiefly under lighter demands, specifically in the history domain and in passages with simple syntax. Mediation modeling provided tentative evidence that this pattern was partly motivational, suggesting that elevated cognitive load appeared to dampen self-efficacy, which in turn diminished performance as an indirect effect most apparent for history content.


\begin{table}[htb]
\centering
\begin{threeparttable}
\caption{Summary of the performance accuracy model.}
\label{tab:full_results_rq4}
\setlength{\tabcolsep}{4pt}
\footnotesize
\sisetup{
    input-signs = {-},
    table-align-text-pre = false,
    table-align-text-post = false
}
\begin{tabular}{
    l % Effects
    S[table-format=2.2]  % Chisq
    S[table-format=1.0]  % Df
    S[table-format=<1.3,table-comparator=true] % p-value
}
\toprule
& {\textit{$\chi^2$}} & {\textit{df}} & {\textit{p}} \\
\midrule
% Main effects
Cognitive load & 0.17 & 1 & {$\phantom{<}.680$} \\
Self-efficacy & 3.32 & 1 & {$\phantom{<}.068$} \\
Domain & 2.10 & 1 & {$\phantom{<}.147$} \\
Modality & 0.14 & 1 & {$\phantom{<}.713$} \\
Linguistic complexity & 0.39 & 1 & {$\phantom{<}.535$} \\
Section & 37.04 & 2 & {$<.001$} \\
Category & 28.48 & 2 & {$<.001$} \\
Major & 0.00 & 1 & {$\phantom{<}.955$} \\
\addlinespace
% Interactions
Cognitive load:Linguistic complexity & 4.78 & 1 & {$\phantom{<}.029$} \\
Domain:Self-efficacy  & 5.34 & 1 & {$\phantom{<}.021$} \\
Linguistic complexity:Self-efficacy  & 4.87 & 1 & {$\phantom{<}.027$} \\
\bottomrule
\end{tabular}

\end{threeparttable}
\end{table}


\begin{figure*}[ht]
  \centering
  \includegraphics[width=0.8\textwidth]{Chapters/study3fig/s3f5.pdf}
  \caption{
  Predicted probability of correct responses. (a) cognitive load $\times$ linguistic complexity interaction. (b) self-efficacy $\times$ domain interaction. (c) self-efficacy $\times$ linguistic complexity interaction.}
  \label{fig:rq4_fig}
\end{figure*}


\begin{table}[ht]
  \centering
  \begin{threeparttable}
    \caption{Mediation analysis: cognitive load $\rightarrow$ self-efficacy $\rightarrow$ performance accuracy.}
    \label{tab:mediation_rq4}
    \setlength{\tabcolsep}{4pt}
    \footnotesize
    \sisetup{
      input-signs = {-},
      table-align-text-pre = false,
      table-align-text-post = false
    }
    \begin{tabular}{
      l
      S[table-format=-1.2]
      c
      S[table-format=2.1, table-space-text-post = \%]
    }
    \toprule
      {Effect} & {Median} & {95\% CrI} & {pd} \\
    \midrule
        \multicolumn{4}{l}{\textit{Overall Mediation}}\\
    {Indirect Effect (CL $\rightarrow$ SE $\rightarrow$ Perf)} & -0.09 & {$[-0.20, 0.01]$} & 96.4\% \\
    {Direct Effect (CL $\rightarrow$ Perf)}    & -0.06 & {$[-0.34, 0.23]$} & 66.1\% \\
    {Total Effect}                     & -0.15 & {$[-0.42, 0.11]$} & 87.0\% \\\addlinespace[0.6ex]
      \multicolumn{4}{l}{\textit{Moderated Mediation by LC}} \\
      {Indirect Effect (Complex)}    & -0.09 & {$[-0.20, 0.01]$} & 96.0\% \\
      {Indirect Effect (Simple)}     & -0.06 & {$[-0.12, -0.01]$} & 98.0\% \\
      {Difference (Complex -- Simple)}         & -0.03 & {$[-0.16, 0.09]$} & 69.0\% \\
      \addlinespace[0.6ex]
      \multicolumn{4}{l}{\textit{Moderated Mediation by Domain}}\\
      {Indirect Effect (History)}             & -0.09 & {$[-0.20, 0.01]$} & 96.0\% \\
      {Indirect Effect (Science)}             & -0.01 & {$[-0.13, 0.10]$} & 61.0\% \\
      {Difference (Science -- History)}        & +0.08 & {$[0.01, 0.15]$}  & 98.0\% \\
     \addlinespace[0.6ex]
      \multicolumn{4}{l}{\textit{Moderated Mediation by Major}}\\
      {Indirect Effect (Arts/Soc.\ Sci.)}     & -0.09 & {$[-0.20, 0.01]$} & 96.0\% \\
      {Indirect Effect (Science)}             & -0.09 & {$[-0.19, -0.01]$} & 99.0\% \\
      {Difference (Science -- Arts/Soc.\ Sci.)} &  0.00 & {$[-0.09, 0.08]$} & 51.0\% \\
    \bottomrule
    \end{tabular}
    \begin{tablenotes}[flushleft]
      \footnotesize
      \item \textit{Note.} CL: Cognitive load; SE: Self-efficacy; Perf: Performance; Arts/Soc. Sci.: Arts/Social Sciences; CrI: Credible Interval; pd: probability of direction. Indirect effects represent the pathway CL $\rightarrow$ SE $\rightarrow$ Performance. All parameters converged (\(\hat{R} \le 1.01\); bulk and tail ESS \(\approx 4600\)).
    \end{tablenotes}
  \end{threeparttable}
\end{table}

\clearpage


\section{Discussion}




\subsection{RQ1: Cognitive Load---Effects of Domain, Modality, and Linguistic Complexity}



The findings regarding cognitive load reveal complex processing dynamics beyond simple additive effects. While the main effects confirmed that science passages, complex language, and video modality each increased cognitive load, the interactions demonstrate how these factors combine in theoretically important ways. The modality effect, in which video increased cognitive load more than text, diverged from multimedia learning predictions \parencite{mayer_multimedia_2002}. This likely stems from two advantages of text presentation. The first is more efficient metacognitive control, as learners can instantly reread unclear passages with minimal cognitive overhead. In contrast, rewinding a video requires multiple interface actions that impose extraneous cognitive load. Second, this text advantage likely reflects the orthographic overlap between Chinese and Japanese writing systems. Although written Japanese also employs the phonographic scripts \textit{hiragana} and \textit{katakana}, its \textit{kanji} characters are largely identical in form and meaning to Chinese \textit{hanzi}, and they carry most of the lexical content in a sentence. Therefore, Chinese readers can extract a substantial portion of a Japanese text's meaning at a glance, a cross-linguistic advantage unavailable in the spoken modality \parencite{machida_2001, matsumoto_2013}. This orthographic convergence helps explain why video imposed greater cognitive load on our learner population. 
%-revision---
However, for learners whose L1 shares no characters with Japanese, this advantage might not only disappear but also reverse. Eye-tracking evidence shows that when the native and target scripts are dissimilar, L2 learners rely far less on written text and more on audio and visual cues, because the written modality itself becomes cognitively taxing \parencite{winke2013}.
%----


The domain $\times$ linguistic complexity $\times$ modality interaction reveals that science and history passages involve fundamentally different processing demands. The post hoc comparisons confirmed a key crossover pattern where simplified science passages presented via video maintained processing demands exceeding those of complex history materials presented as text. This aligns with prior findings that scientific discourse contains an inherent conceptual density that linguistic simplification cannot fully mitigate \parencite{kintsch_text_1994, wolfe_learning_2007}. Notably, the cognitive load ratings produced by complex history passages presented as text (\emph{history + text + complex}) were consistently among the lowest, falling below those of numerous science and video conditions. This indicates that history passages benefit maximally from text presentation, with orthographic advantages offsetting even complex linguistic demands, whereas science passages maintain elevated cognitive load across presentation formats.



Supporting H1, the domain $\times$ linguistic complexity $\times$ major interaction confirmed that the experience of intrinsic load was moderated by expertise. Our post hoc tests revealed a classic expertise-reversal pattern in an L2 context \parencite{kalyuga_expertise_2003}. For arts/social sciences majors, cognitive load was uniformly high for all science content, regardless of whether the syntax was simple or complex. They were overwhelmed by the domain's high conceptual density. For science majors, however, the conceptual load of the science domain was manageable, and their cognitive resources were instead taxed by variations in linguistic complexity. This finding precisely illustrates how expertise shifts the primary source of cognitive load from the conceptual to linguistic processing.


In sum, our findings provide clear support for our hypotheses. Supporting H1, the results demonstrated that expertise plays a critical role in moderating intrinsic cognitive load, with arts/social sciences majors showing a generalized difficulty with the science domain, while science majors were more specifically sensitive to its in-domain linguistic complexity. Supporting H2, the results also confirmed that the effect of modality is conditional on the task's intrinsic load. The cognitive advantage of text was most pronounced for complex history passages but was nullified by the overwhelming conceptual density of science content, as evidenced by the significant crossover effect.



\subsection{RQ2: Self-Efficacy---Effects of Domain, Modality, and Linguistic Complexity}


The self-efficacy pattern largely tracked the cognitive load results, but it also revealed a distinct role for learner expertise in shaping motivational outcomes. First, the significant domain $\times$ modality $\times$ linguistic complexity interaction indicates that the impact of modality depends on intrinsic load, consistent with the conditionality anticipated in H2. In practice, conditions that felt harder (higher cognitive load) tended to yield lower self-efficacy, in line with social cognitive accounts that emphasize mastery experiences and perceived task difficulty as primary drivers of efficacy judgments \parencite{bandura_self-efficacy_1997}. Importantly, we interpret H2 at the interaction level: although the three-way interaction was reliable, some simple modality contrasts within high load cells were small or only trend-level after Tukey correction, so we refrain from strong claims about any single cell.


Second, the significant domain $\times$ linguistic complexity $\times$ major interaction provided clear support for H1 and revealed an asymmetrical pattern of expertise that was distinct from the cognitive load findings \parencite{han_trajectories_2021}. For science majors, the results mirrored the cognitive load pattern: their self-efficacy was highly sensitive to linguistic complexity within their own domain, showing a significant benefit for low linguistic complexity. For arts/social sciences majors, however, the pattern differed. Unlike with cognitive load, their self-efficacy was not sensitive to linguistic complexity, even within their own domain of history. This suggests that while all learners experience greater cognitive effort with complex in-domain material, only those with a high degree of domain-specific training (the science majors) show a corresponding calibration in their self-efficacy.



A plausible interpretation, consistent with attribution theory, is that the two groups attributed their struggle to different causes \parencite{weiner_development_2010}. Specifically, Arts/social sciences majors are likely to make an internal attribution, interpreting the high cognitive load as a sign of personal limitation (``I'm not a science person''), which causes their self-efficacy to decline \parencite{usher_sources_2009}. Science majors, in contrast, are more likely to make an external attribution, interpreting the same cognitive effort as a reflection of the material's inherent difficulty (``This is a complex scientific topic''). This external attribution protects their sense of competence, allowing them to remain confident despite the challenge.


Finally, the main effects for domain, modality, and linguistic complexity on self-efficacy mirrored those for cognitive load (history > science, text > video, simple > complex), underscoring that learners' self-efficacy is tightly coupled to moment-by-moment processing demands. Together, these results suggest that supporting self-efficacy in L2 learning requires aligning modality and linguistic complexity with learners' domain background and current intrinsic load, rather than applying one-size-fits-all multimedia prescriptions.


\subsection{RQ3: Coupling Between Cognitive Load and Self-Efficacy Across Sections}

Our within-participant analysis showed that momentary increases in cognitive load reliably coincided with immediate drops in self-efficacy: the participants reported lower self-efficacy for the next section whenever they experienced higher mental effort in the current section. This pattern aligns with ``dynamic'' models of self-efficacy, which posit that self-efficacy fluctuates in real time based on processing experiences instead of remaining a fixed trait  \parencite{yeo_revisiting_2013, sitzmann_meta-analytic_2013}.


Importantly, the cognitive load $\times$ domain interaction revealed that this coupling was more pronounced in science passages: self-efficacy declined more steeply when cognitive load spiked during science passages than during history passages. Although prior studies examining domain-specific self-efficacy trajectories over longer intervals \parencite{han_trajectories_2021} found that STEM self-efficacy tends to evolve more variably than humanities self-efficacy, suggesting that cognitively demanding domains induce greater volatility. Our section-level data extend this idea to within-task fluctuations, showing that effort surges in science reading yield larger immediate self-efficacy dips than equivalent effort in history reading.


One plausible explanation for the shallower self-efficacy drop observed in the selected history passages is their discourse structure. Both history texts unfold as chronological narratives, which remind readers of the global progression and supply ready-made links between successive propositions. Such narrative sequencing may reduce working memory demands by providing a coherent organizational structure for the reader \parencite{kintsch_text_1994}. By contrast, the selected science texts are organized as hierarchically layered expository prose replete with terminology, definitional clauses, and conditional relations. Readers must weld these discontinuous propositions into a coherent macrostructure, which increases cognitive load. Under attribution theory principles, this is likely to be interpreted as evidence of limited ability, thereby accelerating the drop in self-efficacy \parencite{wolfe_learning_2007}. Our design did not directly assess the participants' attributions. However, attribution theory can be applied to deduce that, when a science section with simpler syntax presented in video form (i.e., \emph{science + video + simple}) still felt taxing, our participants may have interpreted that effort as evidence of their low ability \parencite{weiner_development_2010}, whereas similar effort in a history section did not trigger as large a self-efficacy drop.


Pedagogically, the section-level coupling implies that cognitive- and motivational-support strategies should be integrated. Techniques known to reduce extraneous load, such as segmenting information, signaling key elements, and ensuring spatial contiguity, have been shown to prevent unnecessary working memory spikes \parencite{mayer2003} and may thereby avert the confidence dips that accompany transient overload. Conversely, interventions targeting self-efficacy alone may have short-lived effects if underlying load remains unmanaged; when the instructional design succeeds at controlling cognitive load, self-efficacy remains more stable, but when the design allows load to spike, self-efficacy erodes and downstream comprehension may suffer \parencite{yeo_revisiting_2013}.

%-revision--- results comment 2
The covariate pattern helps contextualize these dynamics. Across the two self-efficacy models, gender had a stable main effect on self-efficacy, with male participants reporting higher confidence than females; this is consistent with the modest gender gap observed in large-scale syntheses of academic self-efficacy \parencite{huang2013}. Length of stay in Japan and baseline Japanese self-efficacy, by contrast, were non-significant once task factors were entered. Although both variables predict broad L2 attainment \parencite{segal2024,mills2007}, the present section-level design suggests that moment-to-moment self-efficacy is driven more by immediate task demands than by these background characteristics.

%----


%-revision---method comment 4


Although exploratory and limited by zero-inflation, the analysis of video control actions provides converging behavioral evidence. Rewind events are widely viewed as a compensatory strategy that learners invoke when the initial pass overloads working memory, rather than as a mere marker of engagement \parencite{schwan2004, merkt2011}. Similarly, the use of pause actions can be interpreted as a strategy to manage transient information and prevent cognitive overload \parencite{schwan2004, merkt2011}. Consistent with these interpretations, learners in the \emph{science + complex} condition not only reported the highest cognitive load and the steepest declines in self-efficacy, but also produced the highest average number of rewinds, a pattern driven by a few participants with extremely high counts. The co-occurrence of high load and low self-efficacy with concentrated re-processing in a subset of learners (\emph{science + complex}) alongside higher typical median rewind use in \emph{history + complex} strengthens the interpretation that complex disciplinary material taxed working memory for some learners while others disengaged. In summary, these within-task dynamics reinforce H3 by illustrating a reactive pathway in which momentary overload immediately undermines efficacy and elicits compensatory action. These insights can inform design features aimed at interrupting vicious cycles of overload and disengagement in L2 multimedia learning.


%---


\subsection{RQ4: Cognitive Load, Self-Efficacy, and Comprehension Performance}

Cognitive load did not uniformly predict comprehension accuracy; rather, its association with accuracy was more negative when linguistic complexity was high, with a suggestive decrease in complex passages. In passages with simple syntax, variations in cognitive load were statistically unrelated to performance, indicating that the participants retained sufficient working memory resources to compensate. However, once syntax became complex, additional mental effort showed a trend toward depressing performance. This pattern aligns with CLT's working memory capacity limitations \parencite{sweller_cognitive_2011} and with existing evidence that syntactic integration costs rise nonlinearly with sentence complexity \parencite{Gibson1998}, an effect to which L2 learners are especially vulnerable \parencite{clahsen_how_2006}.

Self-efficacy displayed a similarly conditional profile. It improved accuracy in history passages and simple-syntax texts but not in science passages or syntactically complex texts. This aligns with prior research showing that science self-efficacy is often lower and more volatile than humanities self-efficacy \parencite{han_trajectories_2021}. Our findings suggest that the high intrinsic load inherent in scientific content \parencite{kintsch_text_1994, wolfe_learning_2007} creates a cognitive ceiling, leaving little room for motivational factors to boost performance. Similarly, when linguistic complexity becomes too demanding, the processing cost may override the benefits of self-efficacy; this finding complements existing evidence that difficulty with complex passages depresses self-efficacy \parencite{wu_self-efficacy_2012}.

%-revision--- 


Supporting integrative models of self-regulated learning \parencite{de_bruin_synthesizing_2020, wang_how_2023}, our Bayesian multilevel mediation provides suggestive evidence of an indirect path overall, with a credible effect only in simple passages. In complex passages, the evidence was suggestive, and the simple-complex contrast was uncertain. Importantly, the estimated magnitudes were similar across simple and complex passages, but only the simple condition was precise enough to be credible. This provides micro-level evidence for the “self-management bottleneck” \parencite{eitel_self-management_2020}, in which processing demands tax the self-efficacy needed to regulate learning. These conditional effects partially support H4: cognitive load impaired accuracy chiefly under high linguistic demands (complex syntax), whereas self-efficacy enhanced accuracy mainly under lighter demands (history, simple syntax).


%-----

A clearer distinction emerged for content domain. History passages showed a suggestive negative mediation, whereas science passages showed no evidence of an effect. Critically, the contrast between domains was itself credible, demonstrating that the ``cognitive load $\rightarrow$ self-efficacy $\rightarrow$ performance'' indirect pathway is stronger for history content. Hence, the higher intrinsic load of science texts \parencite{kintsch_text_1994} may be so demanding that it overwhelms the mediating potential of self-efficacy, leaving only a direct, unbuffered cognitive impact. Finally, we found no evidence that academic major moderated the mediation, indicating that the cognitive--motivational pathway operates similarly regardless of learners' disciplinary background.

The question category further shaped performance. Recall items were answered more accurately than both reasoning and comprehension items, whereas reasoning and comprehension did not differ reliably; this is consistent with existing evidence that different question types draw on partially distinct cognitive resources \parencite{kuechler_why_2010}. This pattern likely reflects the differential cognitive load imposed by each question type under working memory constraints. Furthermore, accuracy declined by the third section of each task, suggesting cumulative cognitive fatigue in an L2 context, where the sustained processing and inhibition of cross-linguistic interference tax attention over time \parencite{van_der_linden_mental_2003}.

%revision---add comment 7
Importantly, neither cognitive load nor self-efficacy moderated category effects; the corresponding interaction terms were non-significant, indicating that the recall advantage holds regardless of a learner's perceived load or self-efficacy. Large-scale assessments have shown that literal-recall questions are answered most accurately, whereas inferential and critical-analysis items (conceptually parallel to our reasoning and comprehension categories) are markedly more difficult for adolescents \parencite{spencer2019}. Discourse-processing theories likewise argue that questions demanding bridging or explanatory inferences impose heavier cognitive-resource requirements than prompts that tap surface retrieval, thereby depressing accuracy on comprehension items when working-memory resources are limited \parencite{graesser1994}. Thus, the present pattern reinforces the broader claim that performance deteriorates as the inferential burden of the task increases.
%---

%-revision----
The logistic mixed-effects model for performance accuracy yielded $R^{2}_{\mathrm{m}} = 0.118$ and $R^{2}_{\mathrm{c}} = 0.174$. Although modest, such values are common for binary item-level reading comprehension, where substantial variance remains at the word, reader, and passage levels. A comparable logistic mixed-effects study of reading comprehension reported higher values ($R^{2}_{\mathrm{m}} = 0.239$, $R^{2}_{\mathrm{c}} = 0.370$) \parencite{pickren2022}. These higher values likely reflect differences in task granularity and measurement (e.g., fewer trial-level stochastic influences), greater stimulus homogeneity, and richer fixed effect specifications that capture more systematic variance. By contrast, studies using continuous outcomes often report much larger $R^{2}$ because continuous scores smooth binomial noise and aggregate over many items, thereby raising the observed $R^{2}$ values \parencite{ortiz2025}. Accordingly, our lower values reflect a fine-grained, binary, item-level design in an L2 context, with unmeasured attributes (e.g., vocabulary knowledge and working-memory capacity) leaving considerable residual variance.


%----

Collectively, these findings reveal an asymmetry in L2 comprehension: Cognitive load threatens performance primarily when linguistic demands exceed capacity, whereas self-efficacy enhances performance mainly in contexts of moderate demand. Therefore, effective instruction should pair load-management techniques with confidence-supporting interventions, tailoring both to the domain and linguistic complexity of the material. The mediation and interaction patterns further support H4 by highlighting how task characteristics moderate the joint influence of cognitive load and self-efficacy on performance accuracy.


\subsection{Synthesis of Findings and Theoretical Implications}
Across RQ1--RQ4, two cross-cutting themes emerged. First, expertise shifted the primary source of difficulty from conceptual density to linguistic processing: science majors were sensitive to in-domain syntactic complexity, whereas arts/social sciences majors experienced uniformly high load in science. Second, modality effects were conditional on intrinsic load rather than uniformly favoring text or video. At the within-person level, higher section-level load predicted lower subsequent self-efficacy, with a steeper coupling in science than in history. Performance costs appeared chiefly under complex syntax, whereas self-efficacy most reliably supported accuracy under lighter demands. These patterns situate our results within cognitive load and social cognitive frameworks, emphasizing that effective design must account for domain, modality, linguistic complexity, and learner expertise. Together, these findings extend CLT and social cognitive models into L2 multimedia contexts by showing that domain expertise shifts cognitive challenge from conceptual to linguistic processing; that cognitive load and self-efficacy form a tightly coupled feedback loop within tasks; and that their interplay exerts both direct and mediated influences on comprehension.


 
\subsection{Practical Significance}
Practically, instructional design for L2 learners should move beyond universal multimedia guidelines. Adaptive platforms could prioritize text for narrative history materials. If a history module is delivered via video and behavioral indicators such as frequent rewinds or prolonged pauses suggest overload (as observed in our logs), the system should offer a simplified text summary of the same content. For science content, segmented videos with on-screen cues could reduce processing demands; if self-efficacy drops noticeably across segments, potentially signaling a self-management bottleneck \parencite{eitel_self-management_2020}, systems might insert reflective prompts or summarizers to scaffold self-efficacy before progressing. When creating course content, particular care should be taken with linguistic complexity, as our findings suggest elevated syntactic demands can exacerbate cognitive load and erode self-efficacy, particularly in science domains or video modalities. Tailoring modality and linguistic complexity to learners' disciplinary background and momentary cognitive-motivational states is essential for aligning self-efficacy with actual task demands.

\subsection{Limitations and future directions}

Several limitations of this research warrant mention. First, our sample comprised only Chinese-speaking learners of Japanese, which may limit generalizability. Thus, replication with other L1--L2 pairings is necessary. Second, the data capture a single 90-min session, providing a cross-sectional snapshot rather than a developmental trajectory. Micro-longitudinal or experience-sampling protocols that track learners across multiple lessons would allow growth-curve or cross-lagged analyses to determine whether early confidence dips, modality adaptation, and load-management interventions have enduring effects. Third, our assessment of L2 competence relied on self-reported background and JLPT levels rather than more direct measures. We did not assess prior formal training in academic Japanese or include scores from standardized listening and reading proficiency tests as covariates, both of which could have influenced comprehension strategies independently of disciplinary major. Finally, experimental testing of targeted interventions, such as adaptive signaling (e.g., dynamic on-screen cues that highlight key terms in response to rewind patterns \parencite{beege2021}) or scaffolded textual overlays (e.g., pop-up summaries triggered by self-efficacy drops below baseline \parencite{haagsman2020}), could be conducted via randomized controlled trials with real-time indicators like pupil dilation or behavioral logs. This would identify methods to disrupt maladaptive load-efficacy cycles in diverse L2 settings. Addressing these gaps will inform the development of adaptive multimedia systems that dynamically monitor learners' fluctuating cognitive and motivational states and optimize support in real time.

\section{Conclusion}


Results demonstrate that domain, modality, and linguistic complexity jointly shape cognitive load, self-efficacy, and comprehension in L2 academic learning, and that disciplinary expertise modulates these effects. Across conditions, science passages, video presentation, and complex syntax increased cognitive load, and these same features tended to depress self-efficacy. Expertise shifted where difficulty was felt: within-domain learners were sensitive to linguistic complexity in their own field, while out-of-domain science content imposed generally high load regardless of syntax. Modality effects were conditional rather than uniform; advantages of text appeared when intrinsic demands were moderate. Within participants, higher load in one section predicted lower self-efficacy in the next section, with a steeper coupling in science than in history. Performance costs emerged chiefly under complex syntax, and self-efficacy related positively to accuracy in lower-demand contexts. In sum, these findings show that effective L2 instruction requires a calibrated approach that manages cognitive load while supporting self-efficacy. This work provides an evidence-based framework for moving beyond one-size-fits-all multimedia principles to design instruction that is dynamically responsive to the specific domain, modality, linguistic complexity, and, crucially, the expertise of the learner.


