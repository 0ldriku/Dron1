%************************************************
\chapter{Integration and Conclusions}\label{ch:integration}
%************************************************

\section{Overview: From Observable Breakdowns to Principled Reallocations}

Across five studies spanning two experiments, this thesis traced how cognitive load shapes L2 production and comprehension. The unifying finding is that L2 performance under load reveals principled resource allocation patterns, not deficits. When demands rise, speakers do not collapse uniformly; they sacrifice phrasal fluency to protect lexical precision and monitoring (Study 1), and listeners adjust which cues they weight when evaluating this speech (Study 2). When readers encounter linguistically complex academic materials, they do not simply ``fail''; they reallocate attention from global, expert strategies to local, survival-level parsing (Study 4), with physiological engagement patterns that can dissociate from subjectively reported difficulty (Study 5). These patterns are adaptive responses to capacity limits, and they follow predictable principles across modalities.

The program comprised two experiments with complementary designs. Experiment 1 manipulated element interactivity in L2 Japanese speaking tasks while holding content constant, then analyzed how speakers redistributed complexity, accuracy, and fluency (Study 1) and how native Japanese listeners evaluated the resulting speech (Study 2). Experiment 2 employed a factorial design crossing linguistic complexity (simple versus complex syntax), content domain (history versus science), and presentation modality (text versus video) to examine how these factors jointly shaped subjective cognitive load, self-efficacy, and comprehension accuracy (Study 3), eye-movement patterns during text reading (Study 4), and continuous pupil dilation trajectories across both modalities (Study 5).

By treating production and comprehension as two views of the same capacity-limited system rather than as separate research domains, this work clarifies when and how processing constraints force trade-offs, what learners prioritize when resources are scarce, and which design features genuinely ease cognitive burden versus merely shifting where that burden is felt. The remainder of this chapter synthesizes theoretical contributions (\autoref{sec:theory-contrib}), methodological advances (\autoref{sec:method-contrib}), practical implications (\autoref{sec:practical}), limitations (\autoref{sec:limitations}), future directions (\autoref{sec:future}), and concluding reflections (\autoref{sec:conclusion}).

\section{Theoretical Contributions}\label{sec:theory-contrib}

This thesis advances understanding in three theoretical domains: Cognitive Load Theory, limited capacity models of language processing, and second language acquisition research. Each contribution builds on the parallel examination of production and comprehension within a unified framework.

\subsection{Extensions to Cognitive Load Theory}

\subsubsection{Intrinsic Load: The Primary Bottleneck}

Across both experiments, intrinsic load emerged as the primary constraint. In production, element interactivity forced specific CAF trade-offs (\autoref{ch:prod-synthesis}). In comprehension, linguistic complexity dominated processing costs (\autoref{ch:comp-synthesis}). This consistency validates CLT's architectural claims for real-time L2 performance \parencite{sweller2010,sweller2019}.

\subsubsection{Expertise as Modulator}
Expertise effects (\autoref{subsec:expertise-modulator}) revealed that disciplinary knowledge modulates rather than eliminates load. It changes what is difficult (domain vs. language) and how attention is allocated. Notably, STEM and humanities training appear to shape linguistic sensitivity differently, with implications for differentiated scaffolding.

\subsubsection{Modality as Alignment}
Study 5's pupillometry revealed a critical dissociation: subjective difficulty and physiological engagement can move in opposite directions. Science majors sustained engagement with video despite rating it harder, while showing gradual disengagement with text. This reframes modality as learner-contingent alignment rather than universal design prescription, extending CLT beyond split-attention principles to schema-format matching. For STEM learners accustomed to dynamic, visual, causal representations, video may sustain cognitive investment even when consciously perceived as harder. For narrative-oriented learners, text's self-paced, low-transience affordances may be superior. 

\subsection{Contributions to Limited Capacity Models}

\subsubsection{CAF Dimensions Form a Joint Allocation Profile}
Study 1's multivariate approach revealed that complexity, accuracy, and fluency do not vary independently under load. Instead, they form a joint allocation profile that shifts systematically: speakers protected lexical precision (lexical density increased) while accepting costs in temporal speed (speech rate decreased) and phrasal elaboration (content words per AS-unit decreased). Accuracy was not sacrificed. This challenges common practices in task-based language teaching research that analyze CAF with separate univariate tests, which can obscure trade-offs and inflate Type I error. The finding that these dimensions covary supports limited capacity accounts that treat production as a resource allocation problem where dimensions compete for attention\parencite{Skehan2009,Robinson2005}. Critically, the covariance pattern itself shifts with task demands.

\subsubsection{Listener Judgments Track Speaker Reallocations}
Study 2 extended this by showing that listeners do not evaluate speech in a vacuum; their judgments reflect the production reallocations documented in Study 1. When task complexity increased, the linguistic correlates of comprehensibility and perceived fluency shifted: features that were diagnostic under low complexity (e.g., final-clause pause ratio, error rate) became joined by additional dimensions under high complexity (e.g., clauses per AS-unit, auxiliary verbs per AS-unit). This demonstrates a perceptual corollary to production trade-offs: when speakers are under heavier load, listeners draw on a broader set of linguistic dimensions to reconstruct the intended message. This bidirectional coupling provides a process-level account of communicative success under load. Production constraints shape what is available for perception, and perceptual weighting adapts to production conditions.

\subsubsection{Eye-Movement Patterns Reveal Adaptive Processing Strategies}
Study 4's eye-tracking analysis distinguished temporal signatures (how long readers look, when they go back) from spatial signatures (how orderly the path is). Linguistic complexity increased fixation counts, reduced forward saccade length, and broadened spatial dispersion (radius of gyration, convex hull area), indicating that readers adopted a more localized, cautious, and spatially dispersed strategy under high load. Critically, domain expertise modulated this pattern selectively: in simple passages, science majors showed more fixations on science texts (not fewer), suggesting schema-driven engagement rather than efficiency gains. This reframing challenges simplistic ``expert = faster'' accounts by showing that expertise manifests as deeper sampling when capacity permits, not as global speed advantages. This finding aligns with disciplinary literacy research showing that expert readers engage more thoroughly with domain-relevant structure\parencite{wineburg1991reading}.

\subsubsection{Physiological Engagement Can Dissociate from Subjective Experience}
Study 5's continuous pupillometry revealed that what learners report as difficult may not correspond to what sustains their physiological investment. Video was rated as subjectively harder overall (Study 3), yet pupil data showed selective effort reductions and fundamentally different engagement trajectories depending on learner background. This dissociation implies that subjective difficulty and sustained cognitive engagement are separable constructs, each with distinct design implications. Instruction that feels ``easy'' may fail to sustain deep processing, while instruction that feels ``hard'' may nonetheless maintain physiological investment when format aligns with disciplinary training.

\subsection{Contributions to L2 Processing Research}

\subsubsection{Japanese-specific Processing Constraints}
The thesis identified language-specific patterns that extend beyond general CLT principles. In production (Study 2), final-clause pausing was the primary temporal predictor of comprehensibility and perceived fluency, not mid-clause pausing as is typical in English. This reflects Japanese's SOV structure, where key morphological and semantic information (main verb, tense/aspect markers, sentence-final particles, conjunctive particles) is reserved for clause ends. Pauses at this juncture force listeners to hold incomplete information in working memory, delaying message resolution. In comprehension, the kanji/hanzi orthographic overlap between Chinese and Japanese provided text advantages for these L1 Chinese learners (Study 3), an advantage unavailable in spoken modality. These findings demonstrate that CLT principles must be adapted to language-specific affordances and constraints.

\subsubsection{Process Measures reveal When and How Load Constrains Processing}
By integrating eye-tracking (Study 4) and pupillometry (Study 5) with subjective ratings (Study 3), the thesis provided a multi-level account of comprehension under load. Subjective ratings captured conscious appraisal of difficulty. Eye movements revealed the micro-temporal dynamics of word recognition and syntactic integration, showing precisely when and where processing became effortful. Pupil trajectories revealed moment-to-moment physiological effort and engagement patterns that aggregate measures would miss. This multi-method convergence addresses a persistent limitation in L2 research: the over-reliance on post-hoc outcome measures that cannot distinguish whether poor performance reflects encoding failure, integration failure, or retrieval failure.

\subsubsection{Bidirectional Load-Efficacy Coupling at Section-Level Granularity}
Study 3’s section level measurement of cognitive load and self-efficacy showed a clear within task coupling in this sample, momentary increases in cognitive load were followed by drops in self-efficacy for the next section, with a steeper coupling in science than in history. This pattern is consistent with motivational research that describes a self management bottleneck, where high load consumes resources needed for both processing and self regulation, and it suggests that such a bottleneck can operate at section level timescales, not only across courses or semesters\parencite{Feldon2019,de2020synthesizing}. The finding that this coupling varied by domain (steeper in science) suggests that certain content types are particularly vulnerable to vicious cycles where load spikes $\to$ efficacy drops $\to$ disengagement $\to$ further load increases.

\section{Methodological Contributions}\label{sec:method-contrib}

The thesis advances L2 research methodology in four ways: multivariate analysis of production, convergent validation of load manipulations, spatial eye-movement indices, and trajectory-based physiological analysis.

\subsection{Multivariate Analysis Preserves CAF Covariance}

Study 1 demonstrated that analyzing CAF in a single multivariate framework reveals trade-off patterns that separate univariate tests obscure. The approach showed that the five representative CAF indicators covaried systematically, with task complexity explaining 4\% of multivariate variance (small but reliable effect). Discriminant analysis further clarified the allocation pattern: higher task complexity was associated with a shift toward lexical precision (positive loading on lexical density) and away from phrasal elaboration and temporal speed (negative loadings on content words per AS-unit and speech rate). This integrated profile would be missed by separate ANOVAs for each measure, which risks Type I inflation, obscures dependencies, and can lead to contradictory interpretations when some indices rise and others fall.


\subsection{Convergent Validation of Cognitive Load Manipulations}

Both experiments implemented multimethod verification before interpreting outcomes, addressing a persistent weakness in L2 task complexity research: the assumption that designed differences necessarily translate to experienced differences. In Experiment 1 (production), participants reported higher effort and showed slower, less accurate secondary-task performance in the high-complexity condition, confirming that the element interactivity manipulation imposed measurably higher cognitive load. In Experiment 2 (comprehension), participants reported higher effort for complex passages, and this was corroborated by pupillometry indices.

This convergent approach follows recent methodological recommendations and repeats the verification rule stated in Chapter 2 and applied in the empirical chapters, namely that outcomes are interpreted only when at least one subjective index and one behavioral or physiological index move in the predicted direction (see \autoref{sec:verification}). The principle is conservative, analyses proceed only when multiple indices align, and interpretations remain proportional to effect sizes when streams diverge. This guards against two risks. First, it prevents attributing performance differences to ``cognitive load'' when no load difference was actually experienced. Second, it prevents over-interpreting effects that appear in one measure (e.g., subjective ratings) but not in behavioral or physiological indices, which can indicate that learners adjusted their subjective standards or disengaged rather than experiencing genuine load relief.

\subsection{Spatial Eye-Movement Indices Complement Temporal Measures}

Study 4 demonstrated that spatial characteristics of eye movements (radius of gyration and convex hull area) provide unique information beyond traditional temporal measures like mean fixation duration. Linguistic complexity and domain both shaped spatial dispersion in ways that were partially independent of timing effects. For instance, history passages produced larger spatial footprints than science passages even when timing differences were modest, suggesting that disciplinary conventions (e.g., sourcing and contextualization in historical reading) influence how attention is deployed across the page.

This spatial perspective addresses a limitation in much eye-tracking research: the focus on ``how long'' (fixation durations) and ``when'' (regression counts) without considering ``where'' and ``how orderly'' (spatial organization of the scanpath). When sentences are harder, gaze paths become less orderly. This loss of orderliness can be quantified without committing to a single optimal sequence, offering a complementary window on how readers manage difficulty. Future L2 reading research should routinely include spatial dispersion indices alongside temporal measures to capture the full dimensionality of attentional reallocation.

\subsection{Trajectory-Based Pupillometry Reveals Time-Localized Effects}

Study 5 modeled complete pupil diameter trajectories using GAMMs, moving beyond the single-peak or area-under-curve summaries common in earlier pupillometry work. This approach revealed effects that aggregate measures would miss: video advantages in specific conditions (simple history, complex science) emerged as early-onset, sustained separations rather than brief spikes; the modality $\times$ major interaction on engagement trend (science majors rising in video, falling in text) was fundamentally about temporal direction rather than about average level.

This trajectory-based logic is increasingly recognized in psychophysiology, where effects can be transient, late-emerging, or modality-specific in their temporal profile. The present application demonstrates its value for L2 research: if demand fluctuates within a section, a continuously sampled physiological signal registers those fluctuations even when summary scores look similar. Future work should adopt this time-resolved approach whenever the theoretical question concerns when and how long an effect persists, not merely whether it exists.

\section{Practical Implications for Instruction and Design}\label{sec:practical}

The findings translate into a clear, evidence-based hierarchy of instructional priorities for L2 learning contexts where both speaking and academic reading must improve together.

\subsection{Hierarchy of Intervention}

\subsubsection{Manage Linguistic Complexity}
Both production and comprehension results converge on this priority. In production, higher element interactivity forced specific trade-offs regardless of proficiency (Study 1). In comprehension, linguistic complexity was the primary driver of processing cost across nearly all measures (Studies 3--5). This is the non-negotiable constraint: high linguistic complexity will override expertise benefits (Study 4) and can trigger the self-management bottleneck where load consumes resources needed for both processing and self-regulation (Study 3).

Actionable guidance: Before implementing any other intervention, instructional materials must control syntactic load. Use shorter sentences, canonical word order, and transparent morphological marking during initial phases. For L2 Japanese specifically, prioritize clause final clarity (Study 2), and avoid long dependency distances in line with the comprehension findings (Study 4). Provide clear morphological cues at clause boundaries, since pauses or ambiguity at these junctures are especially costly for comprehension and for perceived fluency.

\subsubsection{Match Scaffolding to Learner Expertise Profile}
The expertise findings (Studies 3 and 4) from this experiment set provide clear guidance about what to scaffold. For out of domain learners, for example Arts Majors studying the science texts used here, the primary barrier was conceptual density, not linguistic complexity. These learners need domain scaffolding first, explicit definitions, worked examples, advance organizers, and conceptual diagrams. Linguistic simplification helps but is not sufficient. For in domain learners, for example Science Majors reading science texts from this set, the primary barrier shifted to linguistic realization. These learners need language scaffolding, syntactic simplification, function word highlighting, and morphological clarity.

Translating this into practice requires diagnosing learner-content alignment before choosing scaffolds. Mixed-major classrooms require differentiated support: offer conceptual glossaries and schema-building activities for out-of-domain students, and offer linguistic glossaries and sentence-unpacking activities for in-domain students. Recognize that STEM-trained learners may be particularly sensitive to linguistic complexity even within their domain (Study 3 self-efficacy asymmetry).

\subsubsection{Align Modality With Learner Background and Content Type}
The comprehension studies (3, 5) revealed that modality effects are neither simple nor universal. Video was subjectively rated as harder overall (Study 3), yet pupillometry showed selective advantages in specific conditions and fundamentally different engagement trajectories for different learners (Study 5). The critical finding is that Science Majors showed sustained physiological engagement with video (rising pupil trends) but gradual disengagement with text (falling trends), while Arts and Social Sciences Majors showed flat-to-falling trends in both modalities.

Actionable guidance: Do not default to video because it ``seems more engaging.'' Instead:
\begin{itemize}
    \item For STEM learners studying any content (even history), prefer well-designed video that coordinates narration with visual change, segments content into meaningful units, and avoids redundancy. These learners' trained schemas for dynamic, visual, causal representations appear to sustain engagement even when the format feels subjectively harder.
    \item For Arts and Social Sciences learners, especially with narrative content, prioritize text's self-paced, low-transience affordances. These learners may not sustain physiological investment in video format.
    \item For mixed audiences, consider providing both formats and allowing learners to self-select after an initial diagnostic, or sequence modalities (e.g., video for initial exposure to leverage visual scaffolding, then text for consolidation to leverage self-pacing).
    \item Critically: do not use subjective difficulty ratings alone to judge format effectiveness. Learners' conscious appraisals and physiological engagement can dissociate (Studies 3 versus 5).
\end{itemize}

\subsection{Task Design for Production}

Study 1's findings inform speaking task design in several ways. First, increasing element interactivity can be used intentionally to probe lexical precision while anticipating costs in phrasal density and temporal speed. This suggests a pedagogical sequence: use lower-interactivity tasks to develop fluency and phrasal routines, then introduce higher-interactivity tasks to push lexical precision and monitoring, with explicit scaffolding for phrasal packaging so learners do not regress in that dimension.

Second, evaluate speaking with composite CAF profiles, not isolated indices. A speaker who slows down but becomes more lexically precise and maintains accuracy is making a strategic, adaptive choice, not ``getting worse.'' Assessment rubrics should reflect this: reward appropriate trade-offs (e.g., slowing to maintain accuracy under high demands) rather than penalizing any deviation from native-like speed or native-like complexity in isolation.

Third, modality-specific signatures matter for Japanese: Study 2 showed that final-clause pausing and function word accuracy (case particles, auxiliary verbs) are the primary cues Japanese listeners use to judge comprehensibility and fluency, especially under high task complexity. Instruction should therefore target clause-final packaging and morphological precision, not just global fluency.

\subsection{Materials Design for Comprehension}

The comprehension studies (3--5) provide several concrete design levers, the most critical of which is the non-negotiable management of linguistic complexity. Before worrying about multimedia effects or engagement strategies, ensure that syntax is manageable. For L2 Japanese texts, this means: controlling dependency distances, providing clear case-marking, avoiding multiple levels of embedding, and ensuring that clause-final positions are unambiguous. Beyond syntactic simplification, discourse structure should be made visible through predictable information placement, explicit signaling of relations, and consistent formatting. This allows readers to apply domain knowledge without exceeding cognitive capacity. For instance, in CLIL contexts where students face dual challenges of mastering content in L2 \parencite{coyle2010}, structuring texts with clear topic sentences, explicit connectives, and consistent heading hierarchies reduces the load required to build situation models.

For adaptive learning platforms, the dissociation between subjective difficulty and physiological engagement (Study 5) suggests moving beyond universal multimedia guidelines. Systems could prioritize text for narrative history materials while offering simplified text summaries when behavioral indicators suggest video overload. For science content, segmented videos with on-screen cues could reduce processing demands. When self-efficacy drops noticeably across segments, potentially signaling a self-management bottleneck \parencite{eitel_self-management_2020}, systems might insert reflective prompts or brief consolidation activities before progression. Tailoring modality and linguistic complexity to learners' disciplinary background and momentary cognitive-motivational states is essential for aligning self-efficacy with actual task demands.


For video materials, Study 5's temporal trajectory findings suggest that well-coordinated video (where narration and visual change are co-timed) can reduce early-section effort by segmenting and signaling explicitly. This aligns with multimedia segmentation and signaling principles. When using video, break content into meaningful chunks (e.g., 1--2 minute segments), provide clear signals for transitions, and allow learner control (pause/replay). Avoid long, uninterrupted streams that tax sustained attention.

Finally, the results imply a need to monitor engagement, not just difficulty. Study 5 showed that physiological engagement trends (rising vs. falling pupil trajectories) can dissociate from subjective difficulty. In adaptive instructional systems, this implies monitoring both: if a learner reports low difficulty but shows falling engagement (e.g., via behavioral proxies like time-on-task, replay patterns, or gaze dispersion), the material may be failing to sustain deep processing. Conversely, if a learner reports high difficulty but shows sustained engagement, the material may be appropriately challenging.

\section{Limitations}\label{sec:limitations}

Several limitations constrain the generalizability and interpretation of findings, each pointing toward refinements in future work.

\subsection{Sample and Population}

All studies involved Chinese L1 learners of L2 Japanese, with participants in Experiment 2 (comprehension) being international students in Japan. This population is not representative of L2 learners globally. The kanji/hanzi orthographic overlap provided a specific text advantage (Study 3) that would not generalize to learners whose L1 does not share characters with Japanese. Eye-tracking evidence from other populations shows that when L1 and L2 scripts are dissimilar, learners rely less on written text and more on audio/visual cues because the written modality itself becomes taxing. The present text advantages might therefore reverse for learners from non-character-based L1s.

Similarly, all Experiment 2 participants were university students, many at graduate level, with advanced Japanese proficiency (N1--N2). Findings about expertise effects and self-management bottlenecks may not transfer to lower-proficiency learners for whom even ``simple'' syntax imposes high load, or to learners without advanced academic training in any discipline.

\subsection{Task Range and Materials}

The production findings from Studies 1 and 2 are limited by the use of a single argumentative matching task. The specific trade-off pattern (lexical precision prioritized over phrasal fluency) may be task-specific. Other genres (e.g., narrative, procedural) or modalities (writing versus speaking) could yield different reallocations. Study 1's speech samples were also trimmed to the initial 60 seconds, which may not capture performance stabilization patterns that emerge later in extended discourse. Additionally, Study 1's secondary task, while serving to validate the cognitive load manipulation, may have diverted attentional resources and potentially affected speech production itself \parencite{fukuta2015}. Future production studies should explore validation methods that do not interfere with the primary task. 

Study 2's listener judgments were collected from eight highly experienced university-level Japanese instructors. While this likely increased rating consistency, it limits generalizability to less experienced or mixed-L1 listeners. Moreover, neither production study incorporated segmental or prosodic phonetic measures, preventing determination of whether listeners relied on additional acoustic cues beyond the analyzed linguistic features. Future work should diversify rater pools and expand the analytical framework to include phonetic predictors.



Similarly, the comprehension results from Studies 3 through 5 are based on a specific set of expository academic passages. The observed domain $\times$ linguistic complexity $\times$ major interactions observed may not generalize to other content types (e.g., literary texts, procedural instructions). The passages were also relatively short (3 sections per topic, each under 3 minutes for video or self-paced for text), and cumulative fatigue effects across longer study sessions were not examined.

\subsection{Expertise Operationalization}

Academic major was used as a proxy for domain expertise, a quasi-experimental variable that does not permit causal claims. This grouping glosses over considerable heterogeneity: the ``science'' major category included students from chemistry, electrical engineering, and other STEM fields with potentially different reading strategies and schema types. A finer-grained operationalization, such as direct knowledge probes, domain-specific reading strategy assessments, or think-aloud protocols, would provide stronger tests of expertise effects. At the lexical timescale, future tests should align items with individual knowledge (e.g., term-familiarity norms, cloze predictability) to isolate expertise-linked first-pass benefits.

Moreover, the asymmetry in self-efficacy findings (Study 3: science majors showed linguistic complexity sensitivity within science, but arts majors did not show this in history) suggests that the ``expertise'' construct may be qualitatively different across disciplines. STEM training may cultivate attention to linguistic precision and formal structure, while humanities training may prioritize narrative coherence and interpretive flexibility. These disciplinary differences were not directly measured and remain speculative. Moreover, Study 3's assessment of L2 competence relied on self-reported background and JLPT levels rather than direct measures of academic Japanese proficiency, which could have influenced comprehension strategies independently of disciplinary major.


\subsection{Effect Sizes and Precision}

Several effects, while statistically reliable, were modest in magnitude. Study 1's PERMANOVA showed that task complexity explained 4\% of multivariate variance in CAF, a small effect. Study 4's word-level processing findings (first-pass duration, skipping probability) showed domain effects independent of expertise, but effect sizes were small and some interactions were trend-level. Study 5's modality advantages in pupillometry appeared in only 2 of 4 conditions, and the time-integrated contrasts, while significant, reflected moderate standardized differences.

These modest effects are interpretable in two ways. First, they may reflect genuine small-to-moderate impacts of the manipulations, suggesting that other unmeasured factors (e.g., individual working memory capacity, vocabulary knowledge, L2 exposure history) account for more variance. Second, they may reflect conservative, multimethod verification: by requiring convergence across subjective, behavioral, and physiological indices, the design may have filtered out noisier effects that would appear larger with less stringent criteria.

\section{Future Directions}\label{sec:future}

The findings and limitations point toward several productive research directions.

\subsection{Cross-Linguistic and Cross-Cultural Replication}

Do the core patterns of linguistic complexity as bottleneck, expertise as modulator, and modality as alignment generalize across L1--L2 pairings? Replication with learners from non-character-based L1s (e.g., alphabetic European languages) learning Japanese would test whether text advantages reverse when orthographic overlap is absent. Similarly, testing Chinese learners of other L2s (e.g., English, Korean) would clarify whether the linguistic complexity bottleneck is language-universal or whether head-final, case-marked languages like Japanese have unique vulnerabilities (e.g., clause-final ambiguity costs).

Cross-cultural replication is also warranted. The disciplinary literacy findings (science majors showing linguistic complexity sensitivity, arts majors not) may reflect educational systems that emphasize formal analysis in STEM but interpretive flexibility in humanities. Testing learners from educational contexts with different disciplinary norms would clarify whether these patterns reflect universal cognitive constraints or culturally-specific pedagogical practices.

\subsection{Developmental Trajectories}

All studies were cross-sectional snapshots, limiting the findings to a single point in time. A crucial future direction is to adopt longitudinal designs to trace these processes developmentally. Such work could determine whether the CAF trade-off patterns identified in Study 1 stabilize, shift, or become more flexible as learners gain proficiency, and clarify how prioritization patterns differ between novice and advanced learners. A developmental approach would also illuminate the load-self-efficacy coupling from Study 3; it is plausible that this tight coupling weakens as learners build automaticity and confidence. In comprehension, longitudinal tracking is needed to validate the schema-engagement hypothesis from Study 4. This would reveal whether the expert pattern of more fixations on in-domain text emerges gradually with knowledge or appears as a threshold effect once a critical mass of knowledge is acquired.

\subsection{Intervention Studies}

The present work was descriptive and correlational, so a crucial next step is to move from description to intervention by testing whether instructional designs based on these findings actually improve outcomes.

For example, Study 5 showed that pupil trajectories can dissociate from subjective ratings. This finding invites intervention studies testing adaptive scaffolding triggered by physiological signals. Randomized trials comparing adaptive systems that respond to physiological indicators versus systems that respond only to performance errors could test whether this process-level adaptation provides benefits beyond outcome-level adaptation.

A second avenue, based on the production findings, involves explicit training in trade-off awareness. Study 1 showed that speakers make specific CAF trade-offs under load. Pre-post designs could therefore be used to teach this trade-off awareness. This involves explaining that slowing to maintain accuracy is strategic, not a failure. Such designs could then be used to determine whether this metacognitive calibration improves prioritization and mediates load effects. 

For comprehension, future research should employ direct measurement of domain knowledge through validated assessments or knowledge probes rather than relying on major as a proxy. Think-aloud protocols could reveal online schema activation processes that eye-tracking and pupillometry cannot capture, clarifying when domain knowledge genuinely facilitates processing versus when learners merely spend more time engaging with familiar content.


Finally, Study 3's load-self-efficacy coupling suggests interventions based on sequenced complexity with load-efficacy monitoring. Because certain sequences may be vulnerable to vicious cycles (load spike $\to$ efficacy drop $\to$ disengagement), factorial experiments could be designed to test whether instructional sequences that alternate high-load and consolidation phases, or that provide explicit efficacy-supporting feedback, can interrupt these cycles and clarify optimal pacing.



\section{Concluding Reflections}\label{sec:conclusion}

This thesis began with a simple observation: millions of adults communicate daily in second languages they have not fully mastered, experiencing observable breakdowns in both production and comprehension. The research program traced how these breakdowns reflect principled, adaptive responses to the fundamental cognitive constraint of severely limited working memory capacity, rather than random failures or deficits.

Across five studies and two experiments, three core claims emerged and were consistently supported:

\begin{enumerate}
    \item Cognitive load forces trade-offs, not uniform collapse. Speakers do not simply ``perform worse'' under high load; they prioritize lexical precision over phrasal fluency (Study 1), and listeners adapt their evaluative criteria accordingly (Study 2). Readers do not simply ``fail to comprehend'' under linguistic complexity; they shift from global, expert strategies to local, survival-level parsing (Study 4), with physiological engagement patterns that can dissociate from subjective difficulty (Study 5). These patterns are rational reallocations of scarce resources, and effective instruction must recognize them as such rather than penalizing any deviation from idealized native-like performance.
    \item Linguistic complexity is the primary bottleneck; other factors modulate its effects. Across both modalities, intrinsic load, operationalized as element interactivity in production and linguistic complexity in comprehension, was the dominant constraint. Expertise, modality, and domain all mattered, but primarily by changing what was difficult (domain versus language) or when and how difficulty was experienced (temporal profiles, spatial footprints), not by eliminating difficulty entirely. This hierarchy has direct design implications: first, manage linguistic complexity; second, scaffold for expertise alignment; and third, choose modality for disciplinary fit.
    \item Process measures reveal mechanisms that outcomes obscure. Eye movements showed precisely when and where comprehension slowed (Study 4); pupil trajectories revealed moment-to-moment effort and engagement patterns that aggregate measures missed (Study 5); multivariate CAF analysis revealed trade-offs that separate univariate tests obscured (Study 1). These converging indices provided a mechanistic account of how learners cope with load, not merely whether they succeed or fail. Future L2 research should embrace this multi-method, process-oriented approach to move beyond post-hoc outcome measures toward real-time understanding of adaptive behavior.
\end{enumerate}


The findings also clarified a critical dissociation: subjective difficulty and physiological engagement can move in opposite directions (Studies 3 versus 5). Learners' conscious appraisals of ``how hard this feels'' do not always map onto sustained cognitive investment. Instruction that feels ``easy'' may fail to engage deep processing; instruction that feels ``hard'' may nonetheless sustain attention when format aligns with trained schemas. This has profound implications for adaptive learning systems, learner choice architectures, and instructor decision-making: do not trust subjective difficulty ratings alone.

Beyond these empirical contributions, the thesis makes a broader argument for treating production and comprehension as coordinated expressions of the same capacity-limited system. The parallel design allowed direct comparisons: speakers sacrifice phrasal fluency to protect lexical precision (Study 1), and listeners shift their weighting of linguistic cues based on speaker task demands (Study 2); readers sacrifice global strategies when linguistic complexity bottlenecks working memory (Study 4), and self-efficacy tracks these momentary load spikes (Study 3). These are not separate phenomena but two views of the same underlying architecture: a severely limited working memory system that forces prioritization, adaptation, and trade-offs.

Finally, the work underscores that cognitive constraints are not merely technical problems for laboratory study; they have profound equity implications. Instruction that fails to manage cognitive load systematically disadvantages learners with fewer prior resources (less L1-L2 orthographic overlap, less domain background knowledge, less confidence from prior success). By clarifying which design features genuinely reduce avoidable coordination (e.g., integrated glosses, clause-final clarity, segmented video) versus which merely shift where effort is spent, this research provides evidence-based guidance for creating more equitable learning environments. 

The road ahead is clear: replicate across populations and tasks, extend to developmental and interactive contexts, test interventions based on these principles, and continue integrating subjective, behavioral, and physiological indices to understand when, why, and how L2 learners adapt to the fundamental constraint of limited cognitive capacity. The payoff is a second language pedagogy that is grounded in cognitive reality, responsive to learner diversity, and committed to supporting the millions worldwide who navigate daily life in languages they are still learning to master.