\chapter{Introduction and Roadmap}\label{ch:intro-roadmap}



Millions of adults worldwide communicate daily in second languages (L2) they have not fully mastered. In real time communication, these learners experience observable breakdowns. In production, these include mid utterance pauses during lexical retrieval, syntactic simplification, and failure to deploy acquired grammatical structures under pressure. In comprehension, they appear as slower reading speeds, more frequent rereading of text, difficulty following a rapid speaker, or difficulty integrating complex ideas while learning new academic content. These phenomena reflect a fundamental cognitive constraint, namely a limited working memory capacity \parencite{Cowan2001}. This thesis examines how cognitive load, the processing demand imposed on limited working memory, constrains both L2 production and comprehension. Specifically, it investigates how processing demands reallocate limited attention during performance and how instructional design can manage these demands to support fluent, accurate, and meaning-focused performance.

The implications extend beyond individual learner experiences. Second language proficiency has become essential for educational, professional, and social participation in globalized contexts, rendering effective L2 instruction both a practical necessity and an equity concern. Traditional approaches that rely primarily on exposure may overwhelm learners, whose working memory resources must at the same time manage comprehension (e.g., listening, decoding) and production (e.g., planning, formulating). This persistent processing struggle is not only a cognitive problem, it also has motivational consequences. Instruction that fails to account for these constraints can erode learner self-efficacy, leading them to plateau at intermediate proficiency levels or disengage entirely \parencite{Han2004}. This thesis, therefore, investigates both the cognitive and affective impact of processing load.

Cognitive Load Theory (CLT; \citeauthor{sweller1988},\citeyear{sweller1988}) provides a systematic framework for analyzing these constraints. Because working memory capacity is severely limited, language processing faces a fundamental bottleneck. The empirical work in the following chapters tests predictions derived from this account. \autoref{ch:cogload} provides a detailed specification of the working memory architecture, the sources of cognitive load, and classic design effects that follow from capacity constraints.

This chapter proceeds in two parts. \autoref{sec:problem} frames the central problem by applying CLT to L2 production and comprehension. \autoref{sec:researchprogram} details the research program, presenting the research questions that guide the empirical studies and providing a brief chapter-by-chapter roadmap.


\section{Problem Framing}\label{sec:problem}


Adult L2 learners face cumulative linguistic and task demands that strain limited processing capacity. Cognitive load is used here because it unifies speaking and reading under a single capacity-limited account, distinguishes intrinsic from extraneous demands, and supports design-level predictions we can verify empirically. This provides a principled way to link task/material features to online processing and outcomes. With that in view, \autoref{subsec:framework} explains and defines the framework; \autoref{subsec:production-comprehension} apply it to production and comprehension; \autoref{subsec:gaps} poses the questions; \autoref{subsec:purpose} sets scope.

\subsection{Cognitive Load as Framework}\label{subsec:framework}


Working memory is commonly modeled with a central executive and phonological and visuospatial stores \parencite{Baddeley2003}. Cognitive load refers to the total processing demand imposed on working memory during a learning or performance task \parencite{sweller2010}. A critical distinction clarifies why L2 learners face elevated baseline demands: the automaticity of linguistic processing. In first language (L1) use, processes such as lexical access, syntactic assembly, and phonological encoding proceed largely automatically, imposing minimal demand on working memory. In second language (L2) use, these same operations require controlled attention and consume substantial working memory resources until they become automatized through extensive practice \parencite{McLaughlin1990,Segalowitz2010}. This shift has two consequences. First, it elevates baseline intrinsic load: materials that are routine for L1 users impose higher processing demands for L2 learners because controlled attention must be allocated to processes that are automatic in the L1. Second, it increases vulnerability to extraneous load, as any design-imposed demands compete for an already taxed working memory system. These differences make careful instructional design especially critical for L2 learners.




\subsection{L2 Production and Comprehension Under Load}\label{subsec:production-comprehension}

Both production and comprehension draw on the same limited working memory system, although capacity constraints appear differently across modalities, which creates distinct but parallel challenges for L2 learners. In production, capacity limits become visible when planning, formulation, articulation, and monitoring must unfold simultaneously. Under rising demand, learners redistribute attention strategically: they streamline content, select simpler constructions, and tolerate disfluencies to maintain message continuity. These trade-offs among complexity, accuracy, and fluency reflect adaptive responses to limited cognitive resources rather than failures \parencite{housen2009}. Critically, these adjustments carry perceptual consequences: listeners use temporal and acoustic cues to judge whether speech sounds fluent and easy to follow \parencite{KormosDenes2004}. In comprehension, similar constraints operate but surface as different behavioral signatures. As linguistic or conceptual difficulty increases during reading, learners slow their reading rate, advance in smaller increments, and revisit earlier material more frequently \parencite{rayner1998}. These patterns intensify in a second language, where both word recognition and sentence integration draw more heavily on limited attentional resources \parencite{just1992}. Despite these differences in how load manifests, both modalities face the same underlying constraint: a severely limited working memory system that creates predictable processing bottlenecks. This shared foundation motivates the unified treatment developed in subsequent chapters, where detailed empirical evidence for load effects in each modality is presented in \autoref{sec:consequences}.




\subsection{Open Questions and Gaps}\label{subsec:gaps}

Although progress continues, several foundational issues remain. First, the field needs a unified, system-level account across modalities. Production and comprehension draw on the same capacity-limited workspace \parencite{just1992}; the resources that sustain continuity, lexical selection, syntactic assembly, and monitoring in speaking are the same resources that support retention, integration, and coherence in reading and listening. Treating them together is therefore required: changes that help a speaker maintain continuity alter the cues available to listeners, and denser sentences force readers into slower, more piecemeal progress \parencite{Rayner2009}. Yet the literatures are often siloed, so parallel pressure points are rarely tested with shared manipulations and measures, hindering direct comparison and transfer of design principles. This thesis addresses the gap by applying a single capacity framework to parallel investigations of production and comprehension with shared manipulations and common verification.



The second issue is precision about redistribution. Even when demand is successfully manipulated, general rules for how attention rebalances across complexity, accuracy, and fluency under different task conditions remain elusive. In practice, speakers seem to protect continuity when pressure rises and invest in elaboration when pressure eases; but the exact pattern depends on features such as planning opportunities and the kinds of information a task pushes to the foreground. Resource-directing demands, which channel attention to particular content or relations, may shift choices differently from resource-dispersing demands, which increase the amount to be handled, yet the field does not yet have stable predictions for each type across tasks and proficiency levels \parencite{housen2009,EllisYuan2004,Johnson2017}. A related gap concerns perception. Listeners do not evaluate fluency in the abstract. They respond to global timing and continuity, and they may reweight these cues when pressure changes. Solid links exist between timing patterns and perceived fluency; what is less clear is whether that mapping itself shifts when demand is higher, for example, whether listeners tolerate brief disruptions when speaking conditions are heavy \parencite{KormosDenes2004}. Clarifying these contingencies would let task designers anticipate how changes in online load are likely to surface both in production and in what audiences hear.

The third issue is breadth in reading diagnostics and outcomes. Temporal indices indicate when effort spikes, but a fuller account also requires a spatial picture of how attention is organized across the page. When sentences are harder, gaze paths become less orderly; this loss of orderliness can be summarized without committing to a single sequence, offering a complementary window on how readers manage difficulty \parencite{VonderMalsburg2015}. What remains underdeveloped are systematic links from these spatial patterns to comprehension outcomes in L2 settings, and tests of how those links interact with individual differences in lexical access and current exposure. Because L2 reading typically taxes recognition and integration more heavily than L1 reading, differences in experience can shift the entire balance of what must be done moment by moment, altering both speed and the organization of the path through text \parencite{whitford2012second}. Closing this gap requires designs that treat temporal and spatial signals as joint evidence about coordination, then relate that coordination to what readers ultimately retain.

\subsection{Thesis Purpose and Scope}\label{subsec:purpose}

This thesis examines how increases in cognitive load shape two sides of L2 use: production and comprehension. The aim is practical as well as theoretical. By tracing how task features alter moment-to-moment allocation of attention, and by linking those adjustments to outcomes that matter (e.g., comprehensibility, fluency judgments, comprehension accuracy, and self-efficacy) the project distinguishes performance changes that reflect pressure on limited resources from changes that reflect knowledge or ability.

% AFTER 1.1.5 Thesis Purpose and Scope (same spot)


This choice follows accounts that treat cognitive load as motivational cost, predicting that instruction which reduces extraneous demand elevates self-efficacy during learning and that necessary intrinsic demand can also strengthen expectancy when benefits are visible \parencite{Feldon2019}. The research program comprises two experiments that together yield five studies. Using two experiments allows parallel coverage of speaking and academic learning while keeping designs focused and interpretable. Experiment~1 concentrates on how speakers reallocate attention and how listeners respond to those changes. Experiment~2 concentrates on how learners comprehend academic content when linguistic and contextual demands vary. 



\section{Research Program, Questions, and Roadmap}\label{sec:researchprogram}

\subsection{Research Questions Across Studies}\label{subsec:questions}


This research program comprises five linked studies that address complementary questions across production and comprehension. The sequence moves from production under validated cognitive load (Studies~1--2) to comprehension under factorial task demands (Studies~3--5), with each study building on insights from the previous investigation. Studies 1--2 draw on Experiment 1, and Studies 3--5 draw on Experiment 2. Each later study elaborates mechanisms suggested by the earlier one while maintaining a common focus on how limited capacity is reallocated and how those reallocations register in behavior and judgment. The questions are designed to make the logic visible from beginning to end: what learners do when cognitive load rises, what listeners actually notice, how learners handle increased difficulty in academic materials, and how moment-to-moment traces reveal the underlying adjustments.

Experiment~1 manipulates integration demand while holding content constant. Specifically, an element interactivity manipulation increases the number of relations that must be coordinated in real time, while the materials remain otherwise comparable. The logic is straightforward: if capacity limits shape how speech unfolds, tightening integration should force visible reallocations in how speakers balance message transmission, form control, and structural elaboration. Study~1 therefore asks:

\begin{description}
    \item[Study 1:] Which dimensions of complexity, accuracy, and fluency do learners protect under validated load, which do they allow to give way, and under what conditions are concurrent gains possible?
\end{description}


This answers a practical classroom and assessment question: when the same learner sounds different under heavier conditions, what exactly was traded off, and was any improvement achieved at the same time? Study~2 turns to the listener. Using the same recordings, it asks:

\begin{description}
    \item[Study 2:] Which linguistic dimensions best predict comprehensibility and perceived fluency for native listeners, and does the relative weight of those predictors shift with task demand?

\end{description}


Internal reallocations matter only insofar as they become audible; linking production data to listener judgments shows when a timing change that helps the speaker also helps (or hinders) the listener. Together, these two studies treat production and perception as two views of the same event, allowing claims about speaking under load to be tied directly to what audiences actually hear.

Experiment~2 manipulates three factors common in instructional contexts: linguistic complexity (simple versus complex syntax), domain (history versus science), and modality (text versus video), to test their main and interactive effects. This design makes it possible to examine how language difficulty, prior knowledge, and presentation format, each and in combination, influence cognitive load and comprehension. The factor structure separates questions that are often conflated: is it the language itself, the kind of knowledge at stake, or the way information is presented that most strongly drives load? 

Building on this shared experiment, Study~3 analyzes subjective cognitive load and self-efficacy ratings together with comprehension accuracy and asks:


\begin{description}
    \item[Study 3:] How do linguistic complexity, domain, and modality shape subjective cognitive load, self-efficacy, and comprehension accuracy in academic learning contexts?

\end{description}


The goal is to separate the difficulty of the language itself, the kind of knowledge at stake, and the presentation format while modeling confidence alongside performance so that small score differences are not overinterpreted. Study~4 analyzes eye movements in the text modality to characterize reading patterns and asks:


\begin{description}
    \item[Study 4:] Which temporal and spatial eye-movement indices best capture the interplay between linguistic complexity and domain expertise in L2 text comprehension?
\end{description}


The analysis is restricted to the text modality to enable position-anchored outcomes (e.g., fixation positions, regression paths). Considering temporal and spatial signals together provides a fuller picture of how attention is routed through text, linking the task factors from RQ3 to a concrete record of how readers proceed. The study tests whether domain expertise moderates the impact of linguistic complexity on these indices. Finally, Study~5 analyzes the complete pupil-diameter trajectory and asks:

\begin{description}
    \item[Study 5:]  Does continuous pupil diameter provide unique information about moment-to-moment cognitive load beyond subjective ratings and eye movements, and how do these dynamics vary with domain, modality, and linguistic complexity?

\end{description}


The rationale is straightforward: if demand fluctuates within a section, a continuously sampled physiological signal should register those fluctuations even when summary scores look similar. Study~5 thus complements Study~4 by providing a time-resolved physiological perspective that applies across modalities. 

Taken together, the five studies map cognitive capacity constraints from two vantage points. The production strand (Studies~1--2) shows how pressure reshapes delivery and how listeners respond to those changes. The comprehension strand (Studies~3--5) shows how pressure alters the path through information and what is ultimately comprehended.


\subsection{Contributions}\label{subsec:contributions}

This thesis contributes to second language research in three domains: theory, methodology, and instructional practice. The overarching contribution is the application of a unified capacity-based framework to analyze production and comprehension in parallel, with each modality examined using appropriate measurement tools. The result is a coherent account of how processing pressure changes what speakers do and what listeners hear, and how that same pressure changes how readers proceed and what they ultimately learn.

The work extends Cognitive Load Theory by demonstrating that intrinsic load operates as the primary bottleneck in L2 performance, that expertise modulates rather than eliminates processing constraints, and that modality effects depend on alignment between presentation format and learners' trained cognitive schemas. It advances limited capacity models by showing that complexity, accuracy, and fluency dimensions covary systematically under load and that listener judgments adapt to speaker reallocations. It contributes to L2 processing research through language-specific insights, multi-level process measures, and evidence for bidirectional load-efficacy coupling at section-level timescales.

Methodologically, both experiments implement converging manipulation checks and complementary process measures so that inferences rest on multiple, low-intrusion signals rather than a single indicator. Across studies, subjective effort ratings, measures of spare capacity, eye-movement traces, and physiological measures provide converging indices. Analyses proceed only when this verification holds. Key innovations include: (1) multivariate analysis of CAF that preserves covariance patterns and reveals trade-off structures, (2) spatial eye-movement indices (radius of gyration, convex hull area) that complement traditional temporal measures, and (3) trajectory-based pupillometry using GAMMs that reveals time-localized effects missed by aggregate measures. Taken together, these tools privilege interpretability: a manipulation is treated as effective only when multiple sources align, and process traces explain how any observed differences arose.

The findings translate into a hierarchically organized set of evidence-based guidance for L2 instruction. The hierarchy prioritizes linguistic complexity management above all other interventions, followed by expertise-contingent scaffolding (conceptual support for out-of-domain learners, linguistic support for in-domain learners), and modality alignment (matching presentation format to disciplinary training). For production, the work informs task sequencing and assessment practices that recognize adaptive trade-offs rather than penalize deviations from native-like norms. For comprehension, it provides concrete design levers for managing syntactic load, segmenting video materials, and monitoring both subjective difficulty and physiological engagement.



\subsection{Roadmap}\label{subsec:roadmap-ch}


Table~\ref{tab:roadmap} provides a visual overview of the thesis structure, showing how the chapters are organized across four parts. This thesis moves from shared foundations to targeted experiments and then to a synthesized account that connects production with comprehension.



\textsc{Part I: Overview and Framework (Chapters~1--2)} establishes the motivation for the work, positions the contribution within current theory and method, and reviews background that is shared by both modalities. Chapter~1 frames the problem, outlines the research questions, and previews the contributions. Chapter~2 specifies the cognitive architecture that underlies CLT, reviews empirical signatures of load in production and comprehension, describes design principles for managing cognitive demands, and examines the bidirectional relationship between cognitive load and self-efficacy. The goal is to ensure that later chapters employ the same terminology, the same assumptions about working memory and capacity limits, and the same rationale for manipulation checks and process tracing. With this groundwork in place, readers can treat the empirical analyses as applications of a single framework rather than as disconnected case studies.

\textsc{Part II: Production (Chapters~3--6)} addresses how cognitive load shapes L2 speaking and how listeners perceive that speech. Chapter~3 explains the Experiment~1 design and validation procedures. Chapter~4 reports Study~1 on how learners reallocate complexity, accuracy, and fluency under verified load. Chapter~5 reports Study~2, linking native listener judgments to observable timing, prosody, and form cues under different task demands. Chapter~6 synthesizes implications for task design and assessment.

\textsc{Part III: Comprehension (Chapters~7--11)} addresses how cognitive load shapes L2 reading and how learners comprehend academic materials under varying demands. Chapter~7 explains the Experiment~2 design, factors, and validation checks. Chapter~8 reports Study~3, analyzing subjective cognitive load and self-efficacy together with comprehension accuracy. Chapter~9 reports Study~4, analyzing eye movements in the text modality using temporal and spatial, position-anchored indices to characterize how domain expertise moderates linguistic complexity effects. Chapter~10 reports Study~5, modeling the complete pupil-diameter trajectory within each passage section across modalities (text and video) as a continuous index of moment-to-moment effort. Chapter~11 synthesizes when expertise buffers processing costs and when linguistic demands dominate.

\textsc{Part IV: Integration and Conclusions (Chapter~12)} closes the thesis by integrating findings across modalities. Chapter~12 draws explicit links from speaker choices to listener experience and from textual demands to the dynamics of reading, organizing the synthesis around common themes of validated demand and principled resource allocation. The chapter consolidates the theoretical, methodological, and practical contributions of the thesis. It concludes by presenting the study's limitations, identifying future research directions, and offering a final reflection on the central claims and their implications for L2 pedagogy. Throughout, recurring measurement logic and shared terminology help readers carry insights forward from one chapter to the next, making the whole more coherent than the sum of its parts.



\begin{table}[t]
\footnotesize
\centering
\caption{Thesis structure and chapter overview}
\label{tab:roadmap}
\begin{tabularx}{\linewidth}{@{}XX@{}}
\toprule
\multicolumn{2}{@{}c}{\textsc{PART I: Overview and Framework}} \\ \midrule
\multicolumn{2}{@{}l}{\textsc{Chapter 1}: Introduction and Roadmap} \\
& \\
\multicolumn{2}{@{}l}{\textsc{Chapter 2}: Cognitive Load: Architecture and Applications} \\
\multicolumn{1}{@{}l}{2.1: Cognitive Load Framework} & 2.2: Consequences of Cognitive Load \\
\multicolumn{1}{@{}l}{2.3: Factors That Affect Cognitive Load} & 2.4: Measurement of Cognitive Load \\
\multicolumn{1}{@{}l}{2.5: Cognitive Load and Self-Efficacy} & 2.6: Remaining Issues and Thesis Links \\
\midrule
\multicolumn{2}{@{}c}{\textsc{PART II: Production}} \\\midrule
\multicolumn{2}{@{}c}{\textsc{Chapter 3}: Methods for Experiment 1} \\
& \\
\multicolumn{1}{@{}l}{\textsc{Chapter 4}: Study 1} & \textsc{Chapter 5}: Study 2 \\
Prioritization of CAF in Production  & Listener Judgments Under Load \\
& \\
\multicolumn{2}{@{}c}{\textsc{Chapter 6}: Synthesis of Production} \\\midrule
\multicolumn{2}{@{}c}{\textsc{PART III: Comprehension}} \\\midrule
\multicolumn{2}{@{}c}{\textsc{Chapter 7}: Methods for Experiment 2} \\
& \\
\multicolumn{1}{@{}l}{\textsc{Chapter 8}: Study 3} & \textsc{Chapter 9}: Study 4 \\
Cognitive Load and Self-Efficacy in Comprehension & Attentional Allocation in Text Reading\\
& \\
\multicolumn{2}{@{}l}{\textsc{Chapter 10}: Study 5} \\
Temporal Dynamics of Cognitive Load in Comprehension & \\
& \\
\multicolumn{2}{@{}c}{\textsc{Chapter 11}: Synthesis of Comprehension} \\
\midrule
\multicolumn{2}{@{}c}{\textsc{PART IV: Conclusions}} \\\midrule
\multicolumn{2}{@{}l}{\textsc{Chapter 12}: Integration and Conclusions} \\ \bottomrule
\end{tabularx}
\end{table}