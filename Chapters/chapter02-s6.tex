\section{Remaining Questions from Prior Research and Links to Thesis RQs}\label{sec:gaps-and-rqs}

The preceding review established the cognitive architecture underlying Cognitive Load Theory, empirical signatures of load in production and comprehension, design principles for managing processing demands, methods for verifying load, and the bidirectional relationship between cognitive load and self-efficacy. Based on this review, this section synthesizes the main unresolved issues and clarifies how they link to the Research Questions in Chapter~\ref{ch:intro-roadmap}, Section~\ref{subsec:questions}.

A foundational methodological gap is the verification of cognitive load across tasks and populations. Designed task complexity, for example element interactivity, does not always translate into measurably higher demand for a given learner population \parencite{sasayama2016,Lee2019}. Because many studies rely only on subjective scales without objective corroboration, this thesis treats multimethod verification as a prerequisite for interpretation: in production, Experiment 1 (Studies 1 and 2) pair effort ratings with a secondary task; in comprehension, Experiment 2 (Studies 3 to 5) pair effort ratings with pupillometry.

Beyond this methodological imperative, specific empirical gaps remain. In production, while trade-offs among complexity, accuracy, and fluency under pressure are well described, it is not clear how a verified increase in demand alters the weighting of linguistic and prosodic cues in native listeners’ judgments of fluency and comprehensibility. This thesis addresses this by first having RQ1 identify redistribution patterns in speaking under verified load. Then, RQ2 models how judgments of the same recordings are predicted by observable cues, testing whether cue weights shift with task demand.

On the comprehension side, a critical gap exists in understanding how key instructional design choices interact to shape the joint cognitive-motivational landscape for L2 learners. While this chapter established that load and self-efficacy are linked, it remains unclear how theoretically-predicted interactions—such as whether content expertise can buffer processing demands, or how presentation format alters load—jointly influence both cognitive load and self-efficacy, and how this landscape in turn predicts learning. RQ3 is designed to test these main and interactive effects on subjective load, self-efficacy, and comprehension accuracy.

Beyond these outcome-level questions, a parallel gap exists in the process-level dynamics of L2 reading. It is unclear how L2 readers spatially reorganize attention for complex text, or how domain expertise alters this process. Furthermore, the moment-to-moment temporal profile of effort, especially across text and video, is poorly understood, as aggregate measures mask fluctuations. This thesis addresses these process-level gaps. RQ4 investigates how linguistic complexity and domain expertise shape readers' temporal and spatial navigation of text. RQ5 tests whether continuous physiological signals reveal unique, time-resolved effort dynamics across modalities.