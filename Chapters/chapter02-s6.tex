\section{Remaining Questions}\label{sec:gaps-and-rqs}


This chapter established the theoretical and methodological foundations for investigating cognitive load in L2 contexts. It specified the cognitive architecture underlying CLT (\autoref{sec:definitions}), characterized behavioral signatures of load (\autoref{sec:consequences}), reviewed design principles (\autoref{sec:manipulations}), detailed measurement and validation (\autoref{sec:verification}), and examined load–efficacy coupling (\autoref{sec:load-selfeff}). Despite this progress, critical empirical and methodological gaps remain. This section synthesizes those gaps and maps them to the Research Questions presented in Chapter~\ref{ch:intro-roadmap}. A cross-cutting concern is generalizability beyond English. As noted in \autoref{subsec:jp-func-words}, this thesis targets Japanese, a typologically distant language with agglutinative morphology and SOV order, testing whether documented load effects and design principles extend to learners navigating these structural differences. Against this backdrop, three empirical gaps structure the work that follows.


\begin{description}
    \item[Gap 1: Production:] In production, while trade-offs among CAF under pressure are well described (\autoref{subsec:prod-sigs}), it is not clear how a verified increase in demand alters native listeners' weighting of linguistic dimensions when judging comprehensibility and fluency. Study 1 first identifies redistribution patterns in speaking. Study 2 then models how judgments of the same recordings are predicted by observable cues, testing whether cue weights shift with task demand.
    \item[Gap 2: Comprehension:] Although cognitive load and self-efficacy are linked (\autoref{sec:load-selfeff}), most studies measure these constructs over long periods (e.g., semester-long), obscuring short-term fluctuations that could reveal how they influence each other moment-to-moment. We address this by measuring both constructs at the section level within tasks (Study 3).
    \item[Gap 3: Comprehension—process dynamics:] Most studies focus on temporal signatures while neglecting spatial organization of attention in L2 text reading. Moreover, aggregate measures obscure within-passage fluctuations in cognitive load that could reveal how readers allocate attention. Study 4 integrates spatial scan-path metrics with temporal eye-movement indices in text to characterize how attention is spatially organized. Study 5 addresses the masking problem by modeling continuous pupil trajectories across text and video, revealing time-localized cognitive load dynamics that aggregate measures cannot capture.
    
\end{description}