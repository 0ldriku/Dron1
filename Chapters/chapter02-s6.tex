\section{Remaining Questions from Prior Research and Links to Thesis RQs}\label{sec:gaps-and-rqs}


This chapter established the theoretical and methodological foundations for investigating cognitive load in L2 contexts. It specified the cognitive architecture underlying CLT (\autoref{sec:definitions}), characterized behavioral signatures of load (\autoref{sec:consequences}), reviewed design principles (\autoref{sec:manipulations}), detailed measurement and validation (\autoref{sec:verification}), and examined load–efficacy coupling (\autoref{sec:load-selfeff}). Despite this progress, critical empirical and methodological gaps remain. This section synthesizes those gaps and maps them to the Research Questions presented in Chapter~\ref{ch:intro-roadmap}.

\textbf{Gap 1: Verification.} As noted in \autoref{sec:verification}, designed task complexity does not always translate into measurably higher cognitive demand \parencite{sasayama2016,Lee2019}. Heavy reliance on subjective scales leaves open whether observed performance differences reflect cognitive load or other factors. We therefore require converging evidence—secondary-task costs in production and pupillometry in comprehension—before interpreting outcomes.

\textbf{Gap 2: Production—perception under verified load.} While CAF trade-offs under pressure are documented (\autoref{subsec:prod-sigs}), prior work has not tested whether verified demand changes native listeners’ weighting of linguistic and prosodic cues when judging comprehensibility and fluency. We first quantify redistribution under verified load (Study 1) and then model listener judgments of the same productions (Study 2).

\textbf{Gap 3: Comprehension—interactive design effects.} Although load and self-efficacy are linked (\autoref{sec:load-efficacy}), theoretically predicted interactions—content expertise buffering demand (\autoref{subsec:boundaries}) and presentation format altering coordination costs (\autoref{subsec:extraneous})—are rarely tested jointly. We evaluate main and interaction effects of syntactic complexity, domain, and modality on subjective load, self-efficacy, and comprehension accuracy (Study 3).


\textbf{Gap 4: Comprehension—process dynamics.} The review in \autoref{subsec:comp-sigs} highlights temporal signatures (e.g., fixation durations, regressions), but spatial organization of attention remains underexplored in L2 text reading, and aggregate measures mask within-passage fluctuations in cognitive effort. Study 4 integrates spatial scanpath metrics with temporal eye-movement indices in text to characterize how attention is spatially organized. Study 5 addresses the masking problem by modeling continuous pupil trajectories across text and video, revealing time-localized effort dynamics that aggregate measures cannot capture.

