

\section{Cognitive Load and Self-Efficacy}\label{sec:load-selfeff}

Beyond verifying the existence of cognitive load through these objective measures, it is also critical to understand its consequences for learners' affective and motivational states. Cognitive load and self-efficacy, in particular, influence one another in both the short term and across longer spans. Recent syntheses integrate CLT with models of expectancy, value, and cost by treating extraneous load as motivational cost and by documenting direct, task-specific effects of load on self-efficacy \parencite{Feldon2019}. Under this view, reducing avoidable coordination lowers perceived cost and protects concurrent self-efficacy, while calibrated intrinsic demand can raise expectancy when it produces visible learning benefits.

Self-efficacy, an individual's belief in their capability to perform a specific task, is among the most powerful motivational predictors of academic achievement \parencite{bandura_self-efficacy_1997}. Meta-analyses demonstrate that higher self-efficacy predicts greater persistence, more adaptive strategy use, and higher achievement across educational levels and domains \parencite{honicke_influence_2016,richardson_psychological_2012}. Furthermore, longitudinal work has revealed a reciprocal relationship: performance success nurtures subsequent self-efficacy, while elevated self-efficacy feeds forward into future achievement, although performance tends to be the stronger driver \parencite{talsma_i_2018}.

Contemporary integrative accounts position experienced effort as the bridge between object-level processing and meta-level control; within this framework, self-efficacy functions simultaneously as an antecedent to cognitive processing and as an outcome that is updated by the experience of allocating effort under varying demands \parencite{de2020synthesizing,de_bruin_synthesizing_2020,wang_how_2023}. Although strategic behaviors such as planning, monitoring, and control can reduce extraneous load, they simultaneously impose metacognitive demands that consume the same cognitive resources they aim to liberate. The present thesis therefore samples effort and self-efficacy at section-level granularity where applicable, so that inferences respect the temporal profile of load and belief updating.

The following subsections synthesize empirical evidence for bidirectional influences between cognitive load and self-efficacy. Section~\ref{subsec:load-efficacy-links} reviews evidence across multiple time scales. Section~\ref{subsec:design-efficacy} examines instructional features that mediate or moderate this relationship. 

\subsection{Mutual Influences Between Cognitive Load and Self-Efficacy}\label{subsec:load-efficacy-links}

Experimental and longitudinal studies demonstrate that instructional load influences self-efficacy through pathways independent of achievement. For example, in an undergraduate science course, materials designed to reduce extraneous load produced larger pre-to-post gains in self-efficacy than comparison materials \parencite{feldon2018self}. These differences persisted after controlling for performance, suggesting that reduced cognitive burden directly influenced capability beliefs rather than operating solely through improved achievement. This temporal dissociation is consistent with observations that confidence can dip while learners work through difficult content, yet strengthen after sustained, supported success with necessary complexity.

Converging evidence across domains reinforces the robustness of this pattern. Recent physiological studies reveal cognitive load spikes at precise moments: pupillometry shows sharp effort increases during speaker or accent switches in sentence comprehension \parencite{mclaughlin_sequence_2024}, while fixation-locked EEG with eye-tracking captures transient peaks from task-irrelevant decorative images \parencite{scharinger_task-irrelevant_2024}. Diary studies indicate that perceived difficulty spikes precede immediate self-efficacy dips, especially when baseline self-efficacy is low \parencite{stoten_metacognition_2019}.

In creative problem-solving, higher creative self-efficacy corresponds to lower experienced load, and competence-relevant feedback reduces load while sustaining performance; mediation analyses indicate that part of feedback's benefit flows through reduced processing strain \parencite{redifer2021self}. In complex skills acquisition, guided formats such as worked examples or structured multimedia reduce perceived effort for novices and are associated with higher self-efficacy than unguided problem-solving when task complexity is high \parencite{zheng2008effects}. Field studies in authentic learning settings show that academic self-efficacy tracks working-memory-relevant demands, though sample sizes are sometimes small and subgroup differences counsel caution \parencite{vasile2011academic}. These micro-dynamics have cumulative consequences. Sustained high load leads to cognitive fatigue and performance declines \parencite{van_der_linden_mental_2003}. Conversely, optimally managed cognitive load fosters effective schema acquisition \parencite{sweller_cognitive_2011}, creating successful performance experiences that strengthen self-efficacy and adaptive strategy use \parencite{bandura_self-efficacy_1997,britner_sources_2006}.

Competing theoretical accounts exist: a proactive view posits that effective metacognitive monitoring guides regulation and performance \parencite{thiede_2003}, while a reactive view suggests that poor metacognition or maladaptive beliefs prompt disengagement, heightening overload vulnerability \parencite{schwonke_metacognitive_2015}. This latter view is supported by modern CLT, which formally identifies learner-internal states as a source of extraneous load. For example, working memory resources consumed by "intrusive worries about failure" constitute a form of extraneous load that can be reduced by "stress-suppressing activities," thereby freeing capacity for the primary task \parencite{paas2020}. The evidence provides strong support for a reactive pathway in which momentary increases in cognitive load are associated with subsequent declines in self-efficacy. 

\subsection{Instructional Mediators and Moderators of Cognitive Load and Self-Efficacy}\label{subsec:design-efficacy}

Several instructional features systematically shape the relation between experienced cognitive load and self-efficacy by altering where effort is spent and how that effort is interpreted. As demonstrated in the \citeauthor{feldon2018self}'s \citeyear{feldon2018self} study reviewed above, reducing extraneous processing demands through explicit task analysis and clear structure is associated with larger gains in self-efficacy than comparison instruction. In ordinary classroom implementations, a climate that combines clear structure with autonomy support corresponds to lower perceived extraneous and intrinsic load together with a stronger motivational environment that is conducive to self-efficacy growth \parencite{evans2024}.

Sequencing of task complexity moderates this process. Progressions that begin at moderate difficulty and then increase complexity foster germane processing and meta-awareness more reliably than uniformly high difficulty, and these process changes co-occur with gains in interest and performance that support later willingness to invest effort when interactivity increases \parencite{zeitlhofer2024complexity}. Models that integrate cognitive load with self-regulation clarify why these levers matter. Accounts that treat experienced effort as the bridge between processing and monitoring predict that learners require some bandwidth for regulation in order to interpret effort as informative rather than discouraging \parencite{de2020synthesizing}. Crucially, learners with stronger self-efficacy are typically more willing to expend the effort needed for self-regulated learning, thereby preventing overload; in contrast, learners with low self-efficacy often disengage, allowing intrinsic or extraneous load to overwhelm working memory \parencite{schwonke_metacognitive_2015}. Consistent with this prediction, interventions that keep difficulty near a moderate level elicit more adaptive strategy use than uniformly easy or uniformly hard tasks, and observed changes in strategy deployment are mediated by perceived load in an inverted-U-shaped pattern \parencite{seufert2024interplay}. In combination, these findings indicate that reductions in avoidable demand, calibrated sequencing, competence-focused feedback, and preserved regulation bandwidth function as mediators and moderators of the link between cognitive load and self-efficacy rather than as independent instructional goals \parencite{evans2024,Feldon2019}.

The literature reviewed above has several implications for the measurement approach adopted in this thesis. First, given that cognitive load and self-efficacy can fluctuate within a single instructional episode, both constructs are measured at section-level granularity rather than only at task completion \parencite{stoten_metacognition_2019}. This approach allows detection of transient spikes in processing demand and corresponding shifts in confidence that end-of-task ratings alone might miss \parencite{singer_reading_2017}.