
\section{Definitions, Scope, and History}\label{sec:definitions}

This section establishes terminology and scope. The following subsections specify the cognitive architecture on which CLT is built, define the three categories of cognitive load, and explain how classic instructional effects follow from these architectural commitments. Boundary conditions and common misclassifications are identified to ensure precise application, and implications for second language learning are outlined.

\subsection{Cognitive Load and Cognitive Architecture}\label{subsec:architecture}

%TODO: add the figure from baddeley about working memory.
Cognitive Load Theory presumes a specific architecture. New information is processed in a short-lived, capacity-limited working memory, while consolidated knowledge in long-term memory can be accessed with little cost \parencite{sweller2019}. Within working memory, a central executive coordinates a phonological store and a visuospatial store \parencite{BaddeleyHitch1974,Baddeley2002,Baddeley2003}. The focus of attention typically maintains only four meaningful units \parencite{Cowan2001}, which creates a bottleneck for complex cognition, including language. When task demands exceed this capacity, performance changes in systematic ways \parencite{sweller1988}. This architectural distinction has direct instructional implications, because designs that respect working memory limits facilitate schema construction and automation in long-term memory \parencite{sweller2019}.




CLT focuses on culturally acquired, biologically secondary knowledge (such as reading, mathematics, and domain-specific concepts) that requires explicit instruction and for which working memory limitations are especially consequential \parencite{schnotz2007}. Within this framework, \textcite{sweller2010} defines cognitive load as the moment-to-moment demand imposed on working memory during learning activities.


A standard analytical framework distinguishes three categories of cognitive load \parencite{sweller2010}. \textit{Intrinsic load} stems from element interactivity in complex cognitive tasks. It reflects demands that are directly relevant for performing and learning the task. For example, understanding a simple sentence with a single clause imposes lower intrinsic load than understanding a sentence with multiple embedded clauses, because the latter requires coordinating more elements simultaneously. \textit{Extraneous load} arises from activities that are not productive for learning. It can be caused by task design features, learner characteristics such as intrusive thoughts about failure, or environmental factors such as distracting information. For instance, presenting text and related diagrams in spatially separated locations forces learners to mentally integrate information across sources, consuming working memory resources without contributing to schema construction \parencite{ChandlerSweller1992}. \textit{Germane processing} refers to working memory resources productively devoted to intrinsic cognitive load after extraneous load has been minimized. Intrinsic and extraneous load are additive, so reducing extraneous demand frees capacity for schema construction and schema automation \parencite{paas2020}. We interpret ``germane'' as the allocation of freed capacity to intrinsic processing rather than as a separate, addable source. Figure \ref{fig:clt-load-profile} provides a conceptual model of this additive relationship, illustrating how intrinsic, extraneous, and germane load combine dynamically over time. This model also distinguishes instantaneous ``peak load'' from ``average load,'' providing a visual rationale for time-resolved measurement. \textbf{TODO: redraw the fig}

\begin{figure}[ht]
  \centering
  \includegraphics[width=\textwidth]{Chapters/c1c2fig/clt_load_profile.png}
  \caption[Load profile and additive sources]{Instantaneous and average load over time with an assumed capacity limit, showing intrinsic load, extraneous load, and germane load as additive components and indicating free capacity. Adapted from \textcite{paas2020}.}
  \label{fig:clt-load-profile}
\end{figure}


Central to these distinctions is the notion of element interactivity: material is simple when elements can be processed independently and complex when elements must be coordinated simultaneously \parencite{sweller2010}. Increases in the number and strength of such interactions raise intrinsic demands for a given level of prior knowledge. The same lens helps identify extraneous demands, because formats that force unnecessary coordination inflate interactivity without changing what must be learned. This architectural grounding motivates the instructional effects reviewed in the following subsection.


Learning is the construction and tuning of schemas in long-term memory that recode multiple elements as a single functional unit, which reduces experienced element interactivity over time \parencite{sweller1994,paas2003}. As procedures consolidate and operations become automatic, working memory cost declines. In L2 contexts, this implies that fluent lexical access and morphosyntactic routines operate as schema-based chunks that free capacity for message planning and discourse integration. This architecture is foundational for the thesis: the phonological loop and central executive are the primary bottlenecks in L2 speaking (Studies~1--2), while the visuospatial store and central executive are the key constraints in L2 comprehension (Studies~3--5). These limits justify the interactivity manipulations in Experiment~1 and the format and pacing choices in Experiment~2. The next subsection shows how these architectural commitments generate classic format effects that guide design.

\subsection{Element Interactivity and Instructional Effects}\label{subsec:effects}

The development of CLT was initially motivated by evidence that conventional problem-solving approaches impose high working memory demands. Specifically, means-ends analysis requires extensive search processes that consume substantial working memory resources \parencite{sweller1988}. Under these conditions, learners devote capacity to difference reduction and step selection rather than to noticing structure or forming schemas, which explains weak learning despite considerable effort \parencite{sweller1988,sweller2010}.

The formative response was to redesign activities to remove this unnecessary search so that remaining capacity could be invested in schema acquisition and automation. For example, goal-free problems minimize backward search, worked examples present solutions directly for study, and completion problems provide partial solutions that constrain search while still requiring essential processing \parencite{sweller2010,sweller2019}. Converging evidence came from format effects: physically or temporally integrating mutually referring sources (e.g., text and diagrams) was found to be superior to presenting them separately, just as removing redundant or duplicated information improved outcomes \parencite{chandler1991}. These effects follow from element interactivity as defined in Subsection~\ref{subsec:architecture}.

When text and diagrams that refer to one another are separated, learners must search for corresponding elements and hold partial results in working memory while integrating across sources. This imposes extraneous coordination that adds nothing to what must eventually be learned. Physically integrating the sources eliminates this avoidable search, freeing capacity for schema construction. Similarly, when the same information is presented redundantly in multiple formats (e.g., narration plus on-screen text that duplicates the narration), learners must reconcile the streams, again consuming capacity without advancing understanding. In this thesis, we cite the intrinsic–extraneous–germane triad for completeness, but the production strand manipulates intrinsic load and holds extraneous factors constant; the comprehension strand manipulates intrinsic load and presentation modality while holding other extraneous features constant.


The boundary between necessary and unnecessary coordination depends on what the learner already knows. For novices, integrated presentation reduces search and supports learning. For experts, the same integration can become redundant because schemas in long-term memory already link the elements, making external guidance superfluous \parencite{kalyuga2007}. This expertise-reversal effect is a central boundary condition, addressed in detail in Subsection~\ref{subsec:boundaries}. The next subsection specifies these boundaries and clarifies when classifications shift with learner knowledge.

\subsection{Breadth, Boundaries, and Distinctions}\label{subsec:boundaries}

While CLT aims to provide generalizable design principles, applying its core distinctions to specific instructional contexts requires careful justification. Intrinsic, extraneous, and germane demands cannot always be cleanly separated in theory or in practice, since the same feature may be necessary for learning in one context yet superfluous in another \parencite{schnotz2007}. Contemporary accounts emphasize that classifications shift with expertise and goals \parencite{sweller2010,sweller2019}.

A critical boundary condition is that optimal instructional formats vary with learner expertise \parencite{kalyuga2007}. Design features that support novices can become redundant or even counterproductive as learners gain knowledge. For example, a graphic that guides attention for novices may become superfluous once learners have encoded the relevant relations in long-term memory, at which point it imposes extraneous load \parencite{sweller2010}. In adult L2 instruction, this predicts that integrated supports such as glosses should benefit novices but may become redundant as learners internalize vocabulary mappings \parencite{Sweller2017TESL}. Empirically, both outcomes appear. High-knowledge readers can learn more from lower-cohesion materials because they supply bridging inferences, while low-knowledge readers benefit from increased cohesion, the classic reverse cohesion pattern \parencite{mcnamara1996are}. A broader expertise-reversal literature reports that guidance and scaffolds assist novices but can become neutral or counterproductive for knowledgeable readers, consistent with a shift in the optimal balance between externally provided structure and self-generated integration \parencite{tetzlaff2025}. These constraints guide Study~4, which tests whether domain expertise buffers the processing costs of complex language in L2 reading; see Part III, Chapter~9 for design and measures.


\subsection{Consolidation, Replication, and Extensions}\label{subsec:consolidation}

Recent summaries portray CLT as consolidated and refined by replication work and by integration with compatible perspectives. Practical implications follow: manipulations should be justified in terms of predicted changes in element interactivity for a given population, measurement choices should be aligned with the architecture, and contrasts should remain close to the theory's mechanisms \parencite{sweller2019,chenpass2023}. Classic prescriptions remain warranted when their premises hold: physically or temporally integrating mutually referring sources is appropriate when learners would otherwise be forced into avoidable search and mapping, and removing duplicated but nonessential information is advisable when redundancy would squander capacity \parencite{chandler1991,sweller2010}. These prescriptions presume careful attention to expertise because guidance essential for novices can become redundant as knowledge grows, calling for adaptive fading or restructuring of supports \parencite{sweller2010,sweller2019}.

Recent work integrates CLT with motivational theories. From a self-determination perspective, instructional formats that reduce extraneous load and provide clear structure can be combined with autonomy support, yielding lower reported cognitive load and higher engagement \parencite{evans2024}. Reviews of digital and online learning likewise argue for balancing motivational affordances with cognitive constraints and for constructive alignment so that load is managed in service of targeted outcomes \parencite{skulmowski2023}.

Across these extensions, the unifying message is conservative and cumulative: treat the architecture as a constraint, articulate how a design changes coordination demands for specific learners, and evaluate outcomes that include both learning and motivational quality \parencite{sweller2019,sweller2023}. Applied summaries extend these mechanisms to L2 teaching, arguing that, for adult novices, explicit instruction and careful control of redundancy and split-attention are preferable to unguided immersion until schemas begin to consolidate \parencite{Sweller2017TESL}. The contrasts and measures in this thesis follow consolidated prescriptions on integration, redundancy, and segmentation, with expertise treated as a moderator.

\subsection{Implications for L2}\label{subsec:l2-preview}

Although CLT was shaped largely by research in mathematics and science, its architectural claims generalize to domains where biologically secondary knowledge must be learned through capacity-limited working memory \parencite{schnotz2007,sweller2019}. Two implications follow for second language contexts.

First, element interactivity is partly a function of prior knowledge and automation. When lexical access, morphosyntactic parsing, or discourse integration are not yet fluent, the same materials require coordinating more elements at once. By CLT's logic, this raises intrinsic demand for a given learner and makes design-induced coordination pressures more consequential \parencite{sweller2010,sweller2019}.

Second, expertise reversal effect (discussed in Section~\ref{subsec:boundaries}) is expected along proficiency gradients. Guidance that supports novices can become redundant or even disruptive as knowledge grows, because previously separate elements are chunked in long-term memory and no longer require the same working memory resources \parencite{sweller2010,sweller2019,sweller2023}. This preview frames later sections, which analyze L2 tasks through the lens of element interactivity and architectural alignment while considering the conditions under which added guidance shifts from helpful to redundant \parencite{sweller2010,sweller2023}.

Beyond linguistic proficiency, domain expertise also moderates processing demands in academic contexts. University students develop specialized reading strategies within their disciplines \parencite{shanahan2008teaching}. Disciplinary literacy research demonstrates that different academic domains emphasize distinct approaches to text comprehension, with each field developing domain-specific reading strategies and priorities \parencite{wineburg1991reading,goldman2002functional}. This disciplinary expertise predicts differential processing of domain-matched versus domain-mismatched materials, as tested in Studies~3--5.

Differences between phonological and visuospatial channels further imply that nonredundant modality can buffer coordination pressure in L2 tasks when streams are complementary and tightly aligned \parencite{mousavi1995,Baddeley2002}. A parallel implication, made explicit for adult L2 instruction, is that second language learning in adulthood constitutes biologically secondary knowledge and therefore depends on capacity-limited working memory and benefits from explicit guidance rather than unguided immersion at novice levels \parencite{Sweller2017TESL}. This framing clarifies why CLT's architecture and effects transfer to L2 contexts and why design features that reduce unnecessary coordination should be expected to matter in L2 classrooms.

The general complexity principles outlined above gain further nuance in Japanese, where the writing system supplies strong visual cues. The system exploits systematic mappings between script types and grammatical functions. Morphographic kanji encode content words, and phonographic kana mark grammatical morphemes \parencites{kajii2001eye,white2012eye}. This arrangement provides pre-lexical word-boundary cues that facilitate initial fixation targeting. At the same time, the system has vulnerabilities. When mixed-script text already provides segmentation cues, additional formatting changes such as inter-word spacing provide benefits only in all-kana contexts \parencite{sainio2007role}. Visual complexity within characters also matters: words containing high-stroke kanji are fixated more often and read more slowly than words with low-stroke kanji, even when lexical frequency is controlled \parencite{white2012eye}. These properties motivate the spatial indices used later, since visual cues and clause-final packaging in Japanese predict changes in the footprint of the scanpath under load. These implications motivate the spatial scanpath indices and the modality contrasts in Studies~3 to~5.



These architectural commitments from CLT are operationalized in L2 research through domain-specific limited capacity models. In production, \textcite{Skehan2009} and \textcite{Robinson2005} both assume that limited working memory constrains real-time speaking, but they make different predictions about task effects. Skehan's Limited Attentional Capacity Model predicts that speakers must trade off among complexity, accuracy, and fluency (CAF) when demands increase, protecting continuity at the expense of elaboration or control. Robinson's Cognition Hypothesis distinguishes resource-directing task features (e.g., reasoning demands), which can enhance complexity and accuracy despite fluency costs, from resource-dispersing features (e.g., time pressure), which typically depress all dimensions \parencite{Robinson2005}. In comprehension, \textcite{just1992} formalize the same capacity constraint: when linguistic complexity increases storage and computational demands simultaneously, working memory saturates, forcing readers into localized parsing at the expense of global integration. This capacity saturation is evident in their findings that low-span readers, who have a smaller pool of activation, are not only less accurate but also disproportionately slower on the most demanding parts of complex sentences.

The present thesis treats CLT and limited capacity models as complementary. CLT provides the architectural rationale (working memory structure, element interactivity) for why L2 tasks are demanding; limited capacity models specify which performance dimensions (CAF in production, eye-movement patterns in comprehension) will shift under that demand and predict observable trade-offs when capacity is exceeded. Studies~1--2 test CAF reallocations and listener judgments under validated element interactivity; Studies~3--5 test how linguistic complexity and domain expertise jointly shape reading dynamics when capacity is constrained. This integrated framework allows the thesis to move from general architectural claims to specific, testable predictions about L2 performance under load.