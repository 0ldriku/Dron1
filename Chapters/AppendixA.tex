%************************************************
\chapter{Full Model Specifications and Diagnostics}
%************************************************
{
\setlength{\parindent}{0pt}
\section{Science 1 Simple}
\subsection{Section 1}

\begin{CJK}{UTF8}{ipxm}
二つの暗号方式 

暗号技術は、共通鍵暗号と公開鍵暗号の2つに分けられます。共通鍵暗号とは、通信する二者間で同じ秘密鍵を共有して情報を暗号化します。一方の公開鍵暗号は、暗号化で使う鍵と復号に使う鍵が異なるのが特徴です。復号とは暗号化された情報を暗号化前の状態に戻すことです。
共通鍵暗号は、暗号化と復号化に同じ鍵を使用するため、対称鍵暗号とも呼ばれます。一方、公開鍵暗号は、暗号化と復号化に別々の鍵を使用するため、非対称鍵暗号とも呼ばれます。
共通鍵暗号は紀元前より使われてきた暗号化方式ですが、「どうやって秘密鍵を共有するか」という課題があります。 軍事目的で使われていた頃は、通信相手が特定できていたため、出発前にあらかじめ秘密鍵を渡しておくという方法でこの問題を解決していました。

\end{CJK}


\subsection{Section 2}

\begin{CJK}{UTF8}{ipxm}

共通鍵暗号の問題点 

しかし、インターネットのように不特定多数との通信を前提とする社会では、共通鍵暗号は使えません。インターネット上に存在するたくさんの相手のすべてと鍵を共有するのはとても効率が悪く、非現実的です。この課題を解決するのが公開鍵暗号です。公開鍵暗号では、データを暗号化する公開鍵と、暗号を復号する秘密鍵の二つの鍵があります。二つの鍵はペアですが、公開鍵から秘密鍵を推測するのは難しいです。公開鍵暗号で使用する暗号鍵は公開されているため、誰でも暗号化することができます。ただし、この暗号鍵とペアとなる復号鍵を持っている人しか、その暗号化されたデータを復号することはできません。つまり、暗号化されたデータを復号できるのは秘密鍵を持っている人だけです。この方法では、複号化に用いる秘密鍵を相手と共有しなくても多数の人と安全に通信できます。このように、公開鍵暗号はネットワーク社会に求められている暗号です。
\end{CJK}

\subsection{Section 3}

\begin{CJK}{UTF8}{ipxm}
公開鍵暗号と共通鍵暗号の機能と役割 

公開鍵暗号は鍵共有の課題を解決できますが、実際の運用では公開鍵暗号と共通鍵暗号のそれぞれの長所を活かして使い分けられています。
共通鍵暗号と公開鍵暗号を比べると、公開鍵暗号のほうが計算量がはるかに多いです。これにより、公開鍵暗号と共通鍵暗号の役割は大きく違います。通常、大容量のデータを暗号化する時、公開鍵暗号を使うと多大な時間がかかるため、共通鍵暗号を使います。しかし、共通鍵暗号では、暗号化と復号をする人が同じ鍵を共有しておく必要があります。そのため、大容量のデータの暗号化には共通鍵暗号を使用し、安全に共通鍵を共有するために公開鍵暗号を使用します。
以上が、暗号技術に関する基本的な知識とその歴史的背景、そして現代社会での役割についての概説です。情報化社会が進む中で、暗号技術は単なるデータの保護手段であるだけでなく、現代社会のインフラを支える重要な技術となっています。

\end{CJK}

\section{Science 1 Complex}
\subsection{Section 1}

\begin{CJK}{UTF8}{ipxm}

二つの暗号技術 

暗号技術の世界において、一般的な認識として広く知られているところによりますと、共通鍵暗号と公開鍵暗号という二つの主要な暗号方式があります。
前者に当たる共通鍵暗号という概念は、通信を行おうとする二人の通信者が、互いに同一の、言わば秘密鍵とも呼ばれるものを共有するという行為を通じて、その相互間における通信の秘密性を保持するという暗号方式です。
一方で、後者に該当する公開鍵暗号というものは、その暗号化プロセスにおいて利用される鍵と、復号鍵という、暗号化された情報を元の状態に戻すために用いられる鍵が、互いに異なる性質を持つものであるという特徴を有しています。
共通鍵暗号は、その同一の鍵を使用するという特性から、対称鍵暗号という別称がありますが、他方で、公開鍵暗号に関しましては、暗号化と復号化に異なる鍵を用いるという特徴から、非対称鍵暗号とも呼ばれています。
共通鍵暗号は、歴史的に見ますと、紀元前から非常に長い期間にわたって利用されてきた暗号化の形態であると言えるかもしれませんが、その実行に際しましては、「双方が如何にして秘密鍵を安全に共有するか」という、ある種の難問とも呼べるような課題が、常に付随しているということが指摘されています。
軍事的な場面など、ある特定の状況下におきましては、通信の当事者が明確に特定できるような場合、事前に秘密鍵を交換するという手法を用いることで、この問題を解決しました。
\end{CJK}


\subsection{Section 2}

\begin{CJK}{UTF8}{ipxm}
公開鍵暗号のメリット

しかしながら、インターネットという、いわば不特定多数との通信を前提とするような社会的環境におきましては、共通鍵暗号という手法が、必ずしも最適な選択肢とは言えません。なぜならば、インターネット上に存在する、数えきれないほどの多くの通信相手のすべてと鍵を共有するという行為は、効率性の観点から見ますと極めて問題があり、現実的な解決策とは言い難いという見方が、一般的に広く受け入れられているからです。
このような課題に対して、一つの解決策を提示しているのが、公開鍵暗号と呼ばれる暗号方式です。公開鍵暗号におきましては、データを暗号化するための公開鍵と、その暗号を復号するための秘密鍵という、二つの異なる性質を持つ鍵が存在するという特徴があり、これらの二つの鍵は互いにペアの関係にあるものの、公開鍵から秘密鍵を推測することは、計算量的に非現実的であるという性質を有しています。
公開鍵暗号で使用する暗号鍵は、その名の通り公開されるものであるため、誰でもその鍵を用いて暗号化を行うことが可能であるという特徴を持っています。しかしながら、この暗号鍵とペアとなる復号鍵を保持している人物以外は、その暗号化されたデータを復号することができないという制約が存在しており、言い換えますと、暗号化されたデータを復号できるのは、秘密鍵を保持している人物に限定されます。
このような方法を採用することにより、複号化に用いる秘密鍵を相手と共有するという行為を行わなくても、多数の人々と安全に通信を行うことが可能になるという利点があり、このような特性を有する公開鍵暗号は、現代のネットワーク社会において求められている暗号方式です。

\end{CJK}

\subsection{Section 3}

\begin{CJK}{UTF8}{ipxm}
公開鍵暗号と共通鍵暗号の機能と役割 

公開鍵暗号方式は、前述の鍵共有における課題を解決することが可能でございますが、実際の運用におきましては、公開鍵暗号と共通鍵暗号、それぞれの特性が効果的に活用されるよう使い分けられています。共通鍵暗号と公開鍵暗号という二つの暗号方式を比較検討いたしますと、公開鍵暗号の方が、その暗号化および復号化の過程において要求される計算量が、共通鍵暗号と比べて格段に多いです。このような計算量の差異があるため、公開鍵暗号と共通鍵暗号が果たす役割には、顕著な違いがあります。 
通常の状況下におきまして、大容量のデータを暗号化しようとする場合、公開鍵暗号を使用しますと、その処理に多大な時間を要するという問題が生じる可能性があるため、このような場面では共通鍵暗号が使用されるという傾向が一般的です。しかしながら、共通鍵暗号を使用する際には、暗号化と復号化を行う当事者が、同一の鍵を事前に共有しておく必要があるという制約が存在しているということもあります。
このような状況を踏まえまして、大容量のデータの暗号化には共通鍵暗号を使用し、その共通鍵を安全に共有するという目的のために公開鍵暗号を使用するという、二つの暗号方式を組み合わせた手法が広く採用されています。
以上が、暗号技術に関する基本的な知識とその歴史的背景、そして現代社会における役割について、概略的な説明を試みさせていただいたものです。情報化社会が急速に進展する現代において、暗号技術は単にデータを保護するための手段であるという狭義の役割を超えて、現代社会のインフラストラクチャーを支える、極めて重要な技術的基盤としての地位を確立しています。今後も技術の進歩と共に、暗号技術はさらなる発展を遂げ、私たちの日常生活や社会システムにおいて、より一層重要な役割を果たしていくことが予想されます。

\end{CJK}

\section{Science 2 Simple}
\subsection{Section 1}

\begin{CJK}{UTF8}{ipxm}
プライバシー保護データ作成の難しさ 

2006年、米国のNetflix社は匿名化データを利用した、映画推薦アルゴリズムを競い合うコンテストを行いました。
Netflix社は、DVDレンタルの履歴データを公開するために、ユーザIDのような準識別子を仮想的なIDに置換することで、個人情報の保護を図りました。それらのデータには、映画名、その映画に対する評価、登録日が含まれていました。
しかし、ある参加者が公開データと映画のレビューサイトを組み合わせることで、二名の個人が判明できたと発表しました。その方法は、時系列データマイニングの手法を用い、Netflix上のDVDレンタル順と、映画レビューサイトに投稿した順番が等しいものを探しだしました。
つまり、単純な匿名化や識別子の置き換えだけでは、十分にプライバシーを保護できません。このようなことから、Netflix社はデータ漏洩の可能性について調査や訴訟を受けることになりました。

\end{CJK}


\subsection{Section 2}

\begin{CJK}{UTF8}{ipxm}
匿名化技術 

データを匿名化する方法として “k-匿名性”があります。
k匿名性は、ある属性データをもつ個人がk人以上となるように データを加工することで匿名性を確保する技術です。
例えば、職業に関するデータを例とします。「心臓外科医」と「脳神経外科医」という分類項目に医師が1人ずつ存在するとします。ある属性データをもつ個人の数であるkを「2」に設定する場合には、この二人の共通点である外科医という共通属性に置き換える加工をすることで、「外科医の属性には2人いる」というデータになります。 
このk-匿名性は,どの属性に注目するのかによって、できあがる加工データが変わります.データ加工の方法が個体識別属性の削除、または仮名化を行って属性の抽象度を上げる場合、加工対象以外の属性についてはそのまま情報が残されますので、 注目する属性に応じて、抽象化する属性とそのまま残される情報が異なります。


\end{CJK}

\subsection{Section 3}

\begin{CJK}{UTF8}{ipxm}

属性情報推定のリスク 



k-匿名化が完全に満たされていたとしても個人の情報が漏れてしまう可能性があります。
例えば、あるECサイトがk-匿名化した購買履歴データを公開したとします。グループyのすべての人は商品Aを購入したことがあります。ここで、何らかの外部情報源により、ある個人xがこの購買履歴データに含まれ、その中のグループyに属していることが判明したと仮定します。
グループyにはk人のデータが含まれており、個人x がその中のどのデータに相当するかは不明です。しかし、個人xが商品Aの購入という事実を隠していたとしても、個人xがグループyに所属するという事実が判明すると、商品を購入したことがわかります。もしこのようなことが起きたら、個人xは、自分の好みを周囲に知られてしまうかもしれません。
企業側には顧客のプライバシーを保護する責任があり、データ公開の範囲と方法を慎重に検討する必要があります。
有用な情報であればあるほど、匿名性が犯される危険性が高いです。今後は高いプライバシーと有用性を両立するデータ利活用が求められています。


\end{CJK}


\section{Science 2 Complex}

\subsection{Section 1}

\begin{CJK}{UTF8}{ipxm}

プライバシー保護データ作成の難しさ 

2006年、DVDのレンタルサービス大手の米国のNetflix社が匿名化データを利用した映画推薦アルゴリズムを競い合うコンテストを開催しました。同社は、顧客のプライバシー保護に配慮するため、DVDレンタルの利用履歴データにあるユーザーIDという個人を特定する可能性がある準識別子を、実在しない仮想的なユーザーIDに置き換えるという匿名化作業を行うことで、十分なプライバシー保護ができると考えていたのです。
このようにして作成されたデータには、仮想的なユーザーID、映画の名称を示す情報、その映画に対する評価を表す数値、そして利用者が登録を行った日付を示す情報が含まれるという構成でした。
しかし、このコンテストの参加者の一人が、Netflix社によって公開されたデータと、誰もが自由にアクセスできる一般に利用可能な映画レビューサイトの情報を利用して、少なくとも二名の個人を特定することが可能であったという事態が明らかになったのです。この参加者は、時系列データマイニングと呼ばれる手法を使って、Netflix上でのDVDレンタルの順序と、映画レビューサイトへの感想投稿順序の一致パターンを分析することによって、個人の特定に至ったということです。
つまり、単純に匿名化を行うという作業の実施において、識別子の置き換えを行うという行為だけでは、プライバシーという個人の情報を十分に守ることができないという結果が明らかになったのです。このような事態を受けて、Netflix社はデータ漏洩が発生した可能性について、調査や訴訟を受けるという事態に直面することとなったのです。

\end{CJK}


\subsection{Section 2}

\begin{CJK}{UTF8}{ipxm}
匿名化技術 

データを匿名化するという行為を実現する方法として、"k-匿名性"という技術が存在しています。
k匿名性という技術は、ある属性データを持つ個人の数がk人以上となるように、すなわちkという数以上の人数になるようにデータを加工するという作業を行うことによってデータの匿名化を実施し、それによって匿名性を確保するという目的を達成する技術です。
例えば、職業に関するデータという、人々の仕事を示すデータを例として取り上げましょう。「心臓外科医」という職業と「脳神経外科医」という職業という、二つの分類項目に医師が1人ずつ存在するという状況が存在していると設定します。ある属性データを持つ個人の数であるkという値を「2」という数字に設定するという場合には、この二人の医師が持っている共通点である外科医という共通属性に置き換えるという加工を実施することによってデータの変換を行い、その結果として「外科医という属性を持つ人物が2人存在する」という匿名データになります。
しかし、k-匿名性は、どの属性に注目するのかという選択によって、できあがる匿名データが変わってしまうという特徴を持っているのです。データ加工の方法が個体識別属性の削除作業を実施する、または仮名化により属性の抽象度を上げるという操作を行う場合、加工対象以外の属性については、そのままの状態で情報が残されるという状況が発生しますので、注目する属性に応じて、抽象化する属性と、そのまま残されるという状態になる情報が異なるという複雑な状況が生じ得るということがあります。
\end{CJK}

\subsection{Section 3}

\begin{CJK}{UTF8}{ipxm}
属性情報推定のリスク 

k-匿名化手法が完全に実装されている場合であっても、個人に関する情報が推論可能となる状況が存在し得るということがあります。
例えば、あるECサイトがk-匿名化を実施することによって購買履歴データを公開したという状況を想定します。そのデータの中に、グループyという集団に属するすべての人々が商品Aという製品を購入したという共通属性を持っているという前提を設定します。ここで、何らかの外部情報源によって、ある個人xという特定の人物がこの購買履歴データに含まれているという事実が明らかになり、さらにその個人xがグループyに属しているという所属関係が判明したと仮定します。
グループyにはk人という数のデータが含まれているという状況が存在しており、個人xがそのk人のデータのうちのどのデータに相当するのかという対応関係は不明です。しかし、個人xが商品Aを購入したということを隠していたとしても、個人xがグループyに所属するということが判明すると、商品Aを購入したという行為を行ったことがわかるのです。もしこのようなことが起きたら、個人xは、自分の好みが周囲の人々に知られてしまうという状況に陥る可能性があるかもしれません。
このような事例は、企業側が顧客のプライバシーを保護する上で負う責任の重大性を示すものでして、データ公開の範囲および方法に関して、極めて慎重かつ戦略的な検討が必要であるということが指摘されているところです。
さらに、このような問題は、データの有用性とプライバシー保護のトレードオフ関係を明確に示すものでして、一般的に、データセットの情報が詳しくかつ有用であればあるほど、匿名性が侵害される危険性も高まる傾向にあるということが言えるでしょう。今後のデータ利活用においては、高度なプライバシー保護と情報の有用性を両立させることが極めて重要な課題となっています。

\end{CJK}

\section{History 1 Simple}
\subsection{Section 1}

\begin{CJK}{UTF8}{ipxm}

19世紀初頭のアイルランドの人口増加とその背景 
 
1800年に、イギリスがアイルランドを自国に統合する合同法を制定しました。これにより、アイルランドは連合王国の一部となりました。政治的にはイギリスの方針が優先され、アイルランドは連合王国の維持だけに利用されました。1815年にナポレオン戦争が終わると、アイルランドには兵士が戻ったのですが、不景気と穀物の不作で失業者が増え、都市で混乱が起こりました。その後、アイルランドの人口は急増しました。この増加の背景にはジャガイモの栽培がありました。ジャガイモは少しの土地でも育てることができ、貧しい人々に多くの栄養を提供することができました。結婚も早まり、子どもの数も増え、特に農村地帯で人口が急増しました。
\end{CJK}


\subsection{Section 2}

\begin{CJK}{UTF8}{ipxm}

胴枯病の感染 

植物の伝染病はジャガイモ飢饉の原因の一つとされます。1845年9月中旬、疫病が現れ、土の中のジャガイモはすべて腐っていました。8月に収穫して保存していたジャガイモも腐りはじめ、家畜でもそれを食べられない状態になりました。この伝染病は驚くほどの速度アイルランド全土に広がりました。翌年までの半年分の大事な食糧がなくなりました。
1845年11月に地方救済委員会が設立されました。一部の委員は同郷の人々の悲惨な姿を見て、無料で食糧を配ったことがありましたが、政府は救援食料の無料配給を認めませんでした。地方地主からの支援金が集まっているにも関わらず、政府は食糧の無料配給による市場価格の崩壊を恐れ、無料化しなかったのです。政府は人々の救済よりも市場の安定を第一に考えるという自由経済の原則を崩さなかったのです。
\end{CJK}

\subsection{Section 3}

\begin{CJK}{UTF8}{ipxm}
飢饉の再発生

アイルランドでは翌年、再びジャガイモの大飢饉が起こります。
1846年は世界中で穀物の収穫が少なく、食糧確保が世界的な問題となっていました。このような中、前年の冬の間に食糧が尽きたため、春には食糧を買えない貧しい人たちが倉庫に押し寄せ、食糧を奪ったり、小麦やオート麦を運ぶ荷車を襲ったりすることもありました。
貧しい農民たちは飢饉に対して何もできず、政府が経済に対して「自由放任主義」の方針をとり続けていたこともあり、アイルランド人は救済されず、見捨てられてしまいました。
この飢饉の間、100万人もの餓死者が出たのです。しかし、1847年には世界的に穀物が豊作で価格が下がったにもかかわらず、飢えた貧しい人たちには食糧が届きませんでした。食べ物が豊富な場所もあり、飢饉を利用して利益を得ている人もいました。 
こうしたことを考えると、アイルランドのジャガイモ大飢饉は自然災害ではなく、人為的な災害だったと言えます。

\end{CJK}

\section{History 1 Complex}
\subsection{Section 1}

\begin{CJK}{UTF8}{ipxm}

19世紀初頭のアイルランドの人口増加とその背景 

19世紀初頭、1800年に、イギリスが、アイルランドの政治的統合を目的とした法的枠組み、すなわち合同法と呼ばれる法律を制定しました。この法的措置の結果として、アイルランドという地理的および文化的単位は、連合王国という、より大きな国の一部として組み込まれることとなったという経緯がございます。
この政治的統合の過程において、より大きな影響力を持つ側の意向が自然と前面に押し出されることとなり、政策決定の主導権がイギリス側を優先する傾向が顕著であったということが指摘されているところでございまして、アイルランドという地域は、主として連合王国という政治的実体の維持と存続のための手段として利用されるという状況に置かれていました。
1815年に、欧州全土を揺るがしたナポレオン戦争の終結を迎えた際、多くの軍人たちが故郷に帰還しましたが、当時のアイルランドにおける経済的停滞と農業生産、特に穀物の収穫量の不足という複合的な要因により、失業率の急激な上昇が引き起こされ、その結果として都市部では平穏が揺らぐ事態に発展していったのです。
その後、アイルランドの人口が急激な増加を見せたのには、ジャガイモの普及が密接に関わっていたと言えるでしょう。ジャガイモという作物は、比較的狭小な耕作面積であっても十分な収穫量を得ることが可能であり、かつ栄養的にも優れた特性を有しているという特徴があったため、貧しい人たちにとって、効率的かつ効果的な栄養摂取の手段として機能しました。結果として、結婚も早まり、子供の数も増え、特に農村地帯においては、人口が急増しました。


\end{CJK}


\subsection{Section 2}

\begin{CJK}{UTF8}{ipxm}

胴枯病の感染 

19世紀半ばのアイルランドを襲った未曾有の食糧危機、すなわちジャガイモ飢饉と呼ばれる歴史的事象の発端を導いたのは植物の伝染病です。
疫病の発生が、1845年9月中旬に、確認されたということがございまして、土の中のジャガイモが全面的に腐敗するという事態にまで及んだということです。さらに同年8月に収穫され、保存されていたジャガイモも腐り始めました。この腐敗の程度は極めて深刻であり、家畜による飼料としての利用すら困難な状態に陥っていたのです。
この伝染病に関しましては、驚異的速さでアイルランド全土に蔓延したということがございます。その結果として、翌年までの約半年という期間に渡って必要とされる重要な食糧資源が、ほぼ全面的に失われました。
飢饉対策の一環として、1845年11月に、地方救済委員会が設立されました。このような厳しい状況下において、一部の委員は、同郷の人々の悲惨な生活実態を目の当たりにし、政府の規則を意図的に無視して食糧の無償配布を実施するという行動に出た事例が存在していたということがございます。
このような個別的な人道的行為に対しても、また地域の地主層からの支援金が一定程度集まっていたにもかかわらず、政府は一貫して救援食糧の無償配布を容認しないという姿勢を崩さなかったのです。 政府は食糧の無償配布が食糧価格の崩壊をもたらす可能性があると懸念していたのです。
このような政府の対応は、自由経済の原則に強く影響を受けていたと見られ、結果として、人命の救済という緊急性の高い課題よりも、市場の安定性の維持という経済的観点が優先されるという事態が生じ、これが後の飢饉の深刻化につながったとのことです。

\end{CJK}

\subsection{Section 3}

\begin{CJK}{UTF8}{ipxm}
飢饉の再発生

アイルランドでは翌年、再びジャガイモの大飢饉が起こりました。
この年は、世界規模で穀物の収穫量が少なく、食糧の確保という課題が世界的な問題となっていました。このような中、アイルランドにおいては、前年の冬の間中に食糧が尽きたため、春には、食糧を買えない貧しい人たちが倉庫に押し寄せ、強制的に食糧を奪うということや、小麦やオート麦といった穀物を輸送する車両を襲うということがありました。
このような状況下において、貧しい農民にとっては、この飢饉に対して有効な手段を講じることが極めて困難であったということがございます。さらに、当時の政府が経済に対して自由放任主義をとり続けていたこともあり、アイルランドの人は救済されず、実質的に見捨てられたも同然の状態に陥るという悲劇的な帰結になったのです。
この飢饉の間、およそ100万人という膨大な数の命が飢餓によって失われたという事実は、極めて衝撃的であったということが指摘されているところでございます。
アイルランドにおける食糧生産の実態に目を向けますと、ジャガイモ以外の食糧資源に関しては、通常時と大差ない生産量が維持されていました。さらに、1847年には世界的な穀物の豊作により、市場価格が下がりました。このような食糧事情の改善にもかかわらず、飢えた貧しい人たちに対して、十分な食糧供給が実現しなかったのです。
一方で、食糧が豊富にある地域が存在し、飢饉を利用して莫大な利益を得ている者もいたということから、これらの事象を総合的に考察すると、アイルランドの大飢饉は、避けがたい自然災害として分類することは適切ではなく、人間によって引き起こされた人為的な災害であったと結論づけられるでしょう。

\end{CJK}

\section{History 2 Simple}

\subsection{Section 1}

\begin{CJK}{UTF8}{ipxm}
ミグ25の本土侵入

1976年9月6日、ソ連防空軍の複数のミグ25戦闘機が訓練のためチュグエフカ基地を離陸しました。ヴィクトル・ベレンコ中尉が乗った戦闘機は途中で急に方向を変えて高度を下げました。日本のレーダーが午後1時10分にこれを検知し、午後1時20分に航空自衛隊千歳基地のF-4EJ戦闘機が緊急発進しました。
航空自衛隊はミグ25を探しましたが、地上のレーダーは戦闘機の超低空飛行に対応できませんでした。そして、F-4EJのレーダーは地面の反射で乱れやすく、低空の目標を探す能力は低かったです。
ミグ25は航空自衛隊に発見されないまま、市街上空を3度旋回しました。その後、午後1時50分頃に函館空港に強行着陸しました。着陸点を誤り、ドラッグ・シュートを使用したものの、オーバーランして草地で停止しました。燃料は約30秒分しか残っていませんでした。
ミグ25の着陸後20分で北海道警察が到着し、函館空港を封鎖しました。ベレンコ中尉は当日、北海道警察にアメリカへの亡命を求めました。

\end{CJK}


\subsection{Section 2}

\begin{CJK}{UTF8}{ipxm}

亡命後のベレンコ

ベレンコは、函館空港への強行着陸から3日後に、早々とノースウェスト航空の定期便で離日し、アメリカへと向かいました。ミグ戦闘機は分解され、茨城県の航空自衛隊基地に輸送されて徹底的に調べられました。
亡命後、ベレンコは安全のために名前と住所を頻繁に変えました。1980年、アメリカ議会がベレンコに市民権を与える議決を行い、ジミー・カーター大統領が署名して、ベレンコはアメリカの市民権を得ました。1983年、韓国の民間機がソ連に撃墜された事件で、ベレンコはソ連機の飛行士の会話の分析と暗号解読を行いました。
なおソ連側は、ベレンコがアメリカで交通事故にあって死亡したとか、帰国して処刑されたといったプロパガンダ情報を時折流しました。ベレンコは、事件発生から47年後となる2023年まで生き延び、アメリカの高齢者介護施設でふたりの子供に見守られながら亡くなりました。
\end{CJK}

\subsection{Section 3}

\begin{CJK}{UTF8}{ipxm}

事件の影響

この事件はパイロットの亡命が目的だったため、幸い被害は生じませんでした。しかし、攻撃目的だったら日本の防空網は簡単に突破されるという危険性が明らかになりました。この事件がきっかけで、予算の認可が難しかった早期警戒機E-2Cが導入されました。
ソ連はこの事件後、敵と味方を識別するためレーダーサイトの暗号を変更する必要がありました。事件調査でチュグエフカ基地を訪れた委員会はここの生活条件の悪さに驚きました。これがパイロットの待遇改善のきっかけになりました。
アメリカはそれまでにミグ25は主翼や胴体にチタニウム合金を使用したと考えていたのですが、この事件でステンレス鋼板を主に使用したことがわかりました。また、電子機器も時代遅れだったことが判明しました。アメリカはそれまでミグ25を大きな脅威と見ていましたが、ミグ25が速いだけで想定ほど怖くないことがわかりました。このことから米国は自国の技術的優位性を再認識したことにより、より現実的かつ積極的な軍事戦略の採用につながったことです。
\end{CJK}


\section{History 2 Complex}


\subsection{Section 1}

\begin{CJK}{UTF8}{ipxm}
ミグ25の本土侵入

1976年9月6日、ソ連防空軍に所属していた数機のミグ25戦闘機が、チュグエフカ基地を離陸し、訓練飛行を行う予定であったという状況の中で、ヴィクトル・ベレンコ中尉が操縦する一機が、予定されていた演習空域へ向かう過程において、予期せぬ方向への航路変更とともに、飛行高度の急速な低下を遂げたのです。この異常な飛行パターンは、日本国内のレーダーサイトによって探知され、この潜在的な領空侵犯の疑念を抱いた航空自衛隊は、その10分後、すなわち午後1時20分に、航空自衛隊千歳基地に配備されていたF-4EJ戦闘機が緊急発進するということに至ったということでございます。 
航空自衛隊は、この未確認機を全力で捜索しましたが、当時の地上レーダー設備が、戦闘機の超低空飛行という特殊な状況に十分に対応できる能力を有していなかったという技術的制約があったのです。さらに、F-4EJ戦闘機に搭載されていたレーダー装置も、地表面からの反射波による干渉を受けやすく、低高度での目標を探知する能力に一定の限界があったのです。
そうした状況の中で、ミグ25は日本の防空網を潜り抜け、函館空港へと接近したのですが、その後、市街地上空を三度にわたって旋回した後、午後1時50分頃に空港の滑走路上に強行着陸を敢行し、ドラッグ・シュートを使用したものの、着陸時にはオーバーランして滑走路から外れた地点に停止したのです。燃料残量わずか約30秒分という極限状態でした。
ミグ25戦闘機の着陸から約20分後、北海道警察が現場に到着し、函館空港の封鎖という措置を講じたということでございます。そして、事件発生日である同日内に実施された北海道警察による任意聴取で、ベレンコ中尉は自らのアメリカ合衆国への亡命の意向を表明しました。
\end{CJK}


\subsection{Section 2}

\begin{CJK}{UTF8}{ipxm}

亡命後のベレンコ

函館空港への強行着陸という極めて異例の事態が発生してから、わずか3日後に、ベレンコ中尉が、ノースウェスト航空の定期便を利用して日本を後にし、米国へと向けて旅立つことで国外に脱出したのです。一方、ベレンコ中尉が操縦していたミグ戦闘機に関しましては、茨城県の航空自衛隊基地へと輸送され、分解することによって解体作業の実施が行われました。 
ベレンコ中尉の亡命後の生活に目を向けますと、自身の安全を確保するという、身の危険を回避し安全を得るという切実な目的のもと、氏名および居住地を頻繁に変更するという、身元を隠す行動を取っていたということがございます。1980年には、米国の議会が、ベレンコ中尉に対して市民権を付与するという議決を行いました。この議決に、当時の大統領であるジミー・カーター大統領の署名を経て、ベレンコ中尉は米国の市民権を正式に取得することに至ったということです。
1983年に発生した、韓国の民間機がソ連の軍用機によって撃墜されるという事件において、ベレンコ中尉はソ連のパイロット間の会話の分析および暗号の解読という、音声を解析することによって情報を抽出するという重要な任務を遂行したのです。
なおソ連側はベレンコ中尉が米国において交通事故により死亡したという偽の情報や、ソ連に帰国の上、処刑されたという根拠のない報道を広める行動を取ったということが、当時のプロパガンダを示す事例として指摘されているところでございます。
ベレンコ中尉は、あの衝撃的な事件の発生から実に47年もの歳月が経過した2023年に至るまで、生き延びてきました。そして、米国の高齢者介護施設で自身の二人の子供に見守られながら、人生を終えるに至ったのです。
 
 
\end{CJK}

\subsection{Section 3}

\begin{CJK}{UTF8}{ipxm}

事件の影響

当該事件が、1人のパイロットの政治的亡命という個人的な動機に基づくものであったということに鑑みますと、人的被害や物的損害が生じなかったという結果は、不幸中の幸いであったと解釈することが可能かと存じます。しかしながら、この事象が日本の防空態勢に対して投げかけた問題は、決して軽視できないものでした。
具体的には、仮に当該航空機が攻撃的意図を持って日本の領空に侵入した場合、現行の防空網が容易に突破される可能性が高いという脆弱性が明らかになったということがございます。この認識がきっかけとなり、それまで予算の制約により導入が困難とされていた早期警戒機E-2Cの調達が、急遽承認されるに至ったということです。
一方、ソ連側の対応に目を向けますと、この事件を受けて、レーダーサイトの暗号システムの全面的な変更を余儀なくされたという事実がございます。さらに、事件の調査のためにチュグエフカ基地を訪問した委員会が、当該施設における生活環境の悪さに驚いたため、パイロットの処遇が改善されました。
それまでアメリカ側は、ミグ25戦闘機の主翼および胴体部分にチタニウム合金が使用されているという推測を立てていたのでございますが、実機の詳細な分析の結果、主たる構造材としてステンレス鋼板が採用されていたという事実が明らかになったということです。そして、ミグ25に搭載されていた電子機器も、想定を大きく下回る技術水準にあったということも判明いたしました。これらの発見により、それまでミグ25を極めて大きな脅威として認識していたのでございますが、当該機が高速性能に特化した存在であり、総合的な戦闘能力という観点からは、当初の想定ほど脅威的ではないという結論に至ったということでございます。このような認識の変化が、米国が自国の技術的優位性に対する認識を再認識し、より現実的かつ積極的な軍事戦略の採用へとつながったのです。

\end{CJK}
}