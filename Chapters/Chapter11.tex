

%************************************************
\chapter{Comprehension Synthesis (Studies 3, 4, and 5)}\label{ch:comp-synthesis}
%************************************************

\section{Introduction}
This chapter integrates the comprehension findings from Study 3, Study 4, and Study 5. It provides a single, synthesized account of how L2 learners process academic content under varying cognitive load. All three studies draw on the factorial design of Experiment~2 (\autoref{ch:exp2-methods}), which manipulated linguistic complexity, content domain, and presentation modality, while also accounting for learners' disciplinary expertise (academic major).

The goal is to move from discrete outcomes to a unified process model. Study 3 established that these factors jointly shape subjective cognitive load, self-efficacy, and accuracy. Study 4 (text) and Study 5 (text and video) clarified how and when processing unfolds, using eye-tracking and pupillometry to trace the real-time allocation of visual and cognitive attention. By integrating these subjective, behavioral, and physiological streams, this chapter explains how linguistic demands create processing constraints, how expertise modulates (but does not erase) these constraints, and how modality choice interacts with disciplinary alignment to shape moment-to-moment engagement.

\section{Summary of Evidence}
This synthesis draws on three complementary sources of evidence, all from Experiment~2's factorial design. Study 3 captured subjective experience and performance outcomes; Study 4 traced visual attention in real time; and Study 5 indexed physiological effort continuously. Together, they reveal:

\begin{itemize}
    \item \textbf{Study 3 (\autoref{ch:study3}): Subjective Outcomes and Performance.} This study analyzed subjective cognitive load, self-efficacy, and comprehension accuracy. It showed that science content and complex language each increased cognitive load and reduced self-efficacy. The modality effect was conditional rather than uniform: video presentation reduced self-efficacy under high intrinsic load conditions, but not across all load levels. Crucially, it identified two key interactions:
    \begin{enumerate}
        \item \textbf{Expertise moderation:} Learners' academic major interacted with domain and complexity. Science majors, for example, were sensitive to \textit{linguistic complexity} within their science domain, whereas arts/social sciences majors were simply overwhelmed by the \textit{domain} (science) itself, regardless of linguistic complexity. Notably, this pattern was asymmetrical in its effects on self-efficacy: science majors showed clear linguistic complexity sensitivity within science (higher self-efficacy for simple than complex), but arts/social sciences majors showed no reliable linguistic complexity sensitivity even within their own history domain. This suggests that STEM disciplinary training may cultivate greater attention to linguistic form than humanities training does, but this attention is selective and showed up most clearly when linguistic load was manageable.
        \item \textbf{Self-management bottleneck:} The negative impact of cognitive load on performance was most pronounced in complex passages, while the positive impact of self-efficacy was strongest in simple passages. Mediation analyses provided suggestive evidence that high load may consume the resources needed for both processing and self-regulation, with the indirect pathway via self-efficacy appearing strongest in history content. However, the credibility intervals for these indirect effects were wide, indicating that this pattern requires replication.

    \end{enumerate}
    
    \item \textbf{Study 4 (\autoref{ch:study4}): Eye-Tracking (Text).} This study analyzed eye movements in the text-only condition. The dominant finding was the primacy of linguistic complexity in driving processing cost across most measures. Complex syntax systematically increased fixation counts, reduced forward saccade length, and broadened the spatial footprint of reading for all learners. Expertise (major-domain congruence) did not provide a global efficiency buffer; in simple passages, experts showed more fixations, suggesting deeper, schema-driven engagement. This expert strategy, however, was overridden by high linguistic load. Unexpectedly, word-level processing (first-pass duration) showed a domain effect that was independent of expertise: history passages yielded faster lexical access than science passages for both major groups, suggesting that discourse-level differences in information density may create baseline processing differences that are orthogonal to learner expertise.
    
    \item \textbf{Study 5 (\autoref{ch:study5}): Pupillometry (Text and Video).} This study analyzed continuous pupil dilation as a physiological index of effort. It found selective video advantages in two specific conditions (simple history and complex science) where time-integrated pupil dilation was reliably lower for video than text, with early-onset and sustained temporal separations. No video advantage was observed in complex history or simple science. The key finding for understanding engagement dynamics was a Modality $\times$ Major interaction on the temporal trend of effort (see \autoref{ch:study5}, Figure~\ref{fig:rq2-trend-patterns}). Science majors showed a rising pupil trend (sustained engagement) in video but a falling trend (gradual disengagement) in text. Arts/social sciences majors showed flat-to-falling trends in both modalities.
\end{itemize}

\section{A Joint Account of Processing, Expertise, and Engagement}
\subsection{The Primacy of Intrinsic Load: Linguistic Complexity as the Bottleneck}
The most coherent finding across all three studies is that linguistic complexity acts as the primary processing constraint in most measures. This bottom-up limit sets the cognitive "price" of comprehension, which other factors (expertise, modality) can only modulate.

Study 4 provided the clearest mechanical evidence: complex syntax immediately degraded processing across most measures, forcing all readers into a more localized, cautious, and spatially dispersed reading pattern. Fixation counts rose, forward saccades shortened, and spatial footprints broadened under complex syntax, with these effects largely overriding expertise-based differences. This bottom-up cost directly explains the outcome-level findings in Study 3. The mediation pattern, where high load appeared to diminish self-efficacy's positive influence on performance, aligns with a capacity-sharing account. When linguistic complexity is high, working memory is saturated by basic parsing, leaving limited capacity for the top-down expert strategies observed in Study 4's simple passages or for the metacognitive regulation that self-efficacy typically supports.

\subsection{Expertise as a Modulator, Not a Global Buffer}\label{subsec:expertise-modulator}
The synthesis of Studies 3 and 4 refines our understanding of expertise. Disciplinary knowledge is not a simple buffer that makes reading "easier" or "faster." Instead, it changes what is difficult and how a learner allocates attention.

Study 3 showed that expertise changes what is difficult: for non-experts, the domain's conceptual density is the processing constraint; for experts, the constraint shifts to the linguistic realization of familiar concepts. Study 4 showed how experts allocate attention: when linguistic load is manageable (simple text), experts engage more deeply (more fixations), presumably to integrate text with their existing schemas and to fill schema "slots" with text-based propositions. This expert strategy is a luxury that is only possible when linguistic demands are low. As soon as linguistic complexity rises, this expert advantage weakens, and expert patterns converge toward novice-like processing.

Importantly, the asymmetry observed in Study 3's self-efficacy findings suggests that disciplinary training differs in how it shapes linguistic sensitivity. Science majors showed clear self-efficacy benefits from simplified syntax within their domain, whereas arts/social sciences majors showed no parallel sensitivity to linguistic complexity in history. This may reflect fundamental differences in how STEM versus humanities disciplines socialize students to attend to linguistic form versus narrative content.

\subsection{Modality Effects: Dissociation Between Subjective Experience and Physiological Engagement}
The joint findings of Studies 3 and 5 reveal a critical dissociation between conscious appraisal and physiological engagement, clarifying that modality effects are neither simple nor uniform.

\begin{description}
    \item[Subjective experience (Study 3):] Participants reported higher cognitive load for video than for text, a main effect that held across conditions. This likely reflects the transience of spoken language combined with the specific orthographic advantage our L1 Chinese participants had with L2 Japanese text (shared kanji/hanzi characters enabling rapid visual comprehension).
    \item[Physiological effort (Study 5):]Pupillometry revealed a more selective pattern. Video reduced pupil-indexed effort relative to text in only two of four conditions: simple history and complex science. These advantages emerged early and remained sustained throughout sections, suggesting genuine reductions in physiological effort rather than brief integration windows. In the other two cells (complex history and simple science), no video advantage was observed, indicating that format benefits depend on specific alignments between content structure, linguistic demand, and presentation mode.
    \item[Engagement trajectories (Study 5, RQ2):] Most critically, the Modality $\times$ Major interaction on effort trend revealed that video and text produce fundamentally different engagement dynamics for different learners. Science majors showed rising physiological engagement (pupil trend) in video but falling engagement in text across sections. Arts/social sciences majors showed flat-to-falling trends in both modalities (see \autoref{ch:study5}, Figure~\ref{fig:rq2-trend-patterns}). 
\end{description}



This Modality $\times$ Major interaction on effort trends suggests that for STEM-trained learners, video presentation may align with disciplinary schemas that emphasize dynamic, visual, and causal representations, sustaining physiological investment even when the material itself (e.g., history content) falls outside their domain. Text, despite being subjectively preferred overall, may not provide the same schema alignment for these learners, leading to gradual disengagement.

Importantly, Study 3's subjective load ratings did not show a modality $\times$ major interaction. Instead, only main effects and three-way interactions with domain and linguistic complexity were found. This dissociation implies that subjective appraisal (what feels difficult) and physiological engagement (sustained resource allocation) can diverge. Learners may report video as harder overall while simultaneously maintaining deeper physiological engagement when format aligns with their trained cognitive schemas. This dissociation has important instructional implications: what learners report as "easy" may not correspond to what sustains their processing investment.

\section{Implications for Instructional Design and L2 Pedagogy}
This synthesis points to a clear hierarchy for L2 academic reading instruction: (1) manage linguistic complexity first, (2) scaffold according to learner-domain alignment, and (3) choose modality based on disciplinary training. \autoref{ch:integration} integrates these principles with production findings to establish comprehensive design guidance.



\section{Conclusion}
This synthesis establishes three core principles for L2 academic comprehension. First, linguistic complexity acts as the primary processing constraint that can override both expertise and modality advantages. Second, disciplinary expertise changes what is difficult rather than making reading globally easier, with STEM and humanities training shaping linguistic sensitivity differently. Third, subjective difficulty ratings can dissociate from physiological engagement, indicating that format effectiveness must be evaluated through sustained cognitive investment rather than conscious appraisal alone. Chapter 12 integrates these comprehension findings with the production strand to establish a unified account of capacity-limited L2 performance.
