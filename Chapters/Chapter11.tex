

%************************************************
\chapter{Comprehension Synthesis (Studies 3, 4, and 5)}\label{ch:comp-synthesis}
%************************************************

\section{Introduction}
This chapter integrates the comprehension findings from Study 3, Study 4, and Study 5. It provides a single, synthesized account of how L2 learners process academic content under varying cognitive load. Table~\ref{tab:recap-studies345} summarizes the key findings from each study.


\begin{table}[h]
\footnotesize
\centering
\caption{Summary of Studies 3, 4, and 5}
\label{tab:recap-studies345}
\begin{tabular}{p{2.2cm}p{8.5cm}}
\toprule
\textsc{Study} & \textsc{Key Findings} \\
\midrule
Study 3:
Cognitive Load \& Self-Efficacy & 
Harder language and science content both raised mental effort and lowered confidence. A key pattern: STEM majors’ confidence was sensitive to language complexity in science, but Arts and Social Sciences majors were mainly knocked by the science topic itself and showed little sensitivity to language complexity (even in history). This suggests STEM training may cultivate attention to linguistic form. Moment-to-moment load predicted lower next-section confidence. There was some evidence that lower confidence partly carried load effects to performance (strongest in history), though estimates were imprecise. \\
\addlinespace
Study 4: 
Eye-Tracking in Text & 
Harder language increased processing cost: more eye fixations, shorter forward eye movements, and a larger area of gaze. Expertise didn’t make readers globally faster: in easy passages, science majors actually looked more at science texts (deeper engagement), and this advantage disappeared when the language was hard. At the word level, both groups read history words faster than science words. \\
\addlinespace
Study 5:
Pupillometry Across Modalities & 
Video reduced effort only in two cases: simple history and complex science; there was no advantage in the other two. Engagement over time split by major: Science majors ramped up in video but tailed off in text; Arts and Social Sciences were flat-to-falling in both. People’s difficulty ratings and their physiological effort can come apart. \\
\bottomrule
\end{tabular}
\end{table}




The goal is to move from discrete outcomes to a unified process model. Study 3 established that these factors jointly shape subjective cognitive load, self-efficacy, and accuracy. Study 4 (text) and Study 5 (text and video) clarified how and when processing unfolds, using eye-tracking and pupillometry to trace the real-time allocation of visual and cognitive attention. By integrating these subjective, behavioral, and physiological streams, this chapter explains how linguistic demands create processing constraints, how expertise modulates (but does not erase) these constraints, and how modality choice interacts with disciplinary alignment to shape moment-to-moment engagement.


\section{A Joint Account of Processing, Expertise, and Engagement}
\subsection{What Is the Primary Processing Bottleneck?}
The most coherent finding across all three studies is that linguistic complexity acts as the primary processing constraint in most measures. This bottom-up limit sets the cognitive "price" of comprehension, which other factors (expertise, modality) can only modulate.

Study 4 provided the clearest mechanical evidence: complex syntax immediately degraded processing across most measures, forcing all readers into a more localized, cautious, and spatially dispersed reading pattern. Fixation counts rose, forward saccades shortened, and spatial footprints broadened under complex syntax, with these effects largely overriding expertise-based differences. This bottom-up cost directly explains the outcome-level findings in Study 3. The mediation pattern, where high load appeared to diminish self-efficacy's positive influence on performance, aligns with a capacity-sharing account. When linguistic complexity is high, working memory is saturated by basic parsing, leaving limited capacity for the top-down expert strategies observed in Study 4's simple passages or for the metacognitive regulation that self-efficacy typically supports.

\subsection{How Does Expertise Change Processing?}\label{subsec:expertise-modulator}
The synthesis of Studies 3 and 4 refines our understanding of expertise. Disciplinary knowledge is not a simple buffer that makes reading "easier" or "faster." Instead, it changes what is difficult and how a learner allocates attention.

Study 3 showed that expertise changes what is difficult: for non-experts, the domain's conceptual density is the processing constraint; for experts, the constraint shifts to the linguistic realization of familiar concepts. Study 4 showed how experts allocate attention: when linguistic load is manageable (simple text), experts engage more deeply (more fixations), presumably to integrate text with their existing schemas and to fill schema "slots" with text-based propositions. This expert strategy is a luxury that is only possible when linguistic demands are low. As soon as linguistic complexity rises, this expert advantage weakens, and expert patterns converge toward novice-like processing.

Importantly, the asymmetry observed in Study 3's self-efficacy findings suggests that disciplinary training differs in how it shapes linguistic sensitivity. Science majors showed clear self-efficacy benefits from simplified syntax within their domain, whereas Arts and Social Sciences Majors showed no parallel sensitivity to linguistic complexity in history. This may reflect fundamental differences in how STEM versus humanities disciplines socialize students to attend to linguistic form versus narrative content.

\subsection{Do Subjective Difficulty and Physiological Effort Align?}
The joint findings of Studies 3 and 5 reveal a critical dissociation between conscious appraisal and physiological engagement, clarifying that modality effects are neither simple nor uniform.

\begin{description}
    \item[Subjective experience (Study 3):] Participants reported higher cognitive load for video than for text, a main effect that held across conditions. This likely reflects the transience of spoken language combined with the specific orthographic advantage our L1 Chinese participants had with L2 Japanese text (shared kanji/hanzi characters enabling rapid visual comprehension).
    \item[Physiological effort (Study 5):]Pupillometry revealed a more selective pattern. Video reduced pupil-indexed effort relative to text in only two of four conditions: simple history and complex science. These advantages emerged early and remained sustained throughout sections, suggesting genuine reductions in physiological effort rather than brief integration windows. In the other two cells (complex history and simple science), no video advantage was observed, indicating that format benefits depend on specific alignments between content structure, linguistic demand, and presentation mode.
    \item[Engagement trajectories (Study 5, RQ2):] Most critically, the Modality $\times$ Major interaction on effort trend revealed that video and text produce fundamentally different engagement dynamics for different learners. Science majors showed rising physiological engagement (pupil trend) in video but falling engagement in text across sections. Arts and Social Sciences Majors showed flat-to-falling trends in both modalities (see \autoref{ch:study5}, Figure~\ref{fig:rq2-trend-patterns}). 
\end{description}



This Modality $\times$ Major interaction on effort trends suggests that for STEM-trained learners, video presentation may align with disciplinary schemas that emphasize dynamic, visual, and causal representations, sustaining physiological investment even when the material itself (e.g., history content) falls outside their domain. Text, despite being subjectively preferred overall, may not provide the same schema alignment for these learners, leading to gradual disengagement.

Importantly, Study 3's subjective load ratings did not show a modality $\times$ major interaction. Instead, only main effects and three-way interactions with domain and linguistic complexity were found. This dissociation implies that subjective appraisal (what feels difficult) and physiological engagement (sustained resource allocation) can diverge. Learners may report video as harder overall while simultaneously maintaining deeper physiological engagement when format aligns with their trained cognitive schemas. This dissociation has important instructional implications: what learners report as "easy" may not correspond to what sustains their processing investment.

\section{Implications for Instructional Design and L2 Pedagogy}


The synthesis of the comprehension experiments (Studies 3–5) points to a clear hierarchy for designing L2 academic reading instruction:

\begin{description}
    \item[Manage linguistic complexity first:]  This is the primary processing bottleneck and can override other advantages if not controlled.
    \item[Scaffold by learner–domain alignment: ] Scaffolding must be differentiated. Out-of-domain learners need more conceptual support, whereas in-domain learners benefit more from targeted linguistic support.
    \item[Choose modality by disciplinary training:] Modality choice is not universal. The data suggest that video may align better with STEM-trained learners, whereas text aligns better with humanities-trained learners.
\end{description}




These principles, derived from the comprehension strand, are combined with the production findings in \autoref{sec:practical} to establish a comprehensive design framework.

\section{Conclusion}
This synthesis establishes three core principles for L2 academic comprehension. First, linguistic complexity acts as the primary processing constraint that can override both expertise and modality advantages. Second, disciplinary expertise changes what is difficult rather than making reading globally easier, with STEM and humanities training shaping linguistic sensitivity differently. Third, subjective difficulty ratings can dissociate from physiological engagement, indicating that format effectiveness must be evaluated through sustained cognitive investment rather than conscious appraisal alone. \autoref{ch:integration} integrates these comprehension findings with the production strand to establish a unified account of capacity-limited L2 performance.
