%************************************************
\chapter{Experiment 2 Comprehension Questions}

{
\setlength{\parindent}{0pt}
%************************************************
\section{History 1}

\subsection{Section 1}

\textit{Question 1} (Recall)\\
\begin{CJK}{UTF8}{ipxm}
ナポレオン戦争が終わった後、アイルランドではどのような社会問題が発生しましたか?
\end{CJK}
\begin{enumerate}[label=\alph*)]
    \item \begin{CJK}{UTF8}{ipxm}失業者の増加\end{CJK} \textit{(correct)}
    \item \begin{CJK}{UTF8}{ipxm}飢饉の発生\end{CJK}
    \item \begin{CJK}{UTF8}{ipxm}人口の大量流出\end{CJK}
    \item \begin{CJK}{UTF8}{ipxm}ジャガイモの不作\end{CJK}
\end{enumerate}

\textit{Question 2} (Comprehension)\\
\begin{CJK}{UTF8}{ipxm}
1800年、イギリスがアイルランドを統合したことはアイルランドにどのような影響を与えましたか?
\end{CJK}
\begin{enumerate}[label=\alph*)]
    \item \begin{CJK}{UTF8}{ipxm}アイルランドの政治的自主性が制限された\end{CJK} \textit{(correct)}
    \item \begin{CJK}{UTF8}{ipxm}イギリス本土への移住が強制された\end{CJK}
    \item \begin{CJK}{UTF8}{ipxm}アイルランドの経済が衰退した\end{CJK}
    \item \begin{CJK}{UTF8}{ipxm}アイルランドの文化の抑圧が強化された\end{CJK}
\end{enumerate}

\textit{Question 3} (Reasoning)\\
\begin{CJK}{UTF8}{ipxm}
ナポレオン戦争終結後のアイルランドで、産業の衰退と失業者の増加が同時に起きた主な原因は何だと推測できますか?
\end{CJK}
\begin{enumerate}[label=\alph*)]
    \item \begin{CJK}{UTF8}{ipxm}戦争終結に伴う軍需産業の縮小\end{CJK} \textit{(correct)}
    \item \begin{CJK}{UTF8}{ipxm}イギリスによる意図的なアイルランド産業の抑制政策\end{CJK}
    \item \begin{CJK}{UTF8}{ipxm}アイルランド人の勤勉さの欠如と教育水準の低さ\end{CJK}
    \item \begin{CJK}{UTF8}{ipxm}産業革命の影響による手工業の衰退\end{CJK}
\end{enumerate}

\subsection{Section 2}

\textit{Question 4} (Recall)\\
\begin{CJK}{UTF8}{ipxm}
疫病の発生により、ジャガイモはどのような状態になりましたか?
\end{CJK}
\begin{enumerate}[label=\alph*)]
    \item \begin{CJK}{UTF8}{ipxm}葉が黄色くなり枯れた\end{CJK}
    \item \begin{CJK}{UTF8}{ipxm}茎が異常に伸びなかった\end{CJK}
    \item \begin{CJK}{UTF8}{ipxm}イモが腐敗した\end{CJK} \textit{(correct)}
    \item \begin{CJK}{UTF8}{ipxm}小さなイモしかできなかった\end{CJK}
\end{enumerate}

\textit{Question 5} (Comprehension)\\
\begin{CJK}{UTF8}{ipxm}
政府が食料の無料配給を認めなかったことから、何が読み取れますか?
\end{CJK}
\begin{enumerate}[label=\alph*)]
    \item \begin{CJK}{UTF8}{ipxm}政府は自由市場経済の原則を優先した\end{CJK} \textit{(correct)}
    \item \begin{CJK}{UTF8}{ipxm}政府は飢饉の深刻さを理解していなかった\end{CJK}
    \item \begin{CJK}{UTF8}{ipxm}政府は地主の利益を優先した\end{CJK}
    \item \begin{CJK}{UTF8}{ipxm}政府は外国からの圧力を受けていた\end{CJK}
\end{enumerate}

\textit{Question 6} (Reasoning)\\
\begin{CJK}{UTF8}{ipxm}
本文の情報から、疫病の発生が農民たちに与えた影響として、最も直接的に推測できるのはどれですか?
\end{CJK}
\begin{enumerate}[label=\alph*)]
    \item \begin{CJK}{UTF8}{ipxm}翌年の種イモを確保することが困難になった\end{CJK}
    \item \begin{CJK}{UTF8}{ipxm}8月に収穫したジャガイモの販売ができなくなった\end{CJK}
    \item \begin{CJK}{UTF8}{ipxm}翌年の春まで食べる予定だった食糧の大部分を失った\end{CJK} \textit{(correct)}
    \item \begin{CJK}{UTF8}{ipxm}家畜の飼料が不足し、畜産業に打撃を与えた\end{CJK}
\end{enumerate}

\subsection{Section 3}

\textit{Question 7} (Recall)\\
\begin{CJK}{UTF8}{ipxm}
1846年春に発生した食糧をめぐる出来事にはどのようなものがありましたか?
\end{CJK}
\begin{enumerate}[label=\alph*)]
    \item \begin{CJK}{UTF8}{ipxm}農民が食糧を奪う事件が発生した\end{CJK} \textit{(correct)}
    \item \begin{CJK}{UTF8}{ipxm}地主が食糧を独占し、高値で売りつけた\end{CJK}
    \item \begin{CJK}{UTF8}{ipxm}政府が食糧を無料で配給した\end{CJK}
    \item \begin{CJK}{UTF8}{ipxm}農民が自発的に食糧を分け合った\end{CJK}
\end{enumerate}

\textit{Question 8} (Comprehension)\\
\begin{CJK}{UTF8}{ipxm}
アイルランドのジャガイモ大飢饉が「人為的な災害」と考えられる理由はどれですか?
\end{CJK}
\begin{enumerate}[label=\alph*)]
    \item \begin{CJK}{UTF8}{ipxm}社会の混乱により多くの農地で食糧生産ができなくなったから\end{CJK}
    \item \begin{CJK}{UTF8}{ipxm}イギリスの地主が意図的にジャガイモに胴枯病を感染させたから\end{CJK}
    \item \begin{CJK}{UTF8}{ipxm}十分な食糧があったにもかかわらず、適切に分配されなかったから\end{CJK} \textit{(correct)}
    \item \begin{CJK}{UTF8}{ipxm}連合王国内での地域間の経済格差によって食糧が買えなくなったから\end{CJK}
\end{enumerate}

\textit{Question 9} (Reasoning)\\
\begin{CJK}{UTF8}{ipxm}
飢饉時にアイルランドの貧しい人々に食糧が届かなかった主な理由として、最も説得力があるのはどれですか?
\end{CJK}
\begin{enumerate}[label=\alph*)]
    \item \begin{CJK}{UTF8}{ipxm}食糧不足で市場に購入できる穀物がほとんどなかったから\end{CJK}
    \item \begin{CJK}{UTF8}{ipxm}貧しい人々に食糧を購入する経済力がなかったから\end{CJK} \textit{(correct)}
    \item \begin{CJK}{UTF8}{ipxm}イギリス政府がアイルランドへの食糧輸送を全面的に禁止したから\end{CJK}
    \item \begin{CJK}{UTF8}{ipxm}疫病の流行により、食糧の生産と流通に携わる労働力が失われたから\end{CJK}
\end{enumerate}

%==============================================================================

\section{History 2}

\subsection{Section 1}

\textit{Question 1} (Recall)\\
\begin{CJK}{UTF8}{ipxm}
ベレンコはどの空港に着陸したのか?
\end{CJK}
\begin{enumerate}[label=\alph*)]
    \item \begin{CJK}{UTF8}{ipxm}千歳空港\end{CJK}
    \item \begin{CJK}{UTF8}{ipxm}函館空港\end{CJK} \textit{(correct)}
    \item \begin{CJK}{UTF8}{ipxm}北海道空港\end{CJK}
    \item \begin{CJK}{UTF8}{ipxm}チュグエフカ空港\end{CJK}
\end{enumerate}

\textit{Question 2} (Comprehension)\\
\begin{CJK}{UTF8}{ipxm}
なぜ航空自衛隊はミグ25を発見できなかったのでしょうか?
\end{CJK}
\begin{enumerate}[label=\alph*)]
    \item \begin{CJK}{UTF8}{ipxm}ミグ25が非常に高速で飛行していたため\end{CJK}
    \item \begin{CJK}{UTF8}{ipxm}ミグ25が超低空で飛行していたため\end{CJK} \textit{(correct)}
    \item \begin{CJK}{UTF8}{ipxm}日本のレーダーシステムが故障していたため\end{CJK}
    \item \begin{CJK}{UTF8}{ipxm}航空自衛隊が誤った地点を捜していたため\end{CJK}
\end{enumerate}

\textit{Question 3} (Reasoning)\\
\begin{CJK}{UTF8}{ipxm}
ベレンコ中尉がミグ25で日本の空港の上空を旋回していた時に、彼が最も懸念していたと考えられる要素は何でしょうか?
\end{CJK}
\begin{enumerate}[label=\alph*)]
    \item \begin{CJK}{UTF8}{ipxm}着陸時に機体を損傷させ、飛行機の価値を失うこと\end{CJK}
    \item \begin{CJK}{UTF8}{ipxm}撃墜されて民間人に被害を与えること\end{CJK}
    \item \begin{CJK}{UTF8}{ipxm}燃料切れで不時着すること\end{CJK} \textit{(correct)}
    \item \begin{CJK}{UTF8}{ipxm}着陸許可が得られないこと\end{CJK}
\end{enumerate}

\subsection{Section 2}

\textit{Question 4} (Recall)\\
\begin{CJK}{UTF8}{ipxm}
ベレンコ氏の亡命後の安全確保のための方法はなんでしたか?
\end{CJK}
\begin{enumerate}[label=\alph*)]
    \item \begin{CJK}{UTF8}{ipxm}日本政府が公式に保護プログラムを提供した\end{CJK}
    \item \begin{CJK}{UTF8}{ipxm}ベレンコ氏は頻繁に名前と住所を変更した\end{CJK} \textit{(correct)}
    \item \begin{CJK}{UTF8}{ipxm}ソ連政府が亡命者の安全を保証した\end{CJK}
    \item \begin{CJK}{UTF8}{ipxm}ベレンコ氏は公の場に一切現れなくなった\end{CJK}
\end{enumerate}

\textit{Question 5} (Comprehension)\\
\begin{CJK}{UTF8}{ipxm}
1983年の韓国間機撃墜事件の分析にベレンコ氏が協力を求められた理由はどれですか?
\end{CJK}
\begin{enumerate}[label=\alph*)]
    \item \begin{CJK}{UTF8}{ipxm}亡命時に持ち出したミグ25の技術情報が、事件の解析に直接関係していたから\end{CJK}
    \item \begin{CJK}{UTF8}{ipxm}ソ連の通信に関する内部知識を持っていたから\end{CJK} \textit{(correct)}
    \item \begin{CJK}{UTF8}{ipxm}韓国の航空システムに詳しかったから\end{CJK}
    \item \begin{CJK}{UTF8}{ipxm}ソ連のプロパガンダ戦術に詳しかったから\end{CJK}
\end{enumerate}

\textit{Question 6} (Reasoning)\\
\begin{CJK}{UTF8}{ipxm}
ベレンコ氏の亡命後の人生の展開から、冷戦期のアメリカの情報戦略について、最も妥当な推論はどれですか?
\end{CJK}
\begin{enumerate}[label=\alph*)]
    \item \begin{CJK}{UTF8}{ipxm}アメリカは亡命者がソ連に情報を流すことを警戒した\end{CJK}
    \item \begin{CJK}{UTF8}{ipxm}アメリカは亡命者を公の場に頻繁に登場させ、プロパガンダ活動に利用した\end{CJK}
    \item \begin{CJK}{UTF8}{ipxm}アメリカは亡命者の専門知識を活用する戦略をとった\end{CJK} \textit{(correct)}
    \item \begin{CJK}{UTF8}{ipxm}アメリカは亡命者の受け入れを控え、国際的な緊張緩和を優先した\end{CJK}
\end{enumerate}

\subsection{Section 3}

\textit{Question 7} (Recall)\\
\begin{CJK}{UTF8}{ipxm}
事件の数年後、日本はどのような装備を導入したか?
\end{CJK}
\begin{enumerate}[label=\alph*)]
    \item \begin{CJK}{UTF8}{ipxm}早期警戒機\end{CJK} \textit{(correct)}
    \item \begin{CJK}{UTF8}{ipxm}レーダーシステム\end{CJK}
    \item \begin{CJK}{UTF8}{ipxm}対空ミサイル\end{CJK}
    \item \begin{CJK}{UTF8}{ipxm}アメカの戦闘機\end{CJK}
\end{enumerate}

\textit{Question 8} (Comprehension)\\
\begin{CJK}{UTF8}{ipxm}
ミグ25の実際に優れていた性能は何だったと理解できますか?
\end{CJK}
\begin{enumerate}[label=\alph*)]
    \item \begin{CJK}{UTF8}{ipxm}高度な攻撃能力\end{CJK}
    \item \begin{CJK}{UTF8}{ipxm}軽量な機体構造\end{CJK}
    \item \begin{CJK}{UTF8}{ipxm}優れた操縦性能\end{CJK}
    \item \begin{CJK}{UTF8}{ipxm}高速飛行\end{CJK} \textit{(correct)}
\end{enumerate}

\textit{Question 9} (Reasoning)\\
\begin{CJK}{UTF8}{ipxm}
この事件の影響として正しいのはどれですか?
\end{CJK}
\begin{enumerate}[label=\alph*)]
    \item \begin{CJK}{UTF8}{ipxm}アメリカの航空技術の方向性の変化\end{CJK} \textit{(correct)}
    \item \begin{CJK}{UTF8}{ipxm}日ソ対立の激化\end{CJK}
    \item \begin{CJK}{UTF8}{ipxm}日本の防衛体制の脆弱化\end{CJK}
    \item \begin{CJK}{UTF8}{ipxm}米ソの間の軍拡競争の激化\end{CJK}
\end{enumerate}

%==============================================================================

\section{Science 1}

\subsection{Section 1}

\textit{Question 1} (Recall)\\
\begin{CJK}{UTF8}{ipxm}
共通鍵暗号は通信する二者間で何個の鍵を共有しますか?
\end{CJK}
\begin{enumerate}[label=\alph*)]
    \item \begin{CJK}{UTF8}{ipxm}1個\end{CJK} \textit{(correct)}
    \item \begin{CJK}{UTF8}{ipxm}2個\end{CJK}
    \item \begin{CJK}{UTF8}{ipxm}3個\end{CJK}
    \item \begin{CJK}{UTF8}{ipxm}4個\end{CJK}
\end{enumerate}

\textit{Question 2} (Comprehension)\\
\begin{CJK}{UTF8}{ipxm}
公開鍵暗号が非対称鍵暗号と呼ばれる理由として、最も適切な説明はどれですか?
\end{CJK}
\begin{enumerate}[label=\alph*)]
    \item \begin{CJK}{UTF8}{ipxm}暗号化と復号化の処理速度が異なるため\end{CJK}
    \item \begin{CJK}{UTF8}{ipxm}暗号化と復号化に使用する鍵が異なるため\end{CJK} \textit{(correct)}
    \item \begin{CJK}{UTF8}{ipxm}送信者と受信者で異なる暗号化アルゴリズムを使用するため\end{CJK}
    \item \begin{CJK}{UTF8}{ipxm}暗号文と平文の長さが異なるため\end{CJK}
\end{enumerate}

\textit{Question 3} (Reasoning)\\
\begin{CJK}{UTF8}{ipxm}
共通鍵暗号が軍事目的で効果的に使用できた理由として、最も適切なものは何でしょうか?
\end{CJK}
\begin{enumerate}[label=\alph*)]
    \item \begin{CJK}{UTF8}{ipxm}軍事用の暗号化アルゴリズムが特に強力だったから\end{CJK}
    \item \begin{CJK}{UTF8}{ipxm}軍事通信では常に同じ鍵を使い続けることができたから\end{CJK}
    \item \begin{CJK}{UTF8}{ipxm}通信相手が限定されており、事前に鍵を配布できたから\end{CJK} \textit{(correct)}
    \item \begin{CJK}{UTF8}{ipxm}軍事組織では暗号解読の専門家が常に待機していたから\end{CJK}
\end{enumerate}

\subsection{Section 2}

\textit{Question 4} (Recall)\\
\begin{CJK}{UTF8}{ipxm}
ネットワーク社会での共通鍵暗号の課題は何ですか?
\end{CJK}
\begin{enumerate}[label=\alph*)]
    \item \begin{CJK}{UTF8}{ipxm}鍵の生成が難しい\end{CJK}
    \item \begin{CJK}{UTF8}{ipxm}鍵の保管が難しい\end{CJK}
    \item \begin{CJK}{UTF8}{ipxm}鍵の共有が困難\end{CJK} \textit{(correct)}
    \item \begin{CJK}{UTF8}{ipxm}鍵の更新が頻繁に必要\end{CJK}
\end{enumerate}

\textit{Question 5} (Comprehension)\\
\begin{CJK}{UTF8}{ipxm}
公開鍵暗号方式を用いてインターネット上で暗号通信を行う場合、メッセージを暗号化するために使用するのは何ですか?
\end{CJK}
\begin{enumerate}[label=\alph*)]
    \item \begin{CJK}{UTF8}{ipxm}送る側の秘密鍵\end{CJK}
    \item \begin{CJK}{UTF8}{ipxm}受け取る側の秘密鍵\end{CJK}
    \item \begin{CJK}{UTF8}{ipxm}送る側の公開鍵\end{CJK}
    \item \begin{CJK}{UTF8}{ipxm}受け取る側の公開鍵\end{CJK} \textit{(correct)}
\end{enumerate}

\textit{Question 6} (Reasoning)\\
\begin{CJK}{UTF8}{ipxm}
公開鍵暗号がインターネットにふさわしい暗号とされる理由として、最も妥当なものはどれでしょうか?
\end{CJK}
\begin{enumerate}[label=\alph*)]
    \item \begin{CJK}{UTF8}{ipxm}公開鍵暗号の方が暗号化の速度が速いから\end{CJK}
    \item \begin{CJK}{UTF8}{ipxm}公開鍵暗号の方がハッキングに対して安全だから\end{CJK}
    \item \begin{CJK}{UTF8}{ipxm}公開鍵暗号の方が多数の相手と容易に暗号通信ができるから\end{CJK} \textit{(correct)}
    \item \begin{CJK}{UTF8}{ipxm}公開鍵暗号の方が計算量が少なく、効率的だから\end{CJK}
\end{enumerate}

\subsection{Section 3}

\textit{Question 7} (Recall)\\
\begin{CJK}{UTF8}{ipxm}
共通鍵暗号と公開鍵暗号を比較した場合、どちらの計算量が多いですか?
\end{CJK}
\begin{enumerate}[label=\alph*)]
    \item \begin{CJK}{UTF8}{ipxm}共通鍵暗号のほうが多い\end{CJK}
    \item \begin{CJK}{UTF8}{ipxm}公開鍵暗号のほうが多い\end{CJK} \textit{(correct)}
    \item \begin{CJK}{UTF8}{ipxm}両者は比較できない\end{CJK}
    \item \begin{CJK}{UTF8}{ipxm}計算量はほぼ同じ\end{CJK}
\end{enumerate}

\textit{Question 8} (Comprehension)\\
\begin{CJK}{UTF8}{ipxm}
大容量のデータを暗号化する際、公開鍵暗号はどのような役割を果たしますか?
\end{CJK}
\begin{enumerate}[label=\alph*)]
    \item \begin{CJK}{UTF8}{ipxm}鍵の暗号化\end{CJK} \textit{(correct)}
    \item \begin{CJK}{UTF8}{ipxm}データの暗号化\end{CJK}
    \item \begin{CJK}{UTF8}{ipxm}鍵の生成\end{CJK}
    \item \begin{CJK}{UTF8}{ipxm}データの復号\end{CJK}
\end{enumerate}

\textit{Question 9} (Reasoning)\\
\begin{CJK}{UTF8}{ipxm}
本文の内容から、公開鍵暗号と共通鍵暗号を組み合わせて使用する利点について、どのような推測ができますか?
\end{CJK}
\begin{enumerate}[label=\alph*)]
    \item \begin{CJK}{UTF8}{ipxm}暗号化の効率性と安全性を両立できる\end{CJK} \textit{(correct)}
    \item \begin{CJK}{UTF8}{ipxm}暗号化のコストが低くなる\end{CJK}
    \item \begin{CJK}{UTF8}{ipxm}暗号化と復号化の手順が簡略化される\end{CJK}
    \item \begin{CJK}{UTF8}{ipxm}暗号化されたデータの保存容量が大幅に削減される\end{CJK}
\end{enumerate}

%==============================================================================

\section{Science 2}

\subsection{Section 1}

\textit{Question 1} (Recall)\\
\begin{CJK}{UTF8}{ipxm}
Netflix社が匿名化して公開したデータは何に関するものでしたか?
\end{CJK}
\begin{enumerate}[label=\alph*)]
    \item \begin{CJK}{UTF8}{ipxm}映画のレビュー\end{CJK}
    \item \begin{CJK}{UTF8}{ipxm}映画チケットの購入履歴\end{CJK}
    \item \begin{CJK}{UTF8}{ipxm}DVDレンタルの履歴\end{CJK} \textit{(correct)}
    \item \begin{CJK}{UTF8}{ipxm}支払い情報\end{CJK}
\end{enumerate}

\textit{Question 2} (Comprehension)\\
\begin{CJK}{UTF8}{ipxm}
Netflixのコンテストで参加者はどのような方法で匿名データから個人を特定することに成功しましたか?
\end{CJK}
\begin{enumerate}[label=\alph*)]
    \item \begin{CJK}{UTF8}{ipxm}映画チケットの購入履歴から居住地をデータマイニングした\end{CJK}
    \item \begin{CJK}{UTF8}{ipxm}DVDのレンタル履歴からユーザー行動パターンを分析した\end{CJK}
    \item \begin{CJK}{UTF8}{ipxm}映画の評価点数のパターンを解析した\end{CJK}
    \item \begin{CJK}{UTF8}{ipxm}DVDレンタル順とレビュー投稿順の一致を見つけた\end{CJK} \textit{(correct)}
\end{enumerate}

\textit{Question 3} (Reasoning)\\
\begin{CJK}{UTF8}{ipxm}
Netflix社のコンテストで公開された匿名化データから個人が特定された事例が示唆する最も重要なことはどれでしょうか?
\end{CJK}
\begin{enumerate}[label=\alph*)]
    \item \begin{CJK}{UTF8}{ipxm}時系列データは公開すべきではない\end{CJK}
    \item \begin{CJK}{UTF8}{ipxm}映画レビューサイトの情報は匿名化すべきである\end{CJK}
    \item \begin{CJK}{UTF8}{ipxm}公開データの再識別化リスクを考慮すべきである\end{CJK} \textit{(correct)}
    \item \begin{CJK}{UTF8}{ipxm}データマイニング技術の使用を法的に制限すべきである\end{CJK}
\end{enumerate}

\subsection{Section 2}

\textit{Question 4} (Recall)\\
\begin{CJK}{UTF8}{ipxm}
k匿名性において「個人の区別がつかない」状態とは、どのような状態を指しますか?
\end{CJK}
\begin{enumerate}[label=\alph*)]
    \item \begin{CJK}{UTF8}{ipxm}すべての個人情報が完全に削除された\end{CJK}
    \item \begin{CJK}{UTF8}{ipxm}同じ属性値の組み合わせを持つ個人が複数いる\end{CJK} \textit{(correct)}
    \item \begin{CJK}{UTF8}{ipxm}データがすべて暗号化されている\end{CJK}
    \item \begin{CJK}{UTF8}{ipxm}個人を特定できる情報がランダムな値に置き換えられた\end{CJK}
\end{enumerate}

\textit{Question 5} (Comprehension)\\
\begin{CJK}{UTF8}{ipxm}
k-匿名性の特徴として、最も適切なものはどれですか?
\end{CJK}
\begin{enumerate}[label=\alph*)]
    \item \begin{CJK}{UTF8}{ipxm}全ての属性を同じように変える\end{CJK}
    \item \begin{CJK}{UTF8}{ipxm}選んだ属性で結果が変わる\end{CJK} \textit{(correct)}
    \item \begin{CJK}{UTF8}{ipxm}個人を特定できる情報を統合する\end{CJK}
    \item \begin{CJK}{UTF8}{ipxm}データの全ての情報を必ず変える\end{CJK}
\end{enumerate}

\textit{Question 6} (Reasoning)\\
\begin{CJK}{UTF8}{ipxm}
あるデータセットに以下の居住地データが含まれています:日本: 1名、韓国: 1名、タイ: 3名。このデータセットをk=2の基準で匿名化する必要があります。k-匿名性の原則に従って、どのようにこのデータを加工すべきでしょうか?
\end{CJK}
\begin{enumerate}[label=\alph*)]
    \item \begin{CJK}{UTF8}{ipxm}日本と韓国を「東アジア」としてまとめる\end{CJK} \textit{(correct)}
    \item \begin{CJK}{UTF8}{ipxm}日本とタイを「アジア」としてまとめ、韓国データは削除する\end{CJK}
    \item \begin{CJK}{UTF8}{ipxm}韓国とタイを「日本以外」としてまとめる\end{CJK}
    \item \begin{CJK}{UTF8}{ipxm}タイのデータのみを公開する\end{CJK}
\end{enumerate}

\subsection{Section 3}

\textit{Question 7} (Recall)\\
\begin{CJK}{UTF8}{ipxm}
グループyに属する人の「特徴」は何ですか?
\end{CJK}
\begin{enumerate}[label=\alph*)]
    \item \begin{CJK}{UTF8}{ipxm}ECサイトを頻繁に利用する\end{CJK}
    \item \begin{CJK}{UTF8}{ipxm}同じ職業である\end{CJK}
    \item \begin{CJK}{UTF8}{ipxm}同じ商品を購入したことがある\end{CJK} \textit{(correct)}
    \item \begin{CJK}{UTF8}{ipxm}同じ地域に住んでいる\end{CJK}
\end{enumerate}

\textit{Question 8} (Comprehension)\\
\begin{CJK}{UTF8}{ipxm}
なぜ個人xが商品Aを購入したことが明らかになってしまうのでしょうか?
\end{CJK}
\begin{enumerate}[label=\alph*)]
    \item \begin{CJK}{UTF8}{ipxm}グループyのすべてのメンバーが同じ購買パターンを持っているから\end{CJK}
    \item \begin{CJK}{UTF8}{ipxm}公開された購買履歴データが外部情報源と照合されたから\end{CJK} \textit{(correct)}
    \item \begin{CJK}{UTF8}{ipxm}kの値が小さいため、k-匿名性の手法が適切に適用されてなかったから\end{CJK}
    \item \begin{CJK}{UTF8}{ipxm}データセットに含まれる他の属性から商品Aの購入が推測できたから\end{CJK}
\end{enumerate}

\textit{Question 9} (Reasoning)\\
\begin{CJK}{UTF8}{ipxm}
グループyに属する他の人のプライバシーについて、最も適切な説明はどれですか?
\end{CJK}
\begin{enumerate}[label=\alph*)]
    \item \begin{CJK}{UTF8}{ipxm}購買履歴が完全に漏洩する\end{CJK}
    \item \begin{CJK}{UTF8}{ipxm}グループyに属することが判明する\end{CJK}
    \item \begin{CJK}{UTF8}{ipxm}全く影響を受けない\end{CJK} \textit{(correct)}
    \item \begin{CJK}{UTF8}{ipxm}購買傾向が判明する\end{CJK}
\end{enumerate}



}