
\section{Consequences of Cognitive Load}\label{sec:consequences}

Cognitive load manifests in observable behavioral patterns. The following subsections review empirical signatures of increased processing demand in production (Section~\ref{subsec:prod-sigs}) and comprehension (Section~\ref{subsec:comp-sigs}), connecting theoretical constructs to measurable indices.

\subsection{Production Signatures}\label{subsec:prod-sigs}
\subsubsection{Observable Speaking Signatures: Timing, Prosody, Repairs}

Under high task or presentation demands, speakers reallocate limited resources among competing performance dimensions. They may shift resources away from syntactic elaboration or error monitoring to maintain fluency, or conversely, they may slow their speech rate to preserve accuracy—producing the systematic trade-offs documented in L2 research \parencite{housen2009}. The balance among CAF shifts systematically depending on task demands and learner priorities, with continuity often receiving priority because maintaining communicative flow is essential for successful interaction. Explicit guidance that reduces avoidable coordination demands should yield steadier fluency and more reliable self-monitoring \parencite{Sweller2017TESL}. Experimental evidence suggests that task conditions nudge this balance in predictable directions. When online pressure is eased through planning time, clear goals, or materials that reduce last-second problem-solving, learners typically sustain steadier flow and can attempt more elaborated language. These reallocations are the raw material for listener judgments; the same timing choices that help speakers maintain continuity are the cues listeners use to decide whether speech is smooth and easy to follow. When pressure increases, maintaining flow and monitoring form become harder to coordinate, making conservative choices and small breakdowns more likely \parencite{EllisYuan2004}. 

Empirical studies of perceived fluency align with this account: temporal measures such as speech rate, phonation time ratio, and mean length of runs are strong predictors of raters' fluency judgments, indicating that when coordination costs increase, timing variables absorb the impact. These indices capture global timing and rhythm, the steadiness of advance, the length of continuous runs, and the proportion of talking time, all of which align with judgments of smoothness and comprehensibility \parencite{KormosDenes2004}. Converging evidence shows that prosody itself shifts under validated load: in military simulator flights, higher rated cognitive load increased mean F0 by roughly 7--12 Hz, increased intensity by about 1--1.5 dB, and compressed F0 range, indicating that processing pressure is audibly realized in voice as well as in timing \parencite{Huttunen2011AviationProsody}. Monitoring is integral to online speech production; process-proximal indicators such as repetitions, false starts, and reformulations complement global fluency metrics by indexing control costs and time-gaining strategies under load \parencite{albarqi2022}.


\subsubsection{Task Features, Sequencing, and Moderators}


Building on these signatures, task features modulate how attention is redistributed across CAF. Within the resource-directing versus resource-dispersing distinction, increasing resource-directing complexity is often associated with gains in accuracy and complexity, frequently with fluency costs, whereas tightening resource-dispersing constraints tends to depress performance unless support is provided \parencite{Robinson2005}. A limited-attention perspective likewise predicts trade-offs under higher conceptual demands, especially when learners must conceptualize new content while encoding it \parencite{skehan1997}. Recent syntheses report consistent increases in linguistic complexity when tasks include more elements and stronger reasoning demands, more variable effects on accuracy, and typical costs to fluency; consequently, sequencing rather than single-shot contrasts has become a central design focus \parencite{jin2025}.

Sequencing proposals such as SSARC (to stabilize, simplify, automatize, reconstruct, and complexify) make explicit commitments about ordering and expected outcomes over a syllabus \parencite{Robinson2005}. Experimental work on oral performance shows how sequencing interacts with load type. Using a resource-directing manipulation of the number of elements and a resource-dispersing manipulation of planning time, \textcite{chen2025} compared simple-to-complex and complex-to-simple sequences against task repetition. Complex-to-simple advantaged complexity and accuracy; simple-to-complex advantaged fluency. Both sequences exceeded pure repetition for syntactic complexity, and repetition favored lexical complexity and fluency. These patterns are consistent with the idea that conceptual difficulty can be increased to direct attention to form, provided that dispersing constraints are not tightened at the same time \parencite{Robinson2001,Robinson2005}.

Validation cautions remain important. Not all nominally complex tasks impose measurably higher load for a given population. A careful test of an elements manipulation found that only large increases in the number of interacting elements yielded differences on dual-task costs, duration estimates, and mental-effort ratings, with indications that proficiency moderates sensitivity \parencite{sasayama2016}.

Modality and individual differences further condition outcomes. When the same complexity manipulation is administered as speaking and as writing, complexity gains and accuracy costs can appear in both modes, yet baseline trade-offs differ. One study observed that speaking was more accurate but less fluent than writing and that working memory did not predict outcomes once complexity and modality were controlled \parencite{cho2018}. In L2 writing specifically, working memory shows limited direct effects on text-level performance once proficiency is taken into account, and proficiency often emerges as the more reliable predictor across simple and complex versions of the same task \parencite{manchon2023,kuiken2011}.

Classic dispersing supports such as pre-task planning can shift attention to form and buffer fluency costs under higher conceptual load, especially in speaking \parencite{ryu2018}. In practical terms, resource-directing manipulations (e.g., increasing reasoning demands) recruit attention to formulation, typically enhancing structural and lexical elaboration. In contrast, resource-dispersing manipulations (e.g., time pressure) heighten coordination demands, typically depressing temporal fluency \parencite{gilabert2007,albarqi2022}. For example, adding elements or reasoning demands often yields more lexical and syntactic elaboration with fluency costs, while dual-tasking reliably reduces temporal fluency and increases repair behavior \parencite{gilabert2007,albarqi2022}. These signatures define the CAF indices and the listener feature set in Studies~1 and~2.

\subsection{Comprehension Signatures}\label{subsec:comp-sigs}
\subsubsection{Predictive Processing and Language-Specific Constraints}

In comprehension, capacity constraints have well-documented ocular correlates: as task complexity or format-induced coordination increases, first-fixation duration and gaze duration rise on difficult regions, regressions become more frequent, and forward saccades shorten, reflecting effortful local integration rather than fluent parafoveal sampling \parencite{rayner1998}. These patterns follow from element interactivity: formats that inflate avoidable coordination, for example, mapping across separated sources or reconciling poorly aligned cues, raise extraneous demands and manifest in precisely these slowing and backtracking patterns \parencite{sweller2019}. Because these signatures unfold at different timescales, early measures, such as first-fixation duration, are sensitive to recognition and initial lexical access, whereas later measures, such as total fixation time, fixation counts, and regressions, index integration and discourse-level binding. Increases in either set of indices reveal the stage at which coordination costs emerge and whether a manipulation primarily obstructs identification or subsequent unification \parencite{rayner1998}.

Capacity-based accounts clarify why these patterns emerge: when storage and processing demands exceed available resources, progress becomes less efficient and partial results may degrade, which in turn increases reliance on local cues and weakens cross-clause integration \parencite{just1992}. These pressures are commonly amplified in a second language; recognition, meaning selection, and syntactic assembly draw more heavily on attention in the L2, so lines that are routine in the L1 can impose substantial demands in the L2. As a result, differences in current exposure and proficiency modulate efficiency; richer experience lowers the costs of early recognition and later integration, whereas limited experience makes both stages more resource intensive, with slowdowns and rereads more likely \parencite{whitford2012second}.



Understanding how linguistic complexity affects reading requires moving beyond traditional temporal measures to spatial indices that capture how attention is deployed across text. While these temporal measures are foundational, spatial measures reveal how attention is distributed across the text. Recent research shows that these spatial characteristics of eye-movement patterns provide converging evidence for processing difficulty \parencites[e.g.,][]{schad2012zoom,torres2021eye}. Radius of gyration (the root mean square distance of fixations from scanpath centroids) increases with linguistic complexity and correlates more strongly with self-reported effort than mean dwell time \parencite{schad2012zoom,Salvucci2000}. Convex hull area (the polygon encompassing all fixations) correlates with readability measures across different text types \parencite{torres2021eye}. Computational models that incorporate parallel word processing and continuous attentional gradients reproduce these patterns by modeling how processing difficulty broadens attentional gradients and suppresses forward saccade generation \parencite{engbert2002dynamical,engbert2005swift,schad2012zoom}. Including spatial dispersion alongside timing measures therefore strengthens inferences about complexity effects.



A growing body of work also points to discourse-level complexity: texts that lack cohesive devices or exhibit unpredictable topic shifts further inflate fixation durations, and a recent corpus analysis showed that a combined index of lexical rarity, syntactic embedding, and cohesion accounts for roughly one third of variance in L2 eye-movement measures, outperforming traditional readability formulas \parencite{zhang2024modelling}. These effects generalize across typologically diverse languages; large cross-linguistic corpora confirm that dependency-distance penalties apply in head-initial, head-final, and mixed orders alike \parencite{Futrell2020}. Electrophysiological studies show that processing long-distance dependencies is associated with a P600 component, indicating increased syntactic integration costs \parencite{fiebach2002wh}.




\subsubsection{Anticipatory Processing in the Visual World Paradigm}

In the visual world paradigm, listeners use verb semantics and real-world constraints to anticipate upcoming referents, often launching saccades to a target object before it is mentioned; the classic finding that \textit{eat} triggers anticipatory looks to an edible item exemplifies incremental prediction under minimal presentation costs \parencite{altmann1999incremental}. Reviews emphasize that fixations reflect integration of linguistic and visual information in real time, making the visual world paradigm well-suited for testing how load and context shape prediction during comprehension \parencite{huettig2011using}. For late L2 learners, anticipatory behavior can be reduced or delayed relative to native speakers, a difference that arises from frequency biases, competition, and representation quality rather than from a distinct processing architecture \parencite{kaan2014predictive}.

At the predictive processing level, visual world paradigms reveal that L2 learners often fail to use morphosyntactic cues predictively. For instance, one study showed that while native Japanese listeners used a nominative-dative sequence to launch anticipatory eye movements to a theme object before it was mentioned, English-speaking learners of Japanese did not \parencite{mitsugi2016use}. This suggests that the L2 learners did not use case-marker information to generate predictions about upcoming verb arguments. Building on this, \textcite{mitsugi2017incremental} showed that this deficit extends to voice prediction; in a similar visual world study, native listeners exploited the same case cues to anticipate a passive verb, whereas English-speaking learners did not shift their gaze until the voice morphology surfaced. Flexible word order creates additional processing vulnerabilities. Japanese permits object fronting and scrambling, which amplifies dependency distances and forces readers to maintain displaced constituents in working memory. In native speakers, eye tracking reveals disproportionate integration costs at clause-final verbs and increased regressions to case particles when non-canonical orders increase \parencite{ueno2008relative,tamaoka2019eye}. Linguistically complex passages containing multiple embedded structures therefore predict both heightened particle rereading and more dispersed fixation patterns.

\subsubsection{Physiological Indices: Pupillometry and Listening Effort}

Listening research converges on the same conclusion through independent indices of effort. Pupil dilation scales with sentence intelligibility, increasing when the acoustic or linguistic signal is degraded, which indicates greater allocation of limited resources to maintain comprehension \parencite{zekveld2010pupil}. Broader frameworks of effortful listening formalize this relation as a balance between task demands, available capacity, and motivation, and they explain why listeners can report fatigue even when accuracy is maintained \parencite{pichora2016hearing}. Reviews of speech recognition in adverse conditions catalogue how noise, competing talkers, and cognitive load at the receiver jointly compromise segmentation and lexical access; they also motivate design choices such as pacing and redundancy control that reduce extraneous demand \parencite{mattys2012speech}. These online measures and models provide the needed bridge to the present thesis, allowing manipulations to be linked to specific stages of comprehension by predicting longer early measures when decoding is strained and longer late measures when integration fails, and by tying increases in pupil size to the extra effort required to sustain understanding under load \parencite{rayner1998,zekveld2010pupil}. 

However, this relationship is not always linear. Comparable findings have been reported in listening and cognitive-effort studies where pupil size stayed relatively high for tasks that were well within participants’ processing capacity, indicating that larger mean pupil diameter does not always mean the nominal task was more complex \parencite{zekveld2018pupil}. At the same time, pupillometry work on effort regulation shows an inverted U-shaped response: dilation increases with task difficulty up to the point where the task is still manageable, and then declines once the task becomes effectively unmanageable or not worth the effort \parencite{herrmann2024pupil}. This drop at the “too hard” end has also been described in highly demanding or saturating visual/auditory tasks, where reduced dilation is interpreted as disengagement or overload rather than low demand \parencite{minassian2004pupillary}. In that framework, a lower mean pupil diameter in a complex condition can be read as an under-allocation of resources triggered by overload. This suggests learners may stop attempting full integration even though the nominal demand was higher, rather than providing evidence that the complex passages were easier \parencite{vanderwel2018pupil}.

