%************************************************
\chapter{Study 4: Attentional Allocation in Text Reading}\label{ch:study4}
%************************************************

\section{Introduction}


Study 3 established that linguistic complexity, domain, and modality jointly shape cognitive load, self-efficacy, and comprehension accuracy, with complexity effects dominating and expertise producing more selective benefits. However, outcome measures alone cannot reveal the underlying mechanisms—the moment-to-moment reallocation of attention that produces these patterns.

Study 4 addresses this gap by analyzing eye-movements during text-based reading from the same experimental design. Eye-tracking provides millisecond-level traces of visual attention, revealing both the timing and spatial organization of comprehension processes \parencite{rayner1998, Rayner2009}. Critically, this analysis focuses exclusively on the text modality because position-anchored eye-movement measures (e.g., first-fixation duration, regression probability, and spatial dispersion indices) require a stable visual layout where content remains in fixed locations. Video, with its transient display, does not afford these spatial analyses. Study 5 will complement this focus by examining continuous pupillary responses across both modalities.

The central theoretical question is whether disciplinary expertise can buffer the processing costs of linguistic complexity in L2 reading, or whether bottom-up demands overwhelm top-down compensation. To address this, the present study examines three complementary facets of eye-movement behavior. First, we analyze global timing patterns to assess overall reading efficiency (RQ1). Second, we examine word-level processing to test whether expertise facilitates lexical access under varying syntactic demands (RQ2). Third, we quantify the spatial distribution of visual attention to determine how linguistic complexity and expertise shape the footprint of reading (RQ3). Together, these analyses provide a process-level account of the outcome patterns observed in Study 3, clarifying when and how disciplinary knowledge supports L2 academic reading.



\section{Background}

\subsection{Domain Expertise and Disciplinary Literacy}

Disciplinary literacy fundamentally alters how readers approach academic texts, creating domain specific processing strategies that extend beyond general reading skills \parencite{shanahan2008teaching}. Expert readers across disciplines develop specialized attention patterns that reflect the epistemological priorities of their fields. These expertise effects are well documented in cognitive studies of reading. For example, when reading the same historical documents, professional historians read for subtext by treating texts as human and rhetorical artifacts, whereas high-achieving high school students mainly sought to extract information. This expert-novice gap was marked by a key behavior: sourcing, which historians used almost always (approximately 98 percent) and students used less than one-third of the time (approximately 31 percent) \parencite{wineburg1991reading}. Although this comparison involves professionals and adolescents, the underlying epistemic stance shift from information extraction to interrogating subtext and source is a defining feature of disciplinary literacy and is observable among undergraduates in domain-congruent expository reading, the setting most comparable to the materials used in the present study. Scientists, in contrast, focus on causal mechanisms and methodological rigor \parencite{goldman2002functional}.


These expertise effects manifest clearly in eye-tracking research. When instructed to read a text from a specific viewpoint (for example, as a tourist versus a military intelligence analyst), readers fixate longer on and later recall more perspective relevant information than irrelevant information \parencite{kaakinen2005perspective}. Expert readers spontaneously deploy these specialized strategies, demonstrating more focused processing of domain-relevant information while strategically skipping less relevant content, which creates substantial processing advantages such as faster reading speeds, more efficient fixation patterns, and enhanced comprehension for domain-congruent texts \parencite{tothova2025eye}. However, expertise can also create vulnerabilities. Research demonstrates a reverse cohesion effect in which high knowledge readers sometimes benefit from lower textual cohesion because they supply elaborative inferences that deepen understanding \parencite{mcnamara1996are}. Recent work positions disciplinary literacies as distinct constellations of epistemology, inquiry practices, and discourse conventions that vary across fields \parencite{goldman2016}. Large-scale survey data further show at least three separable factors (source, analytic, expressive literacy) each dominating different subject areas \parencite{spires2018}. Together, this work suggests that expertise advantages should be conditional on discourse features and task goals, not uniform across contexts.





\subsection{Linguistic Complexity in L2 Reading: Japanese-Specific Constraints}

As reviewed in \autoref{subsec:comp-sigs}, the well-established eye-movement signatures of linguistic complexity include longer fixations, increased regressions, and wider spatial dispersion as syntactic and lexical demands increase. For L2 readers, these general effects are compounded by language-specific features and reduced automaticity in word recognition and syntactic integration.

The Japanese orthographic and morphosyntactic context presents distinctive challenges. As reviewed in \autoref{subsec:comp-sigs}, L2 learners of Japanese frequently underuse predictive cues such as case markers to anticipate upcoming arguments or voice morphology \parencite{mitsugi2016use,mitsugi2017incremental}. This predictive deficit forces processing toward a reactive mode in which comprehension decisions are delayed until disambiguating information surfaces, typically at clause-final verbs. Flexible word order further amplifies these demands: Japanese permits object fronting and scrambling, which increases dependency distances and requires readers to maintain displaced constituents in working memory \parencite{ueno2008relative,tamaoka2019eye}. Native speakers experience measurable integration costs under such structures, but these effects are substantially amplified in L2 readers, for whom a variety of linguistic complexity features, including lexical rarity, syntactic embedding, and discourse cohesion, contribute additively to processing effort \parencite{nahatame2021text,zhang2024modelling}.

When Japanese-specific complexity is combined with L2 status, processing demands compound in ways that may limit the scope for top-down compensation. The critical question for the present study is whether disciplinary expertise can buffer these bottom-up pressures or whether syntactic integration costs consume finite cognitive resources uniformly, thereby attenuating expert advantages. If expertise provides robust support, domain-congruent readers should maintain more efficient processing even under complex syntax. If integration demands override top-down facilitation, complexity effects should be uniform regardless of domain match.



\subsection{Expertise-Complexity Interactions}
Disciplinary expertise denotes domain-specific literacy practices that guide selection, interpretation, and integration during academic reading. In this study, expertise is operationalized as major and text congruence: science majors reading science texts and arts or social sciences majors reading history texts. Analytically, we test expertise through domain-by-major interactions and simple effects contrasts that compare congruent and incongruent pairings. Two accounts make different predictions for how expertise interacts with linguistic complexity. Capacity-limited views hold that syntactic integration and memory demands draw on shared resources, so as locality increases and constituents must be maintained for longer, resources are reallocated to bottom-up processing and domain-guided strategies are curtailed \parencite{Gibson1998}. Interactive views treat comprehension as jointly shaped by text signals and knowledge, where prior domain knowledge provides expectations about information structure and likely relations among propositions, which can stabilize parsing decisions and support coherence building \parencite{goldman2002functional}. These accounts imply that expertise advantages are conditional, not uniform.

When domain expectations are usable and linguistic demands remain moderate, expert readers should display more selective sampling, longer forward steps, and more economical word-level processing, for example, higher skipping of predictable items \parencite{rayner1998}. As linguistic demands increase, selection becomes uniformly conservative for all readers as resources are absorbed by integration. The Japanese context sharpens these expectations. Because L2 readers often underuse predictive cues like case markers and integration costs accumulate at clause final verbs, processing is pushed toward a reactive mode that limits the headroom in which domain expectations can operate \parencite{mitsugi2016use, tamaoka2019eye}. 


\section{Research Questions}

The reviewed literature thus culminates in a theoretical impasse. On one hand, disciplinary literacy affords expert readers significant processing efficiencies within their domain. On the other, elevated linguistic complexity, compounded by the unique orthographic and syntactic demands of Japanese for L2 learners, imposes substantial cognitive loads that can disrupt fluent comprehension. The critical ambiguity lies at the intersection of these competing forces: are the top-down advantages of expertise robust enough to buffer against bottom-up processing difficulties, or do these difficulties consume finite cognitive resources, thereby neutralizing expert strategies? To adjudicate between these accounts, the present study employs eye-tracking methodology. By capturing millisecond-level data on both the timing (e.g., fixation durations) and spatial distribution (e.g., scan-path geometry) of visual attention, this approach can reveal the real-time dynamics of the expertise-complexity interaction.


\begin{description}
    \item[RQ1:] Does disciplinary expertise (major and text congruence) influence overall reading efficiency, and is this influence moderated by domain and linguistic complexity?

\end{description}

We hypothesize:

\begin{description}
    \item[H1:] Texts congruent with a reader’s disciplinary expertise (major and text congruence) will be read more efficiently than incongruent texts, as indexed by fewer fixations and shorter forward steps; this advantage will be smaller under complex passages than under simple passages.
\end{description}


\begin{description}
    \item[RQ2:] Does disciplinary expertise modulate lexical level processing, and do these effects vary with domain and linguistic complexity?
\end{description}

At the lexical level, we hypothesize:

\begin{description}
    \item[H2:] Expertise-congruent texts will show easier word-level processing than incongruent texts, as indexed by shorter first pass or gaze durations, lower regression probability, and higher skipping probability; this facilitation will be smaller under complex passages.
\end{description}


\begin{description}
    \item[RQ3:] Does disciplinary expertise shape the spatial distribution of visual attention, and are these spatial effects contingent on domain and linguistic complexity?
\end{description}

Regarding the spatial distribution of visual attention, we predict:


\begin{description}
    \item[H3:] Linguistic complexity will enlarge the spatial footprint of reading, as indexed by greater radius of gyration and larger convex hull area; expertise congruence will counteract this by concentrating attention more tightly, with the concentration effect smaller under complex passages.
\end{description}





\section{Method}


\subsection{Data Preprocessing}

Across the remaining participants, 408 trials were processed; 403 trials (98.8\%) met quality control, and 5 trials (1.2\%) were excluded for insufficient data quality (more than 50\% missing samples).

Tokens were segmented with GiNZA \parencite{ginza}. For each token we extracted part-of-speech (POS) tags and lemma forms, then grouped parts of speech into content words (nouns, verbs, adjectives, adverbs) and function words (particles, auxiliary verbs, conjunctions, etc.) . Token frequency (raw and per million words) was obtained from the Balanced Corpus of Contemporary Written Japanese (BCCWJ)\parencite{asada2024NLP}. To map fixations to words, we used a two-step process: a fixation was matched to any token whose bounding box was expanded (by 1.0\% of screen width and 1.7\% of screen height) to account for measurement error, then uniquely assigned to the candidate token with the closest geometric center. This mapping yielded 86.36\% valid text fixations successfully linked to tokens. To align with subsequent analyses, “on text” fixations refer to fixations whose coordinates fell within the predefined text region on the right side of the display.

The following trimming and inclusion rules were applied uniformly across analyses. At the token-level, first pass duration values were log transformed and observations beyond $\pm 2.5$ standard deviations on the log scale were removed, retaining $N=24{,}835$ tokens (34 participants, 3771 unique tokens, 12 sections). At the section-level, spatial metrics used the following rules: radius of gyration ($R_g$) was computed from duration weighted on text fixations; sections with fewer than three fixations were excluded and the top one percent of $R_g$ values was trimmed. Convex hull area (CHA) was computed as the polygon area of the convex hull of unique on text fixation locations; sections with fewer than three unique fixations were excluded and the top one percent of CHA values was trimmed. For mean progressive saccade amplitude, the top one percent of amplitudes was trimmed prior to modeling based on section-level percentiles.


\subsection{Data Analysis}

All analyses were conducted in R (version 4.5.1; \cite{rcoreteam}). We fitted a series of mixed effects models, specifically linear mixed models (LMMs) for Gaussian and log normal specifications using lme4 \parencite{bates2015}, and generalized linear mixed models (GLMMs) for negative binomial, Gamma, binomial, and heteroskedastic specifications using glmmTMB \parencite{mcgillycuddy2025parsimoniously}. Unless otherwise specified, all models included the same experimental fixed effects, namely domain, linguistic complexity, major group, and all two way interactions; we refer to this as the standard fixed effects structure. Omnibus tests used Type III Wald $\chi^2$ tests. Post hoc estimates were obtained with the emmeans package using Tukey adjustment \parencite{emm}. We report odds ratios (OR) for logistic models, incidence rate ratios (IRR) for count models, and ratio of means (ROM) for models with log link or log response, along with percentage change $\Delta=(\text{ratio}-1)\times100\%$.




\subsubsection{RQ1: Global Eye-Movement Patterns}
All section-level dependent variables were modeled with the standard fixed effects structure. Every model included a random intercept for participant. An item intercept was included when supported by the data and removed when it produced singular fits. Model families were chosen to suit the measurement scale and observed distributional shape of each dependent variable. Fixation count used a negative binomial GLMM to accommodate overdispersed count data, and regressive saccade proportion used a binomial GLMM to handle a proportion with varying denominators; both included a random intercept for participant. Mean fixation duration and mean progressive saccade amplitude used Gamma GLMMs with a log link, appropriate for strictly positive and right skewed distributions; both models were weighted by the number of contributing observations (fixations and progressive saccades, respectively) and included random intercepts for participant and item.

\subsubsection{RQ2: POS Effects on Word Processing}
Token-level models included the standard fixed effects structure, nuisance controls for lexical factors, and random intercepts for participant, token, and section. First pass duration used a log normal LMM, since token-level fixation durations are well approximated by a normal distribution after a log transform. These controls included standardized token length and frequency (linear and quadratic terms) and POS group (content versus function). Skipping probability used a binomial GLMM, appropriate for a binary token-level outcome, with the same fixed effects and the same participant, token, and section intercepts.

\subsubsection{RQ3: Spatial Characteristics of Eye-Movements}

Both spatial dispersion measures are strictly positive and right skewed; therefore, analyses used a log scale and accounted for fixation count as a known determinant of dispersion. Radius of gyration (\(R_g\)) used a log normal LMM for \(\log(R_g)\), predicted from \(\log(n_{\text{fix}})\) and the standard fixed effects structure; the model included a random intercept for participant and an item intercept was tested and removed due to zero variance. Convex hull area (CHA) used a heteroskedastic Gaussian GLMM for \(\log(CHA_{\text{px}})\) with a random intercept for participant and a dispersion submodel. The dispersion submodel allowed residual variance to vary with task characteristics and fixation count by including domain by complexity, major group, and \(\log(n_{\text{fix}})\).


Descriptive statistics for all measures are shown in Table~\ref{tab:descriptives}. Omnibus test tables appear in the main text, and complete post hoc results are provided in Supplementary Tables S1–S24.

\begin{table}[!htbp]
\centering
\footnotesize
\caption{Global eye-movement and word-processing descriptives by Domain $\times$ Complexity. Values are means (SD); section-level DVs computed over participant $\times$ section observations; token-level DVs over participant $\times$ token observations.}
\label{tab:descriptives}
\begin{tabular}{lcc}
\toprule
\textsc{Dependent Variable} & \textsc{Simple} & \textsc{Complex} \\
\midrule
\addlinespace
\multicolumn{3}{@{}l@{}}{\textsc{History}} \\
\addlinespace[0.3ex]
\multicolumn{3}{@{}l@{}}{\textit{RQ1 (section-level)}} \\
Fixation count & $146~(70)$ & $261~(94)$ \\
Mean fixation duration (ms) & $351.15~(87.47)$ & $360.94~(36.49)$ \\
Regressive saccades (\%) & $32.78~(11.73)$ & $30.73~(3.84)$ \\
Mean prog. sacc. amp. (px) & $205.76~(67.11)$ & $179.58~(19.71)$ \\
\addlinespace[0.3ex]
\multicolumn{3}{@{}l@{}}{\textit{RQ2 (token-level)}} \\
First pass duration (ms) & $394.57~(183.90)$ & $377.87~(159.68)$ \\
Skipping probability (\%) & $63.58~(48.12)$ & $60.74~(48.83)$ \\
\addlinespace[0.3ex]
\multicolumn{3}{@{}l@{}}{\textit{RQ3 (section-level)}} \\
Radius of gyration (px) & $394.04~(54.17)$ & $426.18~(27.00)$ \\
Convex hull area (px$^2$) & $594895~(181629)$ & $974018~(84283)$ \\
\addlinespace
\midrule
\addlinespace
\multicolumn{3}{@{}l@{}}{\textsc{Science}} \\
\addlinespace[0.3ex]
\multicolumn{3}{@{}l@{}}{\textit{RQ1 (section-level)}} \\
Fixation count & $187~(94)$ & $243~(89)$ \\
Mean fixation duration (ms) & $366.76~(37.79)$ & $367.07~(74.38)$ \\
Regressive saccades (\%) & $31.53~(4.65)$ & $28.64~(7.17)$ \\
Mean prog. sacc. amp. (px) & $177.72~(25.97)$ & $186.33~(70.14)$ \\
\addlinespace[0.3ex]
\multicolumn{3}{@{}l@{}}{\textit{RQ2 (token-level)}} \\
First pass duration (ms) & $393.62~(172.75)$ & $397.85~(189.76)$ \\
Skipping probability (\%) & $54.48~(49.80)$ & $60.69~(48.85)$ \\
\addlinespace[0.3ex]
\multicolumn{3}{@{}l@{}}{\textit{RQ3 (section-level)}} \\
Radius of gyration (px) & $382.24~(22.02)$ & $419.13~(35.58)$ \\
Convex hull area (px$^2$) & $553086~(102295)$ & $922345~(140611)$ \\
\bottomrule
\end{tabular}
\end{table}

\section{Results}
For clarity, domain refers to the text domain (history; science) and major group refers to the participant's academic major (arts or social sciences; science). ``Disciplinary expertise'' is not a separate factor; it is the alignment of these two and is evaluated via the domain $\times$ major group interaction and simple effects that compare congruent and incongruent cells.

\subsection{Model Diagnostics}

All models were evaluated for residual distribution, homoscedasticity, influential observations, convergence behavior, and dispersion using performance and DHARMa \parencite{dharma,performance}. 
For RQ1: the fixation count negative binomial GLMM showed good adequacy (AIC \(=2329.84\); \(R^2_m=0.338\), \(R^2_c=0.704\); uniformity \(p=.123\), dispersion \(p=.056\), zero inflation \(p=1.00\); overdispersion ratio \(=0.598\), \(p=.080\)); an item intercept was removed due to singularity. 
The mean fixation duration Gamma GLMM used section-level weights and included participant and item intercepts; AIC \(=370{,}054.2\); \(R^2_m=0.139\), \(R^2_c=0.894\); uniformity \(p=.667\), dispersion \(p=.140\). 
The regressive saccade proportion binomial GLMM converged with AIC \(=1383.15\); \(R^2_m=.063\), \(R^2_c=.795\); uniformity \(p=.928\), dispersion \(p=.288\), zero inflation \(p=.786\). 
The mean progressive saccade amplitude Gamma GLMM trimmed the top one percent, used section-level weights, and included participant and item intercepts; AIC \(=250{,}451.8\); \(R^2_m=0.115\), \(R^2_c=0.855\); uniformity \(p=.266\), dispersion \(p=.410\).

For RQ2 token-level models: the first pass duration log normal LMM retained \(N=24{,}835\) tokens; AIC \(=24{,}971.9\); \(R^2_m=0.015\), \(R^2_c=0.099\). DHARMa showed the expected large sample deviation from perfect uniformity with no dispersion issue (dispersion \(p=.832\)). The skipping probability binomial GLMM showed AIC \(=82{,}363.8\); \(R^2_m=0.035\), \(R^2_c=0.141\); uniformity \(p=.863\), dispersion \(p=.950\); no observation level random effect was required.

For RQ3 spatial metrics: the radius of gyration log normal LMM trimmed the top one percent, used a participant intercept, and dropped an item intercept due to zero variance; AIC \(=-380.47\); \(R^2_m=0.412\), \(R^2_c=0.739\); uniformity \(p=.817\), dispersion \(p=.932\). The convex hull area heteroskedastic Gaussian GLMM used a participant intercept and a dispersion submodel; AIC \(=-201.86\); uniformity \(p=.662\), dispersion \(p=.114\). Pseudo \(R^2\) is not reported for this heteroskedastic specification. Collectively, these checks support the adequacy of all fitted models.

\subsection{RQ1: Global Eye-Movement Patterns}

\subsubsection{Fixation Count on Text}

The number of fixations on text showed significant main effects of domain, complexity, and major group, alongside trend level interactions for domain $\times$ complexity and domain $\times$ major. As shown in Table \ref{tab:dv1_type3}, fixation counts were higher for science than history passages overall, higher for complex than simple passages, and higher among science than arts majors. Given the trend level interactions, we probed simple effects to characterize these patterns. For Science majors, history texts elicited fewer fixations than science texts (IRR $=0.838$, $\Delta=-16.2\%$, $p=.002$), whereas Arts showed no domain difference (IRR $=0.986$, $\Delta=-1.4\%$, $p=.827$). Both groups made many more fixations for complex than simple passages (Arts IRR $=0.616$, $\Delta=-38.5\%$, $p<.001$; Science IRR $=0.647$, $\Delta=-35.3\%$, $p<.001$). By complexity, domain differences were present for simple text (IRR $=0.739$, $\Delta=-26.2\%$, $p=.018$) but not for complex text (IRR $=1.119$, $\Delta=+11.9\%$, $p=.380$).


\begin{table}[ht]
\centering
\footnotesize
\caption{Type III (Wald) tests for the negative binomial mixed model of fixation counts on text.}
\label{tab:dv1_type3}
\sisetup{
  input-signs = {},
  table-align-text-pre = false,
  table-align-text-post = false
}
\begin{tabular}{l S[table-format=3.3] S[table-format=1.0] c}
\toprule
{Effect} & {$\chi^2$} & {df} & {$p$} \\
\midrule
{Domain} & 4.853 & 1 & {$\phantom{<}.028$} \\
{Complexity} & 112.872 & 1 & {$<.001$} \\
{Major group} & 8.678 & 1 & {$\phantom{<}.003$} \\
{Domain $\times$ Complexity} & 2.963 & 1 & {$\phantom{<}.085$} \\
{Domain $\times$ Major group} & 3.528 & 1 & {$\phantom{<}.060$} \\
{Complexity $\times$ Major group} & 0.341 & 1 & {$\phantom{<}.559$} \\
\bottomrule
\end{tabular}
\end{table}

\subsubsection{Mean Fixation Duration on Text}
Mean fixation duration on text was not directly influenced by a main effect of complexity, but was instead shaped by main effects of domain and major group, as well as significant domain $\times$ major and complexity $\times$ major interactions. As shown in Table \ref{tab:dv2_type3}, on the response scale, within Arts, science passages elicited longer fixations than history (ROM $=0.966$, $\Delta=-3.40\%$, $p<.001$); within Science, the domain difference was not reliable (ROM $=1.008$, $\Delta=+0.76\%$, $p=.191$). For complexity within major, Arts showed no reliable difference (ROM $=0.990$, $\Delta=-0.99\%$, $p=.088$), whereas Science showed slightly longer fixations for simple than complex (ROM $=1.013$, $\Delta=+1.33\%$, $p=.023$). Domain within complexity contrasts were not significant after adjustment.


\begin{table}[ht]
\centering
\footnotesize
\caption{Type III (Wald) tests for the Gamma GLMM of mean fixation duration (ms).}
\label{tab:dv2_type3}
\sisetup{
  input-signs = {},
  table-align-text-pre = false,
  table-align-text-post = false
}
\begin{tabular}{l S[table-format=4.3] S[table-format=1.0] c}
\toprule
{Effect} & {$\chi^2$} & {df} & {$p$} \\
\midrule
{Domain} & 5.424 & 1 & {$\phantom{<}.020$} \\
{Complexity} & 0.081 & 1 & {$\phantom{<}.776$} \\
{Major group} & 5.924 & 1 & {$\phantom{<}.015$} \\
{Domain $\times$ Complexity} & 0.185 & 1 & {$\phantom{<}.667$} \\
{Domain $\times$ Major group} & 1551.845 & 1 & {$<.001$} \\
{Complexity $\times$ Major group} & 468.631 & 1 & {$<.001$} \\
\bottomrule
\end{tabular}
\end{table}

\subsubsection{Regressive saccade proportion}

The proportion of regressive saccades was significantly influenced by domain and complexity, but not by the major group or any interactions between factors. As shown in Table \ref{tab:dv3_type3}, the proportion of regressive saccades was higher for history than science passages and higher for simple than complex passages overall. Because the complexity $\times$ major term trended ($p=.078$), we examined simple effects using a simple/complex contrast: Science majors showed higher odds of a regression in simple than in complex passages (OR$_{\text{simple/complex}}=1.131$, $\Delta=+13.1\%$, $p<.001$), whereas Arts showed no reliable difference (OR$_{\text{simple/complex}}=1.044$, $\Delta=+4.4\%$, $p=.247$). Domain effects within major were small and not reliable, and domain differences within complexity levels were non-significant after adjustment.


\begin{table}[ht]
\centering
\footnotesize
\caption{Type III (Wald) tests for the binomial GLMM of regressive saccade proportion.}
\label{tab:dv3_type3}
\sisetup{
  input-signs = {},
  table-align-text-pre = false,
  table-align-text-post = false
}
\begin{tabular}{l S[table-format=2.3] S[table-format=1.0] c}
\toprule
{Effect} & {$\chi^2$} & {df} & {$p$} \\
\midrule
{Domain} & 4.634 & 1 & {$\phantom{<}.031$} \\
{Complexity} & 13.138 & 1 & {$<.001$} \\
{Major group} & 0.195 & 1 & {$\phantom{<}.659$} \\
{Domain $\times$ Complexity} & 0.813 & 1 & {$\phantom{<}.367$} \\
{Domain $\times$ Major group} & 0.000 & 1 & {$\phantom{<}.988$} \\
{Complexity $\times$ Major group} & 3.105 & 1 & {$\phantom{<}.078$} \\
\bottomrule
\end{tabular}
\end{table}

\subsubsection{Mean Progressive Saccade Amplitude}

Mean progressive saccade amplitude was influenced by main effects of domain and complexity, and these effects were further qualified by a significant domain $\times$ major interaction and a marginal complexity $\times$ major interaction. As shown in Table \ref{tab:dv4_type3}, on the response scale, forward saccades were shorter in science than history for both groups, with a larger history over science advantage in Arts (ROM $=1.078$, $\Delta=+7.83\%$, $p<.001$) than in Science (ROM $=1.060$, $\Delta=+6.00\%$, $p=.001$). Both groups made shorter forward saccades in complex than simple passages, with a slightly larger reduction for science majors (simple over complex ROM $=1.049$, $\Delta=+4.93\%$, $p=.009$) than arts majors (ROM $=1.045$, $\Delta=+4.45\%$, $p=.018$). Domain within complexity contrasts were not reliable after adjustment.

\begin{table}[ht]
\centering
\footnotesize
\caption{Type III (Wald) tests for the Gamma GLMM of mean progressive saccade amplitude (after excluding top one percent).}
\label{tab:dv4_type3}
\sisetup{
  input-signs = {},
  table-align-text-pre = false,
  table-align-text-post = false
}
\begin{tabular}{l S[table-format=2.3] S[table-format=1.0] c}
\toprule
{Effect} & {$\chi^2$} & {df} & {$p$} \\
\midrule
{Domain} & 13.294 & 1 & {$<.001$} \\
{Complexity} & 6.254 & 1 & {$\phantom{<}.012$} \\
{Major group} & 3.775 & 1 & {$\phantom{<}.052$} \\
{Domain $\times$ Complexity} & 0.663 & 1 & {$\phantom{<}.416$} \\
{Domain $\times$ Major group} & 47.575 & 1 & {$<.001$} \\
{Complexity $\times$ Major group} & 3.393 & 1 & {$\phantom{<}.065$} \\
\bottomrule
\end{tabular}
\end{table}


\subsection{RQ2: Word-Level Processing}

\subsubsection{First Pass Duration}

First pass duration at the word-level was primarily affected by text domain and the major group, in addition to a significant quadratic effect of token length. As shown in Table \ref{tab:fpd_type3}, first pass durations were longer for science than history passages overall, and longer for science than arts majors overall. To probe the marginal complexity $\times$ major interaction, simple-effects contrasts showed shorter first pass durations in history than science for both majors (Arts ROM $=0.964$, $\Delta=-3.57\%$, $p<.001$; Science ROM $=0.980$, $\Delta=-2.03\%$, $p=.012$). Complexity effects were negligible for Arts (ROM $=0.999$, $p=.930$) and small but significant for Science (simple/complex ROM $=1.020$, $\Delta=+2.02\%$, $p=.014$). Domain-within-complexity contrasts were not reliable after adjustment.

\begin{table}[ht]
\centering
\footnotesize
\caption{Type III (Wald) tests for the log normal LMM of First Pass Duration.}
\label{tab:fpd_type3}
\sisetup{
  input-signs = {},
  table-align-text-pre = false,
  table-align-text-post = false
}
\begin{tabular}{l S[table-format=2.3] S[table-format=1.0] c}
\toprule
{Effect} & {$\chi^2$} & {df} & {$p$} \\
\midrule
{$z_{\text{loglen}}$} & 0.022 & 1 & {$\phantom{<}.883$} \\
{$z_{\text{loglen}}^{2}$} & 6.426 & 1 & {$\phantom{<}.011$} \\
{$z_{\text{logfreq}}$} & 0.538 & 1 & {$\phantom{<}.463$} \\
{$z_{\text{logfreq}}^{2}$} & 0.455 & 1 & {$\phantom{<}.500$} \\
{POS group} & 0.019 & 1 & {$\phantom{<}.889$} \\
{Domain} & 15.842 & 1 & {$<.001$} \\
{Complexity} & 1.811 & 1 & {$\phantom{<}.178$} \\
{Major group} & 6.717 & 1 & {$\phantom{<}.010$} \\
{Domain $\times$ Complexity} & 0.251 & 1 & {$\phantom{<}.616$} \\
{Domain $\times$ Major group} & 2.098 & 1 & {$\phantom{<}.147$} \\
{Complexity $\times$ Major group} & 3.660 & 1 & {$\phantom{<}.056$} \\
\bottomrule
\end{tabular}
\end{table}


\subsubsection{Skipping Probability}

The probability of skipping a word was influenced by token-level features (length, frequency, POS), experimental factors (domain, major group), and a significant interaction between domain and major. As shown in Table \ref{tab:skip_type3}, within both majors, history showed higher odds of being skipped than science (Arts OR $=1.155$, $p=.032$; Science OR $=1.283$, $p<.001$). Differences by complexity within major were not significant. By complexity, the history over science skipping difference held in simple passages (OR $=1.611$, $p=.005$) but not in complex passages.

\begin{table}[ht]
\centering
\footnotesize
\caption{Type III (Wald) tests for the binomial GLMM of skipping probability.}
\label{tab:skip_type3}
\sisetup{
  input-signs = {},
  table-align-text-pre = false,
  table-align-text-post = false
}
\begin{tabular}{l S[table-format=3.3] S[table-format=1.0] c}
\toprule
{Effect} & {$\chi^2$} & {df} & {$p$} \\
\midrule
{z\_loglen} & 248.545 & 1 & {$<.001$} \\
{z\_loglen$^{2}$} & 8.915 & 1 & {$\phantom{<}.003$} \\
{z\_logfreq} & 3.958 & 1 & {$\phantom{<}.047$} \\
{z\_logfreq$^{2}$} & 3.359 & 1 & {$\phantom{<}.067$} \\
{POS group} & 12.574 & 1 & {$<.001$} \\
{Domain} & 9.350 & 1 & {$\phantom{<}.002$} \\
{Complexity} & 0.893 & 1 & {$\phantom{<}.345$} \\
{Major group} & 9.652 & 1 & {$\phantom{<}.002$} \\
{Domain $\times$ Complexity} & 3.184 & 1 & {$\phantom{<}.074$} \\
{Domain $\times$ Major group} & 8.411 & 1 & {$\phantom{<}.004$} \\
{Complexity $\times$ Major group} & 2.684 & 1 & {$\phantom{<}.101$} \\
\bottomrule
\end{tabular}
\end{table}


\subsection{RQ3: Spatial Characteristics of Eye-Movements}

\subsubsection{Radius of Gyration}
The radius of gyration, a measure of spatial dispersion, was significantly affected by text domain and the major group, but not by linguistic complexity or any interactions. As shown in Table \ref{tab:rg_type3}, spatial dispersion was larger for history than science passages overall, and larger for arts than science majors overall. In light of the trend level complexity $\times$ major term, simple-effects contrasts indicated that the complexity effect was weak overall and reached significance only for Science majors (simple/complex ROM $=0.967$, $\Delta=-3.27\%$, $p=.039$). Domain-within-major effects mirrored the main effect pattern (Science majors: history $>$ science, ROM $=1.039$, $\Delta=+3.93\%$, $p=.008$; Arts: ROM $=1.022$, $\Delta=+2.17\%$, $p=.173$). Domain-within-complexity contrasts were marginal in simple sections and absent in complex sections.


\begin{table}[ht]
\centering
\footnotesize
\caption{Type III (Wald) tests for the log normal LMM of section-level \( \log(R_g) \).}
\label{tab:rg_type3}
\sisetup{
  input-signs = {},
  table-align-text-pre = false,
  table-align-text-post = false
}
\begin{tabular}{l S[table-format=3.3] S[table-format=1.0] c}
\toprule
{Effect} & {$\chi^2$} & {df} & {$p$} \\
\midrule
{$\log(n_{\text{fix}})$} & 113.214 & 1 & {$<.001$} \\
{Domain} & 7.751 & 1 & {$\phantom{<}.005$} \\
{Complexity} & 1.077 & 1 & {$\phantom{<}.299$} \\
{Major group} & 8.524 & 1 & {$\phantom{<}.003$} \\
{Domain $\times$ Complexity} & 1.204 & 1 & {$\phantom{<}.273$} \\
{Domain $\times$ Major group} & 0.632 & 1 & {$\phantom{<}.427$} \\
{Complexity $\times$ Major group} & 3.364 & 1 & {$\phantom{<}.067$} \\
\bottomrule
\end{tabular}
\end{table}


\subsubsection{Convex hull area}

Convex hull area was strongly influenced by main effects of both domain and complexity, with no significant effects involving the major group. As shown in Table \ref{tab:cha_type3}, hull area was larger for history than science passages overall, and larger for complex than simple passages overall. On the response scale, these main-effect directions held across majors, and the history-over-science difference was present in simple passages (ROM $=1.135$, $\Delta=+13.47\%$, $p=.009$) but not in complex passages (ROM $=1.050$, $p=.204$).


\begin{table}[ht]
\centering
\footnotesize
\caption{Type III (Wald) tests for the heteroskedastic Gaussian GLMM of \(\log(CHA_{\text{px}})\).}
\label{tab:cha_type3}
\sisetup{
  input-signs = {},
  table-align-text-pre = false,
  table-align-text-post = false
}
\begin{tabular}{l S[table-format=3.3] S[table-format=1.0] c}
\toprule
{Effect} & {$\chi^2$} & {df} & {$p$} \\
\midrule
{$\log(n_{\text{fix}})$} & 13.951 & 1 & {$<.001$} \\
{Domain} & 25.208 & 1 & {$<.001$} \\
{Complexity} & 400.500 & 1 & {$<.001$} \\
{Major group} & 0.102 & 1 & {$\phantom{<}.749$} \\
{Domain $\times$ Complexity} & 0.947 & 1 & {$\phantom{<}.330$} \\
{Domain $\times$ Major group} & 0.145 & 1 & {$\phantom{<}.703$} \\
{Complexity $\times$ Major group} & 0.177 & 1 & {$\phantom{<}.674$} \\
\bottomrule
\end{tabular}
\end{table}

\subsection{Summary of Main Findings}

Linguistic complexity was the dominant factor across all measures, producing large effects on fixation counts, forward step length, and spatial dispersion. Domain effects were reliable but modest, with history passages associated with longer forward steps and larger spatial footprints than science passages. Major-related effects appeared both as main effects and as targeted interactions, but were smaller than the effects of linguistic complexity and domain. Critically, several patterns diverged from theoretical predictions, particularly for fixation counts (H1) and first-pass durations (H2), which are examined in detail in the Discussion below.

\section{Discussion}

This study reexamines how disciplinary expertise, operationalized as major-by-domain congruence, relates to reading as domain and linguistic complexity vary. Across outcomes, linguistic complexity showed the largest and most consistent associations with navigation and spatial footprint; domain effects were reliable but modest to moderate; and expertise effects, indexed by major, appeared as targeted, context-specific modulations rather than pervasive shifts. Unless otherwise noted, main-effect statements are collapsed across other factors. We position each result relative to prior work and keep interpretation proportional to effect sizes, especially where effects are small or trend-level.



\subsection{RQ1: Global Eye-Movement Behavior}


Linguistic complexity was associated with markedly higher fixation counts for both groups, in line with classic findings that difficulty is accommodated by denser sampling and shorter forward steps \parencite{rayner1998, drieghe2004, engbert2002dynamical, engbert2005swift}. Progressive saccades were shorter in complex than in simple passages, converging with gradient-based accounts in which higher processing demands suppress long forward launches. These convergences underscore that complexity was the dominant factor in global navigation.




Domain effects were selective and moderate. Collapsing across other factors, fixation counts were higher for science than history. Simple-effects patterns revealed that science majors made more fixations on science than on history texts, whereas arts majors showed no domain difference. This pattern does not support H1, which predicted fewer fixations on expertise congruent texts because domain knowledge would make processing more efficient. Instead, it points to a selective expertise effect. This direction is compatible with a schema activation account, in which familiar discourse structures trigger richer representations with more information slots to be filled, which in turn invites additional fixations. This selective pattern is also consistent with Study 3 (\autoref{ch:study3}), where domain and complexity sometimes increased subjective effort while comprehension stayed high, so expertise can lead to deeper processing rather than cheaper processing when the material matches the learner. When readers encounter familiar discourse conventions, they may activate richer mental frameworks with more structural ``slots'' to be filled, for example, hypothesis–methods–results chains in science texts or sourcing and corroboration moves in historical narratives \parencite{goldman2002functional, goldman2016}. This activation invites additional fixations to bind local textual elements to those schema nodes, increasing sampling rather than reducing it \parencite{shanahan2008teaching, wineburg1991reading}. Under this account, domain-congruent expertise manifests not as global efficiency gains but as more thorough engagement with domain-relevant structure when capacity allows.
This reversal highlights a measurement ambiguity: fixation counts alone cannot adjudicate between efficient processing (fewer fixations) and schema-driven elaboration (more fixations). We therefore treat H1 as not supported, and future work should combine fixation metrics with independent indices of comprehension depth or schema access (e.g., think-aloud probes, knowledge checks) to discriminate between these accounts.


Both groups made many more fixations for complex than simple passages. Critically, by complexity level, the domain contrast was present for simple passages but absent for complex passages. This interaction supports a capacity-sharing account: when syntactic demands are moderate, readers with relevant background knowledge allocate additional sampling to in-domain texts; when linguistic load increases, capacity constraints override domain-selective policies \parencite{Gibson1998}.

Mean fixation duration showed domain and major main effects but no overall complexity effect, consistent with prior work showing that difficulty is absorbed more by navigation (where and how often to look) than by large increases in dwell time \parencite{rayner1998}. Within arts majors, fixations were longer in science than in history, consistent with reading outside one's familiar discourse genre \parencite{shanahan2008teaching}. Within science majors, domain differences in mean duration were not reliable. The complexity comparison within science majors showed slightly longer durations in simple than complex passages (ROM = 1.013, $\Delta$ = +1.33\%). This reversal is statistically reliable but very small in magnitude and may reflect compensatory processes under high load (shorter individual fixations paired with many more fixations overall). Given the modest effect size, we treat this as an exploratory finding requiring replication rather than as evidence for expertise-reversal effects \parencite{kalyuga2007, mcnamara1996are}.

Regressive-saccade proportion was higher for simple than for complex passages overall, a pattern that runs counter to the classic finding that difficulty increases regressions \parencite{drieghe2004}. The complexity × major interaction trended ($p = .078$), with science majors showing higher regression odds in simple than complex passages ($OR = 1.131, p < .001$) and arts majors showing no reliable difference. We interpret this unexpected reversal cautiously. One possibility is that our proportion-based measure combined with the strong reduction in forward-step length under high load indicates that rereading under complexity may have taken the form of short, local corrective fixations rather than long backward saccades, thereby reducing the proportion of movements classified as regressive. Alternatively, high linguistic load may have suppressed backward verification entirely as readers adopted a more conservative forward-only strategy. This pattern requires replication and ideally finer-grained classification of saccade types (e.g., short vs. long regressions) to adjudicate among these accounts. Although the omnibus domain effect reached significance, simple effects by major group were not reliable; we therefore avoid substantive domain claims for regressions at the subgroup level.

\subsection{RQ2: Word-Level Processing}


H2 predicted that expertise-congruent texts would show shorter first-pass durations, reflecting facilitated lexical access. The observed pattern contradicted this prediction. After controlling for token length and frequency, first-pass durations were shorter in history than in science for both majors (Arts: $ROM = 0.964$, $\Delta$ = $-3.57\%$, $p < .001$; Science: $ROM = 0.980$, $\Delta$ = $-2.03\%$, $p = .012$). This domain main effect held across major groups, meaning that science majors did not show faster word-level processing in science texts relative to history texts. We therefore treat H2 as not supported, and note that first-pass duration is unlikely to be a clean expertise index when domain-general lexical/conceptual density differs across genres.

Several factors may account for this unexpected pattern. First, the domain effect may reflect discourse-level differences in information density and lexical predictability that are independent of reader expertise. Scientific expository prose, even when syntactically simplified, often employs lower-frequency technical vocabulary and higher propositional density than historical narrative, which could slow initial lexical access for all readers \parencite{goldman2002functional}. Second, the materials manipulation targeted syntactic complexity while holding lexical frequency distributions roughly constant across domains; however, residual differences in conceptual density may have introduced baseline timing differences that swamped expertise-related facilitation. Third, and most fundamentally, the major-by-domain operationalization of expertise may be too coarse to capture word-level facilitation, which likely requires finer-grained alignment between specific lexical items and individual knowledge structures.

The complexity manipulation produced minimal token-level effects: no reliable difference for arts majors and a small simple-over-complex increase for science majors ($ROM = 1.020$, $\Delta$ = $+2.02\%$, $p = .014$). Given the modest size of this effect (approximately $2\%$), we treat it as consistent with but not diagnostic of expertise-sensitive timing adjustments.

Skipping replicated canonical effects of token length and frequency \parencite{rayner1998} and showed higher skipping in history than in science overall. Critically, this domain difference was qualified by a domain × complexity interaction: higher history skipping appeared in simple passages ($OR = 1.611$, $p = .005$) but disappeared in complex passages. This pattern is compatible with a capacity-sharing account in which higher linguistic load raises selection thresholds and reduces the scope for domain-specific parafoveal sampling policies \parencite{Gibson1998}. When demands are lower, readers may rely more on genre expectations and discourse schemas to guide word-skipping decisions \parencite{shanahan2008teaching, goldman2016}; when demands are higher, those expectations are constrained and selection becomes uniformly conservative across readers.

These word-level outcomes were observed while controlling for POS grouping. The domain contrast in first-pass duration persisted without a POS effect in that model, and the skipping results showed an independent POS main effect alongside the domain-by-complexity interaction. This pattern suggests that residual domain influences reflect differences in discourse structure and local cohesion not fully captured by length, frequency, and POS controls \parencite{goldman2002functional}.

Linking RQ2 back to global and spatial behavior, the reduced skipping for science texts in simple passages aligns with the generally shorter forward steps observed in science passages. When complexity rises, scan-paths broaden and forward steps shorten, offering a coherent picture in which word-level selection (skip vs. fixate) and movement planning are jointly tuned by linguistic load and genre expectations \parencite{engbert2005swift, schad2012zoom, torres2021eye}. Overall, complexity remained the dominant driver at the token-level, with major-related modulation appearing as targeted and context-dependent rather than pervasive.

\subsection{RQ3: Spatial Characteristics of Eye-Movements}


H3 predicted that linguistic complexity would enlarge the spatial footprint of reading and that expertise congruence would counteract this by concentrating attention more tightly. The results provided partial support for these predictions but revealed divergent patterns across the two spatial measures.

Convex hull area showed robust effects of both domain and complexity. Hull area was larger in history than in science passages and markedly larger in complex than in simple passages for all readers, with no interactions involving major group. The strong complexity effect generalizes prior demonstrations that processing difficulty broadens the spatial footprint of reading \parencite{schad2012zoom, torres2021eye} and aligns with SWIFT-style accounts in which elevated load expands attentional gradients \parencite{engbert2005swift}. The history-over-science difference is consistent with disciplinary practices that emphasize sourcing and contextualization in historical reading \parencite{goldman2016}. An alternative or complementary explanation is that scientific articles often follow a predictable, modular format such as the IMRAD structure (Introduction, Methods, Results, and Discussion), allowing experienced readers to adopt more spatially constrained scan-paths because they have learned where particular information types reside. Historical narratives, in contrast, may encourage wider visual search as readers integrate contextual information distributed across the page \parencite{goldman2002functional}. Given modest effect sizes and the absence of document-level structural coding in the present design, we treat this account as plausible but not definitive.

Duration-weighted radius of gyration, after controlling for log fixation count, showed domain and major main effects and a small complexity effect that appeared only for science readers (simple/complex ROM = 0.967, $\Delta$ = -3.27\%, p = .039). The divergence from convex hull area requires explanation. Convex hull area indexes the geometric extent of unique fixation landing positions, essentially how much of the page was covered regardless of revisitation patterns. Radius of gyration, in contrast, measures clustering around the duration-weighted centroid conditional on the number of fixations that occurred \parencite{schad2012zoom}. Thus, two readers could have identical hull areas but different radii if one reader distributed fixation time more evenly across landing sites while the other concentrated time in a few locations with occasional excursions.

The pattern observed here suggests that complexity uniformly broadened the footprint for all readers (as indexed by hull area), but did not uniformly shift how tightly fixations clustered around their geometric center once fixation count was controlled (as indexed by radius). Domain was associated with differences in typical spatial layout: history passages elicited both larger hulls and larger radii, consistent with either wider visual search or less centralized fixation distributions. Major group showed a main effect on radius (arts readers had larger radii than science readers overall) but no effect on hull area. This implies that arts and science majors covered similar total areas but differed in how they distributed fixation time within those areas. These divergences underscore the complementary information provided by different spatial indices and suggest that a single spatial measure may not fully capture the multidimensional adjustments readers make under varying discourse and processing conditions.


\subsection{Implications for Theory and Practice}

\subsubsection{Theoretical Implications}

The present pattern supports a capacity-sharing account of expertise in L2 academic reading. Linguistic complexity was the primary driver of navigation and spatial dispersion, with domain and major introducing targeted adjustments that depended on syntactic load and discourse conditions. When bottom-up integration demands are high, they dominate processing and constrain the scope for top-down compensation. When linguistic demands are moderate, readers can leverage domain knowledge to guide attention allocation, though this manifests as more thorough sampling (increased fixations to fill schema slots) rather than global efficiency gains.

This framework reconciles apparently contradictory findings in the literature. Studies reporting strong expertise effects likely employed materials that reduced structural uncertainty and provided clear discourse signals, allowing domain expectations to guide selection efficiently \parencite{wineburg1991reading, goldman2016}. Studies reporting weak or null expertise effects may have used texts with high syntactic demands that exceeded the threshold at which top-down knowledge can meaningfully compensate \parencite{Gibson1998}. The small and context-bound nature of expertise effects observed here clarifies that domain knowledge provides selective advantages that emerge when texts leave attentional headroom, not uniform facilitation across all aspects of processing.

\subsubsection{Methodological Contributions}

Spatial dispersion measures added value beyond classic timing indices. Convex hull area and radius of gyration captured the broadening of the visual field under load and discriminated layout and genre differences that mean fixation duration did not. Critically, these measures provided complementary rather than redundant information: hull area indexed geometric extent, while radius indexed clustering patterns conditional on fixation count. Including multiple spatial indices alongside temporal measures therefore provides a more complete picture of how readers allocate attention during L2 reading. Future work should continue to employ multiple spatial metrics and should explore finer-grained classifications of saccade types (e.g., short vs. long regressions) to resolve unexpected patterns such as the simple-over-complex regression finding observed here.

\subsubsection{Pedagogical Applications}

These findings provide specific guidance for scaffolding in pedagogical approaches such as Content and Language Integrated Learning (CLIL), where students face the dual challenge of mastering new content in a second language. Our results suggest a sequenced approach to materials design with three priorities in order of impact.

First, manage syntactic complexity, since this factor produced the largest and most consistent effects on cognitive processing across all measures. Simplifying sentence structure reduces the baseline processing load that all readers must handle before domain knowledge can provide any advantage (RQ1: fewer fixations, longer forward steps; RQ3: smaller spatial footprint).

Second, make discourse structure visible through predictable information placement, explicit signaling of relations, and consistent formatting. A legible discourse structure allows readers to apply domain knowledge without exceeding cognitive capacity. Within the CLIL framework, this translates to aligning content and language goals and providing focused practice with the connectives, particles, and text-structure markers that realize disciplinary reasoning \parencite{coyle2010}. Clear structure supports both word-level processing (RQ2: enabling strategic skipping of predictable elements) and concentrates the spatial footprint of reading (RQ3: reducing the need for wide visual search).

Third, once syntactic and structural supports are in place, leverage students' domain knowledge by aligning content with their disciplinary backgrounds. For mixed-ability cohorts, pairing simplified syntax with explicit structural cues is likely to yield the greatest gains. However, instructors should recognize that domain expertise manifests as more thorough engagement with familiar structures rather than as global time savings, particularly for learners still developing L2 automaticity.


\subsection{Limitations and Future Directions}



A primary limitation is the use of academic major as an indirect proxy for domain expertise. This grouping is not only quasi-experimental, precluding causal claims, but it also glosses over significant heterogeneity within disciplines; for example, the 'science' category included students from fields as diverse as chemistry and electrical engineering. Generalizability is further constrained by the sample of advanced L2 readers of Japanese, limiting conclusions beyond this population and orthography \parencite{kajii2001eye, white2012eye}. The low marginal \(R^{2}\) values for token-level models also indicate that our manipulations explain a small portion of total variance relative to individual reader and item differences. Methodologically, trend-level findings (e.g., ($p\approx .05$--$.10$) are treated as exploratory, and the necessity of trimming extreme values for spatial measures limits generality. Future work should therefore aim to measure disciplinary knowledge directly rather than relying on proxies, vary linguistic complexity parametrically, and manipulate domain-diagnostic cues to test whether the small expertise sensitivities observed here can be strengthened or reversed. The study also revealed unexpected patterns that diverged from theoretical predictions, notably the finding that expertise-congruent texts elicited more rather than fewer fixations (contrary to H1) and did not show facilitated word-level processing (contrary to H2). While we have offered post-hoc accounts grounded in schema activation and discourse density differences, these explanations require direct testing. Future work should manipulate schema availability experimentally, measure it directly through think-aloud protocols or knowledge probes, and employ materials that systematically vary both linguistic complexity and conceptual density to disentangle their contributions. At the lexical timescale, future work should align stimuli to participant-level knowledge (e.g., normed term familiarity and cloze probability) to cleanly test expertise effects on first-pass duration.




\section{Conclusion}

This study examined how disciplinary training, operationalized as academic major, relates to eye-movement behavior when L2 readers process academic texts that vary in domain and linguistic complexity. Across the measures analyzed, linguistic complexity emerged as the dominant influence on processing. The findings are consistent with a capacity sharing view in which bottom-up demands set the operating regime, and top-down expertise contributes only when materials make discourse structure easy to exploit.

The eye-tracking results showed a consistent pattern of effects. Linguistically complex passages elicited more fixations, shorter forward steps, and broader spatial dispersion for all readers. In contrast, the influence of domain expertise was more subtle and contingent on this processing load. Readers demonstrated modest advantages in their area of expertise, for example slightly higher skipping of predictable items in simple passages, but these benefits were most apparent in syntactically simple texts. As linguistic demands increased, these advantages diminished, suggesting that the costs of parsing can override knowledge based reading strategies.

For L2 instruction and materials design, the most reliable leverage is a sequenced approach. First, manage syntactic complexity, since it was the primary driver of cognitive load. Next, make discourse structure visible through predictable information placement and explicit connectives. Finally, align content with students' domain knowledge once these linguistic and structural supports are in place. Sequencing materials in this order should yield the largest and most transferable gains for comprehension in second language academic reading.
