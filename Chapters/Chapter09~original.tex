%************************************************
\chapter{Study 4: Eye-Movement Evidence for Expertise and Linguistic Complexity}
%************************************************


\section{Introduction}

University students across disciplines develop specialized reading expertise that shapes how they process academic texts. Disciplinary literacy research demonstrates that different academic domains emphasize distinct approaches to text comprehension, with each field developing domain-specific reading strategies and priorities \parencite{shanahan2008teaching, wineburg1991reading, goldman2002functional}. These strategies can substantially improve processing efficiency for in-domain texts but may be less effective when applied outside one's area of expertise \parencite{tothova2025eye}. Throughout, we use ``disciplinary expertise'' to mean readers’ domain specific literacy practices; operationally, expertise is indexed by major and text congruence (a science major reading science, or an arts or social sciences major reading history), which we test via domain-by-major effects.


For second language learners, the demands of disciplinary reading intersect with linguistic constraints. L2 readers must acquire domain-specific strategies while managing the complexity of academic prose \parencite{crossley2007linguistic, riccardi2019}. In Japanese, a multi-component writing system and head-final syntax introduce additional processing pressures \parencite{kajii2001eye,white2012eye}, and these L2-specific constraints can attenuate the very expertise-based strategies that typically support comprehension \parencite{nahatame2021text,zhang2024modelling}.

The interaction between disciplinary expertise and linguistic complexity is therefore theoretically consequential and empirically underexamined, especially in non-alphabetic languages. While complexity effects on reading are well documented \parencite{gibson1998linguistic,traxler2002processing}, less is known about whether expertise buffers these costs or is itself disrupted by them. The present experiment addresses this gap by testing how readers' major relates to eye-movement behavior across familiar versus unfamiliar domains (history, science) and across simple versus complex linguistic realizations. Using eye tracking, we quantify both temporal indices (e.g., fixation counts, durations) and spatial dispersion (e.g., scan-path geometry) to observe real-time dynamics of the expertise–complexity interaction. Given mixed prior findings and the processing demands of L2 Japanese, we treat complexity main effects as the primary expectation and expertise-related moderation as potentially small and contingent on discourse conditions.

\section{Literature Review}

\subsection{Domain Expertise and Disciplinary Literacy}

Disciplinary literacy fundamentally alters how readers approach academic texts, creating domain specific processing strategies that extend beyond general reading skills \parencite{shanahan2008teaching}. Expert readers across disciplines develop specialized attention patterns that reflect the epistemological priorities of their fields. These expertise effects are well documented in cognitive studies of reading. For example, when reading the same historical documents, professional historians read for subtext by treating texts as human and rhetorical artifacts, whereas high-achieving high school students mainly sought to extract information. This expert-novice gap was marked by a key behavior: sourcing, which historians used almost always (approximately 98 percent) and students used less than one-third of the time (approximately 31 percent) \parencite{wineburg1991reading}. Although this comparison involves professionals and adolescents, the underlying epistemic stance shift from information extraction to interrogating subtext and source is a defining feature of disciplinary literacy and is observable among undergraduates in domain congruent expository reading, the setting most comparable to the materials used in the present study. Scientists, in contrast, focus on causal mechanisms and methodological rigor \parencite{goldman2002functional}.


These expertise effects manifest clearly in eye tracking research. When instructed to read a text from a specific viewpoint (for example, as a tourist versus a military intelligence analyst), readers fixate longer on and later recall more perspective relevant information than irrelevant information \parencite{kaakinen2005perspective}. Expert readers spontaneously deploy these specialized strategies, demonstrating more focused processing of domain relevant information while strategically skipping less relevant content, which creates substantial processing advantages such as faster reading speeds, more efficient fixation patterns, and enhanced comprehension for domain congruent texts \parencite{tothova2025eye}. However, expertise can also create vulnerabilities. Research demonstrates a reverse cohesion effect in which high knowledge readers sometimes benefit from lower textual cohesion because they supply elaborative inferences that deepen understanding \parencite{mcnamara1996are}. Recent work positions disciplinary literacies as distinct constellations of epistemology, inquiry practices, and discourse conventions that vary across fields \parencite{goldman2016}. Large scale survey data further show at least three separable factors (source, analytic, expressive literacy) each dominating different subject areas \parencite{spires2018}. Together, this work suggests that expertise advantages should be conditional on discourse features and task goals, not uniform across contexts.

\subsection{Linguistic Complexity and Reading Processing}

Linguistic complexity spans multiple levels of description. At the lexical level, low frequency words reliably lengthen fixations and suppress word skipping \parencite{rayner1986lexical}. At the syntactic level, greater dependency distance between heads and dependents imposes integration costs that surface as longer first pass reading times and more regressions \parencite{gibson1998linguistic,traxler2002processing}. A growing body of work also points to discourse level complexity: texts that lack cohesive devices or exhibit unpredictable topic shifts further inflate fixation durations, and a recent corpus analysis showed that a combined index of lexical rarity, syntactic embedding, and cohesion accounts for roughly one third of variance in L2 eye movement measures, outperforming traditional readability formulas \parencite{zhang2024modelling}. These effects generalize across typologically diverse languages; large cross linguistic corpora confirm that dependency distance penalties apply in head-initial, head-final, and mixed orders alike \parencite{Futrell2020}.

Crucially, these complexity dimensions interact. Linguistically demanding passages trigger a characteristic cascade of behavioral adjustments, more regressions to re read earlier material, lower skipping rates, and wider dispersion of fixations, indicating tightly coupled temporal and spatial responses to processing difficulty \parencite{drieghe2004}. For the present study, this means that any expertise advantage must be evaluated under multi level constraints that jointly shape eye movement behavior.

\subsection{Expertise Complexity Interactions}

We use ``disciplinary expertise'' to denote domain specific literacy practices that guide selection, interpretation, and integration during academic reading. In this study, expertise is operationalized as major and text congruence: science majors reading science texts and arts or social sciences majors reading history texts. Analytically, we test expertise through domain-by-major interactions and simple effects contrasts that compare congruent and incongruent pairings. Two accounts make different predictions for how expertise interacts with linguistic complexity. Capacity limited views hold that syntactic integration and memory demands draw on shared resources, so as locality increases and constituents must be maintained for longer, resources are reallocated to bottom-up processing and domain guided strategies are curtailed \parencite{gibson1998linguistic}. Interactive views treat comprehension as jointly shaped by text signals and knowledge, where prior domain knowledge provides expectations about information structure and likely relations among propositions, which can stabilize parsing decisions and support coherence building \parencite{goldman2002functional}. Empirically, both outcomes appear. High knowledge readers can learn more from lower cohesion materials because they supply bridging inferences, while low knowledge readers benefit from increased cohesion, the classic reverse cohesion pattern \parencite{mcnamara1996are}. A broader expertise reversal literature reports that guidance and scaffolds assist novices but can become neutral or counterproductive for knowledgeable readers, consistent with a shift in the optimal balance between externally provided structure and self generated integration \parencite{tetzlaff2025}. These results imply that expertise advantages are conditional, not uniform.

When domain expectations are usable and linguistic demands remain moderate, expert readers should display more selective sampling, longer forward steps, and more economical word-level processing, for example, higher skipping of predictable items \parencite{rayner1998}. As linguistic demands increase, selection becomes uniformly conservative for all readers as resources are absorbed by integration. The Japanese context sharpens these expectations. Because L2 readers often underuse predictive cues like case markers and integration costs accumulate at clause final verbs, processing is pushed toward a reactive mode that limits the headroom in which domain expectations can operate \parencite{mitsugi2016use, tamaoka2019eye}. 

\subsection{Japanese Orthography and Processing Complexity}

The general complexity principles outlined above gain further nuance in Japanese, where the writing system supplies strong visual cues. The system exploits systematic mappings between script types and grammatical functions. Morphographic kanji encode content words, and phonographic kana mark grammatical morphemes \parencites{kajii2001eye, white2012eye}. This arrangement provides pre lexical word boundary cues that facilitate initial fixation targeting.

At the same time, the system has vulnerabilities. When mixed script text already provides segmentation cues, additional formatting changes such as inter word spacing provide benefits only in all kana contexts \parencite{sainio2007role}. Visual complexity within characters also matters: words containing high stroke kanji are fixated more often and read more slowly than words with low stroke kanji, even when lexical frequency is controlled \parencite{white2012eye}.


Flexible word order creates additional processing vulnerabilities. Japanese permits object fronting and scrambling, which amplifies dependency distances and forces readers to maintain displaced constituents in working memory. In native speakers, eye tracking reveals disproportionate integration costs at clause final verbs and increased regressions to case particles when non canonical orders increase \parencite{ueno2008relative,tamaoka2019eye}. Linguistically complex passages containing multiple embedded structures therefore predict both heightened particle rereading and more dispersed fixation patterns. These properties make Japanese an informative context for assessing whether expertise buffers or is disrupted by complexity.

\subsection{Second Language Processing Constraints}

When Japanese specific complexity is combined with second language status, processing demands compound. At the predictive processing level, visual world paradigms reveal that L2 learners often fail to use morphosyntactic cues predictively. For instance, one study showed that while native Japanese listeners used a nominative dative sequence to launch anticipatory eye movements to a theme object before it was mentioned, the same learner group did not \parencite{mitsugi2016use}. This suggests that the L2 learners did not use case marker information to generate predictions about upcoming verb arguments. Building on this, \textcite{mitsugi2017incremental} showed that this deficit extends to voice prediction; in a similar visual world study, native listeners exploited the same case cues to anticipate a passive verb, whereas English speaking learners did not shift their gaze until the voice morphology surfaced. This pattern forces L2 learners into a more reactive processing mode.

Integration timing for complex syntactic structures also differs systematically between native speakers and L2 learners. While native speakers show significant processing costs when dealing with structures such as syntactic scrambling \parencite{tamaoka2019eye}, these effects are amplified in L2 readers, for whom a variety of linguistic complexity features significantly contribute to processing effort during second language reading \parencite{nahatame2021text,zhang2024modelling}. Taken together, these constraints suggest that expertise advantages for L2 readers may be attenuated when texts require predictive use of case and voice cues or when integration demands are high.

\subsection{Spatial Measures of Processing Difficulty}

Understanding how linguistic complexity affects reading requires moving beyond traditional temporal measures to spatial indices that capture how attention is deployed across text. Metrics such as mean fixation duration, fixation count, and regression rate are canonical indices of cognitive load, revealing the timing of processing difficulty \parencite{rayner1998}. At the word-level, measures such as gaze duration and skipping probability reveal how readers make moment to moment lexical processing decisions during reading. While these temporal measures are foundational, spatial measures reveal how attention is distributed across the text. Recent research shows that these spatial characteristics of eye movement patterns provide converging evidence for processing difficulty \parencites[e.g.,][]{schad2012zoom, torres2021eye}. Radius of gyration, the root mean square distance of fixations from scan path centroids, increases with linguistic complexity and correlates more strongly with self reported effort than mean dwell time \parencite{schad2012zoom, Salvucci2000}.

Complementary spatial measures reach similar conclusions. Convex hull area, the polygon encompassing all fixations, correlates with readability measures across different text types \parencite{torres2021eye}. Electrophysiological studies show that processing long distance dependencies is associated with a P600 component, indicating increased syntactic integration costs \parencite{fiebach2002wh}. Computational models that incorporate parallel word processing and continuous attentional gradients reproduce these patterns by modeling how processing difficulty broadens attentional gradients and suppresses forward saccade generation \parencite{engbert2002dynamical, engbert2005swift, schad2012zoom}. When lexical difficulty or syntactic integration costs increase, simulated attention fields expand, fixation durations increase, and forward saccade amplitudes decrease, which collectively enlarges the spatial footprint of reading. Including spatial dispersion alongside timing measures therefore strengthens inferences about complexity effects.

\subsection{Research Questions}

The reviewed literature thus culminates in a theoretical impasse. On one hand, disciplinary literacy affords expert readers significant processing efficiencies within their domain. On the other, elevated linguistic complexity—compounded by the unique orthographic and syntactic demands of Japanese for L2 learners—imposes substantial cognitive loads that can disrupt fluent comprehension. The critical ambiguity lies at the intersection of these competing forces: are the top-down advantages of expertise robust enough to buffer against bottom-up processing difficulties, or do these difficulties consume finite cognitive resources, thereby neutralizing expert strategies? To adjudicate between these accounts, the present study employs eye-tracking methodology. By capturing millisecond-level data on both the timing (e.g., fixation durations) and spatial distribution (e.g., scan-path geometry) of visual attention, this approach can reveal the real-time dynamics of the expertise-complexity interaction.

\paragraph{Research question “Global timing” (RQ1).}
Does disciplinary expertise (major and text congruence) influence overall reading efficiency, and is this influence moderated by domain and linguistic complexity?

\textbf{H1.} Texts congruent with a reader’s disciplinary expertise (major and text congruence) will be read more efficiently than incongruent texts, as indexed by fewer fixations and shorter forward steps; this advantage will be smaller under complex passages than under simple passages.

\paragraph{Research question “word-level processing” (RQ2).}
Does disciplinary expertise modulate lexical level processing, and do these effects vary with domain and linguistic complexity?

\textbf{H2.} Expertise-congruent texts will show easier word-level processing than incongruent texts, as indexed by shorter first pass or gaze durations, lower regression probability, and higher skipping probability; this facilitation will be smaller under complex passages.

\paragraph{Research question “Spatial dispersion” (RQ3).}
Does disciplinary expertise shape the spatial distribution of visual attention, and are these spatial effects contingent on domain and linguistic complexity?

\textbf{H3.} Linguistic complexity will enlarge the spatial footprint of reading, as indexed by greater radius of gyration and larger convex hull area; expertise congruence will counteract this by concentrating attention more tightly, with the concentration effect smaller under complex passages.



\section{Methods}

\subsection{Experimental Design}

The present study analyzes eye-tracking data from a previously reported experiment [Author Citation]. While the original study examined learning outcomes using the same experimental paradigm, the eye-tracking data collected during those sessions were not analyzed and are reported here for the first time. We employed a mixed factorial design with two within-subject factors---domain (history versus science) and linguistic complexity (simple versus complex), and one between-subject factor---participants' major (science versus arts/social sciences). We treat disciplinary expertise as major and text congruence and test it through domain-by-major interactions and congruence-based simple effects.


\subsection{Participants and Design}
Participants were 36 Chinese international students (25 female, 11 male) recruited from a university population in Japan. Data from 2 participants were excluded from analysis due to data quality issues, resulting in a final sample of 34 participants (24 female, 10 male). All participants were native speakers of Mandarin with Japanese language proficiency (JLPT N1: n=24; N2: n=9; N3: n=1). Majors were categorized as science (n=19; e.g., electrical and electronic engineering, chemistry, materials science) or arts/social sciences (n=15; e.g., psychology, education, media studies). All participants provided informed consent, and the study was approved by [author affiliation]. For analysis, expertise is coded at the section-level as congruent (major matches text domain) or incongruent (major does not match text domain).




\subsection{Data Preprocessing}

Across the remaining participants, 408 trials were processed; 403 trials (98.8\%) met quality control, and 5 trials (1.2\%) were excluded for insufficient data quality (more than 50\% missing samples).

Tokens were segmented with GiNZA, a Japanese NLP library; for each token we extracted part-of-speech tags and lemma forms, then grouped parts of speech into content words (nouns, verbs, adjectives, adverbs) and function words (particles, auxiliary verbs, conjunctions, etc.) \parencite{ginza}. Token frequency (raw and per million words) was obtained from the Balanced Corpus of Contemporary Written Japanese (BCCWJ)\parencite{asada2024NLP}. To map fixations to words, we used a two-step process: a fixation was matched to any token whose bounding box was expanded (by 1.0\% of screen width and 1.7\% of screen height) to account for measurement error, then uniquely assigned to the candidate token with the closest geometric center. This mapping yielded 86.36\% valid text fixations successfully linked to tokens. To align with subsequent analyses, “on text” fixations refer to fixations whose coordinates fell within the predefined text region on the right side of the display.

The following trimming and inclusion rules were applied uniformly across analyses. At the token level, first pass duration values were log transformed and observations beyond $\pm 2.5$ standard deviations on the log scale were removed, retaining $N=24{,}835$ tokens (34 participants, 3771 unique tokens, 12 sections). At the section-level, spatial metrics used the following rules: radius of gyration ($R_g$) was computed from duration weighted on text fixations; sections with fewer than three fixations were excluded and the top one percent of $R_g$ values was trimmed. Convex hull area (CHA) was computed as the polygon area of the convex hull of unique on text fixation locations; sections with fewer than three unique fixations were excluded and the top one percent of CHA values was trimmed. For mean progressive saccade amplitude, the top one percent of amplitudes was trimmed prior to modeling based on section-level percentiles.




\subsection{Data Analysis}

All analyses were conducted in R (version 4.5.1; \cite{rcoreteam}). We fitted a series of mixed effects models, specifically linear mixed models (LMMs) for Gaussian and log normal specifications using lme4 \parencite{bates2015}, and generalized linear mixed models (GLMMs) for negative binomial, Gamma, binomial, and heteroskedastic specifications using glmmTMB \parencite{mcgillycuddy2025parsimoniously}. Unless otherwise specified, all models included the same experimental fixed effects, namely domain, linguistic complexity, major group, and all two way interactions; we refer to this as the standard fixed effects structure. Omnibus tests used Type\~III Wald $\chi^2$ tests. Post hoc estimates were obtained with the emmeans package using Tukey adjustment \parencite{emm}. We report odds ratios (OR) for logistic models, incidence rate ratios (IRR) for count models, and ratio of means (ROM) for models with log link or log response, along with percentage change $\Delta=(\text{ratio}-1)\times100\%$. Model adequacy and dispersion were evaluated with DHARMa and the performance package \parencite{dharma,performance}.




\paragraph{RQ1: Global eye movement patterns.}
All section-level dependent variables were modeled with the standard fixed effects structure. Every model included a random intercept for participant. An item intercept was included when supported by the data and removed when it produced singular fits. Model families were chosen to suit the measurement scale and observed distributional shape of each dependent variable. Fixation count used a negative binomial GLMM to accommodate overdispersed count data, and regressive saccade proportion used a binomial GLMM to handle a proportion with varying denominators; both included a random intercept for participant. Mean fixation duration and mean progressive saccade amplitude used Gamma GLMMs with a log link, appropriate for strictly positive and right skewed distributions; both models were weighted by the number of contributing observations (fixations and progressive saccades, respectively) and included random intercepts for participant and item.

\paragraph{RQ2: POS effects on word processing.}
Token level models included the standard fixed effects structure, nuisance controls for lexical factors, and random intercepts for participant, token, and section. First pass duration used a log normal LMM, since token level fixation durations are well approximated by a normal distribution after a log transform. These nuisance controls included standardized token length and frequency (linear and quadratic terms) and POS group (content versus function). Skipping probability used a binomial GLMM, appropriate for a binary token level outcome, with the same fixed effects and the same participant, token, and section intercepts.

\paragraph{RQ3: Spatial characteristics of eye movements.}

Both spatial dispersion measures are strictly positive and right skewed; therefore, analyses used a log scale and accounted for fixation count as a known determinant of dispersion. Radius of gyration (\(R_g\)) used a log normal LMM for \(\log(R_g)\), predicted from \(\log(n_{\text{fix}})\) and the standard fixed effects structure; the model included a random intercept for participant and an item intercept was tested and removed due to zero variance. Convex hull area (CHA) used a heteroskedastic Gaussian GLMM for \(\log(\texttt{CHA}_{\text{px}})\) with a random intercept for participant and a dispersion submodel. The dispersion submodel allowed residual variance to vary with task characteristics and fixation count by including domain by complexity, major group, and \(\log(n_{\text{fix}})\).




Descriptive statistics for all measures are shown in Table~\ref{tab:descriptives}. 
\begin{table}[!htbp]
\centering

\caption{Global eye-movement and word-processing descriptives by Domain $\times$ Complexity. Values are means (SD); section-level DVs computed over participant $\times$ section observations; token-level DVs over participant $\times$ token observations.}
\label{tab:descriptives}
\begin{tabular}{@{}l*{4}{>{\raggedleft\arraybackslash}p{2.48cm}}@{}}
\toprule
\textbf{Dependent Variable} & \textbf{History Simple} & \textbf{History Complex} & \textbf{Science Simple} & \textbf{Science Complex} \\
\midrule
\addlinespace
\multicolumn{5}{@{}l@{}}{\textbf{RQ1 (section-level)}} \\
\addlinespace[0.3ex]
Fixation count & 146 (70) & 261 (94) & 187 (94) & 243 (89) \\
Mean fixation duration (ms) & 351.15 (87.47) & 360.94 (36.49) & 366.76 (37.79) & 367.07 (74.38) \\
Regressive saccades (\%) & 32.78 (11.73) & 30.73 (3.84) & 31.53 (4.65) & 28.64 (7.17) \\
Mean prog. sacc. amp. (px) & 205.76 (67.11) & 179.58 (19.71) & 177.72 (25.97) & 186.33 (70.14) \\
\addlinespace
\multicolumn{5}{@{}l@{}}{\textbf{RQ2 (token-level)}} \\
\addlinespace[0.3ex]
First pass duration (ms) & 394.57 (183.90) & 377.87 (159.68) & 393.62 (172.75) & 397.85 (189.76) \\
Skipping probability (\%) & 63.58 (48.12) & 60.74 (48.83) & 54.48 (49.80) & 60.69 (48.85) \\
\addlinespace
\multicolumn{5}{@{}l@{}}{\textbf{RQ3 (section-level)}} \\
\addlinespace[0.3ex]
Radius of gyration (px) & 394.04 (54.17) & 426.18 (27.00) & 382.24 (22.02) & 419.13 (35.58) \\
Convex hull area (px$^2$) & 594895 (181629) & 974018 (84283) & 553086 (102295) & 922345 (140611) \\
\bottomrule
\end{tabular}

\end{table}

\section{Results}
For clarity, \texttt{domain} refers to the text domain (history; science) and \texttt{major group} refers to the participant's academic major (arts or social sciences; science). ``Disciplinary expertise'' is not a separate factor; it is the alignment of these two and is evaluated via the domain $\times$ major group interaction and simple effects that compare congruent and incongruent cells.

\subsection{Model diagnostics}

All models were evaluated for residual distribution, homoscedasticity, influential observations, convergence behavior, and dispersion using \texttt{performance} and \texttt{DHARMa}. 
For RQ1: the fixation count negative binomial GLMM showed good adequacy (AIC \(=2329.84\); \(R^2_m=0.338\), \(R^2_c=0.704\); uniformity \(p=.123\), dispersion \(p=.056\), zero inflation \(p=1.00\); \texttt{performance::check\_overdispersion} ratio \(=0.598\), \(p=.080\)); an item intercept was removed due to singularity. 
The mean fixation duration Gamma GLMM used section-level weights and included participant and item intercepts; AIC \(=370{,}054.2\); \(R^2_m=0.139\), \(R^2_c=0.894\); uniformity \(p=.667\), dispersion \(p=.140\). 
The regressive saccade proportion binomial GLMM converged with AIC \(=1383.15\); \(R^2_m=.063\), \(R^2_c=.795\); uniformity \(p=.928\), dispersion \(p=.288\), zero inflation \(p=.786\). 
The mean progressive saccade amplitude Gamma GLMM trimmed the top one percent, used section-level weights, and included participant and item intercepts; AIC \(=250{,}451.8\); \(R^2_m=0.115\), \(R^2_c=0.855\); uniformity \(p=.266\), dispersion \(p=.410\).

For RQ2 token level models: the first pass duration log normal LMM retained \(N=24{,}835\) tokens; AIC \(=24{,}971.9\); \(R^2_m=0.015\), \(R^2_c=0.099\). \texttt{DHARMa} showed the expected large sample deviation from perfect uniformity with no dispersion issue (dispersion \(p=.832\)). The skipping probability binomial GLMM showed AIC \(=82{,}363.8\); \(R^2_m=0.035\), \(R^2_c=0.141\); uniformity \(p=.863\), dispersion \(p=.950\); no observation level random effect was required.

For RQ3 spatial metrics: the radius of gyration log normal LMM trimmed the top one percent, used a participant intercept, and dropped an item intercept due to zero variance; AIC \(=-380.47\); \(R^2_m=0.412\), \(R^2_c=0.739\); uniformity \(p=.817\), dispersion \(p=.932\). The convex hull area heteroskedastic Gaussian GLMM used a participant intercept and a dispersion submodel; AIC \(=-201.86\); uniformity \(p=.662\), dispersion \(p=.114\). Pseudo \(R^2\) is not reported for this heteroskedastic specification. Collectively, these checks support the adequacy of all fitted models.

\subsection{RQ1: Global eye movement patterns}

\subsubsection{Fixation count on text}

The number of fixations on text showed significant main effects of domain, complexity, and major group, alongside trend level interactions for domain $\times$ complexity and domain $\times$ major. As shown in Table \ref{tab:dv1_type3}, fixation counts were higher for \textit{science} than \textit{history} passages overall, higher for \textit{complex} than \textit{simple} passages, and higher among \textit{science} than \textit{arts} majors. Given the trend level interactions, we probed simple effects to characterize these patterns. Science majors made more fixations on \textit{science} than on \textit{history} texts (IRR $=0.838$, $\Delta=-16.2\%$, $p=.002$), whereas Arts showed no domain difference (IRR $=0.986$, $\Delta=-1.4\%$, $p=.827$). Both groups made many more fixations for \textit{complex} than \textit{simple} passages (Arts IRR $=0.616$, $\Delta=-38.5\%$, $p<.001$; Science IRR $=0.647$, $\Delta=-35.3\%$, $p<.001$). By complexity, domain differences were present for \textit{simple} text (IRR $=0.739$, $\Delta=-26.2\%$, $p=.018$) but not for \textit{complex} text (IRR $=1.119$, $\Delta=+11.9\%$, $p=.380$).

\begin{table}[ht]
\centering
\caption{Type III (Wald) tests for the negative binomial mixed model of fixation counts on text.}
\label{tab:dv1_type3}
\begin{tabular}{lccc}
\toprule
Effect & $\chi^2$ & df & $p$ \\
\midrule
Domain & 4.853 & 1 & .028 \\
Complexity & 112.872 & 1 & $<$.001 \\
Major group & 8.678 & 1 & .003 \\
Domain $\times$ Complexity & 2.963 & 1 & .085 \\
Domain $\times$ Major group & 3.528 & 1 & .060 \\
Complexity $\times$ Major group & 0.341 & 1 & .559 \\
\bottomrule
\end{tabular}
\end{table}

\subsubsection{Mean fixation duration on text}
Mean fixation duration on text was not directly influenced by a main effect of complexity, but was instead shaped by main effects of domain and major group, as well as significant domain $\times$ major and complexity $\times$ major interactions. As shown in Table \ref{tab:dv2_type3}, on the response scale, within Arts, science passages elicited longer fixations than history (ROM $=0.966$, $\Delta=-3.40\%$, $p<.001$); within Science, the domain difference was not reliable (ROM $=1.008$, $\Delta=+0.76\%$, $p=.191$). For complexity within major, Arts showed no reliable difference (ROM $=0.990$, $\Delta=-0.99\%$, $p=.088$), whereas Science showed slightly longer fixations for simple than complex (ROM $=1.013$, $\Delta=+1.33\%$, $p=.023$). Domain within complexity contrasts were not significant after adjustment.



\begin{table}[ht]
\centering
\caption{Type III (Wald) tests for the Gamma GLMM of mean fixation duration (ms).}
\label{tab:dv2_type3}
\begin{tabular}{lccc}
\toprule
Effect & $\chi^2$ & df & $p$ \\
\midrule
Domain & 5.424 & 1 & .020 \\
Complexity & 0.081 & 1 & .776 \\
Major group & 5.924 & 1 & .015 \\
Domain $\times$ Complexity & 0.185 & 1 & .667 \\
Domain $\times$ Major group & 1551.845 & 1 & $<$.001 \\
Complexity $\times$ Major group & 468.631 & 1 & $<$.001 \\
\bottomrule
\end{tabular}
\end{table}

\subsubsection{Regressive saccade proportion}

The proportion of regressive saccades was significantly influenced by domain and complexity, but not by the major group or any interactions between factors. As shown in Table \ref{tab:dv3_type3}, the proportion of regressive saccades was higher for \textit{history} than \textit{science} passages and higher for \textit{simple} than \textit{complex} passages overall. Because the complexity $\times$ major term trended ($p=.078$), we examined simple effects using a simple/complex contrast: Science majors showed higher odds of a regression in \emph{simple} than in \emph{complex} passages (OR$_{\text{simple/complex}}=1.131$, $\Delta=+13.1\%$, $p<.001$), whereas Arts showed no reliable difference (OR$_{\text{simple/complex}}=1.044$, $\Delta=+4.4\%$, $p=.247$). Domain effects within major were small and not reliable, and domain differences within complexity levels were non-significant after adjustment.


\begin{table}[ht]
\centering
\caption{Type III (Wald) tests for the binomial GLMM of regressive saccade proportion.}
\label{tab:dv3_type3}
\begin{tabular}{lccc}
\toprule
Effect & $\chi^2$ & df & $p$ \\
\midrule
Domain & 4.634 & 1 & .031 \\
Complexity & 13.138 & 1 & $<$.001 \\
Major group & 0.195 & 1 & .659 \\
Domain $\times$ Complexity & 0.813 & 1 & .367 \\
Domain $\times$ Major group & 0.000 & 1 & .988 \\
Complexity $\times$ Major group & 3.105 & 1 & .078 \\
\bottomrule
\end{tabular}
\end{table}

\subsubsection{Mean progressive saccade amplitude}

Mean progressive saccade amplitude was influenced by main effects of domain and complexity, and these effects were further qualified by a significant domain $\times$ major interaction and a marginal complexity $\times$ major interaction. As shown in Table \ref{tab:dv4_type3}, on the response scale, forward saccades were shorter in science than history for both groups, with a larger history over science advantage in Arts (ROM $=1.078$, $\Delta=+7.83\%$, $p<.001$) than in Science (ROM $=1.060$, $\Delta=+6.00\%$, $p=.001$). Both groups made shorter forward saccades in complex than simple passages, with a slightly larger reduction for science majors (simple over complex ROM $=1.049$, $\Delta=+4.93\%$, $p=.009$) than arts majors (ROM $=1.045$, $\Delta=+4.45\%$, $p=.018$). Domain within complexity contrasts were not reliable after adjustment.

\begin{table}[ht]
\centering
\caption{Type III (Wald) tests for the Gamma GLMM of mean progressive saccade amplitude (after excluding top one percent).}
\label{tab:dv4_type3}
\begin{tabular}{lccc}
\toprule
Effect & $\chi^2$ & df & $p$ \\
\midrule
Domain & 13.294 & 1 & $<$.001 \\
Complexity & 6.254 & 1 & .012 \\
Major group & 3.775 & 1 & .052 \\
Domain $\times$ Complexity & 0.663 & 1 & .416 \\
Domain $\times$ Major group & 47.575 & 1 & $<$.001 \\
Complexity $\times$ Major group & 3.393 & 1 & .065 \\
\bottomrule
\end{tabular}
\end{table}

\subsection{RQ2: word-level processing}

\subsubsection{First pass duration}

First pass duration at the word-level was primarily affected by text domain and the major group, in addition to a significant quadratic effect of token length. As shown in Table \ref{tab:fpd_type3}, first pass durations were longer for \textit{science} than \textit{history} passages overall, and longer for \textit{science} than \textit{arts} majors overall. To probe the marginal complexity $\times$ major interaction, simple-effects contrasts showed shorter first pass durations in \textit{history} than \textit{science} for both majors (Arts ROM $=0.964$, $\Delta=-3.57\%$, $p<.001$; Science ROM $=0.980$, $\Delta=-2.03\%$, $p=.012$). Complexity effects were negligible for Arts (ROM $=0.999$, $p=.930$) and small but significant for Science (simple/complex ROM $=1.020$, $\Delta=+2.02\%$, $p=.014$). Domain-within-complexity contrasts were not reliable after adjustment.


\begin{table}[ht]
\centering
\caption{Type III (Wald) tests for the log normal LMM of First Pass Duration.}
\label{tab:fpd_type3}
\begin{tabular}{lccc}
\toprule
Effect & $\chi^2$ & df & $p$ \\
\midrule
$z_{\text{loglen}}$ & 0.022 & 1 & .883 \\
$z_{\text{loglen}}^{2}$ & 6.426 & 1 & .011 \\
$z_{\text{logfreq}}$ & 0.538 & 1 & .463 \\
$z_{\text{logfreq}}^{2}$ & 0.455 & 1 & .500 \\
POS group & 0.019 & 1 & .889 \\
Domain & 15.842 & 1 & $<$.001 \\
Complexity & 1.811 & 1 & .178 \\
Major group & 6.717 & 1 & .010 \\
Domain $\times$ Complexity & 0.251 & 1 & .616 \\
Domain $\times$ Major group & 2.098 & 1 & .147 \\
Complexity $\times$ Major group & 3.660 & 1 & .056 \\
\bottomrule
\end{tabular}
\end{table}

\subsubsection{Skipping probability}

The probability of skipping a word was influenced by token-level features (length, frequency, POS), experimental factors (domain, major group), and a significant interaction between domain and major. As shown in Table \ref{tab:skip_type3}, within both majors, history showed higher odds of being skipped than science (Arts OR $=1.155$, $p=.032$; Science OR $=1.283$, $p<.001$). Differences by complexity within major were not significant. By complexity, the history over science skipping difference held in simple passages (OR $=1.611$, $p=.005$) but not in complex passages.

\begin{table}[ht]
\centering
\caption{Type III (Wald) tests for the binomial GLMM of skipping probability.}
\label{tab:skip_type3}
\begin{tabular}{lccc}
\toprule
Effect & $\chi^2$ & df & $p$ \\
\midrule
z\_loglen & 248.545 & 1 & $<$.001 \\
z\_loglen$^{2}$ & 8.915 & 1 & .003 \\
z\_logfreq & 3.958 & 1 & .047 \\
z\_logfreq$^{2}$ & 3.359 & 1 & .067 \\
POS group & 12.574 & 1 & $<$.001 \\
Domain & 9.350 & 1 & .002 \\
Complexity & 0.893 & 1 & .345 \\
Major group & 9.652 & 1 & .002 \\
Domain $\times$ Complexity & 3.184 & 1 & .074 \\
Domain $\times$ Major group & 8.411 & 1 & .004 \\
Complexity $\times$ Major group & 2.684 & 1 & .101 \\
\bottomrule
\end{tabular}
\end{table}

\subsection{RQ3: Spatial characteristics of eye movements}

\subsubsection{Radius of gyration}
The radius of gyration, a measure of spatial dispersion, was significantly affected by text domain and the major group, but not by linguistic complexity or any interactions. As shown in Table \ref{tab:rg_type3}, spatial dispersion was larger for \textit{history} than \textit{science} passages overall, and larger for \textit{arts} than \textit{science} majors overall. In light of the trend level complexity $\times$ major term, simple-effects contrasts indicated that the complexity effect was weak overall and reached significance only for Science majors (simple/complex ROM $=0.967$, $\Delta=-3.27\%$, $p=.039$). Domain-within-major effects mirrored the main effect pattern (Science majors: history $>$ science, ROM $=1.039$, $\Delta=+3.93\%$, $p=.008$; Arts: ROM $=1.022$, $\Delta=+2.17\%$, $p=.173$). Domain-within-complexity contrasts were marginal in simple sections and absent in complex sections.


\begin{table}[ht]
\centering
\caption{Type III (Wald) tests for the log normal LMM of section-level \( \log(R_g) \).}
\label{tab:rg_type3}
\begin{tabular}{lccc}
\toprule
Effect & $\chi^2$ & df & $p$ \\
\midrule
$\log(n_{\text{fix}})$ & 113.214 & 1 & $<.001$ \\
Domain & 7.751 & 1 & .005 \\
Complexity & 1.077 & 1 & .299 \\
Major group & 8.524 & 1 & .003 \\
Domain $\times$ Complexity & 1.204 & 1 & .273 \\
Domain $\times$ Major group & 0.632 & 1 & .427 \\
Complexity $\times$ Major group & 3.364 & 1 & .067 \\
\bottomrule
\end{tabular}
\end{table}

\subsubsection{Convex hull area}

Convex hull area was strongly influenced by main effects of both domain and complexity, with no significant effects involving the major group. As shown in Table \ref{tab:cha_type3}, hull area was larger for \textit{history} than \textit{science} passages overall, and larger for \textit{complex} than \textit{simple} passages overall. On the response scale, these main-effect directions held across majors, and the history-over-science difference was present in \textit{simple} passages (ROM $=1.135$, $\Delta=+13.47\%$, $p=.009$) but not in \textit{complex} passages (ROM $=1.050$, $p=.204$).

\begin{table}[ht]
\centering
\caption{Type III (Wald) tests for the heteroskedastic Gaussian GLMM of \(\log(\texttt{CHA}_{\text{px}})\).}
\label{tab:cha_type3}
\begin{tabular}{lccc}
\toprule
Effect & $\chi^2$ & df & $p$ \\
\midrule
$\log(n_{\text{fix}})$ & 13.951 & 1 & $<$.001 \\
Domain & 25.208 & 1 & $<$.001 \\
Complexity & 400.500 & 1 & $<$.001 \\
Major group & 0.102 & 1 & .749 \\
Domain $\times$ Complexity & 0.947 & 1 & .330 \\
Domain $\times$ Major group & 0.145 & 1 & .703 \\
Complexity $\times$ Major group & 0.177 & 1 & .674 \\
\bottomrule
\end{tabular}
\end{table}



\section{Discussion}

This study reexamines how disciplinary expertise, operationalized as major-by-domain congruence, relates to reading as domain and linguistic complexity vary. Across outcomes, linguistic complexity showed the largest and most consistent associations with navigation and spatial footprint; domain effects were reliable but modest to moderate; and expertise effects, indexed by major, appeared as targeted, context-specific modulations rather than pervasive shifts. Unless otherwise noted, main-effect statements are collapsed across other factors. We position each result relative to prior work and keep interpretation proportional to effect sizes, especially where effects are small or trend-level.



\subsection{RQ1: Global eye movement behavior}

Linguistic complexity was associated with markedly higher fixation counts for both groups, in line with classic findings that difficulty is accommodated by denser sampling and shorter forward steps \parencite{rayner1998, drieghe2004, engbert2002dynamical, engbert2005swift}. Progressive saccades were shorter in complex than in simple passages, converging with gradient-based accounts in which higher processing demands suppress long forward launches. These convergences underscore that complexity was the dominant factor in global navigation here.

Domain effects were selective and moderate. Collapsing across other factors, fixation counts were higher for science than history. Simple-effects patterns indicated that science majors made more fixations on science than on history texts, whereas arts majors showed no domain difference. Both groups made many more fixations for complex than simple passages. By complexity, the domain contrast was present for simple passages but not for complex passages. A cautious interpretation is that when syntactic demands are low enough to leave attentional headroom, readers with relevant background knowledge allocate slightly more sampling to in-domain texts \parencite{shanahan2008teaching, wineburg1991reading}. One plausible mechanism is schema activation: familiar discourse conventions activate richer mental frameworks with more “slots” to be filled (e.g., hypothesis–methods–results chains in science; sourcing/corroboration moves in history), which in turn invites additional fixations to bind local textual elements to those schema nodes \parencite{goldman2002functional, goldman2016}. Under higher linguistic load, capacity sharing likely constrains such domain-selective policies \parencite{gibson1998linguistic}.

Mean fixation duration showed domain and major main effects but no overall complexity effect, echoing work where difficulty is absorbed more by where and how often to look than by large dwell-time increases \parencite{rayner1998}. Within arts, fixations were longer in science than in history, consistent with reading outside one's home discourse \parencite{shanahan2008teaching}. Within science, domain differences in mean duration were not reliable, and the complexity comparison showed slightly longer durations in simple than complex passages. This difference is statistically reliable yet very small; this small difference tentatively echoes expertise-reversal accounts \parencite{kalyuga2007, mcnamara1996are, tetzlaff2025} and should not be treated as a robust demonstration.

Regressive-saccade proportion was higher for simple than for complex passages overall, and the complexity$\times$major term trended; accordingly we examined simple effects: the odds of a regression were higher for simple than complex passages among science majors and not reliably different among arts, consistent with difficulty-driven verification being curtailed under higher linguistic load \parencite{drieghe2004}. Although the omnibus domain effect reached significance, the simple effects by major did not; we therefore avoid substantive domain claims for regressions at the subgroup level. Note that this simple $>$ complex pattern runs counter to classic ‘more difficulty → more regressions’ findings; our proportion measure and the strong reduction in forward-step length under higher load suggest that rereading may have been more local (short, corrective fixations) rather than long backward movements in the complex passages.


\subsection{RQ2: word-level processing}

After controlling length and frequency, first-pass durations were shorter in history than in science for both majors (collapsed), consistent with discourse-level predictability and cueing beyond lexical variables \parencite{rayner1998, goldman2002functional}. The complexity manipulation produced only modest token-level effects: none for Arts and a small simple\(>\)complex increase for Science (about 2\%). Given its size, we treat this as consistent-with rather than diagnostic of expertise-sensitive timing adjustments.

Skipping replicated canonical length/frequency influences \parencite{rayner1998} and showed higher skipping in history than in science (collapsed), but by complexity the domain contrast appeared in simple passages and disappeared in complex passages. This pattern is compatible with capacity sharing in which higher linguistic load raises selection thresholds and reduces room for domain-specific parafoveal policies \parencite{gibson1998linguistic}. When demands are lower, readers may rely more on domain expectations (schema-like guidance) to select which tokens can be safely skipped \parencite{shanahan2008teaching, goldman2016}; when demands are higher, those expectations are constrained and selection becomes uniformly conservative.

These word-level outcomes were observed while controlling for POS grouping. The domain contrast in first-pass duration persisted without a POS effect in that model, and the skipping results showed an independent POS main effect but retained the domain-by-complexity pattern. This mix suggests that residual domain influences are not reducible to part-of-speech distribution alone and likely reflect differences in discourse structure and local cohesion not captured by length/frequency/POS controls \parencite{goldman2002functional}.

Linking RQ2 back to global and spatial behavior, the simple-passage reduction in skipping for science aligns with the generally shorter forward steps in science texts. Separately, when complexity rises, scan-paths broaden and forward steps shorten, offering a coherent picture in which word-level selection (skip vs.\ fixate) and movement planning are tuned jointly by linguistic load and genre expectations \parencite{engbert2005swift, schad2012zoom, torres2021eye}. Overall, complexity remained the dominant driver at the token level, and major-related modulation was targeted and subtle.

\subsection{RQ3: Spatial characteristics of eye movements}

Convex-hull area was larger in history than in science and markedly larger in complex than in simple passages for both majors, with no interactions involving major. The strong complexity effect generalizes demonstrations that processing difficulty broadens the spatial footprint of reading \parencite{schad2012zoom, torres2021eye} and is compatible with SWIFT-style attentional-gradient expansion under load \parencite{engbert2005swift}. The history\(>\)science difference is consistent with disciplinary practices that emphasize sourcing and contextualization \parencite{goldman2016}, but it may also reflect differences in information structure. Scientific articles often follow a predictable, modular format (e.g., IMRAD), so experienced readers learn where particular information types reside and can adopt more spatially constrained scan-paths; historical narratives, by contrast, may encourage wider visual search as readers integrate sources and context across the page \parencite{goldman2002functional}. Given modest effect sizes and the absence of document-level coding of structure here, we treat this account as plausible rather than definitive.

Duration-weighted radius of gyration, controlling for $\log(n_{\text{fix}})$, showed domain and major main effects and a small complexity effect that appeared only for science readers (simple/complex ROM \(=0.967\)). The divergence from hull area is expected: hull area tracks the extent of unique landing positions, whereas radius indexes clustering around the weighted centroid conditional on how many fixations occurred \parencite{schad2012zoom}. Put together, complexity broadened the footprint for everyone; domain was associated with differences in typical layout; and major with small geometric adjustments rather than wholesale changes.



\subsection{Implications for theory and practice}

The present pattern suggests a calibrated view of expertise in L2 academic reading. Linguistic complexity was the primary driver of navigation and spatial footprint, with domain and major introducing targeted adjustments that depended on discourse conditions. Theoretically, this favors a capacity sharing account in which bottom-up demands dominate, with limited top-down compensation that emerges when texts leave attentional headroom. The small and context bound nature of expertise effects also clarifies why studies sometimes report mixed results. When materials reduce structural uncertainty and allow expectations to guide selection, experts can concentrate attention more efficiently. When syntactic demands rise, those policies are constrained and group differences shrink. Methodologically, spatial dispersion adds value beyond classic timing measures. Convex hull area and radius of gyration captured broadening of the visual field under load and discriminated layout and genre differences that mean fixation duration did not. Including spatial indices alongside timing therefore provides a more complete picture of how readers allocate attention when processing L2 Japanese. For practice, these findings provide specific guidance for scaffolding in pedagogical approaches like Content and Language Integrated Learning (CLIL), where students face the dual task of mastering new content and a new language. Our results suggest that the first priority should be managing syntactic complexity, since this factor produced the largest and most consistent changes in cognitive processing. To leverage domain knowledge effectively, texts should feature a legible discourse structure through predictable information placement and explicit signaling of relations, allowing readers to apply their expertise without exceeding cognitive capacity. Within the CLIL framework, this involves aligning content and language goals and providing focused practice with the connectives and particles that realize disciplinary reasoning \parencite{coyle2010}. For mixed ability cohorts, pairing simplified syntax with these clear structural cues is likely to yield the greatest gains. Mapped to our RQs: reducing syntactic load chiefly improves global efficiency (RQ1); explicit cohesion supports word-level processing (RQ2); and clear structure/layout concentrates the spatial footprint of reading (RQ3).

\subsection{Limitations and future directions}



A primary limitation is the use of academic major as an indirect proxy for domain expertise. This grouping is not only quasi-experimental, precluding causal claims, but it also glosses over significant heterogeneity within disciplines; for example, the 'science' category included students from fields as diverse as chemistry and electrical engineering. Generalizability is further constrained by the sample of advanced L2 readers of Japanese, limiting conclusions beyond this population and orthography \parencite{kajii2001eye, white2012eye}. The low marginal \(R^{2}\) values for token-level models also indicate that our manipulations explain a small portion of total variance relative to individual reader and item differences. Methodologically, trend-level findings (e.g., \(p\approx .05\)–.10) are treated as exploratory, and the necessity of trimming extreme values for spatial measures limits generality. Future work should therefore aim to measure disciplinary knowledge directly rather than relying on proxies, vary linguistic complexity parametrically, and manipulate domain-diagnostic cues to test whether the small expertise sensitivities observed here can be strengthened or reversed.




\section{Conclusion}

This study examined how disciplinary training, operationalized as academic major, relates to eye movement behavior when L2 readers process academic texts that vary in domain and linguistic complexity. Across the measures analyzed, linguistic complexity emerged as the dominant influence on processing. The findings are consistent with a capacity sharing view in which bottom-up demands set the operating regime, and top-down expertise contributes only when materials make discourse structure easy to exploit.

The eye tracking results showed a consistent pattern of effects. Linguistically complex passages elicited more fixations, shorter forward steps, and broader spatial dispersion for all readers. In contrast, the influence of domain expertise was more subtle and contingent on this processing load. Readers demonstrated modest advantages in their area of expertise, for example slightly higher skipping of predictable items in simple passages, but these benefits were most apparent in syntactically simple texts. As linguistic demands increased, these advantages diminished, suggesting that the costs of parsing can override knowledge based reading strategies.

For L2 instruction and materials design, the most reliable leverage is a sequenced approach. First, manage syntactic complexity, since it was the primary driver of cognitive load. Next, make discourse structure visible through predictable information placement and explicit connectives. Finally, align content with students' domain knowledge once these linguistic and structural supports are in place. Sequencing materials in this order should yield the largest and most transferable gains for comprehension in second language academic reading.

%below is original
%Across L2 Japanese academic reading, linguistic complexity was the dominant influence on eye movement behavior, while domain and major produced selective, context dependent modulations. This pattern supports a capacity sharing view in which bottom-up demands set the operating regime and top-down expertise contributes when materials afford the use of expectations. A second contribution is methodological, spatial dispersion measures captured meaningful variance that timing measures alone did not, strengthening inferences about how difficulty broadens the spatial footprint of reading. Together these results delineate when domain knowledge helps and when it is neutralized, a constraint that theory and pedagogy need to respect.

%Answers to the research questions are as follows. For RQ1, complexity was associated with more fixations and shorter forward steps for both majors. Domain and major showed targeted navigation differences that were clearest in simple passages, for example slightly greater sampling by science majors in science. Mean fixation duration differences were statistically reliable yet small. For RQ2, domain influenced early timing and skipping mainly when linguistic demands were lower, complexity effects at the token level were modest, and major by complexity interactions were limited. For RQ3, convex hull area increased strongly with complexity for all readers, and a small centroidal dispersion adjustment appeared only for science readers.

%Taken together, the study advances an evidence based boundary condition on expertise in L2 academic reading. When syntactic demands are high, expertise advantages are limited. When demands are moderate and discourse structure is legible, expertise can guide selection and concentration of attention. For designers of learning materials, this implies that managing complexity is the first lever, and that drawing on disciplinary expectations is most productive when texts make structure easy to exploit.
